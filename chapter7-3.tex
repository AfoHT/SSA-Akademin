\section{Jonosfärskikten}
\textbf{
HAREC a.\ref{HAREC.a.7.3}\label{myHAREC.a.7.3}
}

Jonosfären har fått namnet från begreppet jon, som är en fri elektron
eller annan laddad partikel. Jonisering- elektrisk uppladdningav
jordatmosfären sker mellan c:a 40 - 400 km över jordytan. Där är
lufttrycket tillräckligt lågt för att joner skall kunna röra sig fritt
under avsevärd tid utan att kollidera och återförena sig till neutrala
atomer.

När en radiovåg passerar genom ett joniseratskikt i atmosfären,
kanvågen ändra riktning, vilket kallas för refraktion. För att
refraktion skall uppstå måste i första hand två villkor uppfyllas, det
är tillräckligt tät jonisering och tillräckligt lång våglängd. Under
``gynnsamma'' omständigheter kan vågorna till och med böjas av ner mot
jorden, vilket är den viktigaste förutsättningen för långväga
radioförbindelser på kortvåg.

Joniseringen av atmosfären är emellertid oregelbunden och varierar
bl. a. med höjden över jordytan, solinstrålning, tidpunkt m.m.  Ett
antal joniserade skikt kan definieras.  Se Bild II 7-7.

\subsection{D-skiktet}

D-skiktet förekommer under den ljusa delen av dygnet på en höjd av c:a
50 - 90 km. På 70 - 90 km höjd orsakas joniseringen huvudsakligen av
röntgenstrålar från solen, medan den kosmiska strålningen har störst
påverkan på 50-70 km höjd. D-skiktet dämpar de infallande
radiovågorna, med största verkan i kortvågsområdets lågfrekventa del
och under de ljusaste timmarna under sommaren.  D-skiktet har dålig
reflexionsförmåga och verkar hindrande på långdistansförbindelser.

\subsection{E-skiktet}

E-skiktet (Kenelly-Heaviside-skiktet) är det lägsta reflekterande
jonosfärskiktet Det förekommer på en höjd av c:a 90-140 km och är mest
koncentrerat på c:a 110 km höjd. E-skiktet alstras av att ultraviolett
strålning joniserar syreatomer. Skiktet reflekterar vågor bäst i
kortvågsområdets lågfrekventa del och är kraftigast under den ljusa
delen av dygnet. På grund av D-skiktets dämpande verkan under de
ljusaste timmarna är E-skiktet mest användbart under grynings- och
skymningstimmarna.

Ett säsongmaximum i reflexionsförmågan inträffar under
sommaren. Förbindelseavstånd på upp till 2000 km är möjliga.

\subsection{Mögel-Dellinger-effekten}

Strålning från gasutbrott på solytan kan jonisera D-skiktet så
kraftigt, att alla radiovågor med frekvenser över c:a 1 MHz dämpas
helt.  Radiotrafik som baseras på vågutbredning via jonosfären är då
omöjlig att genomföra under en tidsrymd av ett antal minuter upp till
flera timmar - det blir ``black out''.

\subsection{Sporadiska E-skiktet}

Den starkare solinstrålningen under sommaren orsakar en kraftigare
jonisering i den lägre jonosfären än under vintern. Inom Eskiktet
bildas då sporadiska tunna molnlika partier med mycket hög
joniseringsgrad och stor reflexionsförmåga, det s.k. sporadiska
E-skiktet (\(E_s\)). Vågutbredningen via \(E_s\) är mycket olika på
olika latituder och är bäst omkring 40:e breddgraden. Mycket goda

$>>>>>$ TODO: här blandas latitud och breddgrad - välj det ena

långväga förbindelser kan uppnås.

Bild II 7-7 Jonosfärskikten

\subsection{F-skiktet}

F-skiktet är det högst liggande jonosfärskiktet. Det förekommer såväl
dag- som nattetid på en höjd av 140-500 km. Den nedre del av skiktet,
140 - 200 km, uppvisar andra variationer än den övre delen. F-skiktet
beskrivs därför som två skikt, \(\mathrm{F_1}\) upp till ca 200 km
höjd och \(\mathrm{F_2}\) över denna höjd.

Liksom E-skiktet, påverkas \(\mathrm{F_1}\)-skiktet kraftigt av
instrålningen från solen. Det når sin högsta joniseringsgrad ungefär
en timme efter högsta lokala solstånd och förekommer endast under
sommaren. Under natten förenar sig \(\mathrm{F_1}\)- och
\(\mathrm{F_2}\)-skikten till ett enda F-skikt.

\(\mathrm{F_2}\)-skiktet är det skikt som varierar mest i tiden och
rummet. Den högsta joniseringsgraden inträffar vanligen sent efter
högsta lokala solstånd, ibland under aftontimmarna.  Skiktets maximala
jonisering är på 250 - 350 km höjd på mellanlatituder och på 350 - 500
km höjd vid ekvatorn. På mellanlatituder ligger den största
elektrontätheten i skiktet högre under natten än under dagen. Vid
ekvatorn är förhållandet omvänt.

Reflexioner i \(\mathrm{F_2}\)-skiktet möjliggör att stora
avstånd kan överbryggas (1 hopp = 3000 - 4000 km).

Bild II 7-8 Jonosfärsutbredning.

\subsection{Höjd till reflekterande skikt}

När en radiovåg, som riktas rakt uppåt, träffar jonosfären kan den antingen
\begin{itemize}
\item absorberas, sugas upp,
\item reflekteras,
\item tränga igenom.
\end{itemize}

Vilket som inträffar beror på den använda frekvensen. Ju högre
frekvensen är på den uppåtriktade radiovågen, desto högre upp i ett
atmosfärskikt kommer avböjningen tillbaka att inträffa. Höjden till
skiktet beräknas ur radiovågens utbredningshastighet och
utbredningstid fram och åter mellan skiktet och jordytan.

\subsection{Kritisk frekvens}
\textbf{
HAREC a.\ref{HAREC.a.7.4}\label{myHAREC.a.7.4}
}

Vid en viss övre frekvens upphör reflexionen i atmosfärskiktet och
vågen går ut i rymden i stället för ner till jordytan. Denna frekvens
kallas den \emph{kritiska frekvensen}, som varierar med
joniseringsgraden i atmosfären. Den kritiska frekvensen är högst vid
högt solfläckstal, såväl i E- som i F-skikten, eftersom
joniseringsgraden då är störst. Den kritiska frekvensen för E-skiktet
varierar mellan c:a 1 - 4 MHz beroende på tidpunkt i solfläckscykeln
och tid på dagen. Den kritiska frekvensen för F-skiktet varierar med
tid på dagen, årstid och skede i solfläckscykeln.  Den kan variera
från 2 - 3 MHz natten under ett solfläcksminimum till 12 - 13 MHz på
dagen under ett solfläcksmaximum.

\subsection{Kritisk vinkel}

Rymdvågen måste träffa ett joniserat atmosfärskikt med en tillräckligt
flack vinkel för att reflekteras, den s.k. kritiska vinkeln. Denna
vinkel är frekvensberoende. Allt eftersom den utsända frekvensen ökas
ytterligare över den kritiska frekvensen, måste vågen träffa
atmosfärskiktet i en allt flackare vinkel för att vågen skall
reflekteras mot jordytan.  att sända ut vågen i mycket flack vinkel
mot \(\mathrm{F_2}\)-skiktet kan långa distanser överbryggas den vid
frekvenser som är upp till 3.5 kritiska frekvensen.

Så snart den kritiska frekvensen är högre än frekvensområdet för ett
amatörband är det alltså möjligt att kommunicera över våg på detta
band. Det kan ske över alla avstånd, allt ifrån skipavståndet till det
som avgörs av utbredningsförlusterna.

\subsection{Högsta användbara frekvens (MUF)}
\textbf{
HAREC a.\ref{HAREC.a.7.6}\label{myHAREC.a.7.6}
}

Radiovågorna vandrar från sändaren till en avlägsen mottagare genom
att reflekteras en eller flera gånger i jonosfären och jordytan. För
detta kan frekvensen inte vara högre än den högsta användbara
frekvensen, \emph{Maximum Usable Frequency - MUF} för en viss
överföringssträcka.

MUF är högst mitt på dagen eller eftermiddag. Allra högst är MUF under
perioder av högt solfläckstal och kan då komma upp till över 30
MHz. Under tidiga morgontimmar sjunker MUF ofta under 5 MHz.

De jonosfäriska förlusterna är lägst nära MUF och ökar snabbt under
dagtid för lägre frekvenser.

Aktuella MUF-data publiceras periodiskt i olika media, men kan också
överslagsberäknas med hjälp av speciella datorprogram.

\subsection{Optimal trafikfrekvens (FOT)}

I praktiken är det av intresse att veta det frekvensområde där
kommunikation bäst kan genomföras.

Rekommenderad övre frekvensgräns för en tillförlitlig radioförbindelse
kallas \emph{optimal traffic frequency} - FOT - och väljs något under
MUF som marginal för oregelbundenheter och turbulens i jonosfären,
liksom för korttidsavvikelser från det förutsagda månatliga
medianvärdet för MUF. FOT är vaniigen ungefär 15\% lägre än MUF.

\subsection{Lägsta användbara frekvens (LUF)}

Ju lägre sändningsfrekvens som väljs, desto mer dämpas vågorna i
jonosfären, intill den frekvens då de inte kan uppfattas. Den lägsta
användbara rekvensen \emph{Lowest Usable Frequency} - LUF - är den
frekvens som ger tillfredsställande kommunikation för en viss
utbredningsväg och vid en viss tidpunkt.

Vid frekvenser under LUF är mottagning inte möjlig eftersom brusnivån
då är för hög.  Ju mer frekvensen höjs över LUF, desto bättre blir
signal-brus-förhållandet.

Till skillnad från MUF, som endast påverkas av de janesfäriska
förhållandena, kan till en del påverkas genom utsänd effekt och
bandbredd. Generellt kan LUF sänkas c:a 2 MHz för varje 10 dB ökning
av E.R.P.

\subsection{Vågutbredningsförutsägelser}

Det görs regelmässiga förutsägelser av de janesfäriska
förhållandena. Fortlöpande fysiska observationer, statistisk och
matematisk bearbetning ligger till grund för förutsägelserna, vilka
bl.a. utnyttjas för att planera radiotrafiken.
Vågutbredningsförutsägelser (propagation forecasts) görs av både
civila och militära institutioner och upplyser om de lämpligaste
frekvenserna och tiderna för olika förbindelsesträckor. Sådana
förutsägelser meddelas i offentliga publikationer, men även i andra,
t.ex. tidskrifter och bulletiner inom amatörradion.

Regional Warning Center (RWC) samlar sol- och geofysiska data och
sänder dagligen Ursigram per telex eller brev (URSI = Union
Radio-Scientifique lnternationale).

Ursigram kan erhållas genom årsabonnemang (tyvärr till högt pris). De
innehåller aktuella mätvärden såsom solfläckstal \(R\), 10 cm solflux
\(F\), magnetiskt index \(K\), gränsdämpningsvärden, även anvisningar
om särskilda händelser (flares, magnetstormar, polarkalottabsorbtion,
Mögel-Dellinger-effekter och liknande) liksom korttidsprognoser och
förvarningar.

Bild II 7-9 Radioprognos för amatörradionbanden på kortvåg

Bild II 7-9 visar en radioprognos ur SSAs medlemstidning QTC. Ny
prognos presenteras periodiskt och behandlar
kortvågsspektrum. Observera det låga solfläckstalet SSN (Sun Spot
Number) på denna bild.

Bild II 7-10 Detalj av radioprognos i II 7-9

\subsection{Solens inverkan på jonosfären}
\textbf{
HAREC a.\ref{HAREC.a.7.5}\label{myHAREC.a.7.5}
}

\subsection{Solaktivitet}

Solen är ett gasklot, i vars inre pågår en ständig kärnreaktion där
väteatomer omvandlas till helium. Vid denna process frigörs en del av
solmaterian som partikelstrålning och elektromagnetisk strålning inom
ett brett frekvensregister, bl.a. kortvågig radiostrålning,
gammastrålning. Solatmosfärens yttre består av två skikt, kromosfären
och koronan. Vissa områden på solens yta har en lägre temperatur och
uppfattas som mörka fläckar - solfläckar. Från kromosfären kastas det
ut gasmassor, s.k. protuberanser, ofta från områden nära solfläckarna.

Det förekommer även kortvariga eruptioner, s.k. flares, som syns som
lysande fläckar i närheten av solfläckarna. Flares sänder ut stark
elektromagnetisk strålning och partiklar. Koronan är solatmosfärens
yttersta skikt. Från denna utstrålas partiklar i form av atomer,
elektroner och protoner, som fångas upp av jordens magnetfält och
skapar polarsken, s.k. aurora. Den ökade partikelstrålningen från
fiares kan orsaka magnetiska oväder med åtföljande radiostörningar och
ökning av polarskenet. Antalet synliga solfläckar står i samband med
solaktiviteten.

\subsection{Solfläckstal}

Ett mått på solaktiviteten är antalet solfläckar, vilket det görs
fortlöpande obseNationer på. Ur detta statistikmaterial beräknas ett
vägt solfläckstal \(R\) (Wolf-talet). Med stöd av solobservationer
under mer än 200 år har det kunnat fastställas att solfläckstalet
varierar någorlunda periodiskt mellan ungefär 200 och 5.

En solfläcksperiod varar mellan c:a 7.5 och 17 år, med ett medelvärde
av c:a 11 år - den s.k. 11-årscykeln. Vid utgången av år 1996
noterades ett så lågt solfläckstal som 5, vilket innebar slutet på
cykel 22.

När cykel 23 nu börjar betyder det bättre möjligheter till DX på
kortvåg under några år.

$>>>>>$ TODO: uppdatera med nuvarande solcykel och förutsättningar

På senare tid har ännu en metod börjat användas för mätning av
solaktiviteten. Då mäts styrkan av radiobruset från solen (solflux
\(F\)) i våglängdsområdet 10 cm.

De båda mätmetoderna ger i huvudsak samma tendenser och det finns ett
statistiskt samband mellan dem.

Vågutbredningen i jonosfären påverkas av solaktiviteten. Under
solfläcksmaximum blir jonosfären starkt joniserad, speciellt F-skiktet
under dagtid. Då reflekteras även vågor med kortare våglängder mot
jonosfären i stället för att passera igenom denna ut i
rymden. 20-metersbandet är då ``öppet'' nästan dygnet runt,
15-metersbandet från före gryningen till efter solnedgången och
10-metersbandet nästan varje dag till efter solnedgången. Långa
förbindelser med mycket låga effekter är möjliga.

Under solfläcksminimum är det emellertid nödvändigt att använda
avsevärt lägre arbetsfrekvens än vid solfläcksmaximum.
20-metersbandet förblir t. ex. inte öppet under hela natten. Öppningar
på 15-metersbandet uppstår endast tillfälligtvis och öppningar på
10-metersbandet är sällsynta.  Goda antenner och högre effekter
används då för att i någon mån kompensera den sämre
vågutbredningen. Vid låg solaktivitet kan de högre banden vara så
tysta, att operatören kan undra om utrustningen verkligen fungerar.
