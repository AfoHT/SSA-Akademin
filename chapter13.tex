\chapter{REGLER OCH TRAFIKMETODER}

\section{Stationsdagbok (loggbok)}
Ändamål
Dina radioförbindelser och övriga händelser
med radiostationen bör antecknas i en
stationsdagbok (loggbok)
Amatörradioverksamheten bygger på förtroende och då är det viktigt att själv kunna
dokumentera sin verksamhet t. ex. i störningssituationer m.m .. Loggen används också för
att kunna visa när man har varit aktiv.
Helt i eget intresse är det ju också trevligt
med en loggbok. Tänk bara på hur bra det är
att ha alla underlag för tävlingar och diplom
m.m. dokumenterade.
Kunna visa hur man för en loggbok
Bilden på nästa sida visar ett exempel på
hur en loggsida (förminskad) kan se ut.
Fundera på följande:
1. Halv tre på eftermiddagen den tionde
oktober gör Ulrik (SM7LQQ) ett allmänt anrop på den lokala repeatern 2-metersbandet.
Karin (SM7UBM) som är på väg hem från
skolan svarar. Ulrik berättar att han precis
har byggt sitt nya slutsteg på 25 W färdigt
och frågar Karin om det hörs någon skillnad
när han kopplar ur det. Efter lite småprat om
allt möjligt säger de 73 till varandra och då
har det gått sju minuter sen de började.
Fyll i loggboken åt U/rik!
2. Gör ett låtsas-QSO med en kurskamrat. Bokstavera era "signaler". För in i loggen.

Allmännauppgifter om motstationen, t.ex.
signalrapport, namn, QTH, motpartens utrustning, QSL-adress o.s.v. brukar också
vara bra att ha med.
Man bör också skriva upp när man har
gjort allmänt anrop, sänt ut bärvåg för prov,
experiment och annat som kan vara av intresse.
Om någon annan radioamatör använder
din station ska du också skriva upp hans/
hennes namn och anropssignal.

Rapportkoder
Man blir ofta ombedd av motstationen att
lämna en s.k. signalrapport på dennes sändning. Omvänt är det bra att få en signalrapport på den egna sändningen.
För rapportering mellan radioamatörer
används RST -koden
För lyssnarrapporter t.ex. till rundradiostationer, förekommer ett kodsystem, som
kallas för SINPO eller SINPFEMO.
Se Appendix J.

Föra in data
Det man skriver upp i loggen är
e
Tiden i början och i slutet av förbindelsen.
Glöm inte datum!
• Motstationens signal.
• Din effekt (ineffekt, PEP eller utstrålad
effekt)
• Frekvensband, ev frekvens.
• Sändningsslag (FM, SSB, CW, paketradio etc).
e
Uppgift om varifrån man sände (eget
QTH).
e
signalrapporter (rapportkoder)
