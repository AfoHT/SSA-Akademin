\chapter{SVENSKA BANDPLANER}
\label{svenska bandplaner}
Vart och ett lands teleadministration utfärdar föreskrifter för amatörradio i sitt land. Dessa
föreskrifter griper naturligtvis över IARU :s bandplaner, vilka endast är rekommendationer för
hur tilldelade frekvensband bör disponeras.
Post- och telestyrelsens föreskrifter PTSFS 1994: 5, (utdrag ur bil. 1-2)
\hilight{TODO: Anpassa till PTSFS 2015:4}

FREKVENSBAND

1810-1850
10100-10150
18068 - 18168
24890 - 24990

kHz
kHz
kHz
kHz

3500-3600
3600-3800
7000-7040
7040-7100
14000 - 141 00
141 00 - 14350
21000-21150
21150 - 21450
28000 - 28200
28200 - 29700

kHz
kHz
kHz
kHz
kHz
kHz
kHz
kHz
kHz
kHz

144-146
432-438
1240- 1300
2300-2450
5650 - 5850
10,0- 10,5
24,0 - 24,25
47,0- 47,2
75,6- 76,0
76-81
142-144
144- 149
241 -248
248-250

MHz
MHz
MHz
MHz
MHz
GHz
GHz
GHz
GHz
GHz
GHz
GHz
GHz
GHz

SÄNDNINGSKLASSER
Tillstånd enligt
certifikatsklass
CEPT1
l, Il (endast J3E)
l, Il, III

1,11,111

l, Il, III

l
1,11,111
l

l, Il, III
l
l, Il, III
l
1,11,111
l

1,11,111, IV
Tillstånd enligt
certifikatsklass
CEPT 1 och CEPT 2
l, Il, III, IV
l, Il, III, IV, V
l, Il, III, IV, V
l, Il, III, IV, V, VI
l, Il, III, IV, V, VI
l, Il, III, IV, V, VI
l, Il, III, IV, V, VI
l, Il, III, IV, V, VI
l, Il, III, IV, V, VI
l, Il, III, IV, V, VI
l, Il, III, IV, V, VI
l, Il, III, IV, V, VI
l, Il, III, IV, V, VI
l, Il, III, IV, V, VI

UTEFFEKT

AMATÖRRADIOSTATUS

1000 w
150 w
1000 w
1000 w

Primär
sekundär
Primär
Primär

1000W
1000 w
1000 w
1000 w
1000 w
1000 w
1000 w
1000 w
1000 w
1000W

Primär
Primär
Primär
Primär
Primär
Primär
Primär
Primär
Primär
Primär

1000 W
1000 W
1000 W
1000 W
1000 W
1000 W
1000 W
1000 W
1000 W
1000 W
1000 W
1000 W
1000 W
1000 W

Primär
Delad primär
sekundär
sekundär
sekundär
sekundär
sekundär
Primär
Primär
sekundär
Primär
sekundär
sekundär
Primär

Primär tjänst har företräde före tjänst med lägre status. Observera att flera tjänster kan ha
delad primär status i ett band, som till exempel i 3500 kHz- och 432 MHz-bandet.
Ovanstående är den föreskrivna statusen för amatörradio i Sverige när denna bok trycktes.
Mer om sändningsslagen per sändningsklass härovan på följande sida och i Appendix E.

G-1

APPENDIX
SSA:s anvisningar 1995: 1, bilaga 1, avseende SSA-tillstånd klass UN och UC
Med stöd av ett särskilt amatörradiotillstånd för SSA, utger SSA anvisningar SSA 1995: 1
avseende utbildningstrafik i de frekvensband som anges i anvisningarnas bilaga 1. Bandplanen m. m. framgår här nedan.
För ändamålet meddelar SSA på anvisade villkor SSA-tillstånd klass UC respektive UN.
Därutöver skalllARV Region 1 bandplan följas i tillämpliga delar.
Märk, att sändning under SSA-tillstånd endast får ske från Sverige!

SÄNDNINGSKlASSER

FREKVENSBAND

UTEFFEKT

AMATÖRRADIOSTATUS

Tillstånd enligt
certifikatsklass

uc

3500-3800
7000-7100
21000- 21450
28000 - 28200
28200 - 29700

144- 146
432-438

kHz
kHz
kHz
kHz
kHz

l
l
l
l
l' Il

100W
100W
100W
100W
100W

Primär
Primär
Primär
Primär
Primär

MHz
MHz

Tillstånd enligt
certifikatsklass
UC och UN
l, Il, III, IV
l, Il, III, IV, V

100W
100W

Primär
Delad primär

Primär tjänst har företräde före tjänst med lägre status. Observera att flera tjänster kan ha
delad primär status i ett band, som till exempel i 3500 kHz- och 432 MHz-bandet.

Sändningsslag per sändningsklass

l PTS föreskrifter (och i tillämpliga fall i SSA:s anvisningar) anges tillåtna sändningslag för
vart och ett frekvensband med avseende på operatörens tillståndsklass. Sändningsslagen
är grupperade i sändningsklasser.
Respektive sändningsklass (grupp) omfattar följande sändningsslag:
Grupp!
A1A,J2B,J2D,F18,F1D,
Grupp Il
A3E, H3E, R3E, J2C, J3E, 8KOOF3E, 8KOOG3E,
Grupp III
H3C, R3C, J3C, J3D, 3KOOC3F, 3KOOF3F,
Grupp IV
A2A,A2D,H2D,R2D,J2D,F1A,F2A,F28,F2D,A3C,F3C, 16KOF3E,
3KOOG3E,
Grupp V
C3F, F3F, 36KOF3E, 36KOG3E,
Grupp VI
F3E, G3E, P1 A, P2A, K1 E.
l Appendix E beskrivs sändningsslagen närmare.

G-2
