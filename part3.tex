\part{REGLER OCH TRAFIKMETODER}

Naturlagar begränsar frekvensområdet för
radiosändningar. Alla radiosändningar måste
ske inom samma utrymme. En del sändningar är riktade till många lyssnare. Andra
sändningar är mellan två personer. Ingen
part vill bli störd av en annan. Intressena är
många. Eftersom radiosändningarna blir fler
och alla tar ett utrymme i anspråk, så är det
nödvändigt att de görs på ett frekvenseffektivt
sätt.

De frekvensband som tilldelas amatörradion är sålunda överenskomna vid internationella konferenser, där teleadministrationer och radiotjänster jämkar samman sina
intressen. Radioamatörerna, representerade av IARU, verkar därvid för sin tilldelning
av önskvärda frekvensområden. De svenska föreskrifterna för amatörradio påverkas
alltså påtagligt av internationella intressen
och överenskommelser.

För att styra detta samråder ländernas
administrationer och radiotjänster om hur
frekvenserna och utrymmet för radiokommunikation skall fördelas och användas.
Överenskommelserna omfattar inte bara
frekvenstilldelning utan även prioriteringar
om nyttjanderätt, sändningsslag, effekter,
räckvidder m.m.

Den internationella telekonventionen ITC - är den överenskommelse på vilken
verksamheten inom den internationella teleunionen -ITU- bygger. Gällande konven-

Liksom det finns lagar och trafikbestämmelser för flyg, sjöfart och landtrafik så regleras sedan mycket länge även radiotrafik av
alla de slag. Utöver nationella regler finns
det mellanstatliga (bilaterala), regionala och
internationella överenskommelser om radiotrafik. Detta gäller även amatörradiotrafik,
som är en internationell radiotjänst.

tion antogs 1982 och ratificerades (lagfästes) av Sverige 1985 (SÖ 1985: 66). Konventionen kompletteras av det internationella Radioreglementet (RR), vilket omfattar
huvudregler som överenskommits mellan
alla länder inom ITU.
Konventionen och radioreglementet är
bindande för alla stater som ratificerat
konventionen. De avsteg, som ett land vill
göra och övriga länder godtar, skrivs in som
s.k. fotnoter. Radioreglementet omfattar alla
radiotjänsters verksamhet, däribland Amatör- och amatörsatellittjänsterna.

OBSERVERA!
Som radioamatör är Du skyldig att följa gällande bestämmelser
för amatörradioanvändning i det land som Du vistas i.
Förvissa Dig om att Du har senaste utgåvan!
Vid osäkerhet- rådfråga PTS!
