\chapter{KOMPONENTER}

\section{Resistorn}
\textbf{HAREC a.\ref{HAREC.a.2.1}\label{myHAREC.a.2.1}}

Allmänt

Strömkretsar består av komponenter med
olika egenskaper. Den vanligaste egenskapen, åtminstone i likströmskretsar, är resistansen. För att få avsedd funktion, så anpassar man resistansen i komponenterna.
Exempel
En krets med strömkälla, lampa, kopplingsledningar och smältsäkring. Kopplingsledningarna mellan komponenterna bör ha
låg resistans och därför lågt spänningsfall
(små förluster). Lampan skall däremot ha
hög resistans och därmed höga förluster för
att kunna bli het och lysa. Smältsäkringen
skall skydda ledningarna från för hög ström.
Säkringen ges därför en resistans, som gör
att den smälter när strömmen överstiger ett
tillåtet värde.
Som hjälpmedel för att fördela spänningar
och strömmar i en krets, så används en
komponenttyp kallad resistor. Dess utmärkande egenskap är resistans- även kallad
ohmskt motstånd.

Enheten Ohm

(Se även kapitel 1)
Resistansen mellan två punkter i en strömkrets är 1 n (uttalas "en åm"), när spänningen mellan punkterna gör att en ström av
1 A (en ampere) flyter i kretsen.
Inom elektroniken används höga resistansvärden och därför även följande multipler av enheten
3
(1 kn) == 106 Ohm
1 kiloohm
1 mego hm
(1 Mn) == 10 Ohm

Resistans i strömledare

För att bestämma resistansen, t. ex. i en tråd,
behöver man veta dess resistivitet, tvärsnittsyta, längd och temperatur.

Resistivitet
Resistivitet är ett materials strömledningsegenskaper. Ett annat namn för resistivitet
är specifik resistans. Symbolen för resistivitet
är p (uttalas rå).

..
. . . ..
ohm ·mm 2
Form e ln for res1St1v1tet ar p =- - - m
Följande formel gäller för beräkning av
resistansen i en strömledare med linjär ström/spänningskaraktär.

R= p!

A

![meter] A [mm

2

]

[p= n·A]
m

Exempel
l = 4 m koppartråd
A== 2 mm 2
p (koppar) == 0.017

l
A

R=p-

R=0.017i=o.034 n
2

Not. Förväxla inte A [tvärsnittsytan] i denna
formel med beteckningen A t. ex. i Ohms lag då A
betecknar strömstyrkan.

Resistiva material

Resistorer kan utföras med olika typer av
resistiva material, vilket bestämmer användningsområdet.
En resistor, vars resistans är oberoende
av ström, spänning och annan yttre påverkan t.ex. temperatur och ljus, sägs ha linjär
karaktär. Om resistansen däremot beror av
yttre påverkan, så sägs resistorn ha olinjär
karaktär.
Man skiljer mellan tre huvudgrupper av
resistiva material. Det kan vara en kropp av
pressat kol eller ett ledande ytskikt på ett
isolerande underlag eller metalltråd på en
isolerande stomme. På senare tid har tillkommit integrerade resistorer, d.v.s. flera
resistorer av resistiva skikt på ett gemensamt isolerande underlag. Här beskrivs i
korthet
resistortyper. Se f.ö. leverantörskataloger.

Utförandeformer

Resistorer kan utföras med fast eller ställbart resistansvärde. Här följer först en översikt över resistorer med olika resistiva material och fast resistansvärde.

112-1

K MP NENTER
Fasta resistorer med linjär karaktär
Massaresistar
Det resistiva materialet består av kolmassa
med bindemedel (kolkomposit). Massan är
bakad till en stav eller ett rör. Anslutningsledningarna är inbakade i materialet.
Massaresistorer är lämpliga för lik- och
växelströmskretsar med låga krav på temperaturberoende och egenbrus. Den homogena kroppen gör att egeninduktansen är
låg. Å andra sidan uppstår vid höga frekvenser en skineffekt, d.v.s. strömkoncentration
vid ytan, som medför viss resistansökning.
Kolfilmsresistor
Det resistiva materialet består av ett kolskikt,
som genom förångning överförts till ett keramiskt rör. Resistansen bestäms av tjockleken på skiktet samt av spiralformade spår i
detta. Genom spiraliseringen tillförs en induktans, men som i någon mån uppvägs av
egenkapacitansen.
Metallfilmresistor
l denna typ är kolfilmen ersatt av ett metallskikt. Eftersom egenkapacitansen är liten,
så är typen lämpad för höga frekvenser.
Tjockfilmsresistor
Det resistiva materialet består en film av bl. a.
metalloxid, som screentrycks på ett keramiskt underlag. Typen har god tålighet mot
pulser och höga temperaturer, men har relativt högt egenbrus. Ytmonterade resistorer
är oftast tillverkade av tjockfilm.
Tunnfilmsresistor
Det resistiva materialet består av en tunn
metallfilm, som genom förångning överförts
till ett underlag av glas eller keramik.
Denna resistertyp har över lag god stabilitet och används ofta i apparater med hög
precision. Egenskaperna vid höga frekvenser är dock inte så bra.
Metalloxidresistor
Denna resistertyp har ett spiralformat skikt
av metalloxid. Temperatur- och spänningsberoendet är måttligt. Tåligheten mot pulser
och höga temperaturer är stor. Typen kan i
någon mån ersätta trådlindade resistorer.

112-2

Resistarnät
Resisternät (integrerade resistorer) består
av flera resistiva skikt på ett gemensamt
isolerande underlag, d.v.s. liknande teknik
som för tjock-och tunnfilmsresistorer.
Trådlindad resistor
Det resistiva materialet är en metalltråd,
som är lindad på en stomme som tål hög
temperatur; det kan var keramik, glas etc.
Tåligheten mot pulser och höga temperaturer är stor.

Fasta resistm·er med olinjär karaktär
Vanligast är att materialet i resistorer har
linjär ström-/spänningskaraktär, men det
finns även sådana med olinjär karaktär. l
resistorer med olinjär karaktär är det ingående materialet av halvledartyp.
Spänningsberoende resistor
- Voltage Dependent Resistar (VOR)
Linjära resistorer påverkas knappast av den
pålagda spänningen. Resistorer av kiselkarbid har däremot en hög resistans vid låg
spänning och omvänt en låg resistans vid
hög spänning. VOR används t.ex. för begränsning av spänningstoppar.
Ljusberoende resistor, fotoresistor
- Light Dependent Resistar (LDR)
Ledningsförmågan i halvledare påverkas inte
bara av värme utan även av ljus. Halvledare
av germanium och särskilt sammansatta
halvledare av kadmiumoxid, blysulfid och
indiumantimonid har särskilt stor ljuskänslighet. Kadmiumsulfid är känsligast för synligt ljus medan andra material är känsligast i
det infraröda området.
Magnetfältberoende resistor (fältplatta)
Resistansen ökar med längden på strömledaren. Denna egenskap används i magnetfältberoende fältplattor. En sådan består
av en keramisk bärarplatta med en yta av
indiumantimon id. l ytan är ytterst smala parallella metallbanor inlagda på ett avstånd av
någon J.lm. Normalt går strömmen kortaste
vägen tvärs över banorna, men när ett magnetfält träffar vinkelrätt mot plattans yta, så

K MP NENTER
avlänkas elektronerna. De får då längre väg
över till nästa metallbana och den totala
resistansen ökar.

Temperaturberoende resistor

Se nedan om NTC och PTC i resistorer.

Temperaturkoefficienten för resistorer
Resistansen i ingående material påverkas
av temperaturen, vaNid det skiljer mellan
materialen.
Amorft kol och de flesta halvledande
materialleder bättre när de är varma- de har
en negativ temperaturkoefficient (NTC). Sådana material finns t. ex. i dioder och transistorer.
Däremot leder metaller och speciella
halvledarmaterial bättre när de är kalla- de
har en positiv temperaturkoefficient (PTC).
Glödtråden i glödlampor och elektronrör är
resistorer med positiv temperaturkoefficient
(PTC). l vissa metallegeringar, som t.ex. i
konstantan, kan resistansen till och med
vara nästan konstant vid varierande temperatur.
Alla material har en temperaturkoefficient,
som anger hur mycket resistansen ändras
per grad. Resistansen vid någon annan temperatur kan därför beräknas med följande
formel, där man sätter in begynnelsetemperaturen [ o] (o c), temperaturändringen [ .1.0]
och temperaturkoefficienten [a].

Rvarm= Rkall\(\pm\) a· il ?J· Rkatt
Resistansändringen är ledet
AR= \(\pm\)a· .1.19- ·Rkatt
Temperaturkoefficienten kan vara positiv (NTC) eller negativ (NTC).
l principscheman har PTC- respektive NTCresistorer symboler som på bilden.
Bild nr 112-1

Variabla resistorer
En resister kan även utföras med variabelt
resistansvärde. Då används endast den andel av det resistiva materialet, som finns
mellan en resistors ena ände och ett uttag
någonstans mellan ändarna. En sådan anordning kallas för reostat. Om en variabel
resister används som spänningsdelare, så
kallas den för potentiometer.

l en potentiometer används dels hela
resistansen mellan ändpunkterna och dels
andelen mellan uttaget och någon av
ändpunkterna.
Uttagets mekaniska utförande beror oftast av hur bekvämt inställningen skall kunna
ske. En potentiometer, där det resistiva
materialet är lagt på en cirkulär bana och
uttaget är fäst vid en axel i banans centrum,
medger enkel inställning med mejsel, ratt
etc. Ett enklare slags uttag är en släpkontakt
eller ett spännband som kan flyttas utmed en
stavformad resistor.

Resistiva material i variabla resistorer

Banan i en variabel resister består i princip
av liknande resistiva material som i en fast
resistor.
Billigast och enklast är en bana av kol,
som är tryck på ett enkelt underlag. Nackdelar är låg efekttålighet, dålig upplösning och
linjäritet, högt brus och kort livslängd. Fördelen är lågt pris.
Bättre än en kolbana är en bana av kolkomposit, d.v.s. kolpulver med bindemedel,
som är tryckt på ett underlag. Nackdel är
högre pris och låg effekttålighet, medan fördelarna är god upplösning, lågt brus och
lång livslängd.
Vill man ha god effekttålighet och temperaturstabilitet, utöver kolkompositens egenskaper, så erbjuder en bana av cermetsådana fördelar. En cermetbana består av en
blandning av metaller och keramik, som
trycks på ett underlag.
Trådlindad bana har främst god tålighet
mot hög effekt. Tålighet vid hög ström genom uttaget är en nannan fördel.

Linjära och olinjära patentiometrar

En potentiometer har en kuNform, varvid
avses resistansändringen som funktion av
uttagets rörelseväg utmed resistansbanan.
KuNformen kan utföras linjär, logaritmisk
etc .. Olinjära kuNor består då oftast av en
följd av linjära segment, som tillsammans
någorlunda motsvarar den önskade olinjära
formen. Ett exempel på det är när en kurvform
anges som linjär/logaritmisk.

112-3

K MP N
Effektutveckling i resistorer
l resistorer utvecklas värme av den ström
som flyter igenom. Värmeutvecklingen sker
enligt Joule's lag, som återges i kapitel1.
Hur mycket effekt i form av värme som
strålas ut från resistorn beror på storleken på
dess yta och egentemperatur samt på omgivningens temperatur. Det finns en övre
gräns för hur stor värme det ingående materialet tål innan det förstörs och eventuellt
fattar eld.
En resistors effekttålighet framgår i vissa
fall av påstämplade värden. l övriga fall är
man hänvisad till kataloguppgifter eller en
bedömning, som ev. kan grundas på höljets
utseende och dimensioner.
standardiserade komponentvärden
Resistorer tillverkas vanligen med standardiserade värden ur någon talserie.
Märkning av resistorer
Resistorer märkas med huvuddata enligt
något system av siffror och bokstäver eller
med en färgkod. Flera olika system tillämpas.
(Se f.ö. leverantörskatalogerför information
om komponentdata, märkning o.s.v.)

2

1
2
3
4

3

4

Allmän symbol
ställbar resistor, potentiometer
Trimbar resister (trimmer)
Automatiskt ställbar resister

Bild 112-1 Schemasymboler för resistorer

112-4

5

5
6
7
8

6

7

Temperaturberoende resistor,
Temperaturberoende resistor,
Spänningsberoende resistor,
Ljusberoende resistor,

8

NTC
PTC
VOR
LDR

K MP NENTER

2.2 Kondensatorn
Allmänt

Följande formler gäller för kapacitansen i
en enkel kondensator med två plattor. När
en kondensator är uppbyggd av n stycken
plattor, ökar kapacitansen med faktorn (n-1 ).
Med vakuum som dielektrikum gäller

Så snart det finns en elektrisk potentialskillnad .,..-en spänning - mellan två kroppar, så
uppstår ett elektriskt kraftfält mellan dem.
Ett sådant fält är lagrad elektrisk energi.
Kropparna måste då isoleras från varandra.
Elektrisk energi lagras mellan olika delar
av en strömkrets, även om de inte är direkt
avsedda för det. Särskilt vid mycket höga
frekvenser har detta stor betydelse för utformningen av en strömkrets. Vid låga frekvenser och likström däremot, har kretsens
utformning mindre inverkan. Då behövs i
stället särskilda anordningar för ta upp eller
avge energi på önskade ställen i strömkretsen.
En sådan anordning kallas kondensator.
Den består i princip av två band eller plattor
med anslutningsledningar samt ett isolerande skikt- dielektricum- däremellan.
Kapacitansen är näst efter resistansen
den vanligaste egenskapen i en strömkrets.

Kapacitans är elektricitetsmängden per volt
där måttenheten är Farad [F]. Eftersom denna
enhet är mycket stor, används inom elektroniken oftast bråkdelar av den.
1 mikrofarad (1 !-!F) = 1o-6 F
1 nanofarad (1 n F) = 1 9 F
1 pikofarad (1 p F) = 1o- 12 F

\section{Kapacitans}
\textbf{HAREC a.\ref{HAREC.a.2.2}\label{myHAREC.a.2.2}}

Kondensatorn i likströmskretsen

Förmågan att lagra elektrisk energi (elektrisk laddning) kallas kapacitans. Ordet kommer från latinets capax, som betyder rymlig,
duglig.
Kapacitansen betecknas i formler med
bokstaven C
•
•
•
•

En kondensators kapacitans bestäms av
ytan på kondensatorns plattor,
avståndet mellan dessa ytor,
den absolutadielektricitetskonstanten c0
den relativa dielektricitetskonstanten
som är den faktor kapacitansen ökar med
när dielektrikum är annat än vakuum.

c,;

Kapacitans, dimension och dielektrikum

Kapacitansen är proportionell med den yta,
som kondensatorplattorna skuggar varandra, och är omvänt proportionell med plattavståndet.

1•

A

C=cod

Med ett godtyckligt dielektrikum gäller

2.
C [Farad]

d [mm]

E [F/m]

Enheten Farad

o-

En kondensator i en likströmskrets har alltid
samma polaritet. Därvid förhåller sig kondensatorns polspänning till dess laddningsmängd.
En ström flyter till kondensatorn och laddar upp den, när den anslutna spänningskällan har högre spänning än kondensatorn. Ju
högre spänningen är, desto större är laddningen. Ju kortare uppladdningstiden är, desto högre effekt utvecklas under den tiden.
När en uppladdad kondensator ansluts
till en krets med lägre spänning, så urladdas
kondensatorn till kretsen. Ju kortare urladdningstiden är, desto högre effekt utvecklas
under den tiden.
Laddningen i en kondensator kan innebära hög polspänning. Om kondensatorns
kapacitet är stor, kan laddningsmängden bli
betydande. Varning för elektriska stötar och
brännskador!

112-5

K
Kondensatorn i växelströmskretsen
l en likströmskrets förhåller sig kondensatorns polspänning till laddningsmängden. l
en växelströmskrets växlar emelltid spänningen och polariteten ständigt och därmed
kondensatorns laddning och polaritet.
Not: Vissa kondensatortyper kan inte användas i rena växelströmskretsar.
Försök
En glödlampa och en kondensator kopplas i
serie med varandra och ansluts till en
växelströmskrets. Med lämpligt valda värden på komponenterna kommer lampan att
lysa upp.
Detta visar att en kondensator inte hindrar elektronflödet i en växelström krets. Man
brukar säga att kondensatorn "släpper igenom växelström", men i stället är det så att
laddningar pendlar mellan kondensatorns
plattor genom den strömkrets som kondensatorn är ansluten till.
Använd för säkerhets skulllåg spänning,
t.ex. den från en ringledningstransformator!

Kapacitiv reaktans
Strömstyrkan i en växelströmskrets beror
bl.a. på hur stor kondensatorns kapacitans
är, d.v.s. på dess kapacitiva reaklans Xc.
Ordet reaktans kommer från latinets re
(åter) agere (verka).
Större kapacitans innebär större förmåga att ta upp elektrisk laddning och ger
därmed en lägre reaktans. Resultatet blir ett
kraftigare elektronflöde. En mindre kapacitans innebär ett svagare elektronflöde.

1
2n:fC

Xc=-- eller

1
mC

Xc=-

[O]
[Hz]
[F]
eller
[MO]
[MHz]
[uF]
(samma sortenheter i alla led)
Exempel:

f = 50 Hz
1
1
xc -- 2 n:fC - 2 n 50 ·1 O·1

1.

2.

112-6

C = 1O J.tF

c = 1oJ.tF

Xc = ?
=318.3 Q

XC = ?

En kondensators reaktans är således
omvänt proportionell med dess kapacitans
och frekvensen i kretsen.
Jämför detta med en induktor där reaktansen är proportionell med frekvensen.
När en ström flyter genom en res istor, så
uppstår det värmeförluster. När ström flyter
genom en ideal reaktans- en induktor eller
en kondensator - uppstår däremot inga
värmeförluster.

Fasförskjutning i en kondensator
Med fasförskjutning menas här den tidsmässiga förskjutningen mellan ström- och
spänningsförloppen. l en kondensator når
nämligen strömmen inte sitt toppvärde samtidigt som spänningen. l en ideal kondensator är spänningen fasförskjuten 90\(\circ\) efter
strömmen.
Förlustvinkel
l praktiken är fasförskjutningen i en kondensator något mindre än 90\(\circ\) på grund av att
laddning läcker igenom dielektrikum. Man
talar om en förlustvinkeL Läckningen kan
ses som en resister som är kopplad parallellt
över kondensatorn.
läckström m.m.
Med det extremt tunna. dielektrikum har
elektolytkondensatorn en mycket högre kapacitet än andra former, men har också
några nackdelar, bl.a. att
• den normalt endast kan användas med
likspänning,
• den har hög förlustfaktor p.g.a.läckström,
• det utvecklas värme av läckströmmen,
vilket skapar övertryck p.g.a. gasbildning.
Utförandeformer
Kondensatorer kan utföras med fast kapacitansvärde. Dielektrikum består då av ett
skikt av glimmer, impregnerat papper o.s.v.
Kondensatorer kan även utföras med variabelt kapacitansvärde. Dielektrikum består
då oftast av luft, men kan även vara ett fast
material.

Fasta kondensatorer
Kondensatorer har oftast namn efter utförande och materialet i dielektrikum
Pappers- och plastkondensatorer
'Plattorna" i dessa typer består av aluminiumremsor med anslutningstrådaL Däremellan finns en pappers- respektive plastremsa
som dielektrikum. För att spara plats, så
rullas det hela ihop och skyddas med en
plastingjutning.
Keramiska kondensatorer
l keramiska kondensatorer består dielektrikum av något keramiskt material. På ömse
sidor om detta sätts en metallbeläggning
med anslutningstrådaL
Glimmerkondensatorer
l denna kondensatortyp består dielektrikum
av tunna glimmerskivor
Elektrolytkondensatorer
Elektrolytkondensatorer har elektroder av
aluminium eller tantal, därpluspolen (anoden)
ges ett mycket tunt oxidskikt Detta är inte
ledande och fungerarsom dielektrikum. Mellan oxidskiktet och minuspolen (katoden)
läggs en elektrolyt med låg resistivitet.
Elekrolytkondensatorer har särskilt högt
kapacitansvärde. Till skillnad från andra kondensatortyper, så är elektolytkondensatorer
polariserade. Utom i ett specialfall innebär
det, att polariteten på den pålagda spänningen inte får kastas om. Flera olika slags
elektrolytkondensatorer finns, såsom våta
och torra aluminiumelektrolytkondensatorer, tantalelektrolytkondensatorer m. fl.

Variabla kondensatorer
Variabla kondensatorer har oftast sitt namn
efter utförandeformen, såsom vridkondensator och trimbar kondensator (trimmer).

Temperaturkoefficient
På liknande sätt som med resistorer, så
påverkas kapaciteten i kondensatorer av
temperaturen. Att sambandet mellan kapacitet och temperatur är viktigt, förstås av att
temperaturkoefficienten i den frekvensbestämmande kapacitansen i en oscillatorkrets
är en av faktorerna för stabil frekvens.
Temperaturkoefficienten a anger kapacitetsändringen pergrad temperaturändring.
Kapacitetsändringen blir då

AG= \(\pm\)ac ·Ck· AiJ
varvid Ck är kapacitetsvärdet vid den lägre
temperaturen (oftast 20\(\circ\)C) och ~t) är
temperaturändringen i grader Kelvin.
Kelvin [K] är den normerade måttenheten
för absolut temperatur.
En ändring med 1 K motsvarar en ändring med 1
Är ac positivt betyder det att kapaciteten
ökar med ökande temperatur.
Är ac negativt betyder det att kapaciteten
minskar med ökande temperatur.
En kondensator som är märkt med N 100
betyder a c = -1 00 ·i 6 l K

oc.

o-

standardiserade komponentvärden
Kondensatorertillverkasvanligen med standardiserade värden ur någon talserie.
Märkning av kondensatorer
Kondensatorer märkas med huvuddata enligt något system av siffror och bokstäver
eller med en färgkod. Flera olika system
tillämpas.
(Se f.ö. leverantörskatalogerför information
om komponentdata, märkning o.s.v.

1
2
3
4

j

T
1

2

3

4

Allmän symbol
Trimbar kondensator (trimmer)
Vridkondensator
Polariserad kondensator,
elektrolytkondensator

Bild fl 2-2 Schemasymboler för kondensatorer

\section{Induktorn}
\textbf{HAREC a.\ref{HAREC.a.2.3}\label{myHAREC.a.2.3}}

Allmänt
När elektrisk ström flyter genom en ledare,
så alstras ett magnetfält omkring den. Så
snart strömmens styrka eller riktning ändras,
uppstår en motsvarande s.k. elektromotorisk kraft (EMK), som motverkar ändringen.
Kraften finns i magnetfältet, som är lagrad
magnetisk energi.
Självinduktion - induktans
Magnetfältets förmåga att alstra en motverkande EMK kallas självinduktion eller induktans. Ordet induktans kommer från latinets
inducere, som betyder införa.
När en ledare, som ingår i en sluten krets,
rör sig i ett magnetfält, så kommer en ström
att flyta genom ledaren på grund av den
EMK (spänning) som alstras. Varje ändring
av strömmen motverkas av det magnetfält
som strömmen själv alstrar.
När det uppstår självinduktion i en ledare,
så kallas ledaren induktor. Självinduktionen
är jämnt utbredd över ledarens hela längd.
När ett större induktansvärde behövs på
något särskilt ställe i strömkretsen, så kan
ledarens längd ökas just där och lindas upp
till en spole med lämplig form. Hela spolen
kallas då för induktor.
Det att ett motverkande magnetiskt fält
alstras omkring en ledare när strömmen i
den ändras, påverkar kretsens egenskaper
och därmed utformning på olika sätt. Vid
snabba strömändringar, t.ex. vid hög frekvens, är motverkan större än vid långsamma
ändringar. Vid konstant likström uppstår
däremot ingen motverkan- självinduktion.
Induktansen är efter resistansen och
kapacitansen den vanligaste egenskapen i
en strömkrets.
Försök med induktion
Försök 1
Bild II 2-3 överst
Ett känsligt vridspoleinstrument kopplas till
en induktor. Instrumentet bör ha noll på
skalans mitt, så att strömriktningen syns. En
permanentmagnet används för att visa att

självinduktion uppstår när magneten förs
fram och tillbaka genom induktorn.
Instrumentet ger utslag när magneten är
i rörelse. Utslaget blir större vid snabbare
hastighetsändring. Utslagsriktningen växlar,
när magneten förs in i respektive dras ut ur
induktorn - det uppstår en växelström.
En växelspänning uppstår över induktorn,
även när den ingår i en strömkrets som sluts
och bryts-alltså utan en magnetsom rör sig.
Försök 2
Bild II 2-3 mitten
Permanentmagneten byts nu mot ännu en
induktor. Utöver den första induktorn, som vi
nu kallar sekundärlindning, kallar vi den nya
induktorn för primärlindning.
När vi släpper ström genom primärlindningen så alstrar den ett magnetfält. Först är
strömmen noll för att sedan ändras till ett
högt värde och därefter återgå till noll. Det
blir en strömstöt
Varje ändring alstrar en mot-emk, som
bygger upp ett magnetfält, först i en riktning
och sedan i den andra. l båda fallen passerar
fältet genom båda lindningarna. Fältet från
primärlindningen inducerar en spänningsstöt i sekundärlindningen. Stöten har en riktning, när primärlindningens strömkrets sluts
och motsatt riktning när den bryts, d.v.s. det
blir en växelspänning. När sekundärlindningen ingår i en sluten krets, uppstår en
växelström genom sekundärlindningen.
Försök 3
Bild II 2-3 nederst
Vad händer när primärlindningen i försök 2
ansluts till en växelspänning, t.ex. med nätfrekvensen 50 Hz? Använd för säkerhets
skull en skyddstransformator mellan nätet
och lindningen!
l sekundärlindningen uppstår då spänningsstötar, vars polaritet i detta fall växlar
100 gånger per sekund. Det uppstår alltså en
växelspänning över sekundärlindningen och
om denna ingår i en sluten strömkrets uppstår det en motsvarande växelström.

112-9

N

R

E

VÄXELSTRÖM

STRÖMSTÖT

STRÖMSTÖT

Fältspole

Primärkrets

Induktionsspole

sekundärkrets
STRÖMSTÖT

l ndu kti ensspole

Pr i märkrets

sekundärkrets
VÄXELSTRÖM

Bild II 2-3 Försök med induktion
112- 1o

K MP NENTER

2

Allmän symbol,
induktor utan kärna
2 Induktor med kärna
3 Trimbar induktor
4 ställbar induktor

4

3

Bild II 2-4 Schemasymboler för induktorer

Olika utföranden
Elektromagneter, drosslar, induktorer för
svängningskretsar, ramantenner o.s.v.
Enheten Henry
Måttenheten för självinduktion är Henry (H)
1 Henry (1 H) är självinduktionen i en induktor, som alstrar en motspänning av 1 volt vid
en strömändring av 1 ampere under 1 sekund.
l formler betecknas induktans med L
Sambandet är
Volt= Henry · Ampere/sekund
1 H är en stor måttenhet. För elektroniktillämpningar används därför ett mer hanterligt format.
Exempel:
1 H= 1000 mH
1 mH = 1 · 1o-3 H
1 mH = 1000 J..LH
1J..LH = 1 · 1o-3 m H= 1 · 1o-s H

Hur induktansen påverkas
Induktansen beror på induktorns mekaniska
dimensioner, antalet lindningsvarv och materialet i kärnan.
Induktansen i en cylindrisk induktor är
proportionell mot tvärsnittsytan, omvänt proportionell mot längden och proportionell mot
kvadraten på lindningsvarvtalet
Induktansen ökar, om induktorn förses
med en kärna av järn och minskar med en
kärna av omagnetisk, ledande metall, t.ex.
koppar, mässing eller aluminium.

Induktiv reaktans
Till skillnad från när en resistor ansluts till en
spänning, så blir strömökningen i en induktor fördröjd. Orsaken är att en induktor inte
bara har en resistans, vilken ju inte påverkas
av strömvariationer, utan har även en induktiv reaktans XL. Ordet reaktans kommer från
latinets re (åter) agere (verka).
Reaktans - växelströmsmotstånd eller
skenbart motstånd - uppträder så länge
som strömmen genom induktorn ändras. En
induktor gör således också motstånd mot
varje strömändring och detta motstånd ökar
med ökande ändringshastighet
En fullbordad pendling i en växelström
kan ses som ett varv i en cirkel - 360\(\circ\) -och
en fullbordad pendling kallas en period.
En period motsvarar omkretsen i en cirkel med radien r, där omkretsen är 2 · n · r
(n = 3.141593 .. ). När strömmen växlar 1
gång/sekund har pendlingen en frekvens [f]
av 1 Hertz [Hz]. Vid 50 växlingar/sekund har
pendlingen en frekvens av 50 Hz o.s.v.
Induktiva reaktansen XL - växelströmsmotståndet i en induktor- är en funktion av
strömmens s.k. vinkelhastighet m= 2 · rc • f
och av storheten av induktansen L.
Den induktiva reaktansen är proportionell mot strömmens frekvens och mot
induktorns induktansvärde. Inga förluster
uppstår i en ideal induktor, d.v.s. en som
teoretiskt saknar resistans.
Sambandet är
X L = 2 · rc · f· L = mL
eller

XL [Q]

f [Hz] L [H]
[MQ]
[MHz] [mH]
(exempel på prefix)

112- 11

K M N
Exempel:
1.
L= 1H
f = 50 Hz
XL = 2 . Jr. 50 . 1= 3 i 4 .Q
2.

L= 1H
f = 5 kHz
XL = 2 . Jr . 5000 . 1= 31400 .Q

Fasförskjutning mellan
och
ström i en induktor
Bild II 3-000 (i kapitel 3)
Med fasförskjutning menas den tidsmässiga förskjutningen mellan ström- och
spänningsförlopp. Strömmen genom en induktor, når inte sitt toppvärde samtidigt som
spänningen över den. Orsaken är växlingarna mellan elektrisk och magnetisk energi i
induktorn.
l en ideal induktor är spänningen fasförskjuten 90\(\circ\) före strömmen. l praktiken är
dock förskjutningen något mindre än 90\(\circ\) på
grund av resistansen i induktorn.
Q-faktor- godhetstal
Q-faktorn kan avse två olika saker, som inte
skall förväxlas. Det är Q-faktorn för en komponent respektive den för en hel strömkrets.
Q-faktorn för en induktor är kvoten av
dess reaktans och serieresistans.
Q

komponent -

xkomponent

R

komponent

Q-faktorn för en hel svängningskrets beror däremot på bredden på det frekvensband som en viss komponentkombination
ger. Q-faktorn för en resonant svängningskrets är därför ett mått på dess selektivitet
(se kapitel 3).
Medan Q-faktorn för en ingående komponent påverkar Q-faktorn för en hel krets,
så gäller inte det omvända.

Yteffekt {skin-effect)
l en ledare av homogent material fördelar sig
en likström likaöverhela tvärsnittet. Strömtätheten för en växelström däremot, minskar i
ledarens mitt och ökar i stället vid ytan. Ju
högre frekvensen är, desto större är strömtätheten vid ytan. Fenomenet kallas yteffekt
(på engelska skin effect) och uppträder i alla
ledare.

112- 12

Det djup i ledarmaterialet där laddningstätheten sjunkit till 37% av värdet vid ytan
kallas skin depth. För koppar är detta djup
c:a 70 mm vid 100 Hz. Vid 1 MHz har djupet
minskat till 0.07 mm och vid 100 MHz till
0.0067 mm. På grund av yteffekten är alltså
materialet i mitten av homogena ledare elektriskt mindre verksamt vid höga frekvenser.
Resistansen blir alltså större för växelström
än för likström, om ledaren är samma
Utöver frekvensen påverkas yteffekten
av ledarmaterialets elektriska och magnetiska ledningsförmåga. För att få låg resistans
i ledare för högfrekvent ström är det viktigt att
omkretsen är stor och att materialskiktet vid
ytan har hög ledningsförmåga. Det är därför
som induktorerna i sändarslutsteg ofta är
försilvrade och består av rör med stor diameter eller av breda band.

Temperaturkoefficient
Liksom med resistorer, så påverkas induktansen av temperaturen. Att sambandet
mellan induktans och temperatur är viktigt,
förstås av att temperaturkoefficienten i den
frekvensbestämmande induktorn i en oscillatorkrets påverkar frekvensstabiliteten.
Eftersom metallen koppar utvidgar sig
vid temperaturökning och induktorns tvärsnittsyta då blir större, så är temperaturkoefficienten vanligen positiv.
Temperaturkoefficienten aL anger induktansändringen per grad temperaturändring.
Induktansändringen blir då
!lL =\(\pm\)aL· Lk ·llfJ

varvid Lkär induktansvärdet vid den lägre
temperaturen (oftast 20 \(\circ\)C) och fl{} är
temperaturändringen i oKelvin.
Kelvin [K] är den normerade måttenheten
för absolut temperatur. En ändring med 1 K
motsvarar en ändring med 1 oc.
Induktorer kan innehålla kärnor av någon
metallegering, vars egenskaper också är
temperaturberoende.
l praktiken kan man knappast påverka
temperaturkoefficienten i en induktor. Eftersom en svängningskrets för det mesta även
innehåller kondensatorer, så kan man t.ex.
kompensera en positiv temperaturkoefficient
i induktorn med en negativ i en kondensator.

PT

NENTER

Förluster i kärnmaterial
När ett magnetiskt växelfält passerar ett
kärnmaterial så kommer atomerna (som är
permanentmagneter) att ständigt inta nya
lägen i materialet i takt med fältets frekvens.
Då uppstårvirvelström mar, s.k. järnförluster,
som dels påverkar materialets ledningsförmåga och dels höjer temperaturen i kärnan och därmed i hela induktorn.

\section{Transformatorn}
\textbf{HAREC a.\ref{HAREC.a.2.4}\label{myHAREC.a.2.4}}

Allmänt

En transformator består av en eller flera
lindningar eller spolar av elektriska ledare.
Lindningarna är magnetiskt kopplade till varandra. Det innebär att de är anordnade så,
att ett magnetfält, som alstrats i någon av
lindningarna, även passerar genom övriga
lindningar.
När en växelspänning läggs över en lindning, kallas den primärlindning. l och omkring primärlindningen alstras då ett magnetiskt fält som växlar i takt med spänningen.
Primärfältet passerar även genom övriga
lindningar- sekundärlindningarna-och alstrar där spänningar och strömmar.
Den s.k. kopplingsfaktorn mellan lindningarna varierarförolika frekvenser, sämre
vid låga frekvenser (hundratals Hz) och bättre vid höga frekvenser (tusentals Hz). Speciellt vid låga frekvenser behövs en bättre
koppling för att avsedd effekt skall kunna
överföras mellan lindningarna. Då kan ledningsförmågan i den magnetiska flödesvägen ökas med hjälp av en järnkärna.

Terminologi
primärkrets
sekundärkrets
primärlindning
sekundärlindning
primärspänning u 1 sekundärspänning u2
primärström i1
sekundärström i2
lindningsvarvtal n primärt n1 sekundärt n2
varvtalsomsättning = !i eller n2

n2

impedansomsättning

ni

z rt

= z = ,i
1

2

2

Den ideala (förlustfria) transformatorn
Transformering av spänning och ström
Transformatorn är obelastad när sekundär-

kretsen är bruten.
Bild 112-6
När primärlindningen ansluts till en växelspänning, induceras det växelspänningar
både över primär- och sekundärlindningarna. Det uppstår även en ström i primärlindningen, men däremot inte i sekundärlindningen när sekundärkretsen är bruten. För
den obelastade transformatorn gäller sambandet

!!J..=!i
u2

n2

d.v.s. spänningen över lindningarna är proportionell med lindningsvarvtalet

2
1 Allmänna symboler
2 Transformator med järnkärna

Bild II 2-5 Schemasymboler för
transformatorer

Utföranden

Transformatorn kan utföras för olika ändamål, t.ex. som
spä n n ingstransformator,
strömtransformator eller
impedanstransformator
Utförandet påverkas även av överförd effekt
och frekvens.

Transformatorn är belastad när sekundärkretsen är sluten.
Bild II 2-7
När någon av transformatorns sekundärlindningar ingår i en sluten strömkrets, uppstår en sekundärström där.
sekundärströmmen alstrar ett magnetfält, som motverkar primärströmmens fält,
hindrar dess växlingar och tar ut energi från
primärkretsen.
Strömförbrukningen på primärsidan ökar
således i proportion till strömförbrukningen
på sekundärsidan. Transformatorn reglerar
själv hur mycket energi som den tar från
strömkällan och lagrar i fältet för att föra över
till sekundärkretsen.

112-15

K MP NENTER
För den belastade transformatorn gäller, att
strömmen genom lindningarna är omvänt
proportionell med lindningsvarvtalet

i1 n2
-=-

i2

n1

Av föregående formler följer att:

Bild II 2-6 Obelastad transformator

Bild II 2-7 Belastad transformator

112-16

Av~=

u1 ·i1 och ~ = u1 ·i1 följer att~=~

Om man bortser från förlusterna i transformatorn, så är den effekt som den tar från
kraftkällan lika med den effekt den avger.

Transformatortillämpningar
Sparkopplade transformatorer

Bild II 2-8
Här ovan har transformatorn beskrivits så att
primär- och sekundärlindningarnas enda
förbindelse med varandra är över ett magnetfält, alltså utan galvanisk förbindelse.
Varje lindning kan emellertid förses med
godtyckliga uttag. Över
av
finns då en spänning i proportion till det
lindningsvarvtal som finns mellan uttagen.
Detta är en metod att spara in på antalet
lindningar. För att t.ex. omsätta
ningen 230 V till 115 V används
spartransformator.

Med en spartransformator kommer olika
strömkretsar i galvanisk förbindelse med
varandra och särskild försiktighet skall därför iakttas vid användning av sparkopplade
transformatorer, p.g.a. risken för elolycksfall. Spartransformatorer bör därför inte användas i amatörradiosamman hang. Säkrast
ärtransformatorer med galvaniskt skilda ledningar och dessutom med speciellt bra isolering och kapsling- s.k. skyddstransformatorer.

Bild II 2-8 Sparkopplad transformator

Strömtransformatorer

Hög sekundärström under låg sekundärspänning kännetecknar en strömtransformator.
Strömtransformatorer används i elektriska svetsningsutrustningar, induktionsugnar
o.s.v. Strömtransformatorer används även
för mätning av höga växelströmmar.
Bild 112-9
Bilden visar principen för en induktionsugn, som är en transformator med en sekundärlindning med endast ett fåtal varv omkring en smältdegel.

Högspänningstransformatorer

Hög sekundärspänning under förhållandevis låg sekundärström kännetecknar en
spänningstransformator.

Högspänningstransformatorer används i
distributionsnät, neonskyltar, tändsystem för
förbränningsmotorer, anodspänningsaggregat för sändare o.s.v.
Bild II 2-i O
Bilden visar en transformator med ett gnistgap i sekundärkretsen för tändning av gas.

och klenspänningstransformatorer

2-1 i
Lågspänningstransformatorer används i lokala distributionsnät, vanligen med spänningen 400/230 V. För ökad säkerhet mot
elektrisk chock krävs dock att vissa apparater drivs med en s.k. klenspänning av högst
50 V över en s.k. skyddstransformator med
förstärkt isolering.

112-17

K MP NENTER

PT
h

12

nz

= n1

= 500

220 Vrv

Bild II 2-9 Strömtransformator

Uz
U1 : : 220 V
n1 =500 ~.....-   

-.J

~

Uz:::: 4 400 V

nz =10 000

Bild II 2-1 O Högspänningstransformator

u,

n1

u2

nz

- : : : : - ::::

Uz

n1 ::: 1000 ,   

--.J

n2

u, : : 220 v

1000
so
--20

~U2::::

1

4,4 v

= 20

Bild II 2-11 Klenspänningstransformator

Sambandet mellan varvtal och impedans

Transformatorn kan även användas för anpassning av impedanser. Impedansen Z i en
lindning är proportionell mot kvadraten av
dess lindningsvarvtal n.

112- 18

Om effekten i sekundärlindningen är lika
stor som i primärlindningen, gäller formeln

zp

n/

zs

ns

-=-2

\section{Halvledardioden}
\textbf{HAREC a.\ref{HAREC.a.2.5}\label{myHAREC.a.2.5}}

Allmänt
l en strömkrets kan av olika anledningar
ström tillåtas att flyta i en riktning men kanske inte i den motsatta. En anordning med
en sådan funktion kallas för diod.

Bild II 2-12 överst
En halvledardiod består av ett P-ledande
och ett N-ledande materialskikt som fogats
samman.

Först bestod en diod av två elektroder i
vakuum (se avsnitt 2.7). Därav namnet
vakuumdiod.
Numera består en diod oftast av någon
halvledare. Därav namnet halvledardiod.

Mellan de båda skikten utbildas ett tunt
gränsskikt, som inte innehåller laddningsbärare. Detta skikt kan vara ledande eller
icke ledande - ett spärrskikt- beroende på
polariseringen.

n

p

l

+
+

+
+ ·.·':'.

+
+

=l

pn - skikt utan
pålagd spänning

[>l

r-~

l
l

f

1.. - - ·~

,. .... - . .

\

p

+
+

c=::>

n

+
+

+
+

PASSRIKTNING

-

c:::::t> hålledning
....,.,.. elektronledning

+

+
+

p
+
+

+
+

pn- skiktet uppl6ses

n
SPÄRRIKTNING
pn -skiktet byggs upp

---,
l
l
l

,....--'

't

, .... l

l

:--

J

-~

\.. ... .,.J

o
o
-

+

Bild II 2-12 Spärrskiktet i en halvledardiod
112-19

K MP
Halvledardiodens karaktär
Framström, temperatur, förlusteffekt,
passriktning
Bild II 2-12 mitten
Förbinder man den positiva polen på en
spänningskälla med P-skiktet i en diod och
den negativa polen med N-skiktet så är
dioden polariserad i passriktningen. Spärrskiktet upplöses då och ström flyter genom
dioden. Elektronerna flyter till den positiva
polen och hålen till den negativa polen.
Backspänning, backström, läckström, spärrriktning
Bild II 2-12 underst
Förbinder man i stället den negativa polen
på en spänningskälla med P-skiktet i en diod
och den positiva polen med N-skiktet så är
dioden polariserad i spärrriktningen. Spärrskiktet blir då ännu kraftigare.
Endast en obetydlig ström l flyter genom dioden i den s.k. spärriktnfhgen även
vid ökande spänning U . Men över en viss
spänning ökar strömm~h snabbt- den s.k.
zenereffekten uppstår. Dioden kan då lätt
förstöras av en alltför hög ström.

Bild II 2-13
Strömmen 10 börjar att flyta när spänningen U0 har nått ett tröskelvärde (vid kiseldioder 0.6 V). När spänningen ökar ytterligare däröver, så ökar även strömmen.
Produkten av spänningsfallet överdioden
och strömmen genom den kallas förlusteffekt. Denna värmer upp dioden. Vid för hög
temperatur förstörs kristallstrukturen. En kiselkristall kan klara upp till 200
medan en
germaniumkristall klarar bara 75 \(\circ\)C.

oc

*
1
2
3

f

2

3

Allmän symbol
Zenerdiod
Kapacitansdiod

Bild 112-14 Schemasymboler för dioder

lo
50 mA

1v

l
l

l

l

l
l

Bild II 2-13 Halvledardiodens karaktäristik

112-20

20 nA

Uo

MP NENTER
Diodtillämpningar

Likriktning är det vanligaste tillämpningen
(se kapitel3). Halvledardioder görs även för
en rad andra ändamål och finns i en mängd
utföranden, såsom
• Dioder för spänningsstabilisering (zenerdiod).
Inom ett visst område är spänningsfallet
över en zenerdiod i en strömkrets i det
närmaste konstant medan strömmen varierar. Denna egenskap kallas zenereffekt
och används för konstanthållning av spänning.
Det finns zenerdioder
många olika
spänningar och effekter.
•

Dioder som variabel kondensator, s.k.
kapacitansdiod (VariCap).
När en diod är polariserad i spärriktningen så bildas det ett spärrskikt Olika polariseringsspänning alstrar olika tjocka
spärrskikt En spärrad diod har på så sätt
egenskaper som liknar dem i en variabel
kondensator. Det finns därför dioder där
reglerbarheten av kapacitansen är speciellt utvecklad.

•

Lysdioder (LED).
Energi frigörs i spärrzonen i en diod som
är polariserad i passriktningen. Det sker
genom rekombination av par av laddningsbärare, varvid det normalt avgår
energi i form av värme.
Vid en viss inblandning av främmande
atomer avgår istället ljus. Spänningfallet
över en lysdiod är ungefär dubbelt så
stort som över en kiseldiod, d.v.s. ungefär 1.5 volt. Strömmen är i proportion med
önskad ljusstyrka och mellan 1O och 50

mA.

•

o.s.v ..

Vakuumdioden jämfört halvledardioden

Bild II 2-15
Bilden visar principen för hur de båda diodtyperna ingår i en strömkrets. Den stora
skillnaden är att arbetsspänningen för en
vakuumdiod är mångfalt högre än den för en
halvledardiod samt att vakuumdiodens en a
elektrod (katoden) behöver hettas upp för att
avge elektroner.

R

PASSRIKTNING

R

R

SPÄRRIKTNING

R

Bild II 2-15 Dioders polarisering i kretsen

\section{Transistorn}
\textbf{HAREC a.\ref{HAREC.a.2.6}\label{myHAREC.a.2.6}}

Allmänt

En transistor består av skikt av halvledarelement som sammanfogats. Vanligt är två Nskikt och ett mellanliggande P-skikt (NPNtransistor) eller två P-skikt och ett mellanliggande N-skikt (PNP-transistor). Skikten är
försedda med anslutningar.

B~

...-----Kollektor
(C)
r

n
p

~

+ + ~-------Bas (B)

+ +
~

n
G

l... Emitter (E)

E

NPN

PNP

FET

Bild II 2-16 Schemasymboler

n

c

n

C

Vanliga transistortyper
N PN-transistorer (bipolära)
PNP-transistorer (bipolära)
FET-transistorer (fälteffekt-)

NPN-transistorer

Halvledarskikten kallas
E emitter
B bas
C kollektor

E BC

Spärrzonerna
Bild II 2-17 överst
Mellan skikten B och E respektive mellan B
och C bildas zoner, vars ledningsförmåga
kan styras elektriskt över anslutningarna.
Bild II 2-17 mitten
Spänningskällan U8 E
Mellan bas och e mitter finns en diodsträcka.
När en positiv spänning läggs på basen och
en negativ spänning på emittern, så polariseras diodsträckans spärrzon i passriktningen. Spärrzonen upplöses då och det flyter en
s.k. basström 16 •

p~B
+ +
n

+

E

Bild II 2-17 Skikten i en bipolär transistor

112-23

K MP NENTER
Bild II 2-17 nederst
Spänningskällan UeE

När en positiv spänning läggs på kollektorn
och en negativ spänning läggs på emittern,
så polariseras diodsträckan i spärriktningen.
Spärrzonen förstärks då och det flyter ingen
ström.
Bild 112-18
Inverkan av både U8 Eoch UeE

Två spänningskällor U8 Eoch UcE ansluts till
en emitterkopplad NPN-transistor.
Ur den starkt dopade emitterzonen strömmar elektronerna in i den svagt dopade
baszonen (spänning: U8 E). De flesta elektronerna blir emellertid inte kvar i basen. De
stöter igenom det tunna basskiktet och når
fram till kollektorskiktet med spänningen U E.
Det flyter en kollektorström.
c
För strömmer l (emitterström), 18 (basström) och le (kollektorström) gäller:
lE= Is+ le där 18 <<le

(<<mycket mindre än)

Kollektorströmmen le kan styras med basspänningen U8 E.
En liten ändring i basspänningen ger stor
förstärkande verkan i kollektorströmmen.

n

c

p + + B
+ +

n

h
FE

hFE
IIie

!ll 8

=IIie
/:lf
B

strömförstärkningsfaktorn
ändringen i kollektorströmmen
ändringen i basströmmen

PNP-transistorer
Ersätter man de två N-skikten i en NPNtransistor med P-skikt och P-skiktet med ett
N-skikt så erhåller man en PNP-transistor.
Uppbyggnad, koppling och användning
av en PNP-transistor motsvarar i övrigt den
för en NPN-transistor. Spänningskällorna
måste emellertid ha motsatt polaritet.

R

Is

R

UcE

E
..,..
lE

la<< lE

lE

=Is+ le

Bild II 2-18 Emitterkopplad transistor

112-24

Förstärkningsfaktor
Om strömmen i ingångskretsen för en transistor ändras, så kan strömmen i utgångskretsen ändras mer. Det blir då en förstärkning.
Av sambandet le= f(/8 ) framgår strömförstärkningsfaktorn ~ eller hFE' som är kvoten av ändringen i utgångsströmmen och i
ingångsströmmen i transistorns aktiva (linjära) område.
Bild II 2-19
För emitterkoppling gäller:

UsE

UcE

l c (mA)

100

5

10

UcE ::0 V

UcE = 5 V

UsE {mV)

Bild II 2-19 Karaktäristika för transistor BC 107

112-25

K
Fälteffekttransistorer

Allmänt
Fälteffekttransistorer (förkortat FET) har
mycket hög ingångsimpedans och styrströmmen blir därför mycket svag. Man säger
därför att en FET är spänningsstyrd.
Även NPN- och PNP-transistorer- kallade bipolära transistorer - styrs med spänning, men dessa typer har en relativt lågt
ingångsimpedans och därför högre styrströ m.
Man säger därför att de är strömstyrda.

D+

sBild II 2-20 Schemasymbol för en FET

Bild 112-21 Skikten i en N-kanal FET

Fälteffekttransistorn har tre anslutningar
(elektroder)
S source (katod)
D drain (anod)
G gate (grind, styre)
Fälteffekttransistorns uppbyggnad
Bild II 2-21
Bilden visar ett N-ledande skikt (även kallat
N-kanal) med elektroderna S och D anslutna
till respektive ändar av skiktet. N-kanalen
passerar mellan två P-ledande skikt förbundna med styrelektraden G.
När en spärrspänning läggs mellan G
och S, så breder spärrskikten ut sig och Nkanalen blir trängre. Läggs en negativ spänning på S och en positiv spänning på D, så
kommer det att flyta en ström i N-kanalen.
Strömmens styrka kan påverkas med spänningen på G.
En liten spänningsändring llU medför
stor ändring av strömmen lll i tf-kanalen.
Detta innebär förstärkning. os

Bild II 2-22
l en MOS-FET är G-elektroden isolerad med
ett kiseloxidskikt Funktionssättet är samma
som för en FET. Drain-strömmen kan ökas
eller minskas med hjälp av en positiv respektive negativ spänning på G.

112-26

Bild 112-22 Skikten i en N-kanal MOS-FET
Resistansen mellan gate och source
För att erhålla en förstärkning med en FETtransistor, sätter man in en resister R0 i
drain-strömkretsen. Över resistorn uppstår
då spänningsändringar i proportion med
strömändringarna.
För att fastställa vilaströmmen och därmed arbetspunkten för samma transistor
sätter man in en resister R i source-strömkretsen. storleken på soufce-resistorn ger
sig av önskad gate-förspänning -U 88 .
 -UGs
R s-

lo

K MP NENTER
Sambandet drain-ström och spänning
Bild 112-23
För att beskriva en FET använder man sig
av karaktäristiska kurvor. Vi har redan presenterat bipolära transistorers in- och utgångsegenskaper i kuNform. Eftersom ingångsströmmen (gateströmmen) i en FET
är praktiskt taget noll, så är en sådan kuNa
utan praktisk mening. l stället framställer
man grafiskt sammanhanget mellan styrspänningen UGs och utgångsströmmen (drainströmmen 10 ). Eftersom det finns N-kanal
FET och P-kanal FET så skiljer polariteten
på UGs för dessa båda typer.

Bild II 2-23 Karaktäristikför N-kanal FET

\section{Elektronrör}
Allmänt
Ett elektronrör består av två eller flera elektroder i en lufttom glaskolv.

Direktupphettad
katod

Indirekt upphettad katod

Allmän
symbol

Bild II 2-24 Schemasymboler för dioder

Vakuumdioden (två.elektrodröret)
Bild 112-24
Dioden innehåller två elektroder
a anod
k katod, med f f= glödtråd (filament)
Anoden skall dra elektronerna från katoden.
Katoden skall avge elektronerna och måste
därför hettas upp.
Upphettningen av katoden görs på något
av följande sätt:
Direkt upphettning, d.v.s. katoden är i sig
själv en glödtråd. En 4- till 6-volts strömkälla
är vanligt.
Indirekt upphettning, d.v.s. en glödtråd
omsluter och hettar upp ett speciellt katodmaterial. En 1.5 till12.6 volts glödströmkälla
är vanligt.
Ed i soneffekten
Bild 112-25
När katoden upphettas lossnar fria elektroner från den och bildar ett moln. Med en
spänning mellan anod och katod, med
anoden positiv, så kommer elektronerna att
dras mot anoden. En s.k. anodström börjar
att flyta.

uh

la/Ua .. karaktäristikan för en vakuumdiod
Bild II 2-26
Bild II 2-25 Edisoneffekten
När anoden ges positiv potential (anodspänning), flyter en elektronström från katod
la l Ua- karaktäristik
till anod (anodström).
la
Om anodspänningen
ua ökar' så ökar anodströmmen la. Varje
par av talvärden representerar en punkt
i ett diagram, som det
på bilden. När anodspänningen ökattill ett
Ua
visst värde, så ökar
(
Al B
inte anodströmmen
l
ytterligare.
l ett melA: Initialströmsområde
lanområde är kurvan
B: Den linjära delen
i det närmaste rak.
C: Mättnadsområde
Bild II 2-26 Diodens karaktäristik

112-29

P NENTER
likriktarverkan

När anoden i en vakuumdiod ges positiv
potential i förhållande till katoden, flyter en
s.k. anodström förutsatt att katoden upphettas så att den avger fria elektroner.
När anoden ges en negativ potential i
förhållande till katoden flyter däremot ingen
anodströ m.
Vakuumdioden kan därför användas för
likriktning av växelströmmar. Den har en
likriktande funktion.

En anodström flyter

Halwågslikriktning.
Bild 112-27
När anoden ges en omväxlande positiv och
negativ potential, en växelspänning, så flyter
anodström under varje positiv halvperiod av
växelspänningen. En likströmspuls uppstår
under varannan halvperiod.
Helvågslikriktning.
Bild 112-28
Med ett elektronrör med dubbla anoder och
en transformator med mittuttag på sekundärlindningen, kan växelspänningens båda
halvperioder utnyttjas så, att anodström flyter i samma riktning under alla halvperioder.

Det flyter ingen ström

En pulserande likström flyter

Bild II 2-27 Halwågslikriktning

1 :a halvvågen

Bild II 2-28 Helvågslikriktning

112-30

2 :a halvvågen

K MP NENTER
Ua
VÄXELSPÄNNING
PAANODERNA

HALVVAGsLIKRIKTNING

laf

!@.

f@.

~

HELVAGsLIKRIKTNING

r
F

Bild II 2-29 Likriktande funktion

Vakuumtrioden (treelektrodröret)

Triodens funktion

g2

TRIOD

PENTOD

Bild II 2-30 Symboler för triod och pentod
Trioden innehåller tre elektroder
a anod
g 1 styrgaller
k katod, med f f= glödtråd (filament)

Det flyter både anod- och gallerström

Bild II 2-31
En triod fungerar som en diod, när styrgallret
ges samma potential som katoden. Valet av
förspänning avgör triodens arbetssätt. styrgallret kan ges positiv, neutral eller negativ
potential (förspänning) i förhållande till katoden. Med styrgallret positivt ökar anodströmmen. Med gallret negativt minskar den.
Trioden har en förstärkande funktion eftersom anodströmmen kan styras med styrgallret. En liten ändring av gallerspänningen
medför stor ändring av anodströmmen. Vid
positiv förspänning flyter det gallerström,
som inte får bli för hög. Vanligen väljs en
negativ förspänning.

Det flyter anodström men ingen gallerström

Bild II 2-31 Elektronstömmen i en triod

112- 31

K MP NENTE
Triodens strömkretsar och strömkällor

Glödströmskrets Anodkrets Gallerkrets
Glödbatteri
Anodbatteri Gallerbatteri
Glödspänning Ut Anodsp. Ua Gallersp. U91
Glödström lt
Anodstr. la Gallerstr. 191
Vanligen används nätdrivna strömkällor i
stället för batterier.
Valet av gallerförspänning är avgörande för
triodens arbetssätt.

Tetroden (fyraelektrodröret)

Denna rörtyp innehållerfyra elektroder. Uppbyggnaden är densamma som pentodens,
men bromsgallret saknas.

Pentoden (femelektrodröret)

Pentaden innehåller fem elektroder.
a
anod
g3
bromsgaller
g2
skärmgaller
styrgaller
g1
k
katod, med f f = glödtråd (filament)
Bromsgallret förbinds med katoden. Skärmgallret ges en potential som är något lägre än
anodspänningen. Broms- och skärmgallren
förhindrar elektronerna att studsa tillbaka till
styrgallret efter anslaget mot anoden.

- - - - - - · - - - Ua.
Bild II 2-32 Karaktäristika för elektronrör

112-32

Karaktäristika för elektronrör
Bild II 2-32

1iU91 -diagram för en triod eller pentod, vid Ua
=konstant
liUa- diagram för en triod, vid U~ 1 =konstant
liUa - diagram för en pentoa, vid U91 =
konstant
Tre kurvor visas i VUa- diagrammen, med
olika värden på U91 = konstant (U 91 är s.k.
parameter).

la

l ug1 - karaktäristik för en triod eller pentod

la

l

18

l U8 - karaktäristik för en pentod

ua- karaktäristik för en triod

Branthet S och inre resistans Ri
Bild 112-33

•

Om man (vid konstant anodspänning)
ändrar gallerförspänningen med värdet
ilU 91 så ändrar sig anodströmmen med
värdet illa.

la

il/
Branthet S = a
ilU91
S [mA/V]

illa [mA]

ilU 91 [V]

Bild 112-34
• Om man (vid konstantgallerförspänning)
ändraranodspänningen medilUasåändras anodströmmen med värdet illa
Inre resistans
Ri [kQ]
•

BRANTHET

R.= ilUa
' illa
ilUa [V]

3mA

Om man vill ändra anodströmmen med
ett yärde illa , så ges det två möjligheter:
- Andra gallerförspänningen med värqet ilU 91
- Andra anodspänningen med värdet
il Ua
Med ändring av gallerförspänningen
med värdet ilU 1 kan man åstadkommasamma anodströmsändring illa som
med en ändring av anodspänningen
med värdet ilUa.

:::;. ... ,...J....J..---Ug1
-4 V
-2V

Bild II 2-33 Branthet

9mA

INRE MOTSTAND Rl
8m A

R;

=

v

10
1 mA

v

10
0,001 A

: : - = - - - = 10000

A

:: 10 000 1l

Bild II 2-34 Inre resistans

112-33

K MP NENTER

PT

Barkhausen's elektronrörsformler
Förstärkningsfaktorn J.l
Följande samband gäller mellan de s.k. rörkonstanterna

J1

= S·Ri

Exempel:
Beräkna~,

om S = 2 mA/V
Svar:~=

20

R = 1o kQ
(~är

~

=?

dimensionslös)

Transistorer jämfört med elektronrör
Transistorer
Fördelar:
Lågt pris- små dimensioner -lång livslängd
-enkel strömförsörjning (g lödström behövs
inte) -låg driftspänning (6V, 12V ...... ).
Nackdelar:
Känslighet för överbelastning och höga
temperaturer.
Elektronrör
Fördelar:
Tålighet mot överbelastning
Nackdelar:
Behöver hög anodspänning
Behöver glödström
Utrymmeskrävande

Ett användningsområde, där elektronrör
ännu är vanliga, är i större sändarslutsteg.
Transistorer ersätter numera nästan helt
elektronrören, men man bör ändå känna
elektronrörens egenskaper och arbetssätt.

\section{Digitala kretsar}

Särskilt under det senaste decenniet har
digital elektronik blivit vanlig i utrustningar
för radio- och telekommunikation. Även inom
amatörradio används nu denna teknik. Ämnet är mycket omfattande. Utvecklingen är
att enkla digitala funktioner snabbt ersätts av
komplexa datasystem. Här redogörs endast
för några grundläggande digitala funktioner.
Amatörradions kärna finns ännu i analogtekniken, där det under ett förlopp kan förekomma många olika storheter, t.ex. spänningar mellan noll och ett högsta värde.
l digitaltekniken förekommer bara ett bestämt antal tillstånd. l det enklaste digitala
systemet finns två tillstånd, t. ex. Ooch 1 eller
Till och Från eller Hög och Låg eller Fel och
Rätt o.s.v. Ett system med två tillstånd kallas
binärt. En lampa som tänds eller släcks med
en enkel strömställare är ett binärt system.
Strömställaren kan ha olika utföranden. Det
kan vara en mekanisk kontakt som är styrd
för hand eller av en reläspol e. Det kan också
vara en transistor eller annan anordning.

Transistorn som strömställare
Bild II 2-3S

Bilden visar två transistorkopplingar. Den till
vänster är en analog förstärkare för växelspänning. Om det på grund av en viss basspänning flyter en kollektorström av 1 mA
och kollektorresistorn är S n, så blir spänningsfallet över den resistorn SV. Eftersom
matningsspänningen är 12 V, så blir då
spänningen 7V mellan kollektorn och minus.

Kopplingen till höger fungerar som en
binär strömställare. Antag att insignalen intar ett avtvå spänningstillstånd, antingen OV
(låg) eller SV (hög). När inspänningen är
t. ex. SV, så flyter så mycket basström genom
basresistorns 1O k.Q, att transistorn blir fullt
utstyrd.
Därmed är spänningen mellan kollektor
och emitter, d.v.s. utspänningen, nära O V
(0.1 till 0.2V beroende på transistortyp). Man
säger då att utgången är låg (L) eller O(noll).
Om däremot inspänningen ärO V, såspärras kollektorströmmen och utspänningen blir
nära SV. Man säger då att utgången är hög
(H) eller 1.
För NPN-transistorn i bilden, gäller att
• hög inspänning ger låg utspänning,
• låg inspänning ger hög utspänning.
Denna logiska funktion kallas inverterande.

NOT-gate eller inverterande grind
Bild 112-36

Logiska funktioner beskrivs med internationella symboler. En ring vid utgången betyder att utspänningens nivå är motsatt inspänningens. Sambandet mellan in- och utnivåerna beskrivs med en sanningstabe/1.

m
1

o

Bild 1/2-36 NOT-gate

+12V

r
l

Bild II 2-35 Transistorn som analog förstärkare respektive digital strömställare

112- 3S

p

K

Villkorskretsar- s. k. grindar

Det finns olika sätt att bygga grindar. idag är
de flesta grindarna elektroniska lösningar.
Därutöver finns elektromekaniska grindar i
form av strömbrytare och reläkontakter.
Föregångarna till de elektroniska televäxlarna (AXE m.fl.) var stora system av
mestadels elektromekaniska reläer.
För att överskådligt förklara arbetssättet
i de vanligaste grindarna, görs det enklast
med reläsymboler. En reläkontakt kan då
motsvara en transistor eller diod. Reläspolar
kan motsvara logiska nivåer i insignaler.
Elektriska kontakter kan vara normalt
öppna och sluter vid påverkan (s.k. slutande
kontakt). Alternativt kan de vara normalt
slutna och öppnar vid påverkan (s.k. brytande kontakt). l kretsscheman visas kontaktlägena vid systemet i vila.
Bild 112-37
Av bilden framgår att samma villkor kan
skapas med slutande alternativt brytande
kontakter. Observera då placeringen av resistorn på kretsens utgångssida i respektive
fall. När resistorn ligger närmast pluspolen
kallas den pull-up. När den ligger närmast
minuspolen kallas den pull-down. l båda
fallen definierar resistorn den logiska nivån

Bild
Sanningstabellen i bilden säger, att när alla
insignaler är 1 så är utsignalen också 1.

ELLER-grind eller OR-gate
Bild 112-38

Sanningstabellen säger, att när en ellerflera
av insignalerna är 1 så är utsignalen också
1. När alla insignaler är O, så är utsignalen O.

OCH INTE-grind
Bild 112-39

NANO-gate

Sanningstabellen säger, att när ingen eller
någon insignal är 1, men inte alla, så är
utsignalen 1 . När alla insignaler är 1, så är
utsignalen O.

INTE ELLER-grind eller NOR-gate
Bild 112-40

Sanningstabellen säger, att när någon eller
alla insignaler är 1, så är utsignalen O. När
alla insignaler är O, så är utsignalen 1.

c
c

A

l
l
H
H

B

c

A

l

l
l
l
H

o o o
o 1 o
1
o o

H

l
H

1

B

1

c

1

Bild II 2-37 OCH-grind (AND-gate)

112-36

A
B

c

MP NENTER
+

ATBT-

R

c
A

B

A

B

c

L
L
H

L

H

L
H

H

H

H
H

L

c
o o o
o 1 1
1 o 1
A

B

1

1

1

c

~fic

Bild 112-38 ELLER-grind (OR-gate)

c
A

B

c

A

B

c

L
L
H

L

H

H

H

1
1
1

H

H

L

o o
o 1
1
o

L

H

1

1

o

~fic
A~
B~c

Bild 112-39 OCH INTE-grind (NAND-gate)

112-37

PT

K MP

A

c
B

A

c

R

B

A

B

c

A

L
L

H

L

H

L

L
L
L

o o 1
o 1 o
1 o o
1
1 o

H
H

H

B

c

Bild II 2-40 INTE ELLER-grind (NOR-gate)

Inverterad ingång
En ingång kan behöva ha en inverterad
funktion i förhållande till de övriga (s.k. low
active). Man kan då göra på följande sätt
med en OCH-grind som exempel.

A

c

A
B

c
c

A

B

L
L

H

H

L

L
L

H
H

L

H

L

c
o o o
o 1 1
1 o o
1 1 o

A

B

Bild II 2-41 Inverterad ingång

112-38

A
B
A
B

c

R

A

B

c

c

A

L

B

L

C

A

L

o o o

L

H

H

H
H

L
H

H
L

o
1
1

B

1

o
1

C

1
1

o

A

B

A

B

C

L
H

H
L

L
L

H

o o

1

H

H

H

1
1

1

L

L

A
B

Bild II 2-42 Exklusiv ELLER-grind

C

o

o
o o
1

1

c

(EXOR-gate)

Bild II 2-43 Exklusiv INTE ELLER-grind
(EXNOR-gate)

Exklusiv ELLER-grind (EXOR-gate)
Bild II 2-42

Exklusiv INTE ELLER-grind (EXNOR·gate)
Bild 112-43

Sanningstabellen säger, att när alla insignaler antingen är 1 eller O, så är utsignalen O.
När en, men inte alla insignaler är i, så är
utsignalen 1.

Sanningstabellen säger, attnär alla insignaler
antingen är 1 eller O, så är utsignalen 1. När
en, men inte alla insignaler är 1, så är utsignalen O.

112-39

K MP N
Grindar med dioder och transistorer
l stället för reläer i grindar använder man nu

ytterst sällan något annat än kombinationer
av dioder, transistorer och resistorer.
Bild II 2-44
Bilden visar en NANO-grind. Den egentliga
grinden består av tre dioder och en resistor.
Två av dioderna är ingångar och den tredje
är utgång. Grinden styr en digitalt arbetande
transistor liksom den i bild Il 2-35. Resultatet
är en s.k. DTL-Iogik.

Bild II 2-45
Även denna bild visar en NAND-grind. Här
består den egentliga grinden av en ingångstransistor med två emittrar, vilka motsvarar
dioderna vid A och B i föregående bild.
Kollektorn i denna transistor motsvarar ingångsdioden till transistorn i bild Il 2-44.
De övriga tre transisitorerna i bild Il 2-45
bildar en s.k. switch, som ger snabb övergång mellan väl definierade logiska nivåer.
Resultatet är en s.k. TTL-Iogik

A

B

A
B

Bild 112-44 DTL-Iogik

Bild 112-45 TTL-Iogik

112-40

ENTE
\section{Integrerade Kretsar}
Allmänt om IC
Att integrera betyder att samla till en enhet,
det kan vara komponenter, funktioner, verksamheter etc. Integration kan ske på olika
nivåer och i många olika sammanhang.
Med integration avses här komponenter
för elektroniska strömkretsar. Särskilt halvledarelement av olika slag samt resistorer
och kondensatorer med små värden kan
framställas med små dimensioner. Många
komponenter kan då samlas i samma hölje.
Komponenter inom ett hölje, avsedda för
en viss funktion kallas integrerad krets (eng.
lntegrated Circuit -/C).
Komponenterna i en IC kan i sin tur vara
del av komponenterna en hel strömkrets.
Redan inom höljet kan komponenter kopplas samman för en viss funktion eller som en
del av strömkretsen. Skrymmande eller effektkrävande komponenter, såsom induktorer, transformatorer o.s.v. får emellertid inte
plats, varför även yttre kopplingar behövs.
Det kan också behövas flera IC i en strömkrets - kanske med innehåll för en annan
funktion.
Integrationsgrad
En integrerad krets är uppbyggd på en basplatta av halvledarmaterial - ett chip. På
plattan framställs, med fototeknik eller etsning, kompletta eller nästan kompletta dioder, transistorer, resistorer och kondensatorer. Metoden, som kallas planarteknik, medger att många komponenter kan få plats på
samma platta.
Den snabba utvecklingen av produktionsmetoder för integrerade kretsar gör alltmer
avancerade system möjliga och dessutom
på allt mindre utrymme. Med avseende på
integrationsgrad används följande begrepp.
SSI

Small Scale Integration innebär något
1O-tal halvledare på samma ch ip.
MSI Medium Scale Integration innebär
något 100-tal halvledare på ett ch ip.
LSI Large Scale integration innebär något
10000-tal halvledare på ett ch ip.
VLSI Very Large Scale Integration innebär
100000 eller fler halvledare.

Olika slags integrerade kretsar
Det finns stora sortiment av både standardiserade och speciella IC, varav det finns två
huvudtyper:
e digitala integrerade kretsar,
"'analoga integrerade kretsar.

Digitala IC

Digitala
arbetar som framgår av namnet
med digitala signalnivåer. De enklaste typerna innehåller en eller flera digitala grindar
(se avsnitt 2.8). Genom att koppla samman
grindar kan man skapa kretsar för ett visst
ändamål. i början av 70-talet byggdes komplicerade system av grindar i SSI- och MSIteknik. Ett sådant system är emellertid inte
flexibelt eftersom eventuella ändringar måste göras "hårdvarumässigt". Det innebär att
kopplingsledningar måste ändras om, kanske hela kretsar bytas ut o.s.v ..
l dagens digitala system används IC i
form av en mikroprocessoreller t.o.m. flera.
En mikroprocessor är en avancerad LSIkrets, som kan programmeras (kopplas upp)
"mjukvarumässigt" inte bara för ett ändamål
utan för många olika. l system med mikroprocessorer behövs också minnesfunktioner. Sådana kan också samlas i LSI-kretsar.
Mikroprocessorn är hjärtat i en dator. Styrd
av ett program (mjukvaran) styr den kringutrustningar med uppgift att inhämta och avge
information - att kommunicera.

Analoga IC

Analoga IC arbetar med analoga signalnivåer, d.v.s. spänningar och strömmar med
många olika nivåer och frekvenser. En analog IC kan därför även arbeta med digitala
signaler.
Analoga IC innehåller en balanserad förstärkare eller flera samt olika slags hjälpkretsar. Med yttre komponenter kan en analog IC ges olika förstärkning och frekvensgång. Gemensamt namn för dessa förstärkare är operationsförstärkare (OP-amp).
CP-förstärkare utförs vanligen i SSI- eller
möjligen MSI-teknik.

112-41

K

N

Kombinerade och speciella IC

Utöver standardiserade IC finns kombinerade och speciella IC.
Exmpel på speciella digitala IC är sådana
för telekommunikationsändamåL
Ett annat exempel på digitala IC är sådana för signalbehandling, såväl på HFsom LF-nivå
Exempel på speciella analoga !C är sådana för radiokommunikationsändamåL
Bortsett från vissa skrymmande komponenter och manöverdonen kan numera t. ex. en
IC innehålla en komplett radiomottagare.
Ett annat exempel på speciella analoga
IC är sådana för hörapparater. Genom programmering anpassas de för det personliga
behovet.

Utvecklingen
Det kan sägas hur ofta som helst. Genom
den fantastiska utvecklingen av mikroelektronik öppnas även för radioamatören möjligheter, som bara för ett par decennier inte
var tänkbart.
Denna utveckling harvidgat utrymmet för
den experimentella verksamhetsom amatörradio i grunden innebär. Hobbyn får sålunda
med tiden en allt större teknisk spännvidd.
Aktuell litteratur
Ökat teknikomfång inom amatörradio ställer
motsvarande krav på litteratur. På senare tid
inbegripes även digitalteknik.
Mest av utrymmesskäl behandlas i denna
faktabok digitaltekniken mycket kortfattat,
men ändå så mycket som nämns i CEPTrekommendationen T/R 61-02. För djupare
studium hänvisas till andra läromedel samt
tillleverantörskataloger.

112-42


