\section{Jämförelse mellan supern och detektormottagaren}
\label{superheterojämförelse}

Principen för detektormottagaren är enkel. I en sådan sker allt från
antenn till dernodulering på samma frekvens,
d.v.s. mottagningsfrekvensen. Signalen går utan frekvensomvandlinng
rakt igenom mottagaren. Nackdelen är att det kan uppstå oönskade
självsvängningar på grund av den höga förstärkningen i
LF-förstärkaren. Vidare är det obekvämt att ställa infrekvensen om det
finns flera förselektionskretsar. Med ett kristallfilter som är en
bättre selekteringskrets kan å andra sidan mottagning endast ske på en
fast frekvens. Detektormottagare byggs inte annat än för
specialändamål eller i enkla utföranden för
t.ex. radiopejlorientering och byggsatser.

En utveckling av detektormottagaren är den direktblandade mottagaren,
vilken ler en uppgift i vissa enklare sammanhang.  Denna mottagartyp
är liksom supern avstämbar med en VFO.

Selektionen i den direktblandade mottagaren sker, i motsats till
detektormottagaren inte i förkretsen utan i ett LF-filter. En nackdel
är fortfarande den oundvikliga spegelfrekvensmottagningen. Vidare kan
HF utstrålas från VFO vid ett olämpligt val av
blandarprincip. Principen med direktblandning används emellertid som
demoduleringsmetod t.ex. i SSB-mottagare.

Superheterodynmottagaren är avstämningsbar på ett enkelt sätt med en
VFO.  selektionen görs i den fast avstämda
MFdelen. Spegelfrekvensdämpning görs med förselektion i kombination
med en lämpligt vald mellanfrekvens.

En nackdel med en superheterodyn är att den är mer komplicerad. Vidare
kan även i supern HF utstrålas från VFO om olämplig blandarprincip
väljs.

Men med en dubbelsuper kan spegelfrekvensmottagning lättare undvikas
p.g.a.  en hög 1:a MF samtidigt som en låg 2:a MF medger en bättre
närselektivitet

Fortfarande är risken för oönskade blandningsprodukter stor vid
olämpligt valda oscillatorfrekvenser.

Fastän komplexiteten är relativt stor redan i en dubbelsuper så är den
ännu större i en trippelsuper.
