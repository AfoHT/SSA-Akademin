Bl

KNI

Grupp l
Bild 1.1
Bild 1.2
Bild 1.3
Bild 1.4
Bild 1.5
Bild 1.6

Morsetecknens uppbyggnad ...................................................... ~ ................... 1
Inlärningsordning för morsetecken ................................................................ 2
Rätt sittställning sett framifrån ....................................................................... 3
Rätt sittställning sett från sidan ...................................................................... 4
Rätta handledsrörelser .................................................................................. 4
Telegrafnyckel ................................................................................................ 5

GRUPP Il
Kapite11Bild II 1-1
Bild 111-2
Bild II 1-3
Bild 111-4
Bild II 1-5
Bild 111-6
Bild II 1-7
Bild II 1-8
Bild II 1-9
Bild II 1-1 O
Bild 111-11
Bild II 1-12
Bild 111-13
Bild 111-14
Bild II 1-15
Bildll1-16
Bild 111-17
Bild 111-18
Bild 111-19
Bild 111-20
Bild II 1-21
Bild 111-22
Bild II 1-23
Bild 111-24
Bild 111-25
Bild II 1-26
Bild II i -27
Bild 111-28
Bild 111-29
Bild 111-30
Bild 111-31
Bild 111-32

Atomernas uppbyggnad ................................................................................. 1
Tankeförsök med kulor i ett rör ...................................................................... 3
Potential och spänning i en strömkrets .......................................................... 5
Formelsnurra" för Ohms och Joules lagar ..................................................... 6
Elektriska kraftfält ........................................................................................ 16
Elektrisk fältstyrka ........................................................................................ 17
Kraftfält omkring magneter .......................................................................... 20
Magnetiska fält omkring strömledare ........................................................... 21
Tillämpade e lektromgneter .......................................................................... 22
Vågor längs en linje ..................................................................................... 25
Vågutbredning på en yta .............................................................................. 25
Vågutbredning i rummet .............................................................................. 25
Elektromagnetiskt spektrum ........................................................................ 26
Polarisation av elektromagnetiska vågor ..................................................... 28
Våginterferens ............................................................................................. 29
Alstring av en sinusformad signal ................................................................ 31
Vektorer och fasförskjutning ........................................................................ 33
Ren sinusvåg och övertonshaltig våg .......................................................... 35
Uppdelning av en signal i grundton och övertoner ...................................... 36
Uppdelning av en fyrkantvåg i grundton och övertoner ............................... 37
Överlagrade spänningar .............................................................................. 38
Modulerade signaler .................................................................................... 40
Modulerande signaler .................................................................................. 42
Sidband vid A3E-modulation ........................................................................ 43
A3E-modulation med toner med olika styrka och frekvens .......................... 44
Amplitudmodulation med morsetecken ........................................................ 46
Sidband vid DSB .......................................................................................... 47
Sidbandsval vid SSB ................................................................................... 48
Sidband!ägen vid SSB ................................................................................. 49
Frekvensmodulation .................................................................................... 51
Sidbandsspektrum vid FM-modulering med 1 sinuston ............................... 53
Effektförhållande .......................................................................................... 57
Bilder -1

l
GRUPP Il

Kapitel2..

Bild II 2-1
Bild 112-2
Bild 112-3
Bild 112-4
Bild II 2-5
Bild 112-6
Bild II 2-7
Bild 112-8
Bild II 2-9
Bild II 2-1 O
Bild II 2-11
Bild 2-12
Bild 2-13
Bild 2-14
Bild 2-15
Bild 2-16
Bild 2-17
Bild 2-18
Bild 2-19
Bild 2-20
Bild 2-21
Bild 12-22
Bild 112-23
Bild 112-24
Bild 112-25
Bild 112-26
Bild 112-27
Bild 12-28
Bild 112-29
Bild 2-30
Bild 2-31
Bild 2-32
Bild 2-33
Bild 2-34
Bild 2-35
Bild 2-36
Bild 2-37
Bild 2-38
Bild 2-39
Bild 2-40
Bild 112-41
Bild 112-42
Bild 112-43
Bild II 2-44
Bild 112-45

Bilder-2

Schemasymboler för resistorer ...................................................................... 4
Schemasymboler för kondensatorer .............................................................. 7
Försök med induktion .................................................................................. 1O
Schemasymboler för induktorer ................................................................... 11
Schemasymboler för transformatorer .......................................................... 15
Obelastad transformator .............................................................................. 16
Belastad transformator ................................................................................ 16
Sparkopplad transformator .......................................................................... 17
Strömtransformator ...................................................................................... 18
Högspänningstransformator ........................................................................ 18
Klenspänningstransformator ........................................................................ 18
Spärrskiktet i en halvledardiod ..................................................................... 19
Halvledardiodens karaktäristik ..................................................................... 20
Schemasymboler för dioder ......................................................................... 20
Dioders polarisering i kretsen ...................................................................... 21
Schemasymboler ......................................................................................... 23
Skikten i en bipolär transistor ....................................................................... 23
Emitterkopplad transistor ............................................................................. 24
Karaktäristika för transistor BC 107 ............................................................. 25
Schemasymbol för en FET .......................................................................... 26
Skikten i en N-kanal FET ............................................................................. 26
Skikten i en N-kanal MOS-FET .................................................................... 26
Karaktäristikför N-kanal FET ........................................................................ 27
Schemasymboler för dioder ......................................................................... 29
Edisoneffekten ............................................................................................. 29
Diodens karaktäristik ................................................................................... 29
Halvvågslikriktning ....................................................................................... 30
Helvågslikriktning ......................................................................................... 30
Likriktande funktion ...................................................................................... 31
Symboler för triod och pentod ...................................................................... 31
Elektronstömmen i en triod .......................................................................... 31
Karaktäristika för elektronrör ........................................................................ 32
Branthet ....................................................................................................... 33
Inre resistans ............................................................................................... 33
Transistorn som analog förstärkare respektive digital strömställare ............ 35
NOT-gate ..................................................................................................... 35
OCH-grind (AND-gate) ................................................................................ 36
ELLER-grind (OR-gate) ............................................................................... 37
OCH INTE-grind (NANO-gate) ..................................................................... 37
INTE ELLER-grind (NOR-gate) ................................................................... 38
Inverterad ingång ......................................................................................... 38
Exklusiv ELLER-grind (EXOR-gate) ........................................................... 39
Exklusiv INTE ELLER-grind (EXNOR-gate) ............................................... 39
DTL-Iogik ..................................................................................................... 40
TTL-Iogik ...................................................................................................... 40

NG
GRUPP Il

Kapitel3=

Bild II 3-1
Bild II 3-2
Bild II 3-3
Bild II 3-4
Bild II 3-5
Bild II 3-6
Bild 113-7
Bild II 3-8
Bild 113-9
Bild 3-1 O
Bild 3-11
Bild 3-12
Bild 3-13
Bild 3-14
Bild 3-15
Bild 13-16
Bild II 3-17
Bild 113-18
Bild II 3-19
Bild II 3-20
Bild 3-21
Bild 3-22
Bild 3-23
Bild 3-24
Bild 3-25
Bild 3-26
Bild 3-27
Bild 3-28
Bild 3-29
Bild 3-30
Bild 3-31
Bild 3-32
Bild 3-33
Bild 13-34
Bild 3-35
Bild 3-36
Bild 3-37
Bild 3-38
Bild 3-40
Bild 3-41
Bild 3-42
Bild 3-44
Bild 3-45
Bild 3-46
Bild 3-47
Bild 3-48

Seriekopplade resistorer ................................................................................ 1
Parallellkopplade resistorer ........................................................................... 1
Resistiv spänningsdelare ............................................................................... 2
Wheatstone 's brygga ..................................................................................... 2
Parallellkopplade kondensatorer ................................................................... 3
Seriekopplade kondensatorer ........................................................................ 3
Magnetiskt kopplade induktorer ..................................................................... 4
Uppladdning av en kondensator .................................................................... 5
Urladdning av en kondensator ....................................................................... 6
Inkoppling av en induktor ............................................................................... 7
Faslägen och effekter i LC-kretsar ................................................................. 8
Seriekrets av L+C+R ................................................................................... 1O
Spänningar i seriekrets L+C+R .................................................................... 1 O
Impedansen och fasvinkeln i seriekrets L+C+R .......................................... 11
Parallellkopplad LC-krets ............................................................................. 12
Seriekopplad LC-krets ................................................................................. 13
Thomson's svängningskrets ........................................................................ 14
Resonansfallet i parallellkrets ...................................................................... 15
Resonansfallet i seriekrets ........................................................................... i 5
Q-värden i parallellkrets ............................................................................... i 6
Bandbredd i parallellkrets ............................................................................ i 6
Högpassfilter ................................................................................................ 22
Lågpassfilter ................................................................................................ 23
Bandpassfilter .............................................................................................. 24
Passfilter ...................................................................................................... 25
Bandspärrfilter ............................................................................................. 25
Spärrfilter ..................................................................................................... 26
Kvartskristall ................................................................................................ 26
Bandfilter med kvartskristaller ...................................................................... 26
Mekaniskt filter ............................................................................................. 27
Kavitetsfilter ................................................................................................. 27
Pi-filter .......................................................................................................... 28
T-filter ........................................................................................................... 28
Halvledardioder ............................................................................................ 29
Halv- och helvågslikriktning ......................................................................... 30
Glättning av likspänning ............................................................................... 31
Likriktarkoppling med spänningsdubbling .................................................... 32
Spänningsstabilisering ................................................................................. 32
Från elektronrör till transistor ....................................................................... 33
Principen för förstärkare med elektronrör respektive transistor ................... 34
Grundkopplingar för elektronrör och NPN-transistor ................................... 35
Förstärkare i klass A .................................................................................... 37
Förstärkare i klass B .................................................................................... 38
Förstärkare i klass C .................................................................................... 38
Frekvensmultipliceringskedja ....................................................................... 39
Slutsteg med en transistor ........................................................................... 40
Bilder- 3

BILD F

TECKNING

Bild II 3-49 Mottaktskopplat slutsteg med transistorer ................................................... 40
Bild II 3-50 Högeffekts lutsteg med en tetrod .................................................................. 41
Bild II 3-51 Hög effektslutsteg med två trioder ................................................................ 41
Bild 3-52 Bestämning av PEP-effekten ....................................................................... 42
Bild 3-53 Linjäritetskontroll vid SSB ............................................................................ 43
Bild 3-54 Linjäritetens betydelse ................................................................................. 44
Bild 3-55 Dioddetektorn .............................................................................................. 47
Bild 3-56 Produktdetektor för AM (A3E) och GW (A 1A) .............................................. 48
Bild 3-57 Amplitudbegränsning vid FM-mottagning .................................................... 50
Bild 3-58 Ideal arbetslinje för diskriminator ................................................................. 50
Bild 3-59 Slope-detektorn ........................................................................................... 50
Bild 3-60 Foster-Seeley detektorn .............................................................................. 51
Bild 3-61 Räknardiskriminatorn ................................................................................... 51
Bild 3-62 PLL-demodulatorn ....................................................................................... 52
Bild 3-63 Svängningar ................................................................................................. 53
Bild 3-64 Elektromekanisk oscillator ........................................................................... 54
Bild 3-65 Elektronisk oscillator (Me i Bner) ................................................................... 54
Bild l 3-66 Oscillator enligt Meissner ............................................................................. 55
Bild II 3-67 Emitterkopplad förstärkare ........................................................................... 55
Bild II 3-68 Komplett Meissneroscillator ......................................................................... 55
Bild II 3-69 Svängningsvillkoret ...................................................................................... 56
Bild II 3-70 Hartiey-koppling ........................................................................................... 56
Bild 3-71 TPTG-koppling ............................................................................................. 56
Bild 3-72 Golpitts-koppling .......................................................................................... 56
Bild 3-73b Förstärkare i Glappkoppling ......................................................................... 57
Bild 3-73a Glapp-koppling ............................................................................................. 57
Bild 3-7 4 Bandspridning .............................................................................................. 57
Bild 3-75 Golpittsoscillator med kristall i parallellresonansfallet .................................. 59
Bild 3-76 Golpittsoscillator med kristall i serieresonansfallet ...................................... 59
Bild l 3-77 Superheterodyn-VFO ................................................................................... 60
Bild 3-78 VFO och VGO jämförs ................................................................................. 61
Bild 3-79 Kapacitansdiod- Varicap ............................................................................ 61
Bild 3-80a Analogi Människa-PLL ............................................................................. 61
Bild 3-80b Oscillator med PLL-styrning ........................................................................ 62
Bild 3-81 PLL-oscillator kombinerad med frekvensblandning .............................. ~ ...... 63
Bild l 3-82 PLL med frekvensdelare .............................................................................. 64
Bild II 3-83 Principer för frekvensblandning ................................................................... 67
Bild II 3-84a Entaktsblandaren ......................................................................................... 68
Bild 3-84b Entaktsblandaren ......................................................................................... 69
Bild 3-85 Mottaktsblandaren ....................................................................................... 71
Bild 3-86 Ringblandaren ............................................................................................. 72
Bild 3-87 Jämförelse mellan olika blandare ................................................................ 73
Bild 3-88 Frekvensspektrum från en Super-VFO ........................................................ 75
Bild 3-89 A3E-modulator ............................................................................................. 77
Bild 3-90 Alstring av J3E (SSB) .................................................................................. 78
Bild 3-91 Alstring av F3E (FM) .................................................................................... 79
Bild 3-92 Alstring av G3E (PM) ................................................................................... 80

Bilder-4

KNING
GRUPP Il
Kapitel4m
Bild II 4-1
Bild 114-2
Bild 114-3
Bild II 4-4
Bild II 4-5
Bild 114-6
Bild 114-7
Bild II 4-8
Bild 114-9
Bild 114-10
Bildll4-11
Bild 4-12
Bild 4-13
Bild 4-14
Bild 4-15
Bild 4-16
Bild 4-17
Bild 4-18
Bild 4-19
Bild 4-20
Bild 4-21
Bild 4-22
Bild 4-23
Bild 4-24
Bild 4-25
Bild 4-26
Bild 4-27
Bild 4-28
Bild 4-29
Bild 4-30

Detektormottagare ......................................................................................... 1
selektion i detektormottagare ........................................................................ 2
Detektormottagare med LF-förstärkare ......................................................... 2
Förbättrad selektion ....................................................................................... 2
Förbättrade HF-egenskaper i detektormottagare .......................................... 3
Hög H F-selektion ........................................................................................... 3
CW i detektormottagare ................................................................................. 3
Mottagare med direkt frekvensblandning ....................................................... 5
De modulering i mottagare med direkt frekvensomvandling - CW-signaler ... 4
De modulering i mottagare med direkt frekvensomvandling- SSB-signaler .. 4
selektionen i direktblandade mottagare ........................................................ 6
Passbandbredd och spegelfrekvenser i direktblandade mottagare ............... 8
Superheterodyn mottagaren i princip .............................................................. 9
Dubbelsuperheteodynen i princip ................................................................ 1O
Panoramamottagare .................................................................................... 12
Anslutning av panoramamottagare till stationsmottagare ............................ 12
Signal- och svepspänningar ........................................................................ 12
Mottagningskonverter UHF till KV ................................................................ 13
Transverter mellan UHF och KV .................................................................. 14
AGG vid AM-mottagning med superheterodynmottagare ............................ 15
AGC vid SSB- och CW-mottagning med superheterodynmottagare ........... 16
Enkelsuper med låg MF och ingen förselektion ........................................... 18
Enkelsuper med låg MF och med förselektion ............................................. 18
Enkelsuper med hög MF ochmed förselektion ............................................ 18
Samtidig för- och närselektion i superheterodynmottagare ......................... 19
MF-bandbredd vid AM (A3E) ....................................................................... 20
MF-bandbredd och passband-tuning vid SSB (J3E) ................................... 21
Olika MF-bandbreder vid CW (A i A) ............................................................ 22
SIN-värde ..................................................................................................... 24
SINAD-värde ................................................................................................ 24

GRUPP Il
KapitelsBild II 5-1
Enstegs sändare ............................................................................................ 1
Bild II 5-2
Flerstegs rak sändare .................................................................................... 1
Bild II 5-3
FM-sändare med frekvensmultiplicering ........................................................ 2
Bild II 5-4
2-bands CW-sändare med frekvensblandning .............................................. 2
Bild II 5-5
2-bands SSB-sändare med frekvensblandning ............................................. 3
Bild II 5-6
Flerbands SSB-sändare med frekvensblandning .......................................... 3
Bild II 5-7
PLL-styrd FM-sändare för VHF ...................................................................... 4
Bild II 5-8
PLL-styrd SSB-sändare för kortvåg ............................................................... 5
Bild II 5-9
Separat sändare och mottagare .................................................................... 7
Bildll5-10 Transeeiver med samma VFO ....................................................................... 7
Bild II 5-11 Direktblandad transeeiver med gemensam VFO ........................................... 7
Bild II 5-12 Kristallstyrd 6-kanals FM-transceiver för VHF ............................................... 8
Bilder- 5

Bl

RTE

IN

~©~

EPT

Bild II 5-13 PLL-styrd FM-transceiver för VHF ................................................................. 9
Bild II 5-14 SSB-transceiver för kortvåg ......................................................................... 11
Bild il 5-15 PLL-styrd SSB-transceiver för kortvåg ........................................................ 12

GRUPP Il

Kapitel 6Bild II 6-1
Bild II 6-2
Bild Ii 6-3
Bild II 6-4
Bild II 6-5
Bild II 6-6
Bild II 6-7
Bild II 6-8
Bild II 6-9
Bild II 6-1 O

Bild II 6-11
Bild II 6-12
Bild II 6-13
Bild 116-14
Bild II 6-15
Bild 116-i6
Bild II 6-i 7
Bild II 6-i 8
Bild II 6-19
Bild II 6-20
Bild II 6-21
Bild II 6-22
Bild II 6-23
Bild II 6-24
Bild II 6-25
Bild II 6-26
Bild II 6-27
Bild II 6-28
Bild II 6-29
Bild II 6-30
Bild II 6-3i
Bild II 6-32
Bild II 6-33
Bild II 6-34
Bild II 6-35
Bild II 6-36

Bilder- 6

Spänning och ström i en halvvågsantenn ...................................................... 2
Matningsimpedansen i en halwågsantenn ................................................... 2
Halvvågsdipol matad med harmoniska övertoner .......................................... 3
Elektrisk förlängning och förkortning av antenner .......................................... 5
Vertikaldiagram för halwågsantenn ............................................................... 5
Antennvinst d Bd i effekt ................................................................................. 6
Antennvinst d Bd i spänning ........................................................................... 6
F/B-förhållande i effekt ................................................................................... 6
F/B-förhållande i spänning ............................................................................. 7
Halwärdesbredder ........................................................................................ 7
Inverkan av polarisation ................................................................................. 8
Omvikt di pol ................................................................................................... 9
GP-antenn ..................................................................................................... 9
G P-antenner med elektrisk längdanpassning .............................................. i O
SVF-kuNor för flerbands G P-antenn ........................................................... i O
W3DZZ-antennen ........................................................................................ i1
Riktbar dipol-antenn ..................................................................................... 13
Vagi-antenner .............................................................................................. 13
Cubical Quad-antenner ................................................................................ i 4
Strålningsdiagram för horisontell Vagi-antenn ............................................. 16
Spänningskopplad A./2-dipol ........................................................................ 17
Strömkopplad A./2-dipol ................................................................................ 17
Samma A/2-dipol på grundfrekvensen respektive 1:a övertonen ................ i 8
Koaxialkabel ................................................................................................ 19
Bandkabel .................................................................................................... i 9
ståendevåg på ledning ................................................................................ 21
SVF-problemet enkelt sett ........................................................................... 21
Balansering -Transformering ...................................................................... 22
Ringkärnebalun ............................................................................................ 23
Koaxialledare som bal un ............................................................................. 23
Sätt att ansluta en matningsledning ............................................................. 24
A./2-fasningsledning ...................................................................................... 24
Förlopp i öppen A./4 transmissionsledning ................................................... 26
Förlopp i kortsluten A./4 transmissionsledning .............................................. 27
A./4 transmissionsledning som svängningskrets .......................................... 28
Antennkopplare ............................................................................................ 29

F RTECKNING
GRUPP Il

Kapitel 7-

Bild
Bild
Bild
Bild
Bild
Bild
Bild
Bild
Bild
Bild
Bild
Bild
Bild

Il 7-1
Från sluten LC-krets till antenn ..................................................................... 1
Il 7-2
Pendlingen mellan E-fält och H-fält .............................................................. 2
Il 7-3
Elementär di pol ............................................................................................. 2
Il 7-4
Ett självständigt E-fält skapas ....................................................................... 3
Il 7-5
E-, H- och s-tälten omkring en antenn (förenklad framställning) .................. 3
Il 7-6
E-, H- och S-fält ............................................................................................ 3
Il 7-7
Jonosfärskikten ............................................................................................. 5
Il 7-8
Jonosfärsutbredning ..................................................................................... 6
Il 7-9
Radioprognos för amatörradiobanden på kortvåg ........................................ 8
Il 7-1 O Detalj av radioprognos i Il 7-9 ...................................................................... 9
Il 7-11 Vågutbredning på kortvåg .......................................................................... 1O
Il 7-12 Markbaserad repeater ................................................................................ 14
Il 7-13 Transponder i rymdsatellit .......................................................................... 14

GRUPP Il

KapitelsBild II 8-1
Bild II 8-2
Bild II 8-3
Bild II 8-4
Bild II 8-5
Bild II 8-6

Mätning av sändareffekt ............................................................................... 3
Presentation av mätvärden ........................................................................... 5
Vridspoleinstrument ...................................................................................... 5
Mjukjärnsinstrument ...................................................................................... 6
Konstlast ....................................................................................................... 6
Fältstyrkemätare ........................................................................................... 7
Bild II 8-7
Kalibreringsoscillator i mottagare ................................................................. 7
Bild II 8-8
Brusmätbrygga ............................................................................................. 7
Bild II 8-9
SVF-meter, princip och inkoppling ................................................................ 8
Bild II 8-1 O Frekvensräknare ........................................................................................... 8
Bild II 8-11
Absorbtionsvågmeter .................................................................................... 8
Bildll8-12 Dip-meter ...................................................................................................... 9
Bild II 8-13 Mätning med dip-meter ................................................................................. 9
Bild II 8-14 Oscilloscop ................................................................................................... 9
GRUPP Il

Kapitei9Bild II 9-1
Bild II 9-2
Bild II 9-3
Bild II 9-4
Bild II 9-5
Bild II 9-6
Bild II 9-7
Bild II 9-8
Bild II 9-9

Nätfilter ......................................................................................................... 4
Lågpassfilter för sändare .............................................................................. 5
Högpassfilter för VHF-/UHF-mottagare ........................................................ 5
ingångsimpedansen i resonanskretsar ......................................................... 6
Spärrfilter för mottagare ................................................................................ 6
sugkretsar för mottagare .............................................................................. 6
Nät- och skärmströmfilter .............................................................................. 7
Phonoingångsfilter ........................................................................................ 7
Högtalarledningsfilter .................................................................................... 7
Bild II 9-10a H F-avkopplat styrgaller ................................................................................. 7
Bild II 9-10b HF-avkopplad bas på tre sätt ....................................................................... 8
Bilder-?

BILD F
Bild II 9-11 Parasitfilter i HF-förstärkare ........................................................................... 8
Bild II 9-12 Nycklingsfilter ................................................................................................. 8

GRUPPIII

Kapitei2Bild III 2-i

Bilder- 8

ITU Regionkarta (ur

........................................................................... 2

SAKREGISTER
A
AiA IIi -36

A3E Iii -36
Absorbtionsvågmeter 118-9
Aut. förstärkn.reglering (AGC) 114-15, 114-16
Allmänna elnätet 111 0-2
Amplitudmodulation 111-35, 113-73
Analog IC 112-41
Anod 112-29
Anodspänning 112-29
Anodström 112-29
Anropssignal 1111-8
Anropssignalers sammansättning 1111-8
Antenner 111 0-5
Antennavstämningsenhet 116-20
Antennkopplare 116-20, -29
Antennlängd 116-1
Antennsystem 116-1
Antennvinst 116-5
Arbetspunkt 113-32
Atomkärna 111-1
Atomstruktur 111-1
Aurora-reflexion 117-13
Avkoppling 119-7
Avstämd matarledning 116-17
Avstörning 119-2
B
Backspänning 112-20
Backström 112-20
Balun 116-22
Balansering 116-22
Bandbredd 111-35, 113-16, -29
Bandfilter med kvartskristaller 113-22
Bandkabel 116- i 9
Bandpassfilter 113-20
Bandspridning 113-53
Bandspärrfilter 113-21
Barkhausen 112-34
Basband 111-35
Baskoppling 113-30
BCI (broadcasting interference) 119-2
Beskrivningskod för sändn.slag 111-36, E-1
Bestämning av PEP-effekt 113-38
Blandningsprodukt 113-64
Blockering 119-2
Branthet 112-33
Bromsgaller 112-32
Bruksföremål 111 0-3
Brusmätbrygga 118-7
Brusspärr 114-17
Bågmått IIi -27

c

CEPT 1112-3
GEPT-rekommendationerna 1112-3
Chi p (i IG-) 112-41
Glapp-koppling 113-53
Golpitts-koppling 113-52
Goulomb 111-11
D
Decibel 111-53, G-1
Delta-anpassning 116-24

Demodulator 113-43
Detektor 113-43
Detektormottagare 114-1
Dielektrikum 112-5
Diffraktion 117-4
Digital IG 112-41
Digital krets 112-35
Dioden 112-i 9
Dignitet B-3
Dioddetektorn 113-43
Dip-meter 118-9
Direktblandad mottagare 114-4
Diskriminator 113-45
Dipol 117-1
DTL-Iogik 112-40
D-skiktet 117-5
Dubbelsuperheterodynmottagare 114-1 O
Död zon (skip zone) 117-11
E
Edisoneffekt 112-2
Effekt 111-53
Effektdämpning 111-53
Effektförhållande 111-53
Effektförstärkning 111-53
Effektändring uttryckt i dB 111-53
Effektiwärde 111-27
Effektivt utstrålad effekt- ERP 116- 6
E-fält 117-2
Ekvation B-1
Electromagnetic Gompatibility (EMG) 119-1
Electromagnetic lnterference- EMIII9-2
Electromagnetic Susceptibility- EMS 119-2
Elektrisk chock 111 0-1
Elektrisk effekt- Enheten Watt 111-6
Elektrisk förkortning 116- 4
Elektrisk förlängning 116-4
Elektrisk längd 116-1
Elektrisk fältstyrka 111-11
Elektrisk laddning 111-11

Ord-1

SAKRE ISTE
Elektrisk ledare 111-1
Elektrisk spänning 111-4
Elektriskt arbete- Enheten Joule 111-6
Elektriskt kraftfält 111-11
Elektromagnet 111-15
Elektromagnetiskt fält 111-21, 22, 117-3
Elektromagnetisk våg 111-23, 117-2, A-1
Elektromotorisk kraft- EMK 111-9
Elektron 111-1
Elektronrör 112-29
Elektronskal 111-1
Elementär di pol 117-2
Elementarladdning 111-11
Elementarpartikel 111-1
ELLER-grind eller OR-gate 112-36
EMC-Iagen 119-1
EME-förbindelse 117-13
Emitterkoppling 113-30
Energi 111-53
Enkelsuperheterodynmottagare 114-18
Entaktsblandare 113-64
E-skiktet 117-5
Exklusiv ELLER-grind (EXOR-gate) 112-39
Exklusiv INTE ELLER-grind (EXNOR gate)
112-39
F
F3E 111-37
Farad 112-5
Faslåsning- PLL 113-57
Förluster i kärnmaterial 112-13
Förlustvinkel 112-6
Fasförskjutning 111-28
Fasförskjutning i en induktor 112-12
Fasförskjutning i en kondensator 112-6
Fasledare 1110-3
Fasläge 111-27
Fasmodulation 111-35, 113-76
Flerbands GP-antenn 116-1 O
Flerbands halwågsantenn 116-11
Fonetiska alfabeten 1111-2
Formelsnurra 111-6
Formler B-1
Foster-Seeley diskriminatorn 113-47
Fram-/backförhållande (antennvinst) 116- 6
Framström 112-20
Frekvens 111-28
Frekvensblandare 113-63
Frekvensdeviation 111-48
Frekvenseffektivitet 111-35
Frekvensfilter 113-17
Frekvensgång 113-29
Frekvensinställning 113-53
Ord-2

Frekvensmultiplikation 113-34
Frekvensmodulation 111-35, 113-75
Frekvensräknare 118-9
Fri elektron 111-1
F-skiktet 117-6
Fysikalisk strömriktning 111-4
Fädning eller signalbortfall 117-11
Fälteffekttransistor- FET 112-23, -26
Fältstyrkemätare 118-7
Färgkoder 1110-3
Förstärkningsfaktorn p, 112-34
Förkopplingsresistans 118-1
Förselektion 114-19
Förstärkning 113-29
Förstärkningsfaktor 112-24
Förstärkt isolering IIi 0-3
G
G3E 111-37
Gallerjordad koppling 113-38
Gallerspänning 112-31
Galvaniskt kopplade induktorer 113-4
Gamma-anpassning 116-24
Gauss 111-19
Germanium 111-2
Glimmerkondensator 112-7
Glättningskretsar 113-25
Glödtråd 112-29
Grekiska alfabetet A- 2
Grundläggande matematik B-1
Grundton 111-31
Grundämne 111-1
Gruppantenn 116-15
Gruppcentral 1110-3
Gruppledning 1110-3
Grålinjeutbredning = gray Iine 117-11
Gränsfrekvens 113-17
Jordplanantenn - GP 116-4

H
H-fält 117-2
Halvledardiod 112-19
Halvledare 111-1
Halwågsantenn 116-1
Halwågslikriktning 112-30, 113-25
Halwärdesbredd 116-7
Harmoniska övertoner 116-2
Hartiey-koppling 113-52
Hastighetsfaktor 116- i 9
Helixfilter 113-23
Helvågslikriktning 112-30, 113-25
Henry 111-19, 112-11

SAKR
Hertz 111-28
HF-förstärkare 113-29
Huth-Kuhn-koppling 113-52
Huvudbärvåg 111-35
Höga spänningar 1110-4
Höga strömmar 111 0-4
Högsta användbara frekvens (MUF) 117-7
Högfrekvens HF 111-28
Högpassfilter 113-17, 119-5

l
IARU:s bandplaner 1111-1 O
IARU Region 1 bandplan HF F-1
IARU Region 1 bandplan VHF/UHF/SHF/
EHF F-3
IC 112-41
Icke sinusformad signal 111-31
Impedans 113-1 O
Impedans i resonant krets 113-14
Impedansanpassning 111-55
Impedansomsättning 112-15
Impedansomvandling 113-30
Impedanstransformator 112-15
Induktans 112- 9
Induktiv reaktans 112-11
Induktorn 112-9
Inkoppling av induktor 113-6
Integrationsgrad 112-41
Integrerad krets (IC) 112-41
lntermodulation 114-24
Internationell nödtrafik 1111-7
Internat. Amatörradiounionen 1111-1 O
Internatradioreglementet (RR) 1111-1. 1112-1
Internationella telekonventionen -ITC 1111-1
Internationella teleunionen-ITU 1111-1
Inre resistans 1129, -33
INTE ELLER-grind eller NOR-gate 112-38
Inverterad ingång 112-38
Inverterande grind 112-35
Isolator i 11-1
Isotropisk antenn 116-5

J

J3E 111-37
Jon 111-4
Jonosfärskikt 117-5
Jordfelsbrytare 111 0-3
Jordning av antennsystem 1110-4

K

Kalibreringsoscillator 118-7
Kantvåg 111-31

ISTER

Kapacitans 112-5
Kapacitansdiod (VariCap) 112-21
Kapacitiv reaktans 112-6
Karaktäristisk impedans Z 116-19
Katod 112-29
Kavitetsfilter 113-23
Kirchhoffs lagar 111-6
Kirchhoff's 1:a lag 113-1
Kirchhoff's 2:a lag 113-1
Kisel 111-2
Klass A 113-33
Klass AB 113-33
Klass B 113-33
Klass C 113-33
Koaxialkabel 116-18
Kolfilmsresistor 112-2
Kollektorkoppling 113-30
Kondensator, pappers-, plast-, elektrolyt-,
keramisk 112-7
Kondensatorn 112-5
Kondensatorn i växelströmskretsen 112-6
Konduktivitet 111-1
Konstgjord andning 1110-2
Konstlast 118-6,
Korsmodulation 114-24
Kortslutningsström 111-9
Kraftförsörjning 113-25
Kristalldetektor 114-1
Kristalloscillatorer 113-55
Kritisk frekvens 117-6
Kritisk vinkel 117-7
Kvartskristall 113-22

l
Laddningsmängd 111-11
Lag om EMC 119-1
Lagen om radiokommunikation 11125
Le-oscillatorer 113-51
Likspänning 111-4
LF-detektering 119-2
LF-förstärkare 113-29
Lineartransponder 13
Linjäritetskontroll vid SSB 113-39
Linjär och olinjär potentiometer 112-3
Ljusberoende resistor, fotoresister 112-2
Ljusvågor 111-23
Logaritmer B-5
Lysdiod 112- 21
Lågfrekvens LF 111-28
Lågpassfilter 113-1 7. 119-4
Läckström 112-6, -20
Lägsta användbara frekvens (LUF) 117- 7
Ord-3

S KRE ISTER
M

Magnetisk flödestäthet 111-1g
Magnetisk fältriktning 111-15
Magnetisk fältstyrka 111-1g
Mag netiskt flöde 111-1g
Mag netiskt fält 111-15
Magnetiskt kopplade induktorer 113-4
Magnetism 111-15
Magnetfältberoende resister 112-2
Markbaserad relästation- repeater 117-i 3
Markvåg 117-1 O
Massaresistar 112-2
Match-box 116-20
Matningsimpedans 116-2
Meissner-koppling 113-51
Mekanisk längd 116-1
Mekaniskt filter 113-23
Mellanfrekvens 114-1 O
Metallers resistivitet A-1
Metallfilmresistor 112-2
Metalloxidresistor 112-2
MF-bandbredd 114-1g
MF-skift 114-22
Miikrofarad 112-5
Mikroprocessor 112-41
Minnesfunktion 112-41
Missanpassning 116-4
Mittmatad halwågsantenn 116- g
Mjukjärnsinstrument 118-2, -6
Modulation 111-35
Modulationsindex 111-48
Modulationssystem 111-35
Modulatorer 113-73
Modulera 111-35
Modulerad signal 111-35
Modulerande signal 111-35
Momentanvärde 111-27
Morsesignalering 11-1
Morsetecknen 11-i
Mottagare 114-i
Mottagningskonverter 114-13
Mottaktsblandare 113-66
Måttenheter A-1
Människokroppen 1110-1
Märkning av kondensator 112-7
Märkning av resister 112-4
Mögel-Dellinger-effekten 117-5

N

Nanofarad 112-5
Nervbaning 11-2
Neutroner 111-1
Ord-4

N-ledning 111-2
Nolledare 1110-3
NOT-gate 112-35
NPN-transistor 112-23
Nukleon 111-1
Nycklingsfilter IJg-8
Närselektion 114-1g
Nödsignaler 1111-7
Nödtrafik 1111- 7

o

Gavstämd matarledning 116-18
OCH INTE-grind eller NANO-gate 112-36
OCH-grind eller AND-gate 112-36
Ohms lag 111-6
Ohms lag vid växelström 113-1 i
Omvikt dipol (folded dipole) 116-4, g
Operationsförstärkare 112-41
Optimal trafikfrekvens (FOT) 117-7
OSCAR-satellit 117-14
Oscillatorer 113-4g
Oscillator med PLL-styrning 113-57
Oscilloskop 118-1 O,

p
Panoramamottagare 114-11
Parabolantenn 116-15
Parallellkopplade kondensatorer 113-3
Parallellkopplade LC-kretsar 113-12
Parallellkopplade resistorer 113-1
Parasitfilter ng-8
Passband-tuning 114-22
Passfilter 113- 2i
Passriktning 112-20
Period Il i -28
Periodtid Il i -28
Permanentmagnet l l i -15
Permeabilitetstal (fältkonstant) 111-i g
Pi-filter 113-24
Pikofarad 112-5
P-ledning- "hålledning" 111-4
PLL- dernodulatom 113-47
PLL-styrd FM-transceiver 115-g
PLL-styrd sändare 115-4
PNP-transistor 112-23, -24
Polspänning 111-g
Post- och telestyrelsen- PTS 1112-5
Post- och telestyrelsens föreskrifter 1112-5
Potenser B-3
Primärlindning 112-i 5
Produktdetektor 113-43
Proton 111-1

ISTER
Pull-down 112-36
Pull-up 112-36
Pulsmodulation 111-35
Påverkan från elektromagnetiska fält 111 0-1
Q

Q-faktorn i en parallellkrets 113-16
Q-faktor- godhetstal 112-12
Q-koden l l 11-3
Quad-antenn 116-4, -12

R

Radioanläggning
Radiolagen 119- i
Radiotrafik vid naturkatastrofer 1111-7
Radiovågor Il i -23
Rak mottagare 114-1
Rak sändare 115-1
Rapportkoder 1113-1, J-1
Reflexion 117-4
Reflexion mot Es (sporadiskt
117-13
Reflexion mot meteor- Meteorscatter 117-13
Refraktion 117-4
Regler och trafikmetoder III i -1
Resistans 112-1
Resistansen i människokroppen 1110-1
Resistiva material 112-1
Resistivitet 112-1
Resister 112-1
Resolution 640 vid WARC 1979 1111-7
Resonansfallet i en parallellkrets 113-14
Resonansfallet i en seriekrets 113-15
Restladdning i kondensatorer 1110-5
Ringblandare 113-66
Ringkärnebalun 116-23
Rymdsatellit-baserad relästation 117-13
Rymdvåg 117- i O
Räknardiskriminatorn 113- 47
Rötter B-4

s

Schemasymboler
för dioder 112-20
för induktorer 112-11
för kondensatorer 112-7
för resistorer 112-4
för transformatorer 112-15
för transistorer 112-23
för elektronrör 112-31
sekundärlindning 112-15
selektion 114-2
selektivitet 114-17

s-enheter D-1
S-fäit 117-2
Seriekopplade kondensatorer 113- 3
Seriekopplade LC-kretsar 113-13
Seriekopplade resistorer 113-1
signalkänslighet 114-23
Signalstyrkemätare (S-meter) 114-17
SINAD 114-24
IC'TI"\.-rn~lf"t signal 111-27
Självinduktion 112-9
Självsvängningsvillkoret 113-52
Skin-effect (yteffekt) 112-12
Skip-avstånd 117-11
Skydd och jordning IIi 0-6
skyddsjordning 1110-3
skyddsledare 111 0-3
Skärmgaller 112-32
Skärmning av elektriskt fält 111-13
Skärmning av magnetiskt fält 111-19
Slope-detektorn 113-45
S/N 114-24
Snabba och tröga säkringar 1110-4
solaktivitet 117-9
Solfläckstal 117-9
Sparkopplad transformator 112-17
Spegelfrekvensdämpning 114-1 O
Spegelfrekvenser i direktblandare 114-7
Spegelfrekvensproblemet 114-17
Sporadiskt E-skikt 117-5
Spänningsnod 116- 2
Spänningsberoende resister 112-2
Spänningsdelare 113-2
Spänningshöjande likriktarkopplingar 113-27
Spänningsstabilisering 113-27
Spänningstransformator 112-15
Spänningsstyrd oscillator (VCO) 113-57
Spänningsändring uttryckt i dB IIi -54
Spärrfilter
, 119-5
Spärriktning
Spärrskikt 112-19
Spärrzon 112-23
stationsdagbok (loggbok) 1113-1
Strålningslob 116-2
Strömbrytare l l 10-2
Strömbuk 116-2
Strömkrets 111-5
Strömshunt 118-1,
Strömtransfermater 112-15
Strömändring uttryckt i dB 111-54
sändning 11
styrgaller 112-31
ståendevågmeter (SVF-meter) 118-8
Ord-5

SAKRE ISTER
Stående vågor 116-20
Störningsproblem 119- 2
Sugkrets, sugfilter 113-21, 119-5
Superheterodynmottagare 114-9
Superheterodynsändare 115-3
Superheterodyn-VFO 113-56
Svenska bandplaner G-1
Svenska repeaterfrekvenser H-1
Svängningar 113-50
säkerhetsåtgärder 111 0-5
Switchade aggregat 113-28
Sändare med frekvensmultiplicering 115-1
Sändarslutsteg 113-36
Sändningsslag 111-35
Särjordning IIi O- 4

T

T-anpassning 116-24
Teckendelii-i,
Teknisk strömriktning lli-4
Temperaturinversion 117-i3
Temperaturberoende resistor 112-3
Temperaturkoefficient 112-7, -i2
Temperaturkompensation 113-62
Tesla 111-i9
T-filter 113-24
Thomson' s svängningskrets 113-i3
Tjockfilmsresistor 112-2
Topp-till-toppvärde lli-27
Toppvärde eller amplitud lli-27
Toppvärdeseffekt P. E. P. IIi-56
Transistorn 112-23
Transeeiver 115-7
Transformatorn 112-i5, 116-22
Transmissionsledning 116-i7, -25
Transverter 114-i4
Troposfären - Troposcatter 117-i2
Trådlindad resistor 112-2
TTL-Iogik 112-40
Tunnfilmsresistor 112-2
TVI (television interference) 119-2

u

Underbärvåg 111-35
Uppladdning av kondensator 113-5
Urkoppling av induktor 113-6
Urladdning av kondensator 113- 5
Utbredningsförlust 117-4
Uttag och stickproppar med jorddon 1110-3

Ord-6

PT
v

Vakuumdiod 112-19, -29
Vakuumtriod 112-31
Vakuumpentod 112-31
Valenselektron 111-1 , -2
Variabel frekvens oscillator- VFO 113-51
VariCap 112-21
Varvtalsomsättning 112-15
Vektor 111-28
Vridspoleinstrument 118-2
Våghastighet 116-1
Våginterferens 111-25
Vågledare 116-i9
Vågpolarisation lli-23, 116- 8
Vågutbredning 111-21,117-1, -10
Vågutbredningsförutsägelser 117-7
Vågutbredningsmodeller 111-21
Vänsterhandsregeln 111-15
Växelspänning 111-4
Växelströmskretsar 113- 8

w

W3DZZ-antennen 116-4
Weber 111-19
Wheatstone's brygga 113-2

x

Inget sökord

y
Vagi-antenn 116-4, -12
Yteffekt (skin-effect) 112-12

z

Zenerdiod 112-21

Å

Åska 1110-6
Ä
Ändmatad halvvågsantenn 116-9

ö

Överhettning 111 0-4
Överlagrad spänning 111-34
Överton 111-31
Övertonskristaller 113-55

RAT UR
The American Radio Relay League, lnc. (ARRL):
The ARRL Handbook for Radio Amateurs.1994, Seventy-First Ed., ISBN 0-87259-171-9
The ARRL Radio Amateur's Library.
Cuno, Hans. H.:
Vorbereitung auf die Amateur-Funk-Lizenzprilfung.
frech-verlag, 1993, 16. Auflage. ISBN 3-7724-5402-X.
Deutscher Amateur-Radio-Ciub e. V. (DARC):
Ausbildungsunterlagen.
DARC Verlag, 1983.
Ekbom, Lennart:
Tabeller och formler N T Te.
Esselte Studium, 2. upplagan, 1986, ISBN 91-24-34594-6.
Experimenterende Danske Radioamat0rer (EDR):
Vejen til sendetilladelsen.
EDR, 6. udgave, 2. oplag, ISBN 87-85149-02-0.
Follbring, Tommy:
Radioteknik för sändareamatörer.
Ljudbandsinstruktioner AB.
Föreningen Sveriges Sändareamatörer (SSA):
Populär Amatörradio. SSA, 1952.
Grundläggande Amatörradioteknik. SSA, 2. upplagan, 1970.
Hall, T; Pederby, Bo; Elmgren, Bo:
Fysikboken för högstadiet.
Esselte Studium, 2. upplagan, 1974, ISBN 91-24-69278-6.
Haraldsson, Tore:
Radioteknik för radioamatörcertifikat
Radio TV KB Haraidsson \& Söner, 8. upplagan 1989, ISBN 91-970362-1-8.
lsännäinen, Antti:
Amatöörtekniikkaa PerusJuakan Kursseille.
Suomen Radioamatööriliitto r.y. (SRAL), 1987, ISBN 951-96056-1-4.
Lindkvist, Olle:
Antenner. 1993, ISBN 99-0830753-3
Radiosändare. 1989
Ljudbandsinstruktioner AB
Lundqvist, Hans; Roos, Olle:
Elektronik.
Esselte Studium. 2. upplagan 1982, ISBN 91-24-32173-7.
Moltrecht, Eckart K. W.:
Amateurfunk-Lehrgang.
Teil1, 2. Auflage 1987,
Teil2, 2. Auflage 1989,
Teil 3, 2. Auflage 1987,
Teil 4, 2. Auflage 1989,
frech-verlag.

ISBN
ISBN
ISBN
ISBN

3-7724::5386-4,
3-7724-6387-8,
3-7724-5388-0,
3-7724-5389-9.

Litteratur- 1

LITTER
Norsk Radio Relre Liga (NRRL):
Radioamaterens ABC. Laarebok i radioteknikk.
NRRL, 1995, 1. utgave- 2. opplag.
Pietsch, Hans-Joachim:
Kurzwellen-Amateurfunktechnik.
Franzis Verlag, 1984, ISBN 3-7723-6592-2.
Rothammel, Karl:
Antennenbuch.
Franckh' sche Verlagshandlung, 1978, ISBN 3-440-04498-X.
de Sousa Pires, Jorge:
Electronics Handbook.
Studentlitteratur, 1989, ISBN 91-44-21021-3.
Svenska Elverksföreningen, Elektriska Installatörsorganisationen (EIO), Elsäkerhetsverket
och Röda Korset:
Livräddning vid e/skada.
Elsäkerhetsverkets Publikationsservice, 1996, ISBN 91-88924-00-9.
Svenska Elverksföreningen, Elektriska Installatörsorganisationen (EIO):
E/kunskap för vardagsbruk, 1994, ISBN 91-76221-04-0.
Händig med el, utg. 2, rev. 1995:08.
Energikontorets Förlagsservice.
Södra Vätterbygdens Amatörradioklubb (SVARK):
Möt världen genom etern.
Föreningen Sveriges Sändareamatörer (SSA), 1993, ISBN91-86368-07-9.
Wallander, Per:
Bli sändaramatör.
PERANT Per Wallander AB, 3. omarbetade upplagan 1995, ISBN 91-86296-06-X.
Lennart:
EL-LARA och RADIOTEKNIK.
TextdeiiSBN 91-86368-05-2.
Bilddel ISBN 91-86368-04-4.
Föreningen Sveriges Sändareamatörer (SSA), 1990.
Wibe~g,

ÖVSV ADXB-OE (Österrike):
Amateurfunk-Lizenzlehrgang. Der Weg zum Amateurfunk.
Orbit, 1982, 25. Auflage, ISBN 3-85216-001-4.

Litteratur- 2
