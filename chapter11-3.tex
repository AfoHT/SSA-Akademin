\section{Trafikförkortningar, vanliga i amatörradio}
Utöver Q-koden och klartext används vid
morsetelegrafering även andra trafikförkortningar. Eftersom det internationella radiospråket är engelska, är förkortningar av engelska ord vanligast.
Förkortningar bör emellertid inte användas i onödan. En ovan operatör vid motstationen kan då få svårt att förstå meddelandet.

Urval för radioamatörer
l CEPT-rekommendation T/R 61-02 nämns
utöver Q-koden följande övriga trafikförkortningar, som berör amatörradio.
Radioamatörerna använder i praktiken
många fler trafikförkortningar än dessa. En
lista kan beställas från SSA:s kansli.

Ett exempel på en avsnitt ur en amatörradiosändning, där trafikförkortningar används
särskilt flitigt:
``gm es tnx vy much om fer ur rprt. u are
cmg in hr ufb. my tx is .... and rx .... anta 3
el beam . condx hr gud mni dx stns hrd . wl
nw nil so tks es 73``
l klartext ser exemplet ut så här:
``good morning and thank you very much
Old Man for your report. You are coming in
here ultra fine business. My transmitter is .....
and receiver .. ... antenna is a 3 element
beam. Conditions here are good many
stations heard. Weil now nothing for you so
thanks and kindest regards``

l reglementsprovet för radioamatörcertifikat ingår frågor om trafikförkortningar.
Förkortning Engelskt uttryck

Svensk betydelse

BK
CQ

avbryt(-a) (sändningen)
allmänt anrop, till alla
telegrafi (A 1A)
från ..... (anropssignal)
``kom''
meddelande, telegram
var god (att .... )
allt uppfattat, mottaget
mottagare
sändare
din, ditt, dina, er

cw

DE
K
MSG
PSE
R
RX
TX
UR

break
``seek you''
continous waves
franska ``de''
come
message
please
received
receiver
transmitter
your

Utöver ovanstående trafikförkortningar upptas i CEPT-rekommendationen även följande bokstavskombinationer, vilka används i
teleprintertrafik i stället för motsvarande
morsetecken, slagna utan tecken mellanrum.
(Strecket ovanför bokstäverna betecknar
att det inte finns något mellanrum).

Vidare upptas i CEPT -rekommendationen bokstavskombinationen RST som en
trafikförkortning. Denna får tydas som en
fråga eller anmodan om signal rapport.
Mer om detta under 13. stationsdagbok
och Appendix J.

AR
sluttecken
+
VA eller SK avslutningstecken @

1111 - 5

RE LE
