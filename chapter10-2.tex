\section{Allmänna elnätet}
\textbf{
HAREC a.\ref{HAREC.a.10.2}\label{myHAREC.a.10.2}
}
\label{jordning}

Elektrisk energi levereras till förbrukarna över
transformatorstationer där högspänning först transformeras till
lågspänning. Från transformatorstationerna förgrenas lågspänningsnätet
till serviceskåp ute i kvarter och byar.

I Sverige är fördelningstransformatorns sekundärlindningar oftast
sammankopplade till ett Y (s.k. Y- eller stjärnkoppling) där
mittpunkten är jordad.

De i Sverige vanligast förekommande 3-fas lågspänningsnäten har
huvudspänningen 400 V (tidigare 380 V) och fasspänningen 230 V
(tidigare 220 V). Spänningen mellan fasledarna kallas för
huvudspänning och spänningen mellan respektive fasledare och
nollledaren kallas för fasspänning.

Bruksföremålen i huset ansluts oftast 1-fasigt, d.v.s. mellan någon av
fasledarnaoch nolledaren. Någorlunda lika belastning mellan faserna är
önskvärd.  Mer effektkrävande apparater som el-pannor och spisar
ansluts därför till alla tre faserna (3-fasigt).
Amatörradioutrustningar ansluts oftast 1-fasigt.

\subsection{Strömbrytare}

Kraftförsörjningen av radiostationens apparater bör ske över en
gemensam huvudströmbrytare, som lätt kan nås. En indikatorlampa får
gärna markera att den brytaren är tillslagen och att stationen är
under spänning.

Informera familjen och övriga i din omgivning om hur den brytaren
fungerar. Det är en säkerhetsåtgärd om något skulle hända.

Apparaternas nätströmbrytare skall vara utförda för den aktuella
arbetsspänningen och ha ett godkänt utförande.

Vid 1-fassystem skall nätströmbrytaren i apparaterna vara 2-polig och
bryta fas- och N-ledare, men aldrig PE-ledaren.

Vid 3-fassystem skall nätströmbrytaren
vara 3-polig och bryta fasledarna, men aldrig
N-ledare och PE-ledare.

\textbf{Kom ihåg, att behörig installatör skall
  anlitas vid ingrepp i fasta installationer.}

\subsection{Liten terminologi vid elinstallationer}
\begin{description}[style=nextline]
\item[Gruppcentral] Den säkringscentral som följer efter elmätaren,
  t.ex. i villor och lägenheter.
\item [Gruppledningar] Ledningar efter en gruppcentral, d.v.s.
  ledningar till belysning, el-spisar, uttag m.m.
\item[Fasledare] En ledare som för fasspänning.
\item[Nolledare (N-ledare)] En ledare som är ansluten till elnätets
  s.k.  nollpunkt (nollskena) och som normalt inte skall föra spänning
  till jord.
\item[Skyddsledare (PE-ledare)] De ledare i kablar och sladdar, som är
  speciellt avsedda för skyddsjordning.
\item[Bruksföremål] Ett i princip flyttbart elanslutet föremål,
  t.ex. handverktyg och radioapparater.
\item[Förstärkt isolering] Vissa bruksföremål tillverkas med en så god
  isolering att de inte behöver skyddsjordas. Så isolerade får
  anslutningsledningen förses med en speciell stickpropp, som passar i
  vägguttag, såväl med som utan jorddon. Sådana bruksföremål är märkta
  med symbolen \(\boxbox\) och får inte ändras så att de kan skyddsjordas.
\end{description}

\subsection{Färgkoder för fas, noll- och skyddsledare}

Isoleringsmaterialet omkring gruppledarna i fasta elinstallationer har
färger som fyller en viktig funktion. Dessa färger får därför aldrig
förväxlas.

Fasledaren har i regel svart färg. N-ledaren (nollan) har blå färg.
Det är till fas- och N-ledarna i vägguttagen, som man kopplar
apparaterna för att få ström. Helst skall uttagen också ha jorddon
d.v.s. en extra kontakt - ett s.k. jordningsbleck. Detta bleck är
anslutet till PE-ledaren (skyddsjorden), som är färgad randigt
gult/grönt.

\emph{En gul/grön ledare är alltid en skyddsjordledare och får endast
  användas för det.} I äldre installationer kan emellertid
skyddsledarens isolering vara t.ex. röd.


\subsection{Uttag och stickproppar med jorddon}

Jorddonet ger förbindelse med elsystemets skyddsjord (PE).

Det är rummets utförande, som avgör om vägg- och lamputtagen där skall
ha uttag med jorddon. Bostadsrum är klassade som inte särskilt
riskfyllda och har därför tidigare inte försetts medlamp- och
vägguttag med jorddon. Vid nybyggnation är emellertid numera alla
uttag försedda med jorddon!

Kök och tvättstugor med ledande plåtbänkar, vattenkranar o.s.v. anses
som riskfyllda rum och måste ha uttag med jorddon.  Samma gäller
källare och liknande andra rum med ledande golv, väggar och
inredningar.

Det är tillrådligt att installera uttag med jorddon för
radiostationen. Observera då, att alla uttag i det rummet skall ha
jorddon!

\subsection{Skyddsjordning}

Att jorda är det vanliga uttrycket för att ansluta ett föremål till
ett jordtag. Metallhöljen på apparater kan av olika anledningar bli
spänningsförande och är då en elsäkerhetsrisk. För att säkert ha
nollpotential på höljena kan de kopplas till jordskenan via
PE-ledaren - d.v.s. skyddsjordas. När man ansluter apparathöljet till
jorddonet, kommer säkringen att bryta strömtillförseln om det blir
isolationsfel mellan en strömförande del och höljet. PE-ledaren får
därför aldrig brytas!

\emph{Om skyddsjordning finns särskilda föreskrifter. Om du inte är
  säker på hur skyddsjordning skall utföras, fråga en behörig
  installatör.}

\subsection{Jordfelsbrytare}

Jordfelsbrytare kallas en brytare som automatiskt bryter spänningen,
när det uppstår överledning till skyddsjordade detaljer -
s.k. jordfel.  Brytaren mäter felströmmen och bryter spänningen innan
strömmen uppnår ett farligt värde, t. ex. 10 mA. Jordfelsbrytare får
inte ersätta skyddsjordning, men kan under särskilda förutsättningar
komplettera skyddsjordningen som en extra säkerhetsåtgärd. Låt
installera jordfelsbrytare!

\subsection{Särjordning}

Särjordning är ett uttryck för att jorda apparater till en separat
jordpunkt, Det görs via separat jordlina till ett jordtag,
d.v.s. jordplåt eller jordspett. Särjordning skall ske på rätt sätt
eftersom det avsedda skyddet annars kan bli en fara.

\emph{Särjordning får ske endast om skyddsjordning till PEN också har
  gjorts. Om du har planer på särjordning, fråga en behörig
  installatör.}

\subsection{Jordning av antennsystem}

I brist på annan jordpunkt är det frestande att ansluta
antennjordledaren till PE-ledarens anslutningsbleck i vägguttaget med
förhoppning att på så sätt få ett bättre HF-jordplan för
antennen. Detta är emellertid ett dåligt exempel på särjordning, som
både kan innebära säkerhetsrisker och medföra störningsproblem.

\subsection{Snabba och tröga säkringar}

Det finns snabba och tröga säkringar. Snabba säkringar är det som
normalt används. Tröga säkringar för samma strömstyrka kan behövas för
apparater som har speciellt hög startström, t.ex. stora
nättransformatorer med toroidkärna. Säkringarna skall kunna bryta
tillräcklig hög spänning, annars blir det en kvarstående ljusbåge i
dem vid säkringsbrott. Använd säkringar med rätta strömvärden. Det är
förbjudet att laga säkringar, vilket naturligtvis kan orsaka både
brand och andra faror.

