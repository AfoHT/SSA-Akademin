\section{Allmänna elnätet}
\textbf{
HAREC a.\ref{HAREC.a.10.2}\label{myHAREC.a.10.2}
}
\label{jordning}

Elektrisk energi levereras till förbrukarna över
transformatorstationer där högspänning först transformeras till
lågspänning. Från transformatorstationerna förgrenas lågspänningsnätet
till serviceskåp ute i kvarter och byar.

I Sverige är fördelningstransformatorns sekundärlindningar oftast
sammankopplade till ett Y (s.k. Y- eller stjärnkoppling) där
mittpunkten är jordad.

De i Sverige vanligast förekommande 3-fas lågspänningsnäten har
huvudspänningen 400 V och fasspänningen 230 V. Spänningen mellan fasledarna
kallas för huvudspänning och spänningen mellan respektive fasledare och
nollledaren kallas för fasspänning.

Bruksföremålen i huset ansluts oftast 1-fasigt, d.v.s. mellan någon av
fasledarna och nolledaren. Någorlunda lika belastning mellan faserna är
önskvärd.  Mer effektkrävande apparater som el-pannor och spisar
ansluts därför till alla tre faserna (3-fasigt).
Amatörradioutrustningar ansluts oftast 1-fasigt.

\subsection{Strömbrytare}

Kraftförsörjningen av radiostationens apparater bör ske över en
gemensam huvudströmbrytare, som lätt kan nås. En indikatorlampa får
gärna markera att den brytaren är tillslagen och att stationen är
under spänning. Informera familjen och övriga i din omgivning om hur
den brytaren fungerar. Det är en säkerhetsåtgärd om något skulle hända.

Apparaternas nätströmbrytare skall vara utförda för den aktuella
arbetsspänningen och ha ett godkänt utförande.

Vid 1-fassystem skall nätströmbrytaren i apparaterna vara 2-polig och
bryta fas- och N-ledare, men aldrig PE-ledaren.

Vid 3-fassystem skall nätströmbrytaren vara 3-polig och bryta fasledarna,
men aldrig N-ledare och PE-ledare.

\textbf{Kom ihåg, att behörig installatör skall
  anlitas vid ingrepp i fasta installationer.}

\subsection{Liten terminologi vid elinstallationer}
\begin{description}[style=nextline]
\item[Gruppcentral] Den säkringscentral som följer efter elmätaren,
  t.ex. i villor och lägenheter.
\item[Gruppledningar] Ledningar efter en gruppcentral, d.v.s.
  ledningar till belysning, el-spisar, uttag m.m.
\item[Fasledare] En ledare som för fasspänning.
\item[Nolledare (N-ledare)] En ledare som är ansluten till elnätets
  s.k.  nollpunkt (nollskena) och som normalt inte skall föra spänning
  till jord.
\item[Skyddsledare (PE-ledare)] De ledare i kablar och sladdar, som är
  speciellt avsedda för skyddsjordning.
\item[Bruksföremål] Ett i princip flyttbart elanslutet föremål,
  t.ex. handverktyg och radioapparater.
\item[Förstärkt isolering] Vissa bruksföremål tillverkas med en så god
  isolering att de inte behöver skyddsjordas. Så isolerade får
  anslutningsledningen förses med en speciell stickpropp, som passar i
  vägguttag, såväl med som utan jorddon. Sådana bruksföremål är märkta
  med fi-symbolen \(\boxbox\) och får inte ändras så att de kan skyddsjordas.
\end{description}

\subsection{Färgkoder för fas, noll- och skyddsledare}

Isoleringsmaterialet omkring gruppledarna i fasta elinstallationer har
färger som fyller en viktig funktion. Tyvärr har användningen av dessa
färger ändrats flera gånger under årens lopp, vilket skapar risker för
förväxling. Dessa färger får därför aldrig förväxlas.

Fasledaren har numera brun färg vid nyinstallation, men har tidigare varit
både svart, grå, vit eller röd.
N-ledaren (nollan) har numera blå färg vid nyinstallation, men har tidigare
varit både svart och vit.
Skyddsledaren (PE-ledare) med gul/grön längsgående randig färgmärkning är
alltid en skyddsjordledare och får endast användas för det ändamålet. I äldre
installationer kan emellertid skyddsledarens isolering vara t.ex. röd.

Det är till fas och N-ledarna i vägguttagen, som man kopplar apparaterna för
att få ström. Helst ska uttagen vara i skyddsjordat utförande dvs. med ett
jordningsbleck. Detta bleck är anslutet till den gul/gröna ledaren för
skyddsjord.

\subsection{Uttag och stickproppar med jorddon}

Jorddonet ger förbindelse med elsystemets skyddsjord (PE).

Det är tidigare rummets utförande som avgjorde om vägg- och lamputtagen skulle
ha uttag med jorddon.

Kök och tvättstugor med ledande plåtbänkar, vattenkranar o.s.v. anses
som riskfyllda rum och måste ha uttag med jorddon.  Samma gäller
källare och liknande andra rum med ledande golv, väggar och
inredningar.

Bostadsrum var klassade som inte särskilt riskfyllda och har därför tidigare
inte försetts medlamp- och vägguttag med jorddon.

Vid nybyggnation är emellertid numera alla uttag är av skyddsjordat utförande!

Det rekommenderas att installera skyddsjordade vägguttag för
radiostationen. Observera då, att alla uttag i det rummet skall vara
skyddsjordade!

\subsection{Skyddsjordning}

Att jorda är det vanliga uttrycket för att ansluta ett föremål till
skyddsjord. Men uttrycket används även lite slarvigt i andra fall utan att
syfta på skyddsjordning av elsäkerhetsskäl.

Metallhöljen på elektrisk utrustning kan av olika anledningar bli
spänningsförande och är då en elsäkerhetsrisk. För att minska risken för
farlig spänningssättning av metallhöljet ansluts höljet till skyddsjord.
Om det blir isolationsfel mellan en strömförande del och höljet kommer
säkringen att bryta strömtillförseln och risken för skada minskar.
\textbf{PE-ledaren får därför aldrig brytas!}

\emph{För skyddsjordning finns särskilda föreskrifter. Kontakta därför en
behörig installatör.}

\subsection{Jordfelsbrytare}

Jordfelsbrytare är en automatisk strömbrytare som snabbt bryter strömmen
då strömmen till eller från en apparat är olika. Detta kan inträffa vid
ett jordfel eller vid överledning i en skyddsjordad apparat eller i andra
fall när inkommande ström och utgående ström genom jordfelsbrytaren inte
är lika stora.

Jordfelsbrytaren skyddar dig:
\begin{itemize}
\item Vid isolations- och jordfel
\item Om chassiet på en apparat blir strömförande
\item Om du kommer åt spänningsförande delar och jord samtidigt
\item Om du använder en apparat på ett felaktigt sätt i våtutrymmen
\item Om du installerat en apparat på att felaktigt sätt
\item Om apparatens kabel skadats
\item Mot och minimerar risken för brand
\end{itemize}

Jordfelsbrytaren \textbf{skyddar inte} för strömmar som går genom fasledare
och neutralledare eller genom fas till fasledare (3-fas).

Jordfelsbrytare får inte ersätta skyddsjordning, men kan under särskilda
förutsättningar komplettera skyddsjordningen som en extra säkerhetsåtgärd.
Vid nyinstallation av bostäder är det numera krav på att minst en
jordfelsbrytare ska installeras. Installera gärna jordfelsbrytare i äldre
anläggningar!

\subsection{Särjordning}

Särjordning är ett uttryck för att jorda apparater till en separat
jordpunkt, Det görs via separat jordlina till ett jordtag,
d.v.s. jordplåt eller jordspett. Särjordning skall ske på rätt sätt
eftersom det avsedda skyddet annars kan bli en fara.

\emph{Om du har planer på särjordning, fråga en behörig installatör.}

\subsection{Jordning av antennsystem}

I brist på annan jordpunkt är det frestande att ansluta
antennjordledaren till PE-ledarens anslutningsbleck i vägguttaget med
förhoppning att på så sätt få ett bättre HF-jordplan för
antennen. Detta är emellertid ett dåligt exempel på särjordning, som
både kan innebära säkerhetsrisker och medföra störningsproblem.

\subsection{Snabba och tröga säkringar}

Det finns snabba och tröga säkringar. Snabba säkringar är det som
normalt används. Tröga säkringar för samma strömstyrka kan behövas för
apparater som har speciellt hög startström, t.ex. stora
nättransformatorer med toroidkärna.

Säkringarna skall kunna bryta tillräcklig hög spänning, annars blir det
en kvarstående ljusbåge i dem vid säkringsbrott. Använd säkringar med rätta
strömvärden. Det är förbjudet att laga säkringar då det kan orsaka brand.

