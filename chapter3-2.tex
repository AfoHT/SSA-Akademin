\section{Frekvensfilter}

%% sid II3-7/138

Frekvensfilter används inom radiotekniken för många olika ändamål, t.ex. för att
\begin{itemize}
\item eliminera störande signaler,
\item  öka avstämningsskärpan (selektiviteten) i mottagare och sändare,
\item framhäva eller dämpa ett sidband i en AM-signal m.m.
\end{itemize}

Beroende på den s.k. frekvensgången, så indelas filtren i flera ``familjer'',
varav de vanliga presenteras här.

Beroende på det tekniska utförandet finns dels s.k. passiva filter vilka
använder extern energi för sin funktion, och dels aktiva filter vilka i princip
är förstärkare som likaledes använder passiva kretsar. Här presenteras
för enkelhetens skull passiva filter.

Traditionella frekvensfilter är vad som kallas analoga. Men nu i dataåldern
börjar även digitala filter vinna intåg. Sådana är dock för komplicerade för
att behandlas här.

\subsection{Högpassfilter}

Bild II 3-22

Ett högpassfilter släpper igenom signaler med höga frekvenser och dämpar dem med
låga frekvenser.

Exempel: En frekvensberoende spänningsdelare som LC-högpassfilter.

Vid låga frekvenser är \(X_C\) stor och \(X_L\) liten. Över XL uppstår då ett
litet spänningsfall - en låg utgångsspänning \(U_a\). Resultatet blir att
låga frekvenser dämpas.

Vig höga frekvenser är \(X_C\) liten och \(X_L\) stor. Över \(X_L\) uppstår då
ett stort spänningsfall- en hög utgångsspänning \(U_a\). Resultatet blir att
höga frekvenser släpps igenom.

\(X_L\) kan bytas ut mot en resister \(R\), men då blir passbandkurvan inte så
brant.

\emph{Gränsfrekvens}

Gränsfrekvensen \(f_g\) beror av kapacitansen \(C\), induktansen \(L\) samt
resistansen \(R\).

LC-högpass:
\begin{gather*}
  f_g = \frac{1}{2π\sqrt{LC}} \\
  f_g\ \text{[Hz]} \quad L\ \text{[H]} \quad C\ \text{[F]}
\end{gather*}

RC-Högpass:
\begin{gather*}
  f_g = \frac{1}{2πRC}
  f_g\ \text{[Hz]} \quad R\ \text{[Ω]} \quad C\ \text{[F]}
\end{gather*}

Räkneexempel:
\begin{enumerate}
\item \(L = 4\ \text{H} \quad C = 1\ \text{µF} \quad f_g =\ ?\)
  \[
  f_g = \frac{1}{2π\sqrt{4\cdot 2 \cdot 10^{-6}}} = \frac{10^3}{2π}
  = 79.62\ \text{Hz}
  \]
\item \(R = 1\ \text{kΩ} \quad C = 10\ \text{nF} \quad f_g =\ ?\)
  \[
    f_g = \frac{1}{2π \cdot 1 \cdot 10^3 \cdot 10 \cdot 10^{-6}}
    = \frac{10^5}{2π} = 15.934\ \text{Hz}
  \]
\end{enumerate}

\subsubsection{Lågpassfilter}

Bild II 3-23

Om induktor och kondensator respektive resister och kondensator i ett
högpassfilter byter plats, så får man i stället ett LC-lågpassfilter respektive
ett RC-Iågpassfilter.

Ett lågpassfilter släpper igenom signaler med låga frekvenser och dämpar dem med
höga frekvenser.

Exempel: En frekvensberoende spänningsdelare som LC-Lågpassfilter.

Vid låga frekvenser är \(X_C\) stor och \(X_L\) liten. Över \(X_L\) uppstår då
ett litet spänningsfall - en hög utgångsspänning \(U_a\). Resultatet blir att
låga frekvenser släpps igenom.

Vig höga frekvenser är \(X_C\) liten och \(X_L\) stor. Över \(X_L\) uppstår då
ett stort spänningsfall - en låg utgångsspänning \(U_a\). Resultatet blir att
höga frekvenser dämpas.

\emph{Gränsfrekvens}

Samma formler används vid beräkning av gränsfrekvensen både i lågpass- och
högpassfilter, således

LC-Lågpass:
\begin{gather*}
  f_g = \frac{1}{2π\sqrt{LC}} \\
  f_g\ \text{[Hz]} \quad L\ \text{[H]} \quad C\ \text{[F]}
\end{gather*}

RC-Lågpass:
\begin{gather*}
  f_g = \frac{1}{2π\sqrt{RC}} \\
  f_g\ \text{[Hz]} \quad R\ \text{[Ω]} \quad C\ \text{[F]}
\end{gather*}

Bild II 3-22 Högpassfilter

Bild II 3-23 Lågpassfilter

\subsection{Bandpassfilter}

Bild 3-24

Ett bandpassfilter släpper igenom signaler bara inom ett frekvensområde medan
signaler inom andra frekvensområden dämpas.

Bandpassfiltret består i enklaste fall av två svängningskretsar av LC-typ, vilka
är avstämda till angränsande frekvenser. Kretsarna är kopplade induktivt,
kapacitivt eller galvaniskt.

Beroende på kopplingsgrad skiljer man mellan underkritisk koppling (lös
koppling), kritisk koppling och överkritisk koppling (fast koppling).

På bilden visas hur passbandet påverkas bl.a. av kopplingsgraden. Lös koppling
liten bandbredd. Kritisk koppling - större bandbredd. Fast koppling - stor
bandbredd.

Bild II 3-24 Bandpassfilter

\subsection{Passfilter}

Bild II 3-25

Passkretsen stäms av till en viss frekvens och erbjuder där en mycket låg
impedans. Passkretsen kopplas i serie med signalvägen och låter signaler med
frekvenser inom filtrets passband att passera.

Bild II 3-25 Passfilter

\subsection{Bandspärrfilter}

Bild II 3-26

Om serie- och parallellkretsarna i ett bandpassfilter byter plats, så får man
i stället ett bandspärrfilter. Ett sådant spärrar signaler inom ett visst
frekvensområde, men släpper igenom signaler utom detta område.

Bild II 3-26 Bandspärrfilter

\subsection{Spärrfilter}

Bild II 3-27

\emph{Spärrkrets}
Spärrkretsen stäms av till en viss frekvens  och erbjuder där en mycket hög
impedans. Spärrkretsen kopplas i serie med signalvägen och spärrar en signal
med samma frekvens som resonansfrekvensen.

Bild II 3-27
\emph{Sugkrets}
Sugkretsen stäms av till en viss frekvens och erbjuder där en mycket låg
impedans. Sugkretsen kopplas parallellt med signalvägen och kortsluter (suger
bort) en signal med samma frekvens som resonansfrekvensen.

Bild II 3-27 Spärrfilter (2 sorter)

\subsection{Kvartskristall}

Bild II 3-28

En kvartskristall, egentligen en slipad skiva av kvarts, kan fungera som en
elektromekanisksvängningskropp (resonator), vars egenskaper liknar dem i en
LC-krets.

Den låga inre resistansen gör att Q-värdet i en kvartskristall är bättre än
10000. Som jämförelse är Q-värdet i en LC-krets oftast sämre än 1000.

\subsection{Bandfilter med kvartskristaller}

Bild II 3-29

Kvartskristaller kan kombineras till filter med önskad bandbredd. Även
utföranden med keramiska resonatorer finns. Resonatorerna är avstämda till var
sin bestämda frekvens och hela komplexet bidrar på så sätt till att bilda
passband eller andra egenskaper på samma sätt som med sammankopplade LC-kretsar.

Bild II 3-28 Kvartskristall

Bild II 3-29 Bandfilter med kvartskristalle

\subsection{Mekaniska filter}

Bild II 3-30

Med en elektromekanisk givare kan man få en kropp (resonator) att svänga på sin
resonansfrekvens. Med ännu en elektromagnetisk givare kan man känna av
svängningarna och återvandla dem till elektriska signaler. Hela anordningen
fungerar som en elektromekanisk resonator, vars egenskaper liknar dem i en
LC-krets.

Resonatorerna kan kombineras till filterkomplex med önskad bandbredd där
resenatorerna är avstämda till var sin bestämda frekvens. Hela komplexet bidrar
på så sätt till att bilda ett passband på samma sätt som med sammankopplade
LC-kretsar. Beroende på tillämpningen finns olika frekvenslägen i intervallet
60-600 kHz.

Mekaniska filter användes mest förr som
mellanfrekvensfilter i högvärdiga radioutrustningar, men har numera till stor del ersatts
av bandfilter med kvartskristaller där arbetsområdet kan ligga avsevärt högre i frekvens.

\subsection{Kavitetsfilter}

Bild II 3-31 Kavitetsfilter

Bild II 3-31

Svängningskretsars dimensioner minskar med ökande frekvens. Vid mycket hög
frekvens kan induktorns varvtal i en LC-krets ha minskat till ett enda varv
samtidigt som kapacitansen inom detta enda varv kan räcka för önskad
resonansfrekvens.

En sådan svängningskrets kan bl.a. ha formen av en ledare mitt inne i en
elektriskt ledande kavitet. Ledarens längd tillsammans med kavitetens insida
bildar induktorn. Mellan ledaren och kavitetens insida råder en kapacitans,
som kan kompletteras/justeras med en extra kondensator.

Inkommande och utgående signaler ansluts till filtrets mittledare över
induktionskondensatorer eller direkt galvaniskt. kavitetsfilter kan kopplas
ihop för att bilda bandfilter, frekvensdelare m.m.

\subsection{Helixfilter}

När ett kompakt kavitetsfilter behövs, så kan man öka reaktansen i mittledaren
både induktivt och kapacitivt genom att utforma den som en spiral (helix).
Detta är dock på bekostnad av Q-värdet. Flera kavitetsfilter kan kopplas ihop
för att bilda bandfilter, spärrfilter m.m.

\subsection{Pi-filter}

Bild II 3-32

För att överföra HF-signaler med bästa verkningsgrad, så är det viktigt med god
impedansanpassning mellan de olika funktionerna. Om anslutningsimpedansen är
lika i båda funktionerna, så behövs inga extra åtgärder. Är impedanserna däremot
olika, så behövs korrigeringsnät (filter).

Ofta är nätet Pi-format och består av induktanser och kapacitanser. Ett
Pi-format nät kan sägas bestå av två L-formade nät ställda mot varandra, där
den seriella delen är gemensam (på bilden en induktor).

\subsection{T-filter}

Bild II 3-33

Ett nät kan också vara T-format och bestå av induktanser och kapacitanser. Ett
sådant nät kan sägas bestå av två L-formade nät ställda ``rygg mot rygg''. Då är
den parallella delen gemensam. På bilden visas två alternativ.

När den parallella delen är kapacitiv, blir huvudkaraktären ett lågpassfilter,
men att impedansanpassning också är möjlig med en induktiv impedansdelning.

När den parallell delen är induktiv blir huvudkaraktären ett högpassfilter, men
att impedansanpassning också är möjlig med en kapacitiv impedansdelning.

Ett Pi- eller T-filter kan fungera som
\begin{itemize}
\item svängningskrets,
\item impedanstransformator (anpassning),
\item balansera ut en reaktans o.s.v.
\end{itemize}

Bild II 3-32 Pi-filter

Bild II 3-33 T-filter
