
\chapter*{Förord}

\section*{Amatörradio}

Amatörradio är en teknisk hobby med inriktning på kommunikation och experiment
med radioanläggningar samt radiovågors utbredning. Det är en verksamhet som
utövas över hela världen av licensierade radioamatörer, även kallade
sändaramatörer.

Syftet med amatörradio är att främja personlig utveckling och internationell
förståelse samt teknisk färdighet och erfarenhetsutbyte inom området.
Amatörradio kan därtill vara en tillgång då samhällets normala resurser för
radiokommunikation behöver förstärkas.

\section*{En hobby med krav}

För att inneha och använda en radioläggning i ett land, krävs tillstånd (licens) från
dess teleadministration. För ett amatörradiotillstånd föreskrivs i det internationella
radioreglementet bland annat handhavandemässiga och tekniska kvalifikationer hos
varje person som önskar använda en amatörradiostation. De nationella teleadministrationerna tillser detta genom kompetensprov.

\section*{Utbildningsställen}

Amatörradioklubbarna bedriver huvuddelen
av utbildningen med radioamatörcertifikat
som mål. Även vissa skolor, militära förband
m.fl. har amatörradio på programmet. l någon utsträckning förekommer även självstudium. Rekrytering av handledare för terminslånga kurser är en nyckelfråga för kursarrangören, liksom målinriktade, anpassade läromedel.

\section*{Andra förutsättningar}

Den svenska teleadministrationen har omdanats på senare tid. Därvid har även
amatörradioanvändningen berörts, främstgenom att provförrättningarna för
amatörradiocertifikat delegerats till av myndigheten utsedda, ideellt arbetande
förrättare. Vidare genom att teleadministrationerna inom CEPT har infört
harmoniserade certifikats- och tillståndsklasser för amatörradio. Främst av
dessa anledningar har det uppstått behov av samordnade hjälpmedel för utbildning
och examinering, vilket amatörradiorörelsen själv har att tillgodose.

\section*{Föreningen Sveriges Sändareamatörer - SSA}

SSA är en ideell förening för personer med intresse för amatörradio.
Verksamheten är religiöst och politiskt obunden. Ett av syftena är att bland
medlemmarna verka för ökade tekniska kunskaper och god radiotrafikkultur för att
därigenom skapa en kår av kunniga radioamatörer. SSA representerar Sverige som
nationell förening i The International Amateur Radio Union (IARU), Region 1.

\section*{Internationell samverkan}

De nationella föreningarna inom IARU samarbetar över nationsgränserna. Ett
exempel är när DARC (Deutscher Amateur-RadioClub e. V.) för några år sedan
ställde sina Ausbildungsunterlagen till SSAs förfogande som källmaterial för
boken El-lära och Radioteknik. Detta material har till stor del kunnat utnyttjas
även i här föreliggande bok.

xi

\clearpage

TACK!

Ett stort tack till alla dem, som på olika sätt bidragit till att förverkliga
boken. Ett särskilt tack till Bengt Falkenberg SM7EQL och Bertil Nordahl SM7CZL,
vilka har varit rådgivare och sakgranskare. Tack också till Ulf Sjöden SM6CVE
för svenska texter för bilderna från Ausbildungsunterlagen.

Författaren

xii
