\chapter{VÅGUTBREDNING}

Elektromagnetisk vågutbredning är energitransport och förutsättningen för all radiokommunikation. Radiovågornas utbredning
på vägen mellan sändare och mottagare
påverkas emellertid på många sätt. Med
vetskap om radiovågornas utbredningssätt
kan man mer metodiskt försöka uppnå önskade radioförbindelser.

\section{Kraftfälten omkring antenner}

För att sända ut och ta emot radiovågor
behövs antenner. Mycket förenklat är en
antenn en elektrisk krets, som består av en
induktor och en kondensator.
Med kondensatorns elektroder helt isärdragna och förminskade har svängningskretsen fått ett mycket annorlunda mekaniskt utseende. Sedan induktorn i LC-kretsen tagits bort, så återstår mekaniskt sett
endast en enkel ledare, men elektriskt sett
finns kretsen ändå kvar. Ledaren med sin
utsträckning är fortfarande en induktor och
ytorna på dess motstående halvor är fortfarande elektroderna i kondensatorn med omgivningen som dielektrikum.
En elektrisk ledare, en stång, tråd etc. är
alltså en elektrisk svängningskrets, vars resonansfrekvens mest bestäms av längden
och tjockleken. Ledaren (antennen) kan kallas dipol- den har två poler. Se Bild II 7-1.
Vissa likheter finns mellan en mekanisk
pendel och en elektriskt svängningskrets.

\subsubsection{Mekanisk pendel}

Energin i en mekanisk pendel växlar mellan
två ytterlighetstillstånd. Det ena är när pendeln just vänder i ett ytterläge. Då innehåller
den mest lägesenergi och ingen rörelseenergi. När pendeln rör sig mot mittläget, så
omvandlas lägesenergin till rörelseenergi. l
mittläget, som är det andra ytterlighetstillståndet, innehåller pendeln mest rörelseenergi och ingen lägesenergi etc ..

\subsubsection{Elektrisk svängningskrets}

Den elektriska svängningskretsen kan sägas motsvara den mekaniska pendeln i den
meningen att det i båda fallen pågår en
ständig pendling mellan lägesenergi och rörelseenergi. Se Bild II 7-2
När strömmen i den elektriska svängningskretsen just upphört för att vända, så
innehåller kondensatorn mestladdning, d. v .s.
det starkaste elektriska fältet mellan elektroderna. Detta fält är lägesenergi. Den utjämningsström som följer, från den ena elektroden över till den andra, omges av ett
magnetiskt fält. Detta fält är rörelseenergi.
Förloppet visas i Bild II 7-2, där det framgår att dipolen omges av det starkaste elektriska fältet vid tidpunkten t=O samt vid t=1/
2T med omvänd polaritet, där T är periodtiden. Vidare att dipolen omges av det starkaste magnetiska fältet vid tidpunkten t=1/
4T samt vid t=3/4T med omvänd strömriktning och fältpolaritet

+

~~+

LTSluten
resonanskrets

Di pol

Bild II 7-1 Från sluten LC-krets till antenn

117- 1

VÅ

-

Max E- fält
Min H- fält

Min E- fält
MaxH-fält

Min E- fält
MaxH-fält

Bild II 7-2 Pendlingen mellan E-fält och H-fält
Med förklaringen av E- och H-fälten som
bakgrund följer nu en enkel framställning av
hur radiovågor uppstår ur dessa fält.
Maxwell påvisade i sina ekvationer bl. a.
sambandet mellan elektroner i rörelse i en
ledare och elektromagnetiska vågor i rummet. Vidare, att elektroner som rör sig med
avtagande eller tilltagande hastighet avger
elektromagnetisk energi.
Hur energi strålar från en ledare kan förklaras med en (tänkt) elementär dipol, som
genomflyts av växelström (Bild II 7-3).

-(])Bild II 7-3 Elementär dipol
Di polen består av två lika stora elektriska
laddningar med motsatt polaritet. När
len matas med en växelström, så rör sig
laddningarna ständigt, omväxlande emot
respektive ifrån varandra. Tänk på två kulor
i var sin ände av en spiralfjäder. Avståndet
mellan laddningarna ändras i takt med styrkan och riktningen på strömmen. Systemet
är alltså under ständig hastighetsändring
(ökning resp. minskning), vilket är förutsättningen för att energi skall strålas ut.
Först är laddningarna nära varandra på
grund av liten laddning. Vid ökande ström
ökar avståndet mellan laddningarna och det

117-2

byggs upp ett mer utbrett och energirikt Efält. Samtidigt byggs även ett H-fält upp
omkring dipolen, vinkelrätt mot E-fälteto.s.v ..
Detta gäller både för en elementär di pol och
en elektrisk ledare med många fria elektroner (verklig antenn).
Formeln för det resulterande s-fältet är
E. vilket visar att den lagrade energin
i di polens närmaste omgivning ökar när avståndet (potentialen) mellan dipolens laddningar ökar.
Bild II 7-4 visar hur ett E-fält byggs upp
omkring en dipol och avskiljs från den. De
visade kraftlinjerna är E- fältet. H-fältet visas
inte, men ligger vinkelrätt mot E-fältet, i cirklar omkring antennen. Se Bild II 7-5.
När dipolens laddningar ändrar riktning
och åter börjar att röra sig emot varandra,
börjar det E-fält som byggts upp att också
byta riktning. Men det kommer inte att falla
tillbaka till dipolens mitt utan sluts till ett eget
kmt!=:lcmn- Maxwells första ekvation. Jämför
med en såpbubbla som lämnat blåsröret.
Omkring dipolen har det nu bildats ett självständigt E-fält, som sin tur alstrar ett eget Hfält.
En period av en elektromagnetisk våg
(ett S-fält) har alstrats och fortsätter att utFörvarje följande period alstras ett
som separeras från antennen och
bildarett
H-fälto.s.v .. Varjegångbildas
alltså en ny "fältbubbla" inne i den föregående, vilken håller på att utvidgas.
Resultatet är ett elektromagnetiskt fält,
d.v.s. en radiovåg.

s=

VÅ UTBREDNING

Bild II 7-4 Ett självständigt E-fält skapas

Bild II 7-5 E-, H- och B-fälten omkring en antenn (förenklad framställning)
Som nämts består en radiovåg av ett högfrekvent elektromagnetiskt fält (S). Det är i
sin tur sammansatt av två andra fält, det
elektriska E- och det magnetiska H-fältet.
Energin i S-fältet fördelas lika mellan E-fältet
och H-fälten, vars krafter korsar varandra
vinkelrätt. S-fältet ligger i plan med både Eoch H-fälten och breder ut sig vinkelrätt mot
dem. s-fältets riktning beror av den inbördes
riktningen på E- och H-fälten.
När E-fältet är vertikalt, sägs vågen vara
vertikalt polariserad. När samma fält är horisontellt sägs vågen vara horisontellt polariserad. När E-fältet roterar i vågfrontens
plan, och därmed även H-fältet, sägs vågen
vara cirkulärt polariserad.

Fälten framställs i text och bild som s.k.
kraftlinjer med pilar som föreställer kraftriktningen. Linjernas längd föreställer fältets
styrka. Bild II 7-6 visar ett avsnitt av en
vågfront S med vertikal polarisation.
