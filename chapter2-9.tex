\section{Integrerade Kretsar (IC)}

\subsection{Allmänt om IC}

Att integrera betyder att samla till en enhet, det kan vara komponenter,
funktioner, verksamheter etc. Integration kan ske på olika nivåer och i många
olika sammanhang.

Med integration avses här komponenter för elektroniska strömkretsar. Särskilt
halvledarelement av olika slag samt resistorer och kondensatorer med små värden
kan framställas med små dimensioner. Många komponenter kan då samlas i samma
hölje.

Komponenter inom ett hölje, avsedda för en viss funktion kallas
\emph{integrerad krets (eng. Integrated Circuit - IC)}.

Komponenterna i en IC kan i sin tur vara del av komponenterna en hel strömkrets.
Redan inom höljet kan komponenter kopplas samman för en viss funktion eller som
en del av strömkretsen. Skrymmande eller effektkrävande komponenter, såsom
induktorer, transformatorer o.s.v. får emellertid inte plats, varför även yttre
kopplingar behövs. Det kan också behövas flera IC i en strömkrets - kanske med
innehåll för en annan funktion.

\subsection{Integrationsgrad}

En integrerad krets är uppbyggd på en basplatta av halvledarmaterial - ett chip.
På plattan framställs, med fototeknik eller etsning, kompletta eller nästan
kompletta dioder, transistorer, resistorer och kondensatorer. Metoden, som
kallas planarteknik, medger att många komponenter kan få plats på samma platta.

Den snabba utvecklingen av produktionsmetoder för integrerade kretsar gör
alltmer avancerade system möjliga och dessutom på allt mindre utrymme. Med
avseende på integrationsgrad används följande begrepp.

\begin{tabular}{lp{6cm}}
SSI & Small Scale Integration innebär något 10-tal halvledare på samma chip. \\
MSI & Medium Scale Integration innebär något 100-tal halvledare på ett chip. \\
LSI & Large Scale integration innebär något 10000-tal halvledare på ett chip. \\
VLSI & Very Large Scale Integration innebär 100000 eller fler halvledare. \\
\end{tabular}

\subsection{Olika slags integrerade kretsar}

Det finns stora sortiment av både standardiserade och speciella IC, varav det
finns två huvudtyper:
\begin{itemize}
\item digitala integrerade kretsar,
\item analoga integrerade kretsar.
\end{itemize}

\subsection{Digitala IC}

Digitala IC arbetar som framgår av namnet med digitala signalnivåer. De enklaste
typerna innehåller en eller flera digitala grindar (se avsnitt 2.8). Genom att
koppla samman grindar kan man skapa kretsar för ett visst ändamål. I början av
70-talet byggdes komplicerade system av grindar i SSI- och MSI-teknik. Ett
sådant system är emellertid inte flexibelt eftersom eventuella ändringar måste
göras ``hårdvarumässigt''. Det innebär att kopplingsledningar måste ändras om,
kanske hela kretsar bytas ut o.s.v..

I dagens digitala system används IC i form av en mikroprocessor eller t.o.m.
flera. En mikroprocessor är en avancerad LSI-krets, som kan programmeras
(kopplas upp) ``mjukvarumässigt'' inte bara för ett ändamål utan för många
olika. I system med mikroprocessorer behövs också minnesfunktioner. Sådana kan
också samlas i LSI-kretsar. Mikroprocessorn är hjärtat i en dator. Styrd av ett
program (mjukvaran) styr den kringutrustningar med uppgift att inhämta och avge
information - att kommunicera.

\subsection{Analoga IC}

Analoga IC arbetar med analoga signalnivåer, d.v.s. spänningar och strömmar med
många olika nivåer och frekvenser. En analog IC kan därför även arbeta med
digitala signaler.

Analoga IC innehåller en balanserad förstärkare eller flera samt olika slags
hjälpkretsar. Med yttre komponenter kan en analog IC ges olika förstärkning och
frekvensgång. Gemensamt namn för dessa förstärkare är operationsförstärkare
(OP-amp). OP-förstärkare utförs vanligen i SSI- eller möjligen MSI-teknik.

\subsection{Kombinerade och speciella IC}

Utöver standardiserade IC finns kombinerade och speciella IC.

Exmpel på speciella digitala IC är sådana för telekommunikationsändamåL

Ett annat exempel på digitala IC är sådana för signalbehandling, såväl på HF som
LF-nivå.

Exempel på speciella analoga IC är sådana för radiokommunikationsändamål.

Bortsett från vissa skrymmande komponenter och manöverdonen kan numera t.ex. en
IC innehålla en komplett radiomottagare. Ett annat exempel på speciella analoga
IC är sådana för hörapparater. Genom programmering anpassas de för det
personliga behovet.

\subsection{Utvecklingen}

Det kan sägas hur ofta som helst. Genom den fantastiska utvecklingen av
mikroelektronik öppnas även för radioamatören möjligheter, som bara för ett par
decennier inte var tänkbart.

Denna utveckling harvidgat utrymmet för den experimentella verksamhetsom
amatörradio i grunden innebär. Hobbyn får sålunda med tiden en allt större
teknisk spännvidd.

\subsection{Aktuell litteratur}

Ökat teknikomfång inom amatörradio ställer motsvarande krav på litteratur. På
senare tid inbegripes även digitalteknik. Mest av utrymmesskäl behandlas i denna
faktabok digitaltekniken mycket kortfattat, men ändå så mycket som nämns i
CEPT-rekommendationen T/R 61-02. För djupare studium hänvisas till andra
läromedel samt till leverantörskataloger.
