\chapter{KRETSAR}

\section{Komponenter i serie}

parallellt

Seriekopplade resistorer
Bild II 3-1
Den totala resistansen av seriekopplade
resistorer är summan av resistanserna

1

1

1

1

R=R-1 +R-2 +R-3 ... Rn
För två resistorer gäller

1
1
1
eller
R R1 R2
För tre resistorer gäller

-=-+-

Strömmen är lika stor genom alla seriekopplade resistorer i strömvägen (ingen avgrening)

l= ~ + /2 + /3 + .....
Det totala spänningen över seriekopplade
resistorer är summan av spänningen över
var och en av dem

U=U1 +U2 +U3 + .....
Spänningen över var och en av seriekopplade
resistorerförhåller sig som deras resistanse r.
För två resistorer gäller

R1

u1
u2

1

1

1

R=------R1~·-R~2·R~3~---R1·R2 + R1·R3 + R2 ·R3
Strömmen förgrenar sig mellan parallellkopplade resistare r. Den totala strömmen är
summan av grenströmmarna
(Kirchhoff's 1 :a lag)
Spänningen är lika stor över var och en av
parallellkopplade resistorer

u= u1 = u2 = u3 =

-=-

R2

.. .

un

(Kirchhoff's 2:a lag)

Parallellkopplade resistorer
Bild II 3-2
Den totala resistansen av parallellkopplade
resistorer är lägre än den lägsta enstaka
resistansen

Grenströmmarna genom parallellkopplade
resistorer fördelar sig omvänt proportionellt
till deras respektive resistanser.
För två resistorer gäller

l

12

R2
R1

,

(r

1

- = - + - + - eller
R R1 R2 R3

-l

11

+

l '
lz

l,....

R1

u1

:r2
l

Bild 1/3-1 Seriekopplade resistorer

u

Gu

~

,,

u1[ R,

,.....

l

1

z

RZ

J11

JUz

J 'z

Bild II 3-2 Parallellkopplade resistorer
113-1

KRETSAR
Spänningsdelare
Bild II 3-3
Spänningsdelare förekommer i flera former.
Bilden visar en spänningsdelare med
resistorer där spänningen U delas upp i
spänningen U1 över resistorn R1 respektive
U2 över R2 • Man kan då t. ex. använda spänningen
för något ändamål.
Ett alternativ till spänningsdelning med
resistorer med fasta värden är patentiometern Det är en variabel spänningsdelare i
form av en resister med ett uttag som kan
flyttas mellan ändanslutningarna.
Om man nu ansluter en apparat parallellt
över R2 , t. ex. ett instrument vars inre resistans motsvaras av Ry, så kommer spänningarna över R1 och R2 att påverkas.
Om Ry är mycket större än R2 , så kan
man bortse från påverkan. För att beräkna
U2 kan man då använda följande formel för
en obelastad resistiv spänningsdelare.

.....l

u1[ R1

~l

u2

U2 =
U
eller
U2 =U·----"--R2 R1 +R2
+R2
Om Ry däremot är av samma storleksordning eller lägre än R2 , så måste man för
att beräkna U2 använda en formel för en
belastad resistiv spänningsdelare, t. ex.
R2 ·Rr
U2 =U·

u
~ lz

21

U

1.,...

~ly
Ry

R2

~ lz

+ly

Bild II 3-3 Resistiv spänningsdelare
Det
ström mellan X och Y när det
finns en potentialskillnad- spänning- däremellan. Bryggan är då i obalans.
Det flyter däremot ingen ström där när
det inte finns en potentialskillnad, d.v.s. när
bryggan är i balans. Balans (mätvärdet) får
man genom justering av den graderade
potentlometern till noll ström. Då gäller sambandet

R2 +Rr
R2 ·Rr
R1+--=-R2+Rr

Härav förstås att t. ex. en spänningsmätning ger olika resultat beroende på den inre
resistansen i voltmetern.

Wheatstone's brygga
Bild 113-4
En speciell tillämpning av spänningsdelare
är en s.k. brygga (Wheatstone's brygga),
som används för att jämföra spänningar.
Bryggan kan ses som två parallellkopplade spänningsdelare varav den ena är en
potentiometer med en skala graderad t. ex.
i n. Den andra spänningsdelaren består av
en resister med känd resistans och en resisto r med okänd resistans, d.v.s. mätobjektet
l ledningen som förbinder de respektive mittuttagen X och Y, finns en amperemeter som
nollströmsindikator.

113-2

JUz

u

Bild 113-4 Wheatstone's brygga

KRETSAR
Spänningsdelare och bryggor har tagits
med för att påvisa att apparater påverkar
varandra när de kopplas samman, vilket är
fallet även vid mätningar.
Spänningsdelning kan även utföras med
kondensatorer och induktorer förutsatt att
det är fråga om en växelströmskrets.
Parallellkopplade kondensatorer
Bild II 3-5
l stället för en enda kondensator kan man
parallellkoppla flera kondensatorer för att
uppnå önskad total kapacitans.
Den totala kapacitansen för parallellkopplade kondensatorer är summan av de
enskilda kapacitanserna.
C=C1 +C2 +C3 ... Cn

Seriekopplade kondensatorer
Bild II 3-6
Den totala kapacitansen för seriekopplade
kondensatorer är lägre än kapacitansen för
kondensatorn med det minsta värdet.

1

C1 = 5 11 F C2 = 1o 11F
C = C1 + C2 = 5 + 1O= 15 J..LF

2.

C1 = 1 nF

eller
C= C1. C2
C1 c2
C1+C2
För tre kapacitanser gäller

c
1

1

C=

1

c1c2c3

C1C2 + C1C3 + C2C3

C1 = 5 J.tF

C2 = 5 pF

C2 = 1o11 F

1
1
1
-=-+c1 c2

C= 5·10 =3! F
5+10
3 Jl

c

+
+

+
+

+

+
+

+

+

+

+
l+

1

-=-+-+-eller
C C1 C2 C3

+

+

1

Räkneexempel:

C= C1 + C2 = 1+ 0.005 = 1.005 nF

+

1

! = 1 +-1-

Räkneexempel:

1.

1

C = C + C + C + . . . . . .. . . ;
2
3
1
För två kapacitanser gäller

+ +l

(1

l+ +

+l

(z

Bild II 3-5 Parallellkopplade kondensatare

+

-· (1

+

Bild II 3-6 Seriekopplade kondensatorer

113-3

KRETSAR
Sammankopplade induktorer
Galvaniskt kopplade induktorer

Induktansvärdetför galvanisktsammankopplade induktorer kan i princip beräknas på
samma sätt som för motsvarande sammankoppling av resistorer.

Galvaniskt seriekopplade induktorer
Förutsatt att magnetfälten från de respektive
induktorerna inte återverkar på varandra d.v.s. inte "kopplar magnetiskt till varandra"
-så gäller:

L= L1 + L2 + L3 + . . . Ln
Räkneexempel:
L1 = 20 mH L2 =50 mH L= ?
L = L 1 + L2 = 20 + 50 = 70 mH

Galvaniskt parallellkopplade induktorer
Förutsatt att magnetfälten från de respektive
induktorerna inte återverkar på varandrad.v.s. inte "kopplar magnetiskt till varandra"
-så gäller:
1
1
1
1
- = - + - + ... -

L

L1

L2

Ln

För två induktorer gäller:
1
:! = ! + eller
L = L; · L2
L L1 L2
L1 +L2
Räkneexempel:
L1 = 50 mH L2 = 60 mH L = ?

Medverkande magnetfält

L1

-

L2

~:~-'Olflf;Motverkande magnetfält

L1

L2

~~M~

-

Bild 113-7 Magnetiskt kopplade induktorer
Bild 113-7
Bilden visar seriekopplade induktorer, vars
magnetfält kopplar till varandra på olika sätt.
"Pricken" vid änden av induktorerna på
bilden markeraratt magnetfälten där har inbördes polarisering.

Magnetiskt kopplade induktorer i serie
Formel:
L=L 1 +L2 $\pm$2M
Räkneexempel:
Två induktorer har en impedans av 20 resp.
1O J..LH och en ömsesidig induktans av 2 JlH.
Induktorerna är kopplade och placerade så
att deras magnetfält verkar med varandra.
Vardera induktansen ökas därför med
M=2J..LH

L=L1 +M+L 2 +M

L= L1 • L2 = 50 . 60 = 3000 ~ 27 m H
L1 +L2
50+60
110

L = 20 + 2 + 1O + 2 J..LH = 34 J..LH

Magnetiskt kopplade induktorer

Räkneexempel:
Två induktorer har en impedans av 20 resp
1O !lH och en ömsesidig induktans av 2!-LH.
Induktorerna är kopplade och placerade så
att deras magnetfält verkar mot varandra.
Vardera induktansen minskas därför med
M=2J..LH

l praktiken anordnas ofta induktorer så, att
deras respektive magnetfält kan återverka
på varandra- s.k. magnetisk koppling.
En ömsesidig induktans M uppstår i
induktorerna på grund av denna koppling.
Den ömsesidiga induktansen ökar eller minskar det resulterande induktansvärdet beroende på om induktorernas magnetfältverkar
med eller mot varandra.
Beräkningen av värdet på "M" är emellertid relativt komplicerad och behandlas ej här.
l stället görs en förenklad framställning.
113-4

L= L1 -M+L 2 -M

L = 20 - 2 + 1O - 2

= 26 JlH

Magnetiskt kopplade induktorer i parallell
Formel:
L= L; ·L2 ·M 2
L1 +L 2 $\pm$2M

~©rNJ

EPT

KRETSAR

Upp- och urladdning av en kondensator

Uppladdning

Bild 113-8
En kondensator C seriekopplas med en resistans R och kopplas in över spänningen U.
Spänningen över kondensatorn stigerfrån
volt till umax
Laddningsströmmen sjunker från lmax till
noll ampere.

u

Uc

o

Spänningen över kondensatorn ökar
exponentiellt uppladdningen.

(1- e-~ J

uC = Umax ·

e

spänningen över kondensatorn efter
en given inkopplingstid
slutspänningen efter minst t= 5-r
inkopplingstiden
2. 718 (e = basen för den naturliga
logaritmen)

l förloppet ingår storleken av resistans
och kapacitans enligt följande samband, som
kallas tidskonstant
r= R· C
C [F] R [Q] s [sek] -r [tidskonstant i sek]
Efter tiden t= 1r från inkopplingsögonblicket har spänningen över kondensatorn
ökat från noll till 63$\circ$/o av maxvärdet
Efter tiden t= 5-r är kondensatorn uppladdad till 99 o/o.

Strömmen från kondensatorn minskar

exponentiellt under uppladdningen.

ic
lmax

strömmen från kondensatorn efter en
given inkopplingstid
begynnelseströmmen

Efter tiden t= 1r från inkopplingsögonblicket har strömmen till kondensatorn
minskat till 37$\circ$/o av maxvärdet
Efter tiden t= 5 r återstår 1 $\circ$/o av strömmens maxvärde.

Uc
100%---

le

U max

-~------:-:::::;..:;;:;;;-~---

o

2T

l max
100% - - - - - - - - - - - - - - - -

o
Bild If 3-8 Uppladdning av en kandensa to
Urladdning

Bild 113-9
En kondensator C urladdas över resistor R.

Spänningen över kondensatorn minskar
exponentiellt under urladdningen.
t

uC =Umax ·e--:r
Strömmen från kondensatorn minskar

exponentiellt under urladdningen. Strömriktningen är motsatt den vid uppladdningen.
t

iC =-lmax · e--:r
Efter tiden t= 1r är kondensatorn urladdad Så, att 37 $\circ$/o av lmax respektive umax
återstår.
Efter tiden t= 5 r är kondensatorn urladdad så, att mindre än i o/o av lmax respektive
umax återstår.

113-5

KRETSAR
Exempel på beräkning av tidskonstanten:
R = 1 kQ
1. C = 1O J.lF

r= R. c = 1·1 0 3 ·1 o·1 o-e = 1o·1 o-3

d.v.s. 1/100 sekund

2. C

= 1000 J.lF

1: =R·

In- och urkoppling av en induktor
Inkoppling
Bild II 3-1 O
En induktor L i serie med en resistans R
kopplas in över en likspänning U.
Spänningen över induktorn ökar från O till

u max·

R = 1 kQ

C= 10 3 ·1 0 3 ·1 o-e = 1 sekund

(Egentligen, induktorns motspänning
minskar så att. .. )
Strömmen genom induktorn ökar från O
till umax·

Strömmen genom induktorn ökar exponentiellt efter inkopplingen

iL

=lmax ·

(1- e-~ J

iL strömmen efter en given inkopplingstid
lmax slutströmmen efter minst t= 5-r

t

e

inkopplingstiden
2.718 (e = basen för den naturliga
logaritmen)

l förloppet ingår storleken av resistans
och induktans enligt följande samband, som
kallas tidskonstant

u

L
R

1:=-

L [H]

Uc

U max

le
Bild II 3-9 Urladdning av en kondensator

113-6

R [O]

s [sek]

1:

[tidskonstant]

Efter en tid av t= 11: från inkopplingsögonblicket har strömmen genom induktorn
ökat från noll till 63$\circ$/o av lmax och motspänningen över induktorn minskat till 37o/o
av maxvärdet

Urkoppling
Spänningskällan kopplas bort från samma
induktor som ovan. En resister är inkopplad
över induktorn. Energin i induktorn avleds
genom resistorn som en ström med motsatt
riktning än vid inkopplingen. Strömmen är
vid urkopplingstillfället lmax = iL och minskar
därefter exponentiellt.

ETSAR
iL
lmax

e

t

strömmen genom induktorn efter en
given urkopplingstid
strömmen i urkopplingsögonblicket

2.718

tiden efter urkopplingsögonblicket

Efter en tid av t= 1r från urkopplingsögonblicket har strömmen genom induktorn
minskat till 37$\circ$/o av maxvärdet
Teoretiskt kan spänningarna och strömmarna aldrig nå ett noll- eller maxvärd e, men
för praktiskt bruk anses detta inträffa efter en
tid av minst 61:.
All den energi som lagras i en induktor
finns i dess magnetfält. När strömmen bryts
eller minskas så återgår energin omedelbart
till kretsen. i en induktor kan det således inte
finnas någon kvarstående energi, vilket det
däremot kan göra i en kondensator.

Under den tid som magnetfältet i en induktor avvecklas eller byggs upp, så induceras en motspänning i den. Denna spänning
är högre än den som finns över induktorn
innan strömmen bryts eller ändras och är
proportionell till den hastighet som ändringen har. När en en strömkrets med induktor
bryts är det vanligt att det i brytögonblicket
bildas en gnista eller ljusbåge över brytarens
kontakter.
Om induktansen är stor och kretsströmmen hög, så skall en stor mängd energi
frigöras på mycket kort tid. Det är därför inte
ovanligt att brytarkontakter bränns eller smälter. l likströmskretsar kan gnistan eller ljusbågen minskas eller undertryckas genom att
en kondensator i serie med en resistor kopplas över kontaktstället Kondensatorn fångar
upp en del av energin i induktorn och resistorn minskar hastighetsändringen.

l~
L

u
'------{ /

1-----4-----l

lL

l Max

37:'

47:'

10~0--------------~~.

'(

Bild II 3-1 O Inkoppling av en induktor

113-7

KRETSAR

PT

Växelströmskretsar
Komponentegenskaper vid växelström
Inom radiotekniken används mycket ofta
svängningskretsar bestående av kondensatorer och induktorer, som är kopplade i serie
eller parallellt med varandra. När svängningskretsens egenfrekvens sätts lika med
frekvensen på den signal som tillförs kretsen, så får kretsen särskilda egenskaper
som används på olika sätt.
För att förstå hur "LC-kretsar" fungerar,
beskrivs först hur de ingående komponenternas resistans, induktans och kapacitans
förhåller sig till varandra, när de kombineras
och kopplas til en växelströmkälla.
Bild II 3-11
Bilden visar amplituden av spänning och
ström vid ett sinusformat förlopp samt den
effekt som då utvecklas. Tidsaxeln är graderad O - 360$\circ$ per period.

Fall a: Förloppen med en resister R.

Med en resister följer ström- och spänningskurvorna varandra tidsmässigt, även
vid riktningsändring. När kurvorna följs åt på
det sättet, sägs de vara i fas med varandra.
Effekt överförs från strömkällan till resistorn. Den effekt som utvecklas i resistorn är,
vid varje tidpunkt av perioden, produkten av
strömmmen och spänningen just då. Eftersom storheterna av spänning och ström är
antingen positiva eller negativa samtidigt, så
blir produkten alltid positiv. Det betyder att
den effekt som utvecklas pulserar två gånger per period mellan ett noll- och maxvärde.
Fall b: Förloppen med en induktor L.

Med en induktor är utvecklingen av ström
och spänning inte samtidig. Vid inkopplingen stiger spänningen genast till maxvärdet
medan strömmen stiger långsammare och
bygger under tiden upp ett magnetfält i induktorn och omkring övriga ledare i kretsen.

u~
a

u~

p

b

u~
c
Bild II 3-11 Faslägen och effekter i L C-kretsar

113-8

p

ETSAR
Strömmen fördröjs alltså i förhållande till
spänningen. Eftersom kurvornas max- och
nollvärden inträffar vid olika tidpunkter, så
heter det att de är ur fas eller fas förskjutna.
En växelström genom en ideal induktor
ärtasförskjuten 90$\circ$ efterspänn ingen. Strömmen når toppvärdet vid tidpunkten 90$\circ$ av
perioden, när spänningen nått ner till noll.
När spänningen minskar, så sjunker strömmen och tar med sig energin i magnetfältet.
Först vid 180$\circ$, när spänningen har nått maxvärdet åt andra hållet, ändrar också strömmen riktning och bygger upp ett nytt magnetfält med motsatt polaritet.
Effekt överförs från strömkällan till induktorn när ström och spänning har samma riktning. När ström och spänning har olika riktning, försöker induktorn i stället "ladda" strömkällan med energi från sitt kraftfält. Det pendlar därför effekt mellan strömkällan och induktorn, varvid effekten i ena riktningen är
lika stor som i andra riktningen.
Sett över en hel period upphäver därför
dessa effekter varandra. Följden blir att en
ideal induktor, i motsats till en resistor, inte
förbrukar någon aktiv effekt. Man säger att
en reaktans, här en induktor, arbetar med
reaktiv effekt.
l praktiken har kretsen även en viss resistans. Därför sätts reaktansens 90$\circ$ fasförskjutna ström samman med resistansens oo
fasförskjutna ström. Resultatet blir en ström,
som är mindre än 90$\circ$ ur fas, och det förbrukas då en viss aktiv effekt i resistansen.

Sedan strömmen passerat noll vid 180$\circ$ eller
0$\circ$/360$\circ$, bygger den upp ett nytt magnetfält
med motsatt polaritet.
Liksom med en induktor överförs effekt
från strömkällan till kondensatorn när ström
och spänning har samma riktning. När ström
och spänning har olika riktning, försöker
kondensatorn i stället "ladda" strömkällan
med energi. Det pendlar därför effekt mellan
strömkällan och kondensatorn, varvid effekten i ena riktningen är lika stor som i andra
riktningen.
Sett över en hel period upphäver därför
dessa effekter varandra. Följden blir att en
ideal kondensator, i motsats till en resistor,
inte förbrukar någon aktiv effekt. Man säger
då, att en reaktans, här en kondensator,
arbetar med reaktiv effekt.
l praktiken har kretsen även en viss resistans. Därför sätts reaktansens 90$\circ$ fasförskjutna spänning samman med resistansens o fasförskjutna ström. Resultatet blir en
spänning, som är mindre än 90$\circ$ ur fas, och
det förbrukas då en viss aktiv effekt i resistansen. Som framgår av bilden blir variationerna i tiden de omvända med kondensator
jämfört med induktor.

o

Fall c: Förloppen med en kondensator C.
Inte heller med en kondensator utvecklas
ström och spänning samtidig. Efter inkopplingen laddar strömmen upp kondensatorn,
d.v.s. bygger upp ett elektriskt fält med en
viss potential (spänning). Spänningen utvecklas långsammare än strömmen - den
blir fasförskjuten
Strömmen till (och från) en ideal kondensator är fasförskjuten 90$\circ$ före spänningen.
När kondensatorn är kopplad till en växelströmskälla, når strömmen toppvärdet vid
tidpunkten 90$\circ$ eller 270$\circ$ av perioden. Spänningen passerar då i båda fallen värdet noll.
När spänningen minskar, så sjunker strömmen och tar energi ur det elektriska fältet.

113-9

KRETSAR
Impedans

Bild II 3-12
Bilden visar en induktor, en kondensator
och en resistor som är kopplade i serie. När
man vill beräkna den resulterande impedansen i kretsen ("totala växelströmsmotståndet"), måste man ta hänsyn till att komponenternas spänningar eller strömmar inte är
i fas med varandra. De arbetar ju inte "i takt".
Att då addera max. värdena ger fel resultat. l stället söker man den s.k. resultanten
av de olika vektorer som motsvarar strömoch spänningsvärden.
Detta kan göras grafiskt eller beräknas.

Liten ordlista:
Impedans- hindra
(lat. impedire).
Resistans- motstå
(lat. resistere).
Del av impedansen,
kallas ibland ohmskt motstånd.
Reaktans- återverka
(lat. reagere).
Del av impedansen,
samlingsord för växelströmsmotstånd.
- Kapacitans- inrymma (lat. capax).
Del av reaktansen.
- Induktans- införa
(lat inducere).
Del av reaktansen.

Bild 113-13
Vi tänker oss att vektorerna i systemet
vrider sig moturs med hastigheten w= 2rc f,
där f är frekvensen och w ärvinkelhastighet.
Eftersom vektorerna har samma frekvens,
så är vektorernas lägen inbördes samma.
Ögonblicksvärdet av respektive vektorer följer en sinuskurva.
Spänningsvektorn i den "induktiva reaktansen" ligger 90$\circ$ före strömmen och spänningen i resistansen. Spänningsvektorn i
den "kapacitiva reaktansen" ligger 90$\circ$ efter
strömmen och spänningen i resistansen.
Vektorerna i dessa två reaktanser är således 2 · 90 = 180$\circ$ åtskilda, d.v.s. motriktade.
Det kallas att de är i mottas.

Hittills har storheterna resistans, induktans
och kapacitans behandlats var för sig, men
i praktiken förekommer de alltid tillsammans
och kallas impedans.
Resistansen är i princip oförändrad vid
ström- eller spänningsändringar. Men när
strömmen genom en ledare eller induktor
liksom spänningen över en kondensator ändras, så tillkommer en reaktans som motverkar förändringarna.
Reaktansen kan från fall till fall vara kapacitiv eller induktiv och ingår i impedansen.
Om ingen reaktans finns, så är impedansen
lika med resistansen.

Bild II 3-14
l bilden visas vektorerna för komponenterna i Bild II 3-12 samt hur man grafiskt
bestämmer inpedansen av dessa vektorer.
Vidare får man fasvinkeln mellan impedansens och resistansens vektor, varav den senare är den s.k. riktfasen för hela seriekretsen.

Bild II 3-12 Seriekrets av L+C+R

'

spänningsfall över R
= strammens fas 1 XL+ Xc +R

,spänningsfall over Xc.
spännir1gsfoll över XL
1

Bild II 3-13 Spänningar i seriekrets L+C+R

113-10

KRETSAR

Bild II 3-14 Impedansen och fasvinkeln i seriekrets L+C+R
Resistansen ritas som en vektor R, som
riktas vågrätt mot höger. Vektorns längd
motsvarar resistansens storhet i ohm.
Den induktivareaktansen ritas på liknande sätt med vektorn XL lodrätt uppåt. slutligen ritas den kapacitiva reaktansen Xc lodrätt neråt.
Man subtraherar de motverkande reaktiva vektorerna XL och Xc från varandra och
avsätter resultatet X på den vertikala axeln,
uppåt om XL är större och neråt om Xc är
större. Den resistiva vektorn R avsätts åt
höger på den horisontella axeln.
Man låter nu vektorerna X och R bilda
sidor i en rätvinklig rektangel. Längden på
rektangelns diagonal är den resulterande
impedansen Z. Fasvinkeln mellan impedans
och resistans kan också avläsas.
Eftersom vektordiagrammet bildar en rätvinklig triangel kan den resulterande spänningen U i kretsen även beräknas med
Pytagoras sats:

Tillämpad på ovanstående vektordiagram
kan satsen skrivas som

ULcR2

Termerna ersätts med följande ekvationer:

UR

= f. R

UL =J. XL =J. mL

1

Uc=f·Xc=l·-

mC

l' Z

2

=l

2

2

R + (hoL -l miG)'

eller

Z= ~R

'+(mL--dc;J eller

Z=~R

2
+(XL -Xct

l en seriekrets är den resulterande
reaktansen negativ (kapacitiv) om Xc är
större än XL och positiv (induktiv) om XL är
större än Xc.

Ohms lag vid växelström

l formler betecknas impedansen med bokstaven Z och reaktansen med bokstaven X.
l båda fallen är sorten Ohm [Q].
Vid beräkning av impedans är Ohms lag
inte direkt tillämplig, eftersom reaktansen i
en induktor eller kondensator uppträder annorlunda i tiden vid ström- respektive spänningsändring än vad resistansen gör.
Om impedansen Z sätts in i Ohms lag, så
fås följande samband som ofta kallas Ohms
lag för växelström, således

vett =lett.

= UR2 + (UL- Uc)2

ULRc =J. Z

z =R'+( mL- ~c)'

z

eller

Uett = lett. -.J R2 + )(2

eller

Uett=lett·~R2 +(XL -Xc) 2

o.s.v.

Av vad som framgått tidigare i detta avsnitt
kan även slutsatsen dras att:

2

Efter division med 12 fås

skenbar effekt=
= ~ (aktiv effekt) 2 +(reaktiv effekt) 2
113- 11

KR
LC- kretsar
Parallellkopplade LC-kretsar
Bild II 3-15
En parallellkopplad LC-krets är ansluten till
växelspänningen U från en signalgenerator
med inställbar frekvens f. Två fall studeras.
Fall i :

f=

fres

signalgeneratorns frekvens f ställs lika
med LC-kretsens resonansfrekvens fres· Då
visar kretsen hög impedans Z mot generatorn. En stark ström cirkulerar i svängningskretsen, men endast en svag ström flyter i
ledningen mellan generator och krets.
Jämför med modellförsöket på bild 3-000.
Fall2:

f> ~es

eller

f< ~es

Frekvensen f ställs högre eller lägre än
kretsens resonansfrekvens fres·
Svängningskretsen visar då en låg impedans Z mot generatorn. En svag ström cirkulerar i svängningskretsen, medan en starkare ström flyter i ledningen mellan generator och krets.
l praktiken finns även en resistans (belastning) parallellt över kretsen och en resistans i serie med induktansen. För enkelhetens skull bortses här från dessa resistanser.
l en parallellkopplad LC-krets är spänningen över induktans och kapacitans densamma. Spänningsvektorn U används därför som s.k. riktfas.

ström i XL

Riktfasen riktas på bilden åt höger. Strömmen le genom kondensatorn är fasförskjuten 90$\circ$ efter U och ritas rakt neråt (vektorerna roterar motsols). Strömmen IL genom
induktorn är fasförskjuten 90$\circ$ före U och
ritas rakt uppåt. Den resulterande reaktiva
strömmen genom kretsen är skillnaden mellan strömmarna le och IL, vilka är motriktade
varandra.
Formeln förparallellkopplade resistanser
kan även användas för parallellkopplade
re aktanser om man tillämpar Pytagoras sats
[A2 + 82 =
således

c2],

i
R

i
R1

1

-=-+-+ ....
R2

Gr ~(~r +(*J
~~ (~J +GJ ~~~, +;
eller

Med R försumbart kan den resulterande
reaktansen av kapacitansen Xe och den
vektormässigt motriktade induktansen XL beräknas på följande sätt
1
1
1
1 XL- X c
X= X c -XL d.v.s. X= -XL. X c eller

X= -XL ·Xc
XL -Xc
l en parallellkopplad LC-krets är den resulterande reaktansen negativ (kapacitiv) om XL
är större än Xe och positiv (induktiv) om XL är
mindre än Xe.

ström i XL

ström i Xc

l "'~~l

u

spänning

1--------IIB»

ström i Xc

Bild II 3-15 Parallellkopplad LC-krets

113- 12

u

Seriekopplade LO-kretsar

Bild II 3-16
En seriekopplad LC-krets ansluts till växelspänningen U från en signalgenerator med
inställbar frekvens f. Två fall studeras.
Fall 1:

f = f,es

signalgeneratorns frekvens f ställs lika
med svängningskretsens resonansfrekvens
fres· Impedansen Z i en seriekrets visar då ett
mycket lågt värde mot generatorn. Det flyter
en stark ström i ledningen mellan generator
och krets.
Fall2:

f< f,es eller f> f,es

Frekvensen f ställs lägre eller högre än
kretsens resonansfrekvens fres·
Eftersom svängningskretsen då visar hög
impedans Z mot generatorn, så flyter endast
en svag ström i ledningen mellan generator
och krets.
l praktiken finns även en resistans i serie
med induktansen liksom en parallellt över
kapacitansen. För enkelhetens skull bortses
här från dessa resistanse r.
Strömmen l är samma genom hela kretsen och strömvektorn l används därför som
s.k. riktfas. Den ritas i bilden åt höger. Om
serieresistansen R varit med, så skulle ett
spänningsfall UR varit inritad i samma riktning som l (i fas med 1). Spänningen över
re aktansen Xc ligger go o efter l och ritad rakt
neråt (vektorerna roterar motsols). Spänningen över reaktansen XL (induktorn) ligger
90$\circ$ före l och ritad rakt uppåt.

Thomson's svängningskrets
Bild II 3-17
Bilden visar en svängningskrets, som består
av en kondensator och en induktor med
förskjutbar järn kärna. En ändring av kärnans
tvärsnitt ändrar den magnetiska ledningsförmågan och därmed induktansen.
Med anordningen kan resonansfrekvensen alltså ställas in så att den blir högre, lika
med eller lägre än den anslutnaspänningens
frekvens. Tre fall undersöks:
XL > X c LA 1 och LA2 lyser upp, en kraftig
ström flyter genom kondensatorn,
XL < X c LA 1 och LA3 lyser upp, en kraftig
ström flyter genom induktorn,
XL= Xc LA2 och LA3 lyser upp, LA1 lyser
inte, en kraftig ström flyter i kretsen
men inte i tilledningarna
XL = X c kallas Thomson's svängningsformel, vilken beskriver resonansfallet
Då är de induktiva och kapacitiva
reaktanserna i kretsen lika stora och tar ut
varandra. Kvar är kretsens resistans, vilken
vi tills vidare betraktar som försumbar.
Således XL =Xc , där

XL

= 2 nfL
1

och

1 - sats
"t .tn.
Xc = 2nfC

2nfL=-2nfC

4n 2 f ·L· C= i

f-=1

f=

4n 2 LC

f [Hertz]

L [Henry]

1
C [Farad]

Formeln gäller både för parallell- och seriekretsar.

Bild II 3-16 Seriekopplad LC-krets

113-13

KRETSAR
Räkneexempel:
Strömriktning: 1 halvvågen -.....
2 halvvågen - - -

L= 100 nH C= 1O pF

f=

f=?

1
2n~1 00 ·1 o-9 ·1 o·1 o-12 = 2n1 o-9 =

109

= - ~ 159 MHz

2n

Impedansen i en resonant krets
En enkel framställning görs av hur impedans, re aktans och resistans förhåller sig
inbördes när en svängningskrets är i resonans. Som exempel används följande
kretsdata: Induktans 200 J.lH, kapacitans
200 p F, förlustresistans 1O Q.
Resonansfallet i en parallellkrets
Parallellkretsen består i sig själv av seriekopplade komponenter, varav XL och Xc
är reaktiva. Vid resonans är dessa lika
stora och motverkande. Inom kretsen är
således den resulterande reaktansen

Därför uppvisar samma krets en yttre
reaktans av

X= -XL ·Xc
XL -Xc

X= XL ·Xc
O

=oo

l praktiken finns i kretsen också en
resistans varför dessa extremvärden inte
uppstår. Inne i en parallellkrets i resonans cirkulerar alltså en stark ström, som
endast begränsas av kretsens resistans.
Bild II 3-18
Bilden visar en parallellkrets där
induktorn har resistansen rL och kondensatorn antas vara förlustfri. Vidare
förutsätts att kretsen är i resonans.
Vid resonans kan termen XL -Xc = O
bytas mot rL i formeln

X= -XL ·Xc
XL -Xc
Bild II 3-17 Thomson's svängningskrets

113-14

förutsatt att rL är försumbart jämfört med
XL.

ETSAR
Därtill är XL= 2nfL och Xc = ~fC
...
L
d .v.s. X L· X c= C som satts m.

2

Parallellkretsens impedans vid resonans kan
då skrivas
Z= XL ·Xc =-LIj
tj·C

Med ovanstående kretsdata blir Z= 100 kQ
Därav framgår, att impedansen i parallellkretsen är en funktion av det s.k. UC-förhållandet samt av kretsens resistiva förluster.

.
Z vtd

resonans:

XL~ Xc
l
--=-c
rL
rL·

l formeln

Z=~r/+(XL -Xc)
Z=~r/ +0 2

blirdå

=!j

Med ovanstående kretsdata blir resonansfrekvensen

1

~ 796kHz
2n LC
Vid resonansfrekvensen blir reaktansen
1000 Q både i induktansen och kapacitansen. Eftersom reaktansernas spänningsfall
är motriktade tar de ut varandra. Kretsens
impedans i resonans blir resistansen rL och
spänningsfallet över kretsen bestäms enbart av rL.

fo =

{[C

Antag att det alstras en spänning av 5 mV
i antennkretsen. Strömmen genom den vid
5
resonans blir då
m V= O. 5 mA.
10 Q
Av strömmen bildas reaktiva spänningar,
d.v.s. 0.5 mA·1 000 Q == 500 mV både över
induktans och kapacitans (som tar ut varandra) och 5 mV över resistansen.

Z vid
resonansfrekvens

2

resonans:

rL

ohm

Bild II 3-18 Resonansfallet i parallellkrets
Resonansfallet i en seriekrets
Bild II 3-19
När en seriekrets är i resonans, så är

XL =Xc

Serieresonans

i
d.v.s. mL = -

mC

eller

resonansfrekvens

frekvens

1

d.v.s. mL--=0

mC

Bild II 3-19 Resonansfallet i seriekrets

113-15

KR
Q-faktorn i en parallellkrets

Bild II 3-20
Godhetstalet Q (=Quality Facto r) kan ses
som den förmåga en svängningskrets har att
lagra energi, d.v.s. förhållandet mellan den
lagrade energin och energiförlusten i kretsen. Energiförlusten yttrar sig som värmeutveckling.

Q= n
2

lagrad energi i kretsen
energiförlusten per period
Energiförluster uppstår både i kretsens
kondensator och induktor, men moderna
kondensatorer har så låga förluster att
induktorn ensam kan anses bestämma Qvärdet, åtminstone i kortvågsområdet

Bandbredd

Bild II 3-21
Bilden visar med en kurva vilket impedansvärde kretsen har vid olika frekvenser.
Impedansens högsta värde är vid frekvensen fres och avtar vid frekvenser som är högre
eller lägre. Vid frekvenserna f1 och f2 är
impedansvärdet t. ex. 70$\circ$/o av maximalvärdet
Med bandbredden b förstås skillnaden mellan impedansvärdena i ett sådant frekvenspar, d.v.s. b= ~ - ~

z

Q= 2nfL =XL

R

R

En växelspänning U 1 ansluts till en
parallellkrets. l resonansfallet uppträder då
en spänning U2 över kondensatorn och
induktorn.
U2 är mycket större än U1 • Ju högre Q är
i kretsen desto större är förhållandet mellan
U2 och U1 •
l kortvågsområdet är det vanligt med ett
Q i storleksordningen 30 - 100.
Ju högre Q är, desto mindre är bandbredden.
När svängningskretsen är i resonans gäller sambandet

Q= ~es

b

Bandbredden ökar (avstämningsskärpan
minskar) vid ökande frekvens på grund av de
större kretsförlusterna.

u

f res
Bild II 3-20 Q-värden i parallellkrets
113- 16

f.

f
Bild 113-21 Bandbredd i parallellkrets

KRETSAR
3.2 Frekvensfilter

Frekvensfilter används inom radiotekniken
för många olika ändamål, t.ex. för att
• eliminera störande signaler,
• öka avstämningsskärpan (selektiviteten) i
mottagare och sändare,
• framhäva eller dämpa ett sidband i en AMsignal m.m.
Beroende på den s.k. frekvensgången,
så indelas filtren i flera "familjer", varav de
vanliga presenteras här.
Beroende på det tekniska utförandet finns
dels s.k. passiva filter vilka använder extern
energi för sin funktion, och dels aktiva filter
vilka i princip är förstärkare som likaledes
använder passiva kretsar. Här presenteras
för enkelhetens skull passiva filter.
Traditionella frekvensfilter är vad som
kallas analoga. Men nu i dataåldern börjar
även digitala filter vinna intåg. Sådana är
dock för komplicerade för att behandlas här.

Högpassfilter
Bild II 3-22

Ett högpassfilter släpper igenom signaler
med höga frekvenser och dämpar dem med
låga frekvenser.
Exempel: En frekvensberoende spänningsdelare som LC-högpassfilter.
.. Vid låga frekvenser är Xc stor och XL liten.
Over XL uppstår då ett litet spänningsfall- en
låg utgångsspänning ua. Resultatet blir att
låga frekvenser dämpas.
Vig höga frekvenser är Xc liten och XL
stor. Over XL uppstår då ett stort spänningsfall- en hög utgångsspänning ua. Resultatet
blir att höga frekvenser släpps igenom.
XL kan bytas ut mot en resister R, men då
blir passbandkurvan inte så brant.

Gränsfrekvens
Gränsfrekvensen f9 beror av kapacitansen
C, induktansen L samt resistansen R.

f

LC-högpass:

=

g

f9 [Hz]

C [Farad]

lågpassfilter

Bild II 3-23
Om induktor och kondensator respektive
resister och kondensator i ett högpassfilter
byter plats, så får man i stället ett LC-Iågpassfilter respektive ett RC-Iågpassfilter.
Ett lågpassfilter släpper igenom signaler
med låga frekvenser och dämpar dem med
höga frekvenser.
Exempel: En frekvensberoende spänningsdelare som LC-Iågpassfilter.
.. Vid låga frekvenser är Xc stor och XL liten.
Over XL uppstår då ett litet spänningsfall- en
hög utgångsspänning ua. Resultatet blir att
låga frekvenser släpps igenom.
Vig höga frekvenser är Xc liten och XL
stor. Over XL uppstår då ett stort spänningsfall -en låg utgångsspänning ua. Resultatet
blir att höga frekvenser dämpas.

Gränsfrekvens
Samma formler används vid beräkning av
gränsfrekvensen både i lågpass- och högpassfilter, således
LC-Iågpass: ~

f9 [Hz]

L [H]

RC-Iågpass: ~

f9 [Hz]

1

= 2 rc{LC
C [F]

1

= 2 rcRC

C [F]

R [Q]

C [Farad]

f =g

f9 = ?

1
10 3
~=
=-=79.62 Hz
2rc~ 4. 2rc ·1 o-B
4n
2) R= 1 kQ C= 1O n F f9 =?
1
10 5
~=
3
9 = = 15.934
2rc·1·10 ·10·102rc

1
2rc-fLC

L [Henry]
RC-högpass:

Räkneexempel:
1) L = 4 H C = 1 J.lF

1
-

2rcRC
R [Ohm]

113-17

KRETSAR

L

LC- HÖGPASS

Ua

Ekvivalentschema

Xc
R
RC -

HÖGPASS

Ekvivalentschema

Låga frekvenser dämpas

L

Höga frekvenser släpps igenom

Bild II 3-22 Högpassfilter

113-18

fg
Passbandskurva

f

KRETSAR

~©~CEPT

L

XL

Te

Ue

Uu

Ue
Xc

Ua

o

o
LC- LAGPASS

o

o

R
CJ

Ue

Ekvivalentschema

o

le

R

Ua.

I

o

Ue

Ekvivalentschema

RC- LAGPASS

L

o-------1 t ' r n r r •

o

IV Ile IV
o

Ua

Xc
o

-----o

Låga frekvenser släpps igenom

100

Ue

70

fg

o,..I..(-o

f

Passbandskurva

HöQa frekvenser dämoas

Bild II 3-23 Lågpassfilter

113- 19

KRETSAR
Bandpassfilter

Bild 3-24
Ett bandpassfilter släpper igenom signaler
bara inom ett frekvensområde medan signaler inom andra frekvensområden dämpas.
Bandpassfiltret består i enklaste fall av
två svängningskretsar av LC-typ, vilka är
avstämda till angränsande frekvenser. Kretsarna är kopplade induktivt, kapacitivt eller
galvaniskt.

Beroende på kopplingsgrad skiljer man
mellan underkritisk koppling (lös koppling),
kritisk koppling och överkritisk koppling (fast
koppling).
På bilden visas hur passbandet påverkas
bl.a. av kopplingsgraden. Lös koppling liten bandbredd. Kritisk koppling - större
bandbredd. Fast koppling- stor bandbredd.

KOPPLINGSSÄTT

Induktivt

Kapacitivt

Galvaniskt

fr
Underkritisk koppling

Bild II 3-24 Bandpassfilter

113-20

Kritisk koppling

f

fr
överkritisk koppfing

f

KRETSAR
Passfilter

Bild 113-25
Passkretsen stäms av till en viss frekvens
och erbjuder där en mycket låg impedans.
Passkretsen kopplas i serie med signalvägen och låter signaler med frekvenser inom
filtrets passband att passera.

fr
Bild II 3-25 Passfilter

Bandspärrfilter

Bild II 3-26
Om serie- och parallellkretsarna i ett bandpassfilter byter plats, så får man i stället ett
bandspärrfilter. Ett sådant spärrar signaler
inom ett visst frekvensområde, men släpper
igenom signaler utom detta område.

o

uin

I

o

I

CJ

I

I

uut
o

uut
o

Bild II 3-26 Bandspärrfilter

Spärrfilter

Bild II 3-27
Spärrkrets
Spärrkretsen stäms av till en viss frekvens
och erbjuder där en mycket hög impedans.
Spärrkretsen kopplas i serie med signalvägen och spärrar en signal med samma
frekvens som resonansfrekvensen.

Bild 113-27
sugkrets
sugkretsen stäms av till en viss frekvens och
erbjuder där en mycket låg impedans. sugkretsen kopplas parallellt med signalvägen
och kortsluter (suger bort) en signal med
samma frekvens som resonansfrekvensen.

113-21

KRETSAR

o

o

uin

o

uut

uut

·SPÄRRKRETS

I

o

I

uin
o

f

o

fr

uut
o

SUGKRETS

Bild II 3-27 Spärrfilter (2 sorter)

Kvartskristall
Bild II 3-28

Bandfilter med kvartskristaller

En kvartskristall, egentligen en slipad skiva

av kvarts, kan fungera som en elektromekanisksvängningskropp (resonator), vars egenskaper liknar dem i en LC-krets.

Den låga inre resistansen gör att Q-värdet i en kvartskristall är bättre än 10000.
Som jämförelse är Q-värdet i en LC-krets
oftast sämre än 1000.

Bild II 3-29
Kvartskristaller kan kombineras till filter med
önskad bandbredd. Även utföranden med
keramiska resonatorer finns.
Resonatorerna är avstämda till var sin
bestämda frekvens och hela komplexet bidrar på så sätt till att bilda passband eller
andra egenskaper på samma sätt som med
sammankopplade LC-kretsar.

j
[:=J

T
Schemasymbol

Ekvivalentschema

Bild II 3-28 Kvartskristall
113-22

Bild II 3-29 Bandfilter med kvartskristalle

KRETSAR
Mekaniska filter

Bild II 3-30
Med en elektromekanisk givare kan man få
en kropp (resonator) att svänga på sin resonansfrekvens. Med ännu en elektromagnetisk givare kan man känna av svängningarna och återvandla dem till elektriska signaler. Hela anordningen fungerar som en elektromekanisk resonator, vars egenskaper liknar dem i en LC-krets.
Resonatorerna kan kombineras till filterkomplex med önskad bandbredd där resenatorerna är avstämda till var sin bestämda frekvens. Hela komplexet bidrar på så
sätt till att bilda ett passband på samma sätt
som med sammankopplade LC-kretsar.
Beroende på tillämpningen finns olika frekvenslägen i intervallet 60-600 kHz.
Mekaniska filter användes mest förr som
mellanfrekvensfilter i högvärdiga radioutrustningar, men har numera till stor del ersatts
av bandfilter med kvartskristaller där arbetsområdet kan ligga avsevärt högre i frekvens.

Bild II 3-31 Kavitetsfilter
Inkommande och utgående signaler ansluts till filtrets mittledare över induktionskondensatorer eller direkt galvaniskt.
kavitetsfilter kan kopplas ihop för att
bilda bandfilter, frekvensdelare m.m ..

Helixfilter

Kavitetsfilter
Bild II 3-31

Svängningskretsars dimensioner minskar
med ökande frekvens. Vid mycket hög frekvens kan induktorns varvtal i en LC-krets ha
minskat till ett enda varv samtidigt som kapacitansen inom detta enda varv kan räcka
för önskad resonansfrekvens.
En sådan svängningskrets kan bl.a. ha
formen av en ledare mitt inne i en elektriskt
ledande kavitet. Ledarens längd tillsammans
med kavitetens insida bildar induktorn. Mellan ledaren och kavitetens insida råder en
kapacitans, som kan kompletteras/justeras
med en extra kondensator.

Givare

När ett kompakt kavitetsfilter behövs, så kan
man öka reaktansen i mittledaren både induktivt och kapacitivt genom att utforma den
som en spiral (helix). Detta är dock på bekostnad av Q-värdet. Flera kavitetsfilter kan
kopplas ihop för att bilda bandfilter, spärrfilter m.m ..

Resonanskroppar

Avkännare

]!~CJ··C) C)EJ]~
Bild II 3-30 Mekaniskt filter

113-23

KRETSAR
Pi-filter

Bild II 3-32
För att överföra H F-signaler med bästa verkningsgrad, så är det viktigt med god impedansanpassning mellan de olika funktionerna. Om anslutningsimpedansen är lika i båda
~~nktionerna, så behövs inga extra åtgärder.
Ar impedanserna däremot olika, så behövs
korrigeringsnät (filter).
Ofta är nätet Pi-format och består av
induktanser och kapacitanser. Ett Pi-format
nät kan sägas bestå av två L-formade nät
ställda mot varandra, där den seriella delen
är gemensam (på bilden en induktor).

T-filter

Bild II 3-33
Ett nät kan också vara T -format och bestå av
induktanser och kapacitanser. Ett sådant
nät kan sägas bestå av två L-formade nät
ställda "rygg mot rygg". Då är den parallella
delen gemensam. På bilden visas två alternativ.
När den parallella delen är kapacitiv, blir
huvudkaraktären ett lågpassfilter, men att
impedansanpassning också är möjlig med
en induktiv impedansdelning.
När den parallell delen är induktiv blir
huvudkaraktären ett högpassfilter, men att
impedansanpassning också är möjlig med
en kapacitiv impedansdelning.

Ett Pi- eller T-filter kan fungera som
• svängningskrets,
• impedanstransformator (anpassning),
• balansera ut en reaktans o.s.v.

o

·O

l

el
I

w f •

Bild II 3-32 Pi-filter
113-24

L

J t 'f , '

le
I

o

o

(

(

o

o

Bild II 3-33 T-filter

3.3

Kraftförsörjning

Den elektriska energi, som behövs för elektronikutrustningar, hämtas från det allmänna elektricitetsnätet, ett batteri eller en ackumulator. Vissa batterityper kan återuppladdas och kallas då ackumulator.
Batterier och ackumulatorer avger en
nominell spänning, som beror av de ingående materialen och givetvis av laddningstillståndet. Moderna utrustningar för amatörradio är utförda för 12 V likström och försörjs
vanligen från ett nätanslutet kraftaggregat
På så sätt kan mobila radioutrustningar även
försörjas från startackumulatorn i fordonet.
Handburna radioutrustningarförsörjs från
en inbyggd ackumulator, som laddas från
stationär laddare.
Äldre stationära radioutrustningar drivs
nästan alltid med nätanslutna kraftaggregat
med en eller flera transformatorer och likriktare. Alternativt kan samma transformators
sekundärsida vara försedd med flera lindningar för olika spänningar och strömkretsar.
Det allmänna elnätet i Sverige levererar
växelspänning med frekvensen 50Hz. Nätspänningen för hushållsändamål är numera
400/230 v.
Tidigare importerade utrustningar i marknaden kan vara utförda för andra nätspännings- och skyddsjordningssystem än vad
som nu tillämpas i Sverige. Försiktighet med
sådan utrustning rekommenderas.

Halvm och helvågslikriktning m. m.
Bild II 3-34
Likriktning av spänningar och strömmar i en
krets görs med "elektroniska ventiler", som
släpper igenom ström endast i den s.k. passriktningen och stoppas i spärriktningen. En
sådan strömventil kallas för diod och kan
vara av typen vakuumrör eller halvledare. l
moderna konstruktioner används uteslutande halvledardioder i likriktarkopplingar.

+

!>l

Passriktning

[>l

+

Spärriktning

Bild II 3-35
Halvvågslikriktning
Vid halvvågslikriktning släpps endast varannan halwåg av en växelspänning igenom. l
den strömkrets, som bildas av transformatorns sekundärlindning, dioden och belastningen, flyter därför ström endast under
varannan halvperiod.

Helvågslikriktning
l följande kopplingar med två respektive fyra

dioder släpps varje halwåg av transformatorns växelspänning igenom så att alla halvvågor får samma polaritet. Ström flyter genom belastningen i samma riktning under
varje halvperiod. Följande sätt att anordna
helvågslikriktning är vanliga:
• Med två dioder och mittuttag på transformatorns sekundärlindning. Den ena dioden och ena lindningshalvan släpper igenom ström till belastningen under ena halvperioden. Den andra dioden och andra
lindningshalvan under följande halvperiod
o.s.v.
• Med fyra dioder (s.k. Graetz-brygga) och
inget mittuttag på transformatorns sekundärlindning, släpper dioderna 1 och 3 igenom ström under den ena halvperioden.
Dioderna 2 och 4 släpper igenom ström
under följande halvperiod o.s.v.

Glättningskretsar
Bild II 3-36
Efter likriktningen har växelspänningen omvandlats till en pulserande likspänning som
kan "glättas". Efter likriktarna ansluts då ett
glättningsfilter, som t. ex. kan bestå av laddningskondensatorn CL, induktansen L (s.k.
drossel) och glättningskondensatorn C 8 .
Parallellt över denna kondensator ligger för
elsäkerhetens skull en urladdningsresister
med hög resistans alltid inkopplad.
Säkerhetsresistorn skall ladda ur kondensatorerna, när kraftaggregatet inte är
anslutet till strömförsörjningen och belastningen. Säkerhetsresistorn (eng. bleede()
skall vara av trådlindad typ och kunna tåla
fyra gånger sin egen effektförbrukning.

Bild II 3-34 Halvledardioder
113-25

KRETSAR

HALVVAGSLIKRIKTNING

u~~a:u~
HELVAGSUKRIKTNING

a - med 2 dioder

b- med 4 dioder
(Graetzkoppfing)

1 :a halvvågen

Diod 1 och 3 i passriktning

u
2 :a ha tvvågen

Diod 2 Diod 2 och 4 i passriktning

Bild II 3-35 Halv- och helvågslikriktning

113-26

KRETSAR

HALVVÅGSUKRIKTARE MED GLATTNINGSFILTER

GRAETZKOPPLING MED GLATTNINGSFILTER

Bild II 3-36 Glättning av likspänning
l obelastat tillstånd är spänningen över
gånger högre
laddningskondensatorn
än effektivvärdet på transformatorns sekundärspänning. När en transformator i tomgång har ett effektiwärde av 230 V över
sekundärlindningen, så blir spänningen över
= 325 V.
säkerhetsmotståndet 230

.J2

.J2

Spänningshöjande likriktarkopplingar
Vid likriktning av växelspänningar enligt någon av ovanstående metoder behövs en
sekundärspänning från transformatorn av
minst samma storlek som den önskade likspänningen. Önskas en högre likspänning,
t.ex. den dubbla, men med samma sekundärspänning på transformatorn, så måste en
speciell likriktarkoppling användas.
Bild 113-37
Bilden visar en spänningsfördubblande
koppling. Under 1 :a halwågen laddas kondensator C 1 upp. Under 2:a halwågen laddas kondensator C2 upp. Kondensatorerna
är kopplade i serie och den ena kondensatorn
hinner inte bli urladdad under tiden som den
andra kondensatorn blir uppladdad. Följden
blir att belastningen ser kondensatorernas
spänningar som seriekopplade och därmed

har en spänningsfördubbling erhållits. Det
finns även kopplingar för flerdubbling av
spänningar, vilket bl. a. används för att alstra
accelerationsspänningen för TV-bildrör.

Spänningsstabilisering

Bild II 3-38
Utspänningen från ett kraftaggregat tillåts i
många fall att endast variera mellan vissa
värden, fastän inspänningen och strömuttaget varierar mycket. Ett vanligt sätt att hålla
konstant spänning är att anordna en automatisk spänningsdelare efter glättningsfiltret
Glimlampan och zenerdioden har egenskapen att spänningsfallet över dem är i det
närmaste konstant inom ett visst strömområde. Glimlampor arbetar på högre spänningar och används i utrustningar med elektronrör. Zenerdioder arbetar på de lägre
spänningar som används i dagens elektronik.
stabiliseringen tillgår så att t. ex. zenerdioden får ingå som aktiv del i en spänningsdelare, som består av en resister i serie med
belastningen och zenerdioden parallellt med
den. Zenerdioden tar upp variationerna i
belastningsströmmen, varvid spänningen
113-27

KR

1 :a Halvvågen

2:a Halvvågen

Bild II 3-37 Likriktarkoppling med spänningsdubbling
över spänningsdelarens uttag blir stabiliserad. Vid större strömuttag kan zenerdioden
inte ensam ta upp hela den effekt som den
reglerar bort. l stället tas effekten upp av en
eller flera transistorer som i sin tur regleras
av zenerdioden.
l vissa fall behövs i stället en reglerad
utström från kraftaggregatet Även för detta
ändamål används kopplingar med zenerdioder och transistorer.
Senare utvecklingsformer är s.k. switchade aggregat. l sådana regleras spänningen eller strömmen genom sönderhackning (switching). Genom att förändra förhållandet mellan till- och frånslagstiderna kan
man skapa det önskade medelvärdet. Metoden ger hög verkningsgrad. Switch-frekvensen är i storleksordningen 20 kHz eller högre. På grund av den högre frekvensen
krävs mindre kondensatorer i switchade
aggregat. Sådana kraftaggregat kan emellertid ge störningar, varför effektiv avstörning behövs.

113-28

Ostabiliserad
spänning in

Stabiliserad
spänning ut

Bild II 3-38 Spänningsstabilisering

3.4 Förstärkare

Allmänt
Bild Ii 3-40
Elektronrör och transistorer är de aktiva komponenter som används i oräkneliga elektroniska kopplingar för alstring av signaler, för
förstärkning och blandning av signaler, för
multiplicering av signalfrekvenser etc.
Transistorn presenteras i avsnitt 2.6 och
elektronröret i avsnitt 2.7.
Först förekom endast elektronrör. Dessa
har emellertid på några få decennier nästan
helt ersatts av transistorer. Elektronrör används dock fortfarande, särskilt i effektförstärkare för sändare. Det finns därför skäl att
här behandla såväl elektronrör som transistorer.

Förstärkning
Med förstärkning avses här kvoten av amplituden i utgående och inkommande signal,
varvid frekvensgången har inverkan.
Frekvensgång
Förstärkare arbetar endast inom ett visst
frekvensområde, vilket kan skilja från fall till
fall.
Bandbredd
Det frekvensområde där förstärkaren arbetar med fulla data kallas bandbredd. Bandgränserna uttrycks som en nedre och övre
gränsfrekvens, där signalnivån avviker från
ett givet värde, vanligen med högst 3 dB.
För LF-förstärkare för amatörradiobruk
är kravet på bandbredd litet; inom ett band
av 300 Hz till 3 kHz uppnås godtagbar
återgivningskvalitet för taL Bandbredden
bestäms främst av kondensatorer i kretsen
avsedda för överföring och avkoppling.
HF-förstärkare används för signaler med
hög frekvens, typiskt 100kHz och däröver.
Det finns s.k. bredbandig a förstärkare för ett
stort frekvensområde, men även avstämda
förstärkare för smala frekvensband.

Huvudegenskaper hos förstärkare
LF- och HF-förstärkare
Bild II 3-41
Med LF-förstärkare menas förstärkare som
arbetar med signaler i det lägre frekvensområdet, typiskt upp till c:a i 00 kHz. LF-förstärkare är mycket vanliga såväl i mottagare
som sändare. Utöver de aktiva komponenterna (transistorer, elektronrör ) är kondensatorer och resistorer de viktigaste passiva.

Med H F-förstärkare menas förstärkare som
arbetar med signaler med högre frekvenser
än dem i LF-området. Även HF-förstärkare
är mycket vanliga såväl i mottagare som
sändare. De används t.ex. i mottagarnas
ingångs- och mellanfrekvenssteg, liksom i
sändarnas oscillatorer, signalberedningssteg
och slutsteg.
Utöver de komponenter, som även finns
i LF-förstärkare, används kombinationer av
frekvensberoende komponenter såsom
induktorer och kondensatorer.

Anod

·~·

Galler

.p

Katod

n

Kollektor
Bas
Emitter

npn
Bild II 3-40 Från elektronrör till transistor

113-29

KRETSAR

o[]

o[]
Katod kopp l i ng

Emitterkoppling

Bild II 3-41 Principen för förstärkare med elektronrör respektive transistor

Grundkopplingar för förstärkarsteg
Bild 113-42

l det föregående har redan visats att en av
polerna i ingången respektive utgången i en
förstärkare är gemensam. l ovanstående
bild är rörförstärkarens katod den gemensamma polen -därav namnet katodkoppling.
På liknande sätt är N PN-transistorns emitter
gemensam -därav namnet emitterkoppling.
På ett liknande sätt kan någon annan pol
vara gemensam. Man får då i stället en
baskoppling eller kollektorkoppling.
Beroende av kopplingsätt fås olika egenskaper. På nästa sida visas tre olika grundkopplingar för ett elektronrör (triod) respektive en NPN-transistor.
l praktiken känns en grundkoppling igen
på vilken elektrod som är avkopplad till Opotential över en kondensator.

Emitterkoppling används för LF och HF
när hög förstärkning eftersträvas. Eftersom
effektförstärkningen är produkten av
spännings- och strömförstärkningen, så fås
en effektförstärkning av mellan 200 till50000
gånger. Nackdelen med denna koppling är
den ibland låga ingångsimpedansen och
den relativt låga gränsfrekvensen.
Baskoppling använd som H F-förstärkare pågund av sin höga gränsfrekvens och
goda isolation mellan in- och utgång.
Kollektorkoppling används när hög ingångsimpedans och utgångsimpedans önskas. Denna koppling har emellertid ingen
spänningsförstärkning, men kan användas
för s.k. impedansomvandling.

Grundkopplingarnas typiska egenskaper vid NPN-transistor
Egenskap

Emitterkoppling

Baskoppling

Z in

medel

1 KQ

liten

son

stor

100 kn

Z ut

medel

10 kQ

stor

100 kQ

liten

50 kQ

ström-

stor

100 ggr

<1

0.9 ggr

stor

100 ggr

spänning-

stor

100 ggr

stor

100 ggr

<1

0.99 ggr

effekt-

mycket stor 10000 ggr

stor

100 ggr

stor

100 ggr

motfas

medfas

oo

medfas

oo

Kollektorkoppling

Förstärkning

Fasläge

113-30

180$\circ$

ETSAR

Ut
In

In

Katodkoppling

~

In

!

l

j

Emitterkoppling

l
In

Ut

Ut

Gallerkopp li ng

j

Ut

In

Ut

Baskoppling

~

In

!

In,

In

Anod kopp! in g

Kollektorkoppling

Bild II 3-42 Grundkopplingar för elektronrör och NPN-transistor

113-31

KR

EPT
stabilisering av arbetspunkten

För att en förstärkare skall kunna arbeta på
avsett sätt måste arbetspunkten, d.v.s.
arbetsströmmens vilavärde ställas rätt.
Det gör man genom att placera en förspänning över den styrande elektroden i
elektronröret eller transistorn i fråga.
l en katodkopplad rörsförstärkare innebär det att styrgallret skall ges en viss negativ spänning i förhållande till katoden. Det
kan man göra t.ex. med en separat spänningskälla eller vanligare med en avkopplad
res istor mellan katod och minuspolen (jord).
l en emitterkopplad transistorförstärkare
innebär det att basen skall ges en viss positiv
spänning i förhållande till emittern. Det kan
man göra t.ex. med en separat spänningskälla eller vanligare med en avkopplad resisto r mellan emittern ochminuspolen samt en
resistiv spänningdelare mellan plus- och
minuspolen.

Klass A-, B- och C-förstärkare
Arbetspunkt
Arbetspunkten för förstärkare väljs olika,
beroende på önskat arbetssätt. En olämpligt
vald arbetspunkt resulterar i förvrängning av
utsignalens form i förhållande till insignalens
form, s.k.distorsion. Distorsion uppstår även
vid överstyrning, d.v.s. när insignalens amplitud är för stor för att kunna återges med
oförändrad form, även om arbetspunkten är
rätt vald.
Med avseende på arbetspunktens läge
klassas därför förstärkare på sätt som framgår av följande diagram för elektronrör. En
emitterjordad NPN-transistor får anses mest
motsvara elektronrörkopplingen här nedan.
Anodströmmen la motsvaras då närmast av
kollektorströmmen le och styrgallerspänningen U~ 1 av spänningen UsE· Den stora skillnaden år att styrgallerspänningen i dessa fall
alltid är negativ medan bas/emitterspänningen är positiv. styrspänningens relativa läge
(arbetspunkten) mellan olika arbetsklasser
är emellertid lika.

113-32

KRETSAR
Klass A
Bild II 3-44
Klass A är ett arbetssätt i linjära LF- och HFförstärkarsteg, t.ex. i mottagare samt AMoch SSB-modulerade sändare. Vilavärdet
på strömmen i huvudkretsen, den s.k. arbetspunkten, placeras i mitten på den rakaste delen av styrkaraktäristikan (1=0.5 lmax).
Därmed fås låg distorsion. Verkningsgraden
är upp till 50 $\circ$/o.

ningskretsen. En resonanskrets med högt
Q-värde behövs som utgångskrets varvid
amplituddistorsion inte framstår som besvärande vid CW och FM. Med hjälp av en
utgångskrets kan frekvensmultiplicering utföras med förstärkare i klass C.
(På följande tre bilder är IR=anodviloström).

Klass AB
Klass AB är ett godtagbart linjärt arbetssätt
för AM- resp. SSB-modulering, men med en
lägre viloström. Arbetspunkten ligger mellan
den för klass A och B. Ett linjärt arbetssätt
enligt klass A är visserligen önskvärt vid
SSB, men verkningsgraden är lägre. Klass
AB är en kompromiss med bättre verkningsgrad utan en alltför stor distorsion.
Klass B
Bild II 3-45
Klass B är ett olinjärt arbetssätt med en låg
vilaström i förhållande till !max• d.v.s. arbetspunkten ligger i nederkant av styrkaraktäristikans nedre krökta del. Verkningsgraden
är upp till 67o/o. Trots det används klass 8 i
linjäraLF-och H F-förstärkarsteg t. ex. i SSBsändare.
Om klass B skulle tillämpas i ett slutsteg
med endast ett rör eller en transistor skulle
större delen av uteffekten förloras i splatter,
d.v.s. som förvrängda signaler långt vid sidan om den egentliga nyttosignalen. Ett sätt
att undvika det är att använda en avstämd
utgångskrets med högt Q-värde. Linjär förstärkning kan också erhållas med två mottaktkopplade rör eller transistorer i klass B.
Utgångskretsen behöver då inte vara avstämd av linjäritetsskäl.
Klass C
Bild 113-46
Klass C används i HF-förstärkarsteg i FM-,
CW- och AM-sändare. Arbetssättet är kraftigt olinjärt. Vilaströmmen är noll, d.v.s. arbetspunkten ligger på den negativa delen av
styrkaraktäristikan. Endast toppen av den
ena halvvågen av insignalen återges och i
starkt förvrängd form. Verkningsgraden är
upp till80$\circ$/o. Övertonerna dämpas av sväng-

~järt

l

Förvrängning
(distorsion)

genom fel arbetspunkt

Bild If 3-44 Förstärkare i klass A

113-33

KRETSAR

Bild II 3-45 Förstärkare i klass B

Frekvensmultiplicering
Bild 113-47

Frekvensmultiplicering kan användas för att
skapa en högre frekvens än den som avges
av oscillatorn. Oscillatorn följs då av ett eller
flera frekvensmultiplicerande förstärkarsteg
som arbetar i klass C.
l utgången av ett frekvensmultiplicerande steg måste finnas en svängningskrets,
som är avstämd till önskad frekvens, d.v.s.
överton av insignalen. Denna överton förstärks i efterföljande förstärkarsteg, vilket
också kan vara frekvensmultiplicerande.
Ju högre multiplikationsfaktorn är, desto
högre förspänning krävs för att svängningskretsen i utgången skall svänga obehindrat.
Med hög multipliceringsfaktor i ett enda steg
dämpas signalen då så mycket att en hög
förstärkning behövs i efterföljande steg. l
praktiken anordnas därför en kedja av frekvensdubblande och frekvenstripplanda
Den totala multipliceringsfaktorn är faktorerna för vartdera steget multiplicerat med varandra.
Som exempel visar bilden blockschemat
för en VHF-sändare med oscillatorkristaller i
8 MHz-området. Som räkneövning kan andra kristallfrekvenser sättas in för beräkning
av den slutliga sändningsfrekvensen. l frekvensmultiplicerande sändare kan även slutsteget arbeta i klass C, vilket är vanligt i
sändare förtelegrafi eller FM-telefoni. För att
då förhindra utsändning av alla de övertoner

113-34

Bild II 3-46 Förstärkare i klass C
som alstras i förstärkarkedjan, så förses
slutstegets utgång med en svängningskrets
som är avstämd till sändningsfrekvensen.
Övertonsdämpningen kan förbättras }'1terligare med ettefterföljande lågpassfilter. Overtoner för 144 MHz är 288 MHz, 432 MHz
o.s.v.
Frekvensmultiplicering behöver nödvändigtvis inte göras med ett förstärkarsteg i
klass C. En diod har nämligen olinjär karaktäristik och därmed alstras det övertoner i de
strömmar som passerar genom den. En av
dessa övertoner kan filtreras fram och förstärkas. T.ex. finns det frekvenstripplingssteg byggda kring en speciell typ av kapacitansdiod- varaktordiod. Vanliga frekvensområden för s.k. varaktortripplare är 144/
432 MHz och 432/1296 MHz.
Såväl signalen från en kristalloscillator
som den från en VFO kan multipliceras till en
högre frekvens.
Förr täckte VFO i amatörradiosändarna
oftast frekvensområdet 3.5-3.8 MHz. Med
en så vald VFO-frekvens kunde alla upplåtnafrekvensband för amatörradio nås med
frekvensmultiplicering. De ursprungliga amatörradiobanden i KV-området ligger fortfarande harmoniskt relaterade av detta skäl.
Således
2 • 3.5 = 7 MHz
2 · 2 · 3.5 = 14 MHz
2 • 3 · 3.5 = 21 MHz
2 · 2 · 2 · 3.5 = 28 MHz

KRETSAR

PT
-~ 1%1

T

CD

r·---{I>J---~--{g
Olinjär förstärkare

~····
f

\

f"' " 24 HHz

= 8 MHz

!f

8 MHz

+övertoner

~---·----

·-·lliJ--..--------..·--· ---

r

------------o

.,..............

n = 3 · 3 · 2 = 18

fs : : n. fa
fs = 18 · fa

[ii}

fq = 8, ... MHz

l

f :.: ..., ... MHz

f = B, ... MHz

= 24MHz

24 MHz

.
f re kvenstnpp 1are

[o

DJtAAAMAft

!Lfl[} .....

f= ... , ... MHz

f = .. , ... MHz

f = ... , ... MHz

Fyll i frekvensvärdena för styrkristallen och beräkna resterande frekvensvärde
i multiplikatorkedjan

Bild 113-47 Frekvensmultipliceringskedja
Vid frekvensmultiplicering flerfaldigas inte
bara oscillatorfrekvensen utan även variationerna i den. Om t.ex. VFO-frekvensen i
området 3.5 MHz ändras med 50 Hz, så
ändras utfrekvensen i området 28 MHz med
2 · 2 • 2 = 400 Hz. Alla frekvenser i signalen
multipliceras på detta sätt. Amplitudmodulerad telefoni kan därför inte överföras genom
en frekvensmultipliceringskedja utan att talet förvrängs.

Se5.3

113-35

KRETSAR
Sändarslutsteg
Slutsteg med en transistor
Bild 113-48

Transistorslutsteg för HF byggs vanligen
emitterkopplade p.g.a. den högre effektförstärkningen.
Bilden visar ett sådant förstärkarsteg.
Kollektorbelastningen består av en svängningskrets. För att anpassa transistorns kollektorimpedans till svängningskretsens impedans, har kollektorn anslutits till ett uttag på
svängningskretsens spole.
Orossel Dr och kondensator C fungerar
som en HF-mässig avkoppling av strömförsörjningen.
Uteffekten tas ut från svängningskretsen
över en kopplingslindning med samma impedans som belastningen.
För linjär återgivning krävs drift i klass A
eller möjligen klass AB.
Or* +U

Bild II 3-48 S!utsteg med en transistor

slutsteg med två transistorer
Bild 113-49
Ett mottaktkopplat (e ng. pus h-pull) förstärkarsteg i klass B har god verkningsgrad
samtidigt som det är nöjaktigt linjärt för SSB
i amatörradio. l ett slutsteg med endast en
transistor skulle denna behöva klara fyra
gånger så stor förlusteffekt
P.g.a. de låga impedansvärdena i transistoriserade förstärkarsteg används transformatorer, vilka inte är frekvensselektiva
och därför inte dämpar övertoner. Med mottaktkopplingen alstras dock inte jämna övertoner. För övertonsdämpning används fast
avstämda bandpassfilter, ofta ett per frekvensband, mellan drivsteg och slutsteg samt
mellan slutsteg och antenn.
För noggrann anpassning till antennen
behövs en antennkopplare - s.k. matchbox
- med ett n-, T- eller L-kopplat LC-filter.
Att ett slutsteg är "bredbandsavstämt" är
således en fråga om definitioner.
Högeffekts/utsteg med en tetrod
Bild II 3-50
Bilden visar ett effektslutsteg för HF med ett
elektronrör, en s.k. tetrod, i katodkoppling.
Det kan även vara en triod eller en pentod.
Med LC-kretsen i styrgallerkretsen filtreras (selekteras) önskade signalfrekvens ut
ur signalerna från föregående steg.
Orossiarna Dr spärrar HF och kondensatorerna C1 , C2 och C3 kortsluter (avkopplar)
HF till jord. Allt för att hindra HF att komma in
i kraftaggregatet

bredband
U2

Bild 113-49

113-36

omkopplingsbart för
vart och ett av banden

KRETSAR
HF-förstärkare kan råka i oönskad självsvängning. Orsakerna kan vara många, bl. a.
dålig avkoppling av matningsspänningar, induktiv och/eller kapacitiv återkoppling i kretsarna m.m.
Återkopplingsvägar både före och efter
röret kan bilda oavsiktliga svängningskretsar, som genererar självsvängning, ofta på
mycket höga frekvenser t.ex. i VHF-området. Sådana s.k. parasitsvängningar kan
stoppas/dämpas med UHF-drosslar (UHF
Dr) omedelbart intill röranslutningarna.

En åtgärd mot självsvängning i elektronrör är en motkopplingsväg från anod till styrgaller över en trimningsbar s.k. neutraliseringskondensator CN.
slutstegets utgångskrets kan utformas
på olika sätt. Bilden visar ett numera vanligt
sätt, det s.k. n-filtret (utläses pi-), som fungerar som
• en svängningskrets som är avstämd till
sändningsfrekvensen,
• ett övertonsdämpande lågpassfilter,
• anpassning mellan rörets utgångsimpedans och antenntilledningens impedans.

UKV-drossel

Cp

Bild II 3-50 Högeffekts/utsteg med en tetrod

- Ug1

I

I I

I
Bild II 3-51 Högeffekts/utsteg med två trioder

113-37

KRETSAR
Högeffekts/utsteg med två gallerjordade trioder (elektronrör)
Bild II 3-5i
Gallerjordad koppling innebär att elektronrörets styrgaller ligger på HF-mässig nollpotential medan styrsignalen matas in på
katoden. likspänningen mellan katod och
styrgaller väljs så att rörets arbetspunkt blir
den avsedda.
Gallerjordad koppling passar särskilt för
slutsteg med höga effekter, men fordrar en
högre styreffekt än andra kopplingar. l gengäld "överförs" styreffekten till utgången via
röret och ingår där i uteffekten. l gallerjordad
koppling är kapacitansen låg mellan katod
och anod, d.v.s.mellan in- och utgång. Därmed är risken för självsvängning betydligt
mindre än i ett katodjordat steg.
Uteffekten kan ökas genom att parallellkoppla två eller flera rör, som då skall ha så
lika data som möjligt. Uteffekten står i direkt
proportion till antalet rör.
Flera parallellkopplade rör medför emellertid ökade totala rörkapacitanser, ökade
kapacifanser i kopplingsledningarna m.m.,
vilket är till nackdel vid höga frekvenser.
Ett enda slutrör för hela effekten är emellertid dyrare än flera små med tillsammans
jämförbar effekt. Mottaktkoppling av två rör
(eng. "push-pull") i st.f. parallellkoppling har
en fördel i högre förstärkning, men nackdelar i mer komplicerad bandomkoppling av
svängningskretsar m.m. l moderna rörutrustade slutsteg för amatörradio förekommer
därför endast ett slutrör eller flera parallellkopplade. Utgångskretsen är i regel ett nfilter med manuell eller automatisk avstämning.

Slutsteg med elektronrör jämfört med
transistoriserade slutsteg

Ett slutsteg med transistorer är kompakt och
skaktåligt och använder bara klenspänningar. Det är därför särskilt vällämpat för portabelt och mobilt bruk.
Men transistorer är känsliga för överbelastning. Redan ytterst kortvarig överbelastning eller överspänning kan förstöra dem.
Transistorer är också känsliga för termisk
överbelastning. Särskilt vid höga effekter i
trånga utrymmen är det nödvändigt med god
kylning, eventuellt med fläkt.

113-38

Ett slutsteg med elektronrör är inte så
skaksäkert, men är mycket okänsligare i
övriga avseenden. En nackdel är att det
behövs extra effekt för uppvärmning av rörens katoder samt höga anodspänningar,
som är farliga vid ovarsamhet. P.g.a. behovet av flera olika spänningar är även strömförsörjningen för ett slutsteg med elektronrör
mer komplicerad och omfångsrik.

Bestämning av PEP-effekten
Bild 113-52
Moduleringsspänningens topp-toppvärde
Uss mäts lämpligen med ett oscilloscope.
För korrekt belastning vid mätningen används en konstlast
Med topp-toppvärdet känt kan man med
följande formler beräkna
toppvärdet (amplituden

U - Uss

effektiwärdet

Us
Uett= {2.

s- 2

och

Effekten vid moduleringstopparna, s.k.
PEP (Peak Envelope Power), kan beräknas
med följande formler

ue~
k.
P,PEP =
R respe t1ve
P,
PEP=

Us~
SR

PA

os c i lloscope

Bild II 3-52 Bestämning av PEP-effekten

KRETSAR
linjäritetskontroll vid SSB

Bild II 3-53
Linjäriteten i en SSB-sändare kan kontrolleras med ett oscilloscop. Sändaren moduleras då med två övertonsfria toner.

slutsteget bör först belastas med konstlast
upp till max tillåten effekt. Resultatet jämförs
därefter med antennen som last.

SSB-TVATONSSl GNAL
Oscillogram

u

Spektrum

Spektrumanalys

(tid/spänn i ngsdiagra m l
normal SSB-bandbredd

-----t

l.lL 

f

Idealisk linjär förstärkning (klass A- utan överstyrning)

Nästan linjär förstärkning (klass AB- utan överstyrning)

Olinjär förstärkning - för låg vilaström

överstyrning (klippning)
SSB-SIGNAL VID TAL ("aaah")

Linjär förstärkning

Överstyrning (klippning)

Bild II 3-53 Unjäritetskontro/1 vid SSB
113-39

PT

KRETSAR
Linjäritetens betydelse i förstärkare
Bild II 3-54
Förstärkningen bör ske med god verkningsgrad och minsta möjliga förvrängning, så att
det alstras ett minimum av oönskade frekvenser inom minsta möjliga bandbredd.
Linjär förstärkning innebär att den är lika
över hela det aktuella frekvensområdet Frekvensgången måste därför vara så rak som
möjligt. Med tilltagande olinjäritettillkommer
nämligen allt fler oönskade frekvenser.
Det uppstår blandningsprodukter av högre ordning vid olinjär förstärkning. Genom
förvrängning p.g.a. olinjär förstärkning uppstår ömsesidiga summa- och skillnadsfrekvenser av de modulerande frekvenserna.

Varje sådan blandningsprodukt blandar
sig additivt och subtraktivt med grundfrekvenserna till ytterligare blandningsprodukter
av näst högre ordning.
Dessa är:
0
blandningsprodukter i LF-området och
deras övertoner, vilka undertrycks i efterföljande HF-krets,
e
grundfrekvenserna och deras harmoniska
övertoner, som alla ner till 1 :a harmoniska dämpas kraftigt av efterföljande
H F-krets,
0
alla summa-och skillnadsfrekvenser av
de förstnämnda frekvenserna.

l ngångssignal

l  l 
Utgångssignal vid linjär förstärkning

Utgångssignal vid olinjär förstärkning

Olinjär karaktäristik

Linjär karaktäristik

la

Ug

Bild II 3-54 Linjäritetens betydelse

113-40

Ug

KRETSAR
l området för nyttafrekvenserna kallas
dessa produkter för intermodulationsprodukter och ger talförvrängning.
Utanför nyttafrekvenserna uppfattas
intermodulationsfrekvenserna som störningar och kallas splatter. På grund av det
lilla frekvensavståndet till nyttasignalen kan
den intermodulation, som alstrats i slutsteget
inte filtreras bort i efterhand.
Vid linjär drift uppträder grundfrekvensernas övertoner och intermodulationsfrekvenser endast svagt inom och utom överföringsbandet och kommer knappast att uppfattas som inkräktande på annan radiotrafik.
De svaga övertonerna kommer också att
dämpas tillräckligt i n-filtret och eventuella
ytterligare övertonsfilter.

tion till utstyrningsgraden. ALG-spänning
återförs till drivsteget och reglerar dess uteffekt så att överstyrning av slutsteget inte
sker. l transistoriserade slutsteg skapas ALGspänningen genom likriktning av slutstegets
utspänning. l rörslutsteg börjar styrgallret
dra ström, när styrgallerspänningen blir positiv i signaltopparna, vilket används för att
styra ALG-spänningen. När ALG-regleringen
sätter in, är överstyrningen således redan ett
faktum. Överstyrning kan ske både på LFoch HF-nivå.
En orsak till övermodulering är för stor
amplitud på den modulerande signalen. Detta
kan bl.a. avhjälpas med inställning av
mikrofonförstärkaren och riktig mikrofonhantering.

Utstyrningskontroll av slutsteg

slutstegets linjära utstyrningsområde överskrids, om ingångssignalens amplitud blirför
stor. Då ökar utgångssignalens amplitud inte
mycket mer, men utgångssignalens toppar
blir tillplattade (s.k. klippta). Det betyder att
slutsteget är överstyrt.
Vid överstyrning uppstår signalförvrängningar, som medför intermodulation, förvrängt tal, splatterstörningar och övertoner.
Den extra effektökning som uppnås med
överstyrning förbrukas i stort sett till signalförvrängning och kommer inte nyttasignalen
till godo. Överstyrning skall därför undvikas.
Drivstegets uteffekt får inte vara så stor
att slutsteget blir överstyrt. Ett slutsteg med
jordad katod blir fullt utstyrt redan vid en
driveffekt av ett fåtal watt. Ar uteffekten från
drivsteget större, än vad som behövs för full
utstyrning av slutsteget, och driveffekten inte
kan regleras ner, så måste en dämpsats
kopplas in mellan drivsteg och slutsteg. En
sådan dämpsats kan behöva ta upp en betydande effekt, från en vanligt förekommande
amatöradiosändare upp till1 00 watt PEP.
Ett slutsteg med jordat galler fordrar en
större driveffekt, varvid risken för överstyrning i slutsteget är något mindre och de
förebyggande åtgärderna inte så omfattande.
Linjära slutsteg innehåller oftast en funktion kallad ALG (Automatic Load Gontrol),
som kontinuerligt känner av driveffektens
inverkan på slutsteget När driveffekten blir
för hög, alstras en kontrollspänning i propor113-41

KRETSAR

113-42

KRETSA
3.5 Detektorer - Demodulatorer
Allmänt

Sändaren omvandlar informationen i lågfrekventa signaler till högfrekvens som kan
strålas ut från en antenn. l mottagningsanläggningen återvandlas informationen, vilket kallas demodulation eller demodulering.
l mottagare som är specialiserade för ett
sändningsslag, används bara en typ av demodulator medan mottagare för flera sändningsslag, AM, SSB/CW, FM etc. har flera
demodulatorer. Det finns många typer och
namn på demodulatorer, t.ex. detektor, diskriminator. Här beskrivs några av dem.

AM-detektorer
Dioddetektorn AM (A3E)
Bild II 3-55

Bilden visar en superheterodynmottagaredär den sista M F-kretsen är induktivt kopplad till demoduleringsdioden. Den amplitudmodulerade M F-signalen visas som ett amplitud/tid-diagram.

Dioden klipper antingen de negativa eller
positiva halvvågorna, beroende på hur den
är vänd - polariserad.
LF-signalen filtreras ut ur de högfrekventa pulserna med ett LF-Iågpassfilter.
LF-signalen är nu överlagrad på en likspänning. l talpauserna sänds bara bärvågen och då lämnar AM-demodulatorn bara
likspänning, som skiljs från LF-förstärkaren
med en kondensator. Kondensatorn släpper
bara igenom LF-signalen, som förstärks.

Produktdetektorn SSB (J3E)
Bild 113-56
Det finns flera metoder att dernodulera en
SSB-signal, såsom fasningsmetoden, filtermetoden och den s.k. tredje metoden. Filtermetoden är numera är den allra vanligaste
och beskrivs här.
En SSB-signal med undertryckt bärvåg
består av endast ett sidband. Det andra
sidbandet och bärvågen undertrycks i sändaren.

Blockschema på en
AM-super (A3E)

frän sista MF-steget

u

MF

t
A3E- demoduleringsförlopp

Bild II 3-55 Dioddetektorn

113-43

LSB
l . USB
9001,5 ~ ~ 8998,5
kHz
kHz

SSB - mottagare {J3E)

Produktdetektor

u

MF

u

M F-fi lterkurva

SSB-Slgnal
( USB}

LF

==>300Hz, 1kHz och 3kHz

8998,8;
8999,5 och

9 001,5 kHz

u

u

MF

LF

BFO

~~----~~~~,------~f

8998,5;
9001,5
9000,5 och
kHZ
9001,2 kHz

•

300 Hz, 1kHz och 3 kHz

Bild 113-56 Produktdetektor för AM (A3E) och CW (A 1A)

113-44

ETSAR
Vid dernoduleringen av SSB-signalen
alstras i mottagaren en signal som ersättning för den bärvåg som undertrycktes i
sändaren. Det undertrycktaandra sidbandet
ersätts inte.
l en mottagare med direktblandning blandas SSB-signalen med VFO-signalen, varvid en del av blandningsprodukterna faller ut
på LF-nivå.
l en superheterodynmottagare däremot,
blir SSB-signalen först blandad med en VFOsignal och som resultat erhålls en mellanfrekvens MF. Den till MF omvandlade signalen förstärks, filtreras och blandas med en
lokal BFO-signal i ytterligare en blandare,
kallad produktdetektor. Några blandningsprodukterna faller ut på LF-nivå. Ett lågpassfilter följer därför efter detektorn för att filtrera
ut LF-signalerna
Numera består produktdetektorn vanligen av en ringblandare, som i ett omvänt
förlopp även kan användas vid DSB-modulering i en sändare. Bilden visar dernoduleringen av en SSB-signal som innehåller tre
LF-toner
CW-/SSB-detektorer CW (A 1A)
Även telegrafi-signaler, även kallat CW, blir
demodulerade när M F-signalerna och BFOsignalen blandas i en produktdetektor.
Till skillnad från SSB är det vid CW inte nödvändigt med en given skillnad mellan MFoch BFO-frekvenserna. Frekvensskillnaden
påverkar bara överlagringstonens frekvens,
men inte läsligheten av CW-budskapet.
Många moderna mottagare har en fast
BFO-frekvens för CW, som ger en 800 Hzton vid rätt frekvensinstälining. l stället för
lågpassfiltret för SSB, används ibland ett
bandpassfilter, som bara släpper igenom
CW-signaler i frekvensområdet 800 Hz -en
idealfrekvens för god läsbarhet av morsetecken.

FM- och PM-detektorer
Bild 113-57
Vid vinkelmodulering överförs informationen
enbart genom frekvens- eller fasvariationer
i bärvågen. De amplitudvariationer som kan
uppstå före dernoduleringen är ej önskvärda
i detta sändningsslag. Av den anledningen
finns i FM-mottagare en amplitudbegränsare (limiter) före diskriminatorn (se följande
bild). Frekvensvariationerna i den FM-modulerade signalen omvandlas därefter av
detektorn till LF-spänning som motsvarar
det utsända talet.
Bild 113-58
Dernoduleringen skall ske med mottagaren inställd mitt på avsedd sändarfrekvens.
Ett hjälpmedel för det är en indikator, som vid
rätt inställning visar värdet noll. Positivt eller
negativt utslag anger att inställningen är för
högt respektive för lågt i frekvens. En sådan
indikator fanns i tidiga FM-mottagare. Nu
används i stället en AFC (Automatic Frequency Contro l) som själv ställer in mottagaren omsändarfrekvensen är tillräckligt nära.
Slope-detektorn - Diskriminatorn FM (F3E)
Bild II 3-59
Två svängningskretsar är kopplade induktivt
till den sista M F-kretsen. Resonansfrekvensen för dessa båda kretsar är något högre
respektive något lägre än mellanfrekvensen. De signalspänningar som uppträder
över svängningskretsarna likriktas och seriekopplas med varandra med motsatt polaritet.
När de båda svängningskretsarna matas
med samma frekvens, kommer likspänningarna att ta ut varandra. När frekvensen avviker uppåt i frekvens, kommer kretsen med
den högre resonansfrekvensen i kraftigare
svängning än den andra kretsen och avger
högre likriktad spänning. När frekvensen
awiker nedåt i frekvens, skiftar de båda
kretsarna roller, och den resulterande likriktade spänningen skiftar till motsatt polaritet.
Vid växelvisa frekvensändringar i MF,
överoch undervilofrekvensen, blir resultatet
en växelspänning ut från likriktarnas utgångsfilter, som är LF-signalen.

113-45

KRETSAR
F M-mottagare ! F3E)

MF-förstärkare och

2 · ( L1 f max
= 2 · ( 3 kHz

+ fLFmax>'·,.      /
+

3 kHz)

= 2 · 6 kHz = 12 kHz

Bild II 3-57 Amplitudbegränsning vid FM-mottagning

u
variationer i utgängsspänningen =
LF-växelspänning

f
frekvensvariationer i
den frekvensmodulerade bärvägen

Bild II 3-58 Ideal arbetslinje för diskriminator

MF-mittfrekvens

u

fres2

"'-...

''

'\

fres1

fres1

'

\

\

\

''

f

' ' ....
sista MFsvängkrets

Bild II 3-59 Slope-detektorn

113-46

till
L F-förstår kara

f res 2

ETSAR
Foster-Seeley diskriminatorn
Bild 113-60
Denna tidiga dernodulater har god linjäritet,
om den föregås av en god amplitudbegränsare, men har tämligen dålig känslighet.
Sista MF-förstärkarsteget avslutas med
en transformator vars båda lindningar ingår
i svängningskretsar avstämda till MF. MFsignalen överförs från primär- till sekundärsidan dels med induktion och dels med en
kondensatortill mitten av sekundärlindningen. Signalen delas på så sätt i två grenar
med en fasförskjutning av +90$\circ$ resp. -90$\circ$.
Signalerna i grenarna likriktas varför sig och
sammanlagras i ett RC-nät.
Om M F-signalen inte devierar så är LFspänningen i grenarna lika, men eftersom
grenspänningarna har motsatt polaritet så
tar de ut varandra och LF-signalen blir noll.
När M F-frekvensen devierar av modulering,
så ökar signalamplituden i den ena grenen
och minskar i den andra. LF-signalens amplitud blir då proportionell med frekvensdeviationen.

svängkrets

Bild II 3-60 Foster-Seeley detektorn

MF-förstärkare

och begränsare

från blandan~

Jl

Räknardiskriminatorn
Bild II 3-61
En mono-flopvippa påverkas att slå över av
fyrkantpulserna från de amplitudbegränsade
FM-signalerna.
En sådan vippa är en digitalkoppling som,
när den matas med en godtyckligt lång spänningspu!s, ändå kommer att leverera en spänningspuls med konstant längd. För varje
positiv halvvåg leverar mono-flopvippan en
impuls av konstant längd. Tidsavståndet
mellan pulserna kommer att vara proportionella till FM-signalens frekvens. Vid varierande frekvens kommer impulserna med
varierande tidsavstånd. Ett lågpassfilter filtrerar ut lågfrekvensen ur signalen och en
pulserande likspänning kvarstår. Med denna likspänning laddas kondensatorn upp till
ett medelvärde. Vid en högre frekvens av
lika långa pulser blir medelvärdet högre än
vid en lägre pulsfrekvens.
Svängningarna på likspänningen är LFsignalen, överlagrad på en likspänning. Utan
en mono-flopp med lika långa pulser hade
medelvärdet varit konstant. Man kan säga
att FM-signalen blivit omvandlad till en PLMsignal (pulslängdmodulerad signal).
PLL - demodulatorn
Bild II 3-62
Den frekvensmodulerade MF-signalen och
en VCO-signal matas in i en fasjämförare.
V GO-frekvensen följer frekvensändringarna
FM-signalen Avstämningsspänningen för
VCO är en likspänning. Den modulerande
LF-spänningen är överlagrad på denna likspänning.

monaflop

amptitud·
begränsad
FM-signal
medelvärde

Bild fl 3-61 Räknardiskriminatorn
113-47

KR
M F-förstärkare
och begränsare
frän blandare

avstämningsspänning

Bild 113-62 PLL-demodulatorn
LF-frekvenserna är för låga för att kunna
reglera VCO-frekvensen, men via en kondensator kan de styra LF-förstärkaren.
De båda sista metoderna lämpar sig speciellt för demodulering av F1-signaler. F1signaler kan "demoduleras" i en SSB-mottagare eftersom det uppstår en rytmisk svängning i tonhöjden. Denna frekvensmodulerade
ton kan sedan dernoduleras på sätt som
beskrivits.
Det finns ytterligare sätt att dernodulera
FM-signaler. Gemensamt för alla är, att de
fungerar bättre ju lägre mellanfrekvensen är.
Därför utförs de flesta FM-mottagare som
dubbel- eller trippelsuprar, med låg MF.

113-48

KRETSAR
3.6 Oscillatorer

Alstring av svängningar
Ordet aseiiiare (lat.) har betydelsen svänga
och den företeelse eller anordning som skapar en svängning kallas oscillator. Vid alla
slags svängningar sker växelverkan mellan
olika energiformer. Svängningar förekommer i olika former. Det kan vara vibrationer i
en kropp, en pendel som svänger, rörelser i
gaser och vätskor, elektriska laddningar i en
strömkrets o.s.v.
Mekanisk pendel
Bild II 3-63
Energiinnehållet i en pendel växlar mellan
lägesenergi (potentiell energi) och rörelseenergi (kinetisk energi). Lägesenergin är
störst i pendelns ytterlägen och minst i mittläget. Omvänt är rörelseenergin störst i mittläget och minst i ytterlägena.

Stöt till en pendel bara en gång så att den
börjar pendla, men utslagen blir allt mindre.
Pendeln utför en dämpad svängning därför
att det förbrukas energi under pendlingen.
Dämpningen kommer av att energi förloras av friktionen i upphängningspunkten och
av luftmotståndet.
Stöt nu till pendeln varje gång, som den
pendlar tillbaka. För lika stora utslag varje
gång- en odämpad svängning- fordras det
upprepade energitillskott som precis kompenserar förlusterna.
Villkoren för att en svängning skall fortgå
(vara odämpad) är att energitillskotten
• kommer vid rätt tidpunkt,
• har rätt riktning- polaritet,
• kompenserar förlusterna.

PENDEL

\

l
Epot

A

t
DAMPAD SVÄNGNING

ODÄMPAD SVÄNGNING

A

=

utslag från viloläge

Bild II 3-63 Svängningar

113-49

KRETSAR
Alstring av mekaniska svängningar
Bild II 3-64
En elektrisk ringklocka med självbrytande
kontakt är ett exempel på en enkel elektromekanisk oscillator. Denna bild visar hur
kontakten ersatts med ett elektronrör. En
mer "tidsenlig" lösning med transistor hade
naturligtvis också kunnat användas.
När anodspänningen kopplas till, så börjar anodström att flyta genom trioden från
katod till anod och genom elektromagneten.
Magneten drar då till sig bladfjädern, som
kopplar styrgallret till en negativ förspänning.
Den negativa förspänningen stryper anodströmmen och magnetfältet upphör. Bladfjädern släpper då från magneten och kopplar
bort styrgallret från förspänningen. Anodström börjat att flyta igen, varvid elektromagneten drartill sig bladfjädern o.s.v. Förloppet
kallas självsvängning.
Anordningen som alstrar svängningarna
kallas generator eller oscillator. Frekvensen
är det antal svängningar per sekund som
oscillatorn alstrar, i detta exempel bladfjäderns svängningshastighet.

Amperemetern
visar den pulserande
anodströmmen, som är en likström.
Amperemetern A 2 visar växelströmmen
svängningskretsen.
1 Halvvågen

Ua
POSITfVT GALLER:
- i anodkretsen flyter en ström
{A1 visar mot höger)
- i svängningskretsen flyter en ström
(A 2 visar mot höger)

2 Halvvågen

....--~

~--------~+

U9

,A1

,

-F-------~

V
~----~-

Bild II 3-64 Elektromekanisk oscillator

Alstring av elektriska svängningar
Bild II 3-65
En elektrisk svängningskrets är motsvarigheten till ett mekaniskt föremål i svängning.
Bilden visar en elektronisk oscillator med en
LC-krets i anodkretsen till en triod och som
är induktivt återkopplad till styrgallret

113-50

t

+~--~

Ua
NEGATIVT GALLER:
- ingen ström flyter i anodkretsen
(A 1 visar noll)
- i svängningskretsen flyter en ström
i motsatt riktning
(A2 visar mot vänster)

Bild 113-65 Elektronisk oscillator (Meissner)

KRETSAR
En elektronisk oscillator är en förstärkare, vars utsignal återförs till ingången så
att förstärkaren råkar i självsvängning -det
blir en s.k. positiv återkoppling.
Elektroniska oscillatorer används både i
mottagare och sändare. Funktionsprincipen
är lika i båda fallen. Vad som möjligen skiljer
är användningssättet Det finns många oscillatorkopplingar varav några beskrivs här.
Självsvängning kan demonstreras akustiskt med en mikrofon, en förstärkare och en
högtalare. Resultatet blir att det genereras
en ton. Ljudet från högtalaren är tryckvågor
som påverkar mikrofonen och omvandlas
där till elektriska signaler. Dessa går genom
förstärkaren tillbaka till högtalaren och omvandlas åter till tryckvågor som mikrofonen
uppfattar o.s.v. Det uppstår självsvängning,
ett tjut, som beror på akustisk återkoppling.
Om förstärkaren kompletteras med ett frekvensfilter i form av en svängningskrets, så
blir "tjutet" i stället en ton med samma frekvens som filtrets resonansfrekvens.

Bild II 3-66 Oscillator enligt Meissner
Bild II 3-67
Förstärkaren kan t.ex. vara en emitterkopplad transistorförstärkare enligt bilden.
Kopplingskondensatorerna Ck är nödvändiga för att förhindra kortslutning av de likspänningar som pestämmer arbetspunkten
för transistorn. A andra sidan kan växelspänningssignalerna passera till och från
transistorn.

:r
ck

r-

.,.--1}-----o.
utgång

LC-oscillatorer
Variabel frekvens oscillator- VFO
En oscillator med inställbar frekvens kallas
för VFO (variabel frekvensoscillator). Förutom frekvensstabilitet fordras också, att
noggrann inställning och avläsning av frekvensen skall kunna göras.
En Le-oscillator är urtypen för en oscillator med variabel frekvens. Meissner-kopplingen är lätt att urskilja och används här för
att beskriva grundprincipen för en oscillator
i stort. Bl.a. Colpitts- och Glapp-kopplingarna har emellertid bättre stabilitet och
inställbarhet i återkopplingsledet
Meissner-koppling
Bild 113-66
Bilden visar en Meissner-oscillator, som består av en LC-svängningskrets med återkopplingsspole och en förstärkare. Magnetfältet mellan induktansen i svängningskretsen och återkopplingsspolen är polariserat
så att en förändring i utsignalen medverkar
till självsvängning. (Motsatsen är motkoppling.)

f
Bild II 3-67 Emitterkopplad förstärkare
Bild II 3-68
Återkopplingsvägen görs i detta fall så, att
svängningskretsen kopplas parallellt över
förstärkaringången. Aterkopplingsspolen
fungerarsom förstärkarens kollektorresistor.

r

Bild II 3-68 Komplett Meissneroscillator

113-51

KRETSAR
Självsvängningsvillkoret

Bild II 3-69
Självsvängning i en förstärkare uppstår genom återkoppling. signalspänningen ain över
ingången blir förstärkt med faktorn A. När
som i bild 113-68 förstärkaren är emitterkopplad, blirutsignalen fasvriden 180$\circ$ i förhållande till insignalen. Fasvridningen a=180$\circ$ betecknas här med minustecken, alltså blir
förstärkningen -A.
På förstärkarens utgång fås en signalspä~ning out ~ed sambandet
Uut

=-A.

uin

En del av utsignalen återförs (återkopplas) till ingången. l en Meissner-oscillator
sker återkopplingen med en induktor, som är
induktivt kopplad till svängningskretsens induktor.
Kvoten k mellan den återkopplade signalspänningen ok och signalspänningen out
på förstärkarens utgång kallas återkopplingsfaktor. Den återkopplade spänningen
ok fasvrids så att den kommer "i fas med "
med insignalen. För den återkopplade signalen fås då sambandet

ok= -k. out

Tillräcklig signalspänning från utgången
måste återföras till ingången för att det skall
uppstå självsvängning. Det sker när den
återkopplade signalspänningen ak är minst
lika stor som ingångsspänningen Din och är i
rätt fasläge, d.v.s. i detta exempel
Ok~ Din
eller -k· Du!A eller k~ 1/A
Självsvängningsvillkoret blir
k~ 1l A eller k· A ~ 1
Ett k· A ::::: 3 är önskvärt för att oscillatorn
skall svänga igång snabbt.

Bild II 3-69 Svängningsvillkoret
113-52

Hartiey-koppling

Bild II 3-70
Återkopplingen sker galvaniskt över ett uttag
på induktorn i oscillatorns LC-krets.

Bild II 3-70 Hartiey-koppling
Huth-KO hn- eller TGTP-koppling (tuned grid
- tuned plate)

Bild II 3-71
Kopplingen är en förstärkare med LC-kretsar både på in- och utgång. Båda ~retsarna
är avstämda till samma frekvens. Aterkopplingen sker över de inre kapacitanserna
mellan elektronrörets elektroder resp. mellan transistorns materialskikt Denna koppling är av flera skäl inte särskilt vanlig.

{I];

----+---

Y. C>
Bild /13-71 TPTG-koppling
Golpitts-koppling

Bild 113-72
Återkopplingen sker över en kapacitiv spänningsdelare, som ingår som en del av oscillatorns LC-krets.

Bild II 3-72 Golpitts-koppling

Glapp-koppling

Bild 113-73
Denna koppling är en variant av Colpittskopplingen. Vridkondensatorn för frekvensinställningen är seriekopplad med spänningsdelarens kondensatorer. Glapp-oscillatorns
frekvensstabilitet är god.

Bild II 3-73a Glapp-koppling

Vi utvecklar denna beskrivning vidare.
Vridkondensatorn samt en fast och en trimningsbar kondensator är kopplade parallellt
med varandra. Alla tre kondensatorerna är i
sin tur seriekopplade med den kapacitiva
spänningsdelaren C3 C4 • Förstärkarens ingång är kopplad till den övre anslutningen av
C3 • Utgången från oscillatorns förstärkare
återkopplas över dämpresistorn Rct till mitten
av spänningsdelaren c3c4 (återkopplingskretsen).

..---+----+e v

Bild II 3-73b Förstärkare i Glappkoppling

Förstärkarens arbetspunkt bestäms av
spänningsdelaren R1 R2 • Ingen kopplingskondensator behövs eftersom det enbart
finns kondensatorer mellan förstärkaringång
och jord.
Kondensatorn C6 avkopplar kollektorn på
transistor T1 HF-mässigt till jord. Förstärkaren är alltså kollektorkopplad.

Kondensatorn C7 kopplar oscillatorns utsignal till buffertsteget För frekvensstabilitetens skull stabiliseras spänningen 8 V med
en lC-krets som avkopplas HF-mässigt med
en kondensator.

Frekvensinställning och bandspridning
Bild II 3-74
Att ställa in frekvensen i en Le-oscillator
gjordes förr oftast med en vridkondensator.
l moderna mottagare och sändare används
i stället en s.k. varicap, som styrs med en
likspänning.
Med en svängningskrets med endast en
induktor och en vridkondensator, skulle alla
amatörradiobanden endast vara smala områden utspridda på en mekanisk skala, d.v.s.
över vridkondensatorns hela kapacitansområde, varvid kapacitansen kan varieras med
förhållandet 1:5 a 1:1 O, tex. 10-50 pF a 10100 pF.
För att i stället få vart och ett av amatörradiobanden utspridda över större delen av
skalan kan man ordna med bandomkoppling
och s.k. bandspridning. Man parallellkopplar
då en relativt stor fast kapacitans med vridkondensatorns relativt lilla kapacitans. Den
totala kapacitansvariationen i LC-kretsen blir
då liten, trots att kondensatorns hela kapacitansområde utnyttjas. Resultatet blir en frekvensskala med större upplösning, d.v.s. bättre avläsningsnoggrannhet
Bandspridning kan också ordnas med
två seriekopplade kondensatorer, varav den
större görs variabel. Typiskt värde på vridkondensatorn i en kortvågsutrustning är då
100-500 pF och den fasta kondensatorn
mycket mindre än så.

nrno
Bild II 3-74 Bandspridning

113-53

113-54

ETSA
3.6 Kristalloscillatorer
Kvartskristaller i oscillatorkopplingar

En Le-oscillators frekvensstabilitet begränsas av de ingående komponenternas egenskaper. När mycket bättre stabiltet än så
krävs, speciellt inom stora temperaturområden, så är kvartskristallen en svängningskrets med bättre data. Kvartskristallens höga
Q-värde ger också en renare signal.
l en kristalloscillator (CO, Grystal Oscillator) är en kvartskristall det frekvensbestämmande elementet i stället för en LC-krets. l
övrigt kan samma kopplingsprinciper som
för en LC-VFO användas.
Kristallen kan utföras så att den svänger
antingen som en serie- eller parallellresonanskrets. Märk, att en kristall svänger på
något olika frekvens beroende på om den fås
att fungera som serie- eller parallellkrets.
Den högre frekvensen är den som vanligen
används.

Bild II 3-75
l parallellresonansalternativet kopplas
kristallen parallellt över oscillatorns återkopplingsled. Den minsta dämpningen av
den återkopplade signalen fås när signalens
frekvens är lika kristallens resonansfrekvens.
Kristallens reaktans är då som högst.
Parallellt över kristallens inre induktans
ligger dess inre seriekopplade kapacitanser
C och Cw Yttre kapacitanser (en trimbar och
två fasta kondensatorer i serie) är kopplade
parallellt över den inre anslutningskapacitansen Cw
Om den trimbara kapacitansen ändras,
så påverkas kristallens resonansfrekvens.
Man säger då att man "drar" kristallen inom
ett litet frekvensområde. Kristallens och oscillatorns egenskaper avgör hur stort området kan vara. Om kristaller dras för mycket,
så kan resonansfrekvensen bli ostabil.
Den relativa frekvensändringen uppgår
till högst 1o-4 = 0.01 $\circ$/o. Formel:

t, f

re a

IV

.. d .
absolut ändring
re vensan nng=
~ k
resonans1re vens

~ k

Övertonskristaller
Bild II 3-76
l serieresonansalternativet kopplas kristallen in i serie med oscillatorns återkopp!ingsled. Den minsta dämpningen av den
återkopplade utgångssignalen fås, när signalens frekvens är lika som kristallens resonanfrekvens. Kristallens reaktans är då som
lägst. S.k. övertonskristaller används för oscillatorfrekvenser över ca 20 MHz.

Bild II 3-75 Golpittsoscillator med kristall
i parallellresonansfallet

Bild II 3-76 Golpittsoscillator med kristall
i serieresonansfallet
113-55

KRETSAR

PT

Övertonskristallernas dimensioner är lika
grundtonskristallernas, men snittas ut annorlunda och slipas för att svänga på önskad
udda överton. En övertonskristall har övertonens frekvens instäm pi ad i höljet och kristallen förutsätts arbeta i oscillatorkopplingar
som seriekrets. Genom att låta kristaller
svänga på sin överton undviker man en svår
tillverkningsprocedur, nämligen att slipa
mycket tunna kristallskivor.
En övertonsoscillator måste alltid innehålla en svängningskrets som är avstämd till
den överton som anges på kristallen.
Modellförsök: En instrumentsträng sätts i
svängning på sin grundton genom en knäppning mitt på strängen. En knäppning på en
punkt bort från mitten får strängen att svänga
på en överton i stället.
Superheterodyn-VFO
Bild II 3-77
En enkel LC-VFO är inte tillräckligt frekvensstabil i ett högt frekvensläge, t.ex. 144-146
MHz. Man kan då använda en speciell koppling, som är en kombination av LC-VFO och
CO, kallad super-VFO.
l en super-VFO blandas en låg, variabel
frekvens från en VFO med en hög frekvens
från en CO. Ordet super kommer från
superheterodyne = överlagring, blandning.
En VFO arbetar stabilare på låg frekvens
medan en CO fortfarande arbetar stabilt
även på högre frekvenser, dock inte så högt
som vi behöver här. l vårt exempel arbetar
därför VFO i området 8-1 O MHz och CO på
17 MHz. VFO-signalen blandas med en fast
signalfrekvens, som är GO-signalen 17 MHz
multiplicerat med 8, d.v.s. 136 MHz.

VFO 8 -

10 MHz

co 17 MHz
Bild II 3-77 Superheterodyn- VFO

113-56

Ett bandpassfilter filtrerar fram den önskade blandningsprodukten, som ligger i
frekvensområdet 144-146 MHz. Resultatet
blir en hög frekvens, som är både variabel
och stabil.
Fördelar:
Frekvensstabiliteten hos en super-VFO
är mycket bättre än hos en enkel VFO, som
arbetardirekt i VHF-området. En super-VFO
är dessutom mycket brusfattigare än en
PLL-VFO, vilken beskrivs här nedan.
Nackdelar:
Vid frekvensblandning uppstår oönskade blandningsprodukter, vilka visserligen
dämpas av bandpassfilter, men som det är
omöjligt att undertryckta helt. Bl. a. alstras en
svag spegelfrekvens, som vandrar från 128
till 126 MHz, samtidigt som den önskade
blandningsprodukten vandrar från i 44 till
i 46 MHz. Risken för att spegelfrekvensen
förstärks och sänds ut måste elimineras,
vilket kan göras med effektiva bandpassfilter. Se vidare i avsnitt 3.8 om frekvensblandni ng.

3. 7

Oscillatorer med faslåsning - Pll

En kristalloscillator (CO) arbetar med god
frekvensstabilitet Frekvensen som är fast
bestäms av styrkristallen.
En LC-oscillator arbetar däremot inom
ett frekvensområde (VFO), som bestäms av
en LC-krets. Dennas frekvens är emellertid
mindre stabil än den med styrkristalL
l en PLL (Phase Locked Loop) kan god
frekvenstabilitet och stort frekvensområde
förenas. En PLL är en sluten krets för elektrisk styrning av en oscillator, så att dess
frekvens är både stabil och variabel.

Spänningsstyrd oscillator (VCO)
Bild 113-78
En VFO, vars frekvens kan styras med en
likspänning, kallas VCO (Voltage Controlied
Oscillator). l svängningskretsen i en VCO
fyller en kapacitansdiod (varicap, variable
capacitor) samma uppgift som den mekaniskt variabla kondensatorn i en VFO.
Bild 113-79
När en motriktad spänning läggs på dioden, så bildas ett spärrskikt i dioden, så att
zonerna med fria laddningsbärare isoleras
från varandra likt kondensatorplattor. Spärrskiktets tjocklek (c:a 1/1000 mm) beror av
spänningen över dioden. Vid hög spänning
är spärrskiktet tjockt, vilket motsvarar "stort
plattavstånd" och liten kapacitans. Vid låg
spänning är skiktet tunt, vilket motsvarar
"litet plattavstånd" och stor kapacitans.
Med en kapacitansdiod i svängningskretsen, i stället för en mekaniskt variabel kondensator, så behövs ytterligare två komponenter. D rossel n Dr hindrar högfrekvenssignalen att överlagras på styrkretsens likspänning. Då skulle bl.a. svängningskretsens godhetstal försämras (förlorad HF-energi innebär dämpning). Omvänt hindrar kondensatorn C att dioden och spärrspänningen kortsluts genom induktorn. Oscillatorfrekvensen ställs in med den variabla likspänningen U. Av en VFO har det blivit en VCO.

Bild II 3-78 VFO och VCO jämförs
Oscillator med PLL-styrning
Bild II 3-80
Människan jämför och reglerar förlopp utifrån givna fakta. Det kan liknas med PLLkretsens sätt att jämföra det inbördes fasläget mellan signalen från en VCO (är-värdet)
och signalen från en CO (bör-värdet).

......---:=::::;:::=! M .

. . Jiijo MHZJ/
Jamfora .
,
Låg spänning

stor kapacitans

Hög spänning
liten kapacitans

Varicap·
diod

Bild 113-79 Kapacitansdiod- Varicap

ata

:.~,:..c ~~go' O
Bild II 3-BOa Analogi

Människa-PLL

113-57

KRETSAR
Som resultat av jämförelsen justeras styrspänningen så att är- och bör-frekvenserna
hålls lika. En sådan reglerkrets består av
digitala komponenter.
Fasjämföraren levererar en cykliskt justerad styrspänning till kapacitansdioden i
VCO. Eftersom denna spänning ändras
språngvis, så avrundas förloppet så att frekvensändringarna blir mjuka. Avrundningen
sker med ett RC-filter där kondensatorn antar ett medelvärde av den pulserande utgångsspänningen frånjämföraren. Om VGOfrekvensen är för låg, så levererar jämföraren en positiv spänning. styrspänningen på
kapacitansdioden stiger då med en hastighet som bestäms av filtrets tidskonstant

Kapacitansen i kapacitansdioden minskar med ökande spänning, eftersom spärrskiktet blir tjockare och frekvensen på VCO
stiger.
När signalen från VCO åter är lik referenssignalen från CO, till fasläge och frekvens, så ökar utgångsresistansen i fasjämföraren. Lågpassfiltrets kondensator behåller
då sin laddning och styrspänningen till VCO
ändras inte t.v. Skulle frekvensen på VCO
vara för hög så blir jämförarens utgång
lågohmig och filtrets kondensator urladdas
med den hastighet som bestäms av tidskonstante n. Den sjunkande styrspänningen
medför att kapacitansdiodens spärrskikt blir
tunnare, kapacitansen tilltar och VCO-frekvensen sjunker tills en
ny fas- och frekvenslikhet uppnåtts.

~--------~--~----~

rv Utfrekvens

Lågpassfi !ter

fvco < fco

Referensfrekvens
(bör-värde)

fvco > fco

vco~~

co~~

VCO-si gnat och GO-signal, formad som kantvågor att jämföras i fasläge
Ua  

----.~

u
VCO-frekvensen för låg

VCO-frekvensen för hög

Bild 113-BOb Oscillator med PLL-styrning

113-58

PLL-oscillator i kombination med frekvensblandning
Bild II 3-81
Signalen f 1 från en VCO
alstrar en sändningsfrekvens i bandet 144146 MHz. Denna blandas med signalen f 2
(136 MHz), som är en
multiplicerad GO-frekvens. Blandningsprodukten f 1 - f2 filtreras
fram, d.v.s. en signal i
området 8-1 OMHz som
påförs en fasjämförare.
Utsignalen från en VFO,
som är variabel inom
samma frekvensområde 8-1 O MHz, påförs
också fasjämföraren.
Utsignalen från jämföraren är en likspänning, som beror av frekvensskillnaden mellan
blandningsprodukt och
VFO-signal. Jämförarens utsignal ändras
uppåt eller nedåt, beroende på frekvensfelets
riktning.

KRETSAR
V GO-frekvensen bestäms av en likspänningsnivå, som styrs av jämförarens utsignal. Vid varje frekvensändring i VCO, kommer systemet att sträva mot frekvensskillnaden noll i fasjämföraren vilket gör att sändningsfrekvensen hålls vid rätt värde.
Fördelar med en PLL-oscillator: Den har
samma frekvensstabilitet som en VFO eftersom denna även här arbetar på en låg frekvens. Till skillnad mot en super-VFO finns
inga sidafrekvenser i PLL-oscillatorn, eftersom VCO alstrar nyttafrekvensen direkt.
Nackdelar med en PLL-oscillator: Den har
högre brusnivå än en super-VFO. Frekvensstabiliteten är sämre än den för en PLLoscillator med CO och programmerbar
frekvensdelare.
PLL med programmerbar frekvensdelare
Bild II 3-82
Med PLL blir frekvensen på utsignalen från
en VCO låst till referensfrekvensen från en
CO. l princip fås en VCO med samma frekvensstabilitet som en CO, men också lika
svår att ändra frekvensen på. Med en frekvensdelare i fasregleringsslingan (PLL) kan
emellertid utfrekvensen ändras, medan CO
fortfarande avger samma referensfrekvens.
En frekvensdelare är en digital krets, som

räknar svängningar eller pulser upp till ett
valt tal för att återställas till i och börja om
igen. Vid varje återställning avges en utpuls.
Vid en delning med två avges en utpuls för
varannan inpuls. Vid delning med i 5 avges
en utpuls för var i 5:e in puls o.s.v.
Genom att välja delningstal i PLL kan
arbetsfrekvensen i VCO ställas in stegvis,
där varje steg är så stort som en referensfrekvens. signalfrekvensen från vco delas
med det valda delningstalet och resultatet
jämförs med referensfrekvensen från CO.
Varje avvikelse från likhet med referensfrekvensen kommer att medförajustering av
V GO-frekvensen.
Om man t. ex. vill täcka 2-metersbandet i
steg om 25 kHz, så väljer man den referensfrekvensen samt att delaren kan fås att dela
sändarens utfrekvens med vilket som helst
av talen 5760, 576i, 5762 o.s.v. upp till
5840. Om t.ex. delningstalet 5820 valts, så
kommer jämförarens styrspänning att styra
V GO-frekvensen till i 45500 kHz. Delarens
utfrekvens blir då i 45500/5820 = 25 kHz,
vilket motsvarar referensfrekvensen. l detta
exempel styrs alltså sändarens utfrekvens
så att den alltid blir i steg om 25kHz.

vco

144 - 146 MHz
.........
144 - 146 MHz
~~-------------------r--------------------Q

f avstämningsspänning

[Q

8 - 10 MHz
uppblandad VCO-frekvens

8- 10 MHz
referensfrekvens
LP-filter

fasjäm·
förare

d

~
VFO

Bild 113-81 PLL-oscillator kombinerad med frekvensblandning

113-59

KR
Men PLL-oscillatorn
brusar
förhållandevis
144 MHz till 146 MHz
starkt jämfört med en
VCO G !--------.---------·-------<>
VCO och speciellt jäm"""-'
rv 144 till 146 MHz
fört
med en CO. VCO}-----<>
svängningskretsen har
nämligen ett relativt lågt
Programmerbar delare i med tumhjul
n ( 5760 till 5840)
f eller annat
Avstämnings
godhetstal eftersom en
spänning
kapacitansdiod belastar kretsen mer än en
mekanisktvariabel kondensator.
Med det lägre god25kHz
hetstalet blir svängLågpass f i !ter
Referensfrekvens
ningskretsen ett mindre
bra filter för dämpning
Arbetsfrekvens Delningstal n Referensfrekvens
av oscillatorbruset Kapacitansdioden tillför
144 000 kHz
: 5760
= 25 kHz
: 5761
= 25 kHz
144 025 kHz
dessutom ett elektron145 500 kHz
: 5820
= 25 kHz
brus. Därtill kommerdet
146 000 kHz
: 5840
= 25 kHz
s.k. fasbruset från frekvensdelaren och PLL
VCO-frekvens
Med svängningskretsens låga godhetsVCO-frekvens,
tal är frekvensstabilite4-kantformad
ten i en VCO inte så bra
som den i en kristallosVCO-frekvens,
j 2-delad
cillator, utan faktiskt
sämre än den i en VFO.
Trots det är långtidsBild II 3-82 PLL med frekvensdelare
stabiliteten god i en
VCO, närden ingår i en
PLL, eftersom att frekvensen hålls ständigt
För- och nackdelar med PLL -oscillatorn
efterjusterad. PLL kan däremot inte åstadPLL-oscillatorn har nästan samma frekvenskomma en lika bra korttidsstabilitet Ett fasstabilitet som en kristalloscillator och frekvensen är inställbar i steg. Till skillnad mot jämförelseförlopp omfattar ju redan tiden för
en period av referensfrekvensen. Och det
en VFO med mekaniskt inställbar frekvens,
kommer att förflyta en multipel av denna
så är den PLL-styrda VGO-oscillatorns frekvens elektroniskt inställbar. Detta underlät- kortaste tid innan styrspänningen kan återställa VCO-frekvensen igen. Detta beror på
tar utformning och placering av reglage etc.
att kondensatorn i regleringsslingans lågför frekvensinställning, frekvensminne och
passfilter först måste laddas upp under ett
automatisk frekvensavsökning.
antal perioder innan reglering sker.
Först när den PLL-styrda oscillatorn kom
Dessa kortvariga frekvensavvikelser är
till användning i handapparater och mobila
en typ av frekvensmodulation som leder till
apparater blev det möjligt med frekvenstäckning över ett helt band med bibehållet fasbrus från PLL-oscillatorn och som kan
störa. Det är dock endast i extrema fall som
krav på små dimensioner. Som jämförelse,
skulle en inbyggnad av säg 80 till800 stycken fasbruset verkar störande eftersom det i
kanalkristaller i en traditionell kristallstyrd
moderna apparater reduceras till en accepapparat vara en mycket platskrävande, dyr- tabel nivå genom noggrann skärmning och
filtrering.
bar och opraktisk lösning.

1\ J\ 1\ ~1\ ;

r-uLflIlj
LJL 

113-60

Faktorer som påverkar frekvensstabilitet

Sändarens frekvens skall hållas så stabil
som möjligt. En ostabil sändare är inte godtagbar och skapar svårigheter inte bara för
de radiostationer som deltar i förbindelsen
utan även för radiotrafiken på närliggande
frekvenser.
En frekvensstabil oscillator skall ha följande
egenskaper:
stabil mekanisk uppbyggnad
Skakningar från underlaget t.ex. vid mobilt
bruk, vibrationer från en transformatorkärna
etc. kan försämra oscillatorns frekvensstabilitet
Frekvensbestämmande komponenter såsom fasta och variabla kondensatorer, spolar etc skall vara stabilt monterade, trimkärnorna i spolarna fixerade o.s.v.
Förbindningarna får inte tillåtas att böja
sig eller vibrera. Apparatstommen måste
vara tillräckligt styv för att inte ändra formen
och därigenom medföra frekvensändringar
vid hantering o.s.v.
God elektrisk uppbyggnad och högt Q-värde

i svängningskretsarna

Alla elektriska förbindningar måste vara så
korta som möjligt och löd- och kopplingsställen fullgoda. Induktorer och kondensatorer i
svängningskretsarna måste vara förlustfattiga och högvärdiga i övrigt så att signalen
blir så ren som möjligt från oönskade sidafrekvenser.
Återkopplingen i oscillatorn skall vara så
fast (kraftig) att självsvängningen är stabil.
Men för att få en renast möjlig signal får
kopplingen inte vara så fast, att svängningskretsarna blir alltför belastade och deras
godhetstal för lågt.
Avskärmande kapslingar
Svängningskretsar skall skärmas från yttre
kapacitanstillskott (t.ex.från en hand) Det
görs med skiljeväggar och komponentkapslingar av metall. Skärmningarna förhindrar också oönskad koppling mellan oscillatorn och efterföljande förstärkare genom elektriska och magnetiska fält.

stabila drivspänningar
Ostabila drivspänningar medför frekvensändringar. l en oscillator med transistorförstärkare beror ostabiliteten på förändringar
mellan skikten i en transistors diodsträcka.
Skikten fungerar nämligen som "kondensatorplattor" och spärrskiktet där emellan som
dielektrikum. Tjockleken av spärrskiktet och
därmed "plattavståndet" står i förhållande till
den spänning som läggs över transistorn.
Den spänningsberoende kapacitansen i transistorn är ansluten till svängningskretsen via
kopplingskondensatorn.
Eftersom kapacitansen i transistorn är en
del av svängningskretsen, så påverkar den
resonansfrekvensen. Denna egenskap kan
vara till besvär, men kan även användas för
att på ett enkelt sätt ändra oscillatorns arbetsfrekvens.
Se Kapacitansdiod och PLL-oscillatorn.
Buffertsteg
En oscillator i en radiosändare kan bestå av
ett enda förstärkarsteg, som alstrar högfrekventa elektriska svängningar. Vanligen tas
endast små effekter ut från en så enkel
sändare, normalt mindre än en watt. Utan
särskilda åtgärder, som t. ex. att använda en
styrkristall, är nämligen frekvensen inte särskilt stabil och olämplig för kommunikationsändamål.
Särskilt varierande belastning över oscillatorns utgång medför frekvensändring. Oscillatorn bör därför ges en så låg och stabil
belastning som möjligt. Ett buffertsteg kopplas därför in efter oscillatorn. Det bör ha hög
ingångsimpedans för att belasta oscillatorn
så lite som möjligt. Det skall också kunna
lämna tillräcklig driveffekt till efterföljande
förstärkare och bör därför ha låg utgångs impedans. Det måste dessutom arbeta linjärt
(se klass A-drift, bild 113-44) för att inte alstra
övertoner och därmed förvränga oscillatorsignalen. Bild II 3-42 visar ett buffertsteg i
kollektorkoppling, vilken har dessa egenskaper.

113-61

KRETSAR
Temperaturkompensation och termostater
Det alstras alltid förlustvärme i elektriska
apparater och även i en oscillator. Vid uppvärmningen utvidgas spolar och kondensatorer i svängningskretsarna, vilket leder till
frekvensändringar. Även spärrskiktskapacitansen i transistorerna är temperaturberoende. Det totala temperaturberoendet kan
kompenseras genom ett antal åtgärder.
Oscillatorn bör monteras så långt bort
som möjligtfrån övriga värmealstrande komponenter. Den avskärmande kapslingen omkring oscillatorn skall vara så tjockväggig
och värmeisolerande som möjligt. Inbyggnad i en termostatreglerad kapsling är ett
ännu bättre alternativ.
Komponenterna bör ha uppnått drifttemperaturför användningen. Oscillatorn bör
därför värmas upp under åtminstone 15 minuter.

Frekvensstabilitet och oscillatorbrus

Frekvensstabiliteten i kristalloscillatorer är
ca 100 gånger bättre än den är i LCoscillatorer. Likaså är utgångssignalen från
kristalloscillatorer renare från s.k. fasbrus
Gitter). Varje oscillator avger nämligen även
oönskade signaler med frekvenser som ligger omkring utgångssignalens nominella
frekvens.
Oscillatorn är ju en förstärkare, vars utgångsspänning delvis återkopplas till ingången i medtas. Detta innebär att utgångssignalen förstärks lavinartat till ett maximum, omväxlande med att den dämpas lavinartat till
ett minimum. Utan yttre påverkan befinner
sig alltså förstärkaren i ett självsvängningstillstånd mellan två yttervärde n. l återkopplingsvägen placeras ett filter som frekvensbestämmande element, t.ex. en LC-krets
eller en kvartskristall.
Återkopplingen blir starkast på filtrets
resonansfrekvens, vilket medför att oscillatorn svänger bäst där. Eftersom filtret oundvikligen har en viss bandbredd, så kommer
även ett spektrum av andra frekvenser tätt
omkring resonansfrekvensen att släppas
igenom. De oönskade frekvenserna omkring
den nominella kallas för brus.

113-62

l moderna konstruktioner används oftast
PLL-oscillatorer. På grund av sin funktion
pendlar deras frekvens alltid något. Hur
mycket beror bl.a. på loop-filtret. Alltså är
frekvensen egentligen ett mycket litet band
av flera frekvenser varav en framträder mest.
Försök:
Volymkontrollen i en lågfrekvensförstärkare
utan insignal vrids till maximum. Det kommer att höras ett brus i högtalaren, som
huvudsakligen kommer från ingångsstegets
transistorer. När en mikrofon ansluts måste
volymkontrollen vridas ner och då hörs bruset
mindre. Men bruset finns ändå där på en
lägre nivå och överlagras på insignalen från
mikrofonen.
Även i en högfrekvensoscillator överlagras bruset på insignalen. Men ju högre godhetstalet är i svängningskretsen, t.ex. en
kristall, desto smalare är filtrets bandbredd,
desto kraftigare blir brusundertryckningen
och desto merframhävs den önskade signalen. P.g.a. det större godhetstalet i svängningskretsen, och därmed den mindre bandbredden, så brusar alltså en kristalloscillator
mindre än en LC-oscillator.
En nackdel med kristalloscillatorn är att
dess frekvens inte kan ändras inom ettstörre
område. Önskas flera valbara frekvenser
från en kristalloscillator måste flera kristaller
användas tillsammans med något slags
omkopplingsanordning (kanalväljare).
Komponentmängden i en kristalloscillator
är mindre än i en VFO, men i apparater för
flera frekvenser uppvägs denna fördel av
merkostnaden för flera kristaller och kanalväljaren.
Kristalloscillatorn har många användningsområden, där en frekvensstabil och
brusfattig signal önskas och där platsbrist,
skakningar m.m. utesluter användning av en
LC-VFO.

ETSAR
3.8 Frekvensblandare

Grundprinciper

En anordning som blandar signaler för att
skapa andra kallas som namnet säger för
blandare. Blandare används både i mottagare och sändare och funktionsprinciperna
är lika i båda fallen. Vad som skiljer i stort är
hur de används.
Det finns många blandarkopplingar varav de vanligaste beskrivs här. Enkla typer
med vissa nackdelar ställs mot sådana som
är mer komplicerade, men har fördelar.

Bild II 3-83
När en linjär förstärkare matas med två
signaler så sammanlagras de. Den resulterande signalen vid varje tidpunkt är den
förstärkta summan av de inmatade signalerna.
När en olinjärförstärkare matas med två
signaler så blandas de med varandra. Förutom ingångssignalerna uppträder genom
blandningen ytterligare signaler på förstärkarutgången, så kallade blandningsprodukter.

ADDITION AV TVÅ SIGNALER
l ngångssignaler:
Växetspänningar med frekvensen f1

Utsigna!:

u

v.

u

och frekvensen f2

t U1 + U2

Frekvensspektrum

f,

v

l

f2

Endast ingångsfrekvenserna
uppträder i utsigna!en

BLANDNING AV TVÅ SIGNALER

u

u1~1ngångssignaler
fl

Utgångssignal

--·- -----··--.. ·-········,..-t

.....-

Frekvensspektrum

Det uppträder ytterligare frekvenser
utöver ingångsfrekvenserna

symbol för en blandare

f1~ Utgångbia f1, f2, f1
f2

t- f2, f2- f1

fz

>f1

Bild II 3-83 Principer för frekvensblandning

113-63

KR
Det finns ingen förstärkare i kopplingen.
signalspänningarna adderas genom att de
två transformatorernas sekundärlindningar
är seriekopplade. Dioden "förvränger" kraftigt summaspänningens kurvform. Beroende
av hur dioden är polariserad (vänd i kopplingen) blir den negativa eller den positiva
halwågen bortskuren.
Signalen på blandarens utgång, alltså
efter dioden, innehåller bl.a. frekvenserna f 1 ,
f2 , f2 +f1 , f2 -f1 • Den lägsta frekvensen f1 kan
lättast påvisas genom att ansluta ett lågpassfilter till blandarens utgång.

Två av blandningsprodukterna är särskilt
intressanta, det är summan och skillnaden
av ingångssignalernas frekvenser. De oönskade, övriga blandningsprodukterna filtreras bort med en avstämd krets eller ett
bandpassfilter.

Entaktsblandaren
Bild 113-84a
Vi kan övertyga oss om, att de fyra blandningsprodukterna verkligen uppstår. Först
undersöker vi den enklaste blandaren, som
är ett olinjärt element i form av en diod.

u1

l ngångssignaler:
f1

u

:J u, I~T
Ut+ U2

~J~ . .

Utgångssig nal

Ua

t

~~---t
u

(utan LP-filter)

l

U

Framtagning av lägsta frekvens fl

Bild II 3-84a Entaktsblandaren

113-64

Frekvensspektrum

L

b

f1

1 Lt.~

1 11
'.


1

KRETSAR

:J:

t

.
[) l "
N-r-,l· ·- ·r llillnnwnlliLI
n nn "

l

U

l

T.I} l"< lil~
lm~

=: - ---

-t

!

~~~~w~~

Frekvensspektrum

f2 tf 1

'---'-----'-....L--lll.-, f

(utan "'ngkmt.}

med "ängk""

-t

-t

Framtagning av frekvenserna f2, f1 + f2, f2 - f1

u

k~.
Al . -~ /'--- .-- /t+,r·~\ -· / /' . ,.
···'\ ::.

,

u

l

u

l

Signal med f2 - fl
(skilinadsfrekvens)

t

Signal med f2 + f1
(summafrekvims)

--~-

-::y-'(-

l

~-\---1--+-4-·-.~~~<}~ ~.-·.t···. ~-~---·-t ~~~na;;r :e11 f:d~r:~e
--+-- /

...

- 1T\--

UPtAJ\ f\ ~--t
u ,..--""

l

l
1

--,, .,

t--+---+-+-+--

'',..

-·

//

l
l·

-j:~·:"'-1---'

/

l

l

Signal med f2

/

:'.\, /

''< -..

-•·l

Alla tre signalerna adderade

l

Bild II 3-84b Entaktsblandaren

113-65

KRETSAR
Resultatet kan studeras med ett oscilloskop. Liksom på bilden ser man då att kondensatorn laddas upp till den positiva halvvågens toppvärde och med gott närmevärde
följer kurvformen på f 1 •
Bild II 3-84b
En svängningskrets med lämplig bandbredd, och som är avstämd till resonans
frekvensen f2 , ansluts nu till blandarens
gång. En signal med frekvensen f2 kan då
urskiljas och studeras i oscilloskopet. Svängningskretsen tillförs energi under de positiva
halvvågorna. Energin i svängningskretsen
kompletterar med den negativa halvvågen,
varvid en del av kretsens energi förbrukas.
Därför har de positiva och negativa halvvågorna inte samma amplitud (toppvärde).
Det syns i oscilloskopet hur amplituden
"svävar". Av detta dras slutsatsen att signalen består av fler frekvenser än f2 • Signalen
är sammansatt av f2 , f2 +f1 och f2 -f1 • Signalen
f1 ligger utanför svängningskretsens selektiva område och är därför bortfiltrerad (undertryckt). Blandningsprodukterna f2 +f 1 och f2 -f1
har båda en mindre amplitud än f2 •
Att det finns olika grundtoner och blandningsprodukter kan bevisas med en ännu
smalare svängningskrets med variabel frekvensavstämning, se bildens nedre del.
Vi har hittills utgått från entaktsblandaren.
Mer utvecklade blandartyper, såsom mottaktblandaren och ringblandaren, producerar färre blandningsprodukter.

Mottaktsblandaren
Bild II 3-85
Mottaktblandaren har två dioder, till skillnad
motentaktsblandarens enda diod ...................,,.,...
formatoremas ena lindning har mittuttag.
Ingången E1 ligger på den ena transformatorns primärlindning.lngången E2 1iggeröver
de båda mittuttagen. Utgången ligger på den
andra transformatorns sekundärlindning.
Ingången E1 matas med en signal med
en låg frekvens f. Eftersom en av de båda
dioderna alltid spärrar, så flyter det ingen
ström. De streckade pilarna visar endast i
vilken riktning strömmen kunde flyta, om de
spärrande dioderna vore öppna. Men så
länge som ingen signal ligger på ingång E2 ,
uppträder ingen signal på utgången.
113-66

Signalen på
avlägsnas och i stället
matas ingången
med en hög frekvens F.
Under den positiva halvvågen är de båda
dioderna öppna och genom båda flyter lika
stor ström. De båda transformatorernas
lindningshalvor genomflyts av lika ström i
motsatt riktning och då upphäver magnetfälten i lindningshalvorna varandraoch ingen
uppträder på utgången.
När signaler läggs
båda ingångarna
händer följande:
Dioderna öppnar och stänger i takt med
signalen på ingång E2 , med frekvensen F.
Den mycket svagare signalen på ingång E1 ,
med frekvensen f, kan alltefter polaritet passera diod D 1 eller D 2 • På återvägen överlagras signalen från E1 på signalen från E2 •
Strömmarna i lindningshalvorna är olika
stora. Då uppträder en signal på utgången.
Efter blandaren följer ett filter som endast
släpper igenom de önskade blandningsprodukterna F + f eller F - f.

Ringblandaren
Bild 113-86
Ringblandaren består av fyra dioder, som är
riktade åt samma håll i en "diodring".
ingången
matas med en signal med en
låg frekvens f. Till skillnad mot i mottaktsblandaren flyter en ström genom Di och D4
resp. D2 och 0 3 , men inte genom utgångstransformatorn. Ingen signal finns på utgången så länge som signalen F saknas.
Signalen på E 1 avlägsnas och i stället
matas ingången E2 med en hög frekvens F.
Till skillnad mot i mottaktsblandaren flyter en
ström genom dioderna Di och D2 resp. D3
och D4 och då upphäver magnetfälten i
transformatorernas lindningshalvor varandra. Ingen signal finns på utgången, så länge
som signalen f saknas.
När signaler läggs på båda ingångarna
händer följande:
De fyra dioderna kommer att öppna och
stänga parvis. Som i mottaktblandaren överlagras strömmen från ingång E1 på den
ström som dioderna öppnar för.
Här utnyttjas båda halvperioderna av F.
Strömmarna i lindningshalvorna blir olika
stora. På utgången uppträder då en signal.
Efter blandaren följer ett filter som släpper
igenom de önskade blandningsprodukterna.

KRETSAR
Bara signalen

e1

med frekvensen f ligger på
01

Bara signalen E2 med frekvensen f ligger på

båda dioderna öppnar

båda dioderna spärrar

Båda signalerna ligger på

Bild If 3-85 Mottaktsblandaren

113-67

R
Bara signalen E1 med frekvensen f ligger på

o,

Bara signalen E2 med frekvensen f ligger på

Båda signalerna ligger på

non no

lUlU

Bild II 3-86 Ringblandaren

113-68

KRETSAR

Signal utan svängkrets

Entaktsblandare
Frekvensspektrum

u

ut
nttöl\ 6tL.... n/\ nA~a

"

~t

~o~v~v~v~v~v~v~-~v~v\ v~v*v~v~o~v~~

li lL. . ~ ~~ ~ 11.
--
Signal med svängkrets

Signal utan svängkrets

Mottaktsblandare

u

.....

l

l..

+

LL

•

+

, 

f

Frekvensspektrum

u

eller

Signal utan svängkrets

Signal med svängkrets

;t;

il

M

Ringblandare

l

LL

M

-..
LL
M

ll

;;;

..

LL
M

... f

Frekvensspektrum

Bild 113-87 Jämförelse mellan olika blandare

113-69

KRETS R
Jämförelse av blandare

Bild 113-87
Bilden visar de tre beskrivna grundkopplingarna och de jämförs med avseende på frekvensspektrum på utgången.
Videntaktsblandaren uppträder summafrekvensen f+ F och skillnadsfrekvensen Ff, vidare ingångsfrekvenserna f och F, deras
övertoner 2f, 3f, 4f o s v, 2F, 3F, 4F o s v,
liksom deras blandningsprodukter F$\pm$ 2f, F$\pm$
3f o.s.v., 2F $\pm$f, 2F $\pm$ 2f, 2F $\pm$ 3f o.s.v.
Vid mottaktblandaren saknas frekvensen F och dess övertoner. Vidare bortfaller
de jämna övertonerna av frekvensen f.
Vid ringblandaren bortfaller ännu fler icke
önskvärda signaler, nämligen ingångssignalerna f och F och alla deras övertoner.
Endast blandningsprodukter av udda övertoner uppträder.
På bilden visas det fallet att frekvensen f
är mycket låg och då ligger blandningsprodukterna mycket nära varandra i frekvens.
Videntaktsblandaren filtrerar svängningskretsen ut frekvenserna F+ f, F- f, och F. Vid
mottakt- och ringblandaren saknas däremot
frekvensen F, den filtrerade signalen innehåller endast blandningsprodukterna F + f
och F- f. Om dessa båda blandningsprodukter är väl åtskilda eller svängningskretsen
har en bättre selektionsförmåga, då blir enbart summafrekvensen F + f eller skillnadsfrekvensen F- f framfiltrerad.
Vi har visat tre typer av blandare med
passiva komponenter. Sådana innehåller
olinjära dioder (germanium- eller kiseldiade r).
Det finns även blandare med aktiva komponenter, d.v.s. elektronrör eller transistorer
(bipolära, FET, MOSFET), men det skulle
leda för långt att gå in på alla olika lösningar.
l kapitlen 4. Mottagare och 5. Sändare beskrivs hur frekvensblandning används för
modulering och demodulering.

Icke önskade övertoner och blandningsprodukter

Varje olinjärt arbetande funktionssteg alstrar förutom nyttafrekvenser även icke önskade signaler med andra frekvenser. Både
önskade och icke önskade signaler kan bestå av övertoner eller blandningsprodukter
(skillnads- och summatoner) eller bådadera.
113-70

Vissa av signalerna filtreras fram för att
utgöra nyttosignaler. Andra signaler filtreras
bort, så att t. ex. utsändning inte sker på fel
frekvenser.
Bild 113-88
l ett tidigare avsnitt har vi beskrivit en s.k.
super-VFO. Vi skall nu undersöka vilka blandningsprodukter som uppstår i en sådan. De
två mest uppenbara frekvenserna är blandningsprodukterna (summan) i området 144146 MHz och (skillnaden) i området 128-126

MHz.

Ut från blandaren finner vi ingångsfrekvensen 136 MHz och dess övertoner 272
MHz, 408 MHz o.s.v. såväl som VFO-signalen och dess övertoner. På bilden är VFOfrekvensen 8 MHz och dess övertoner inritade, d.v.s. 16 MHz, 24 MHz, 32 MHz o.s.v.
Tyvärr bildar också de båda ingångssignalernas övertoner blandningsprodukter vilket bilden visar.
Bandpassfiltret släpper igenom nyttafrekvensen, men dämpar alla övertoner och
blandningsprodukter. Detta är enklare ju
längre ifrån nyttasignalen de icke önskade
signalerna ligger. l vårt exempel faller VFOsignalens övertoner inom bandpassfiltrets
passband på följande sätt:
15 · 9.6 = 144 MHz till15 · 9.733 = 146 MHz
16 · 9.0 = 144 MHz till16 · 9.125 = 146 MHz
17 · 8.471 =144 MHz till17· 8.588 = 146 MHz
18 · 8.0 = 144 MHz till18 · 8.111 = 146 MHz
Eftersom det här handlar om 15 :e-18 :e
övertonerna, så blir amplituderna så små, att
vi kan bortse från dem.
Det är viktigt med goda filter i signalbehandlande funktionssteg. En god regel är att
på ett tidigt stadium filtrera bort oönskade
övertoner och blandningsprodukter-helst i
varje steg -så att onödigt komplexa signaler
undviks. Det är också viktigt med frekvensvalet, så att oönskade blandningsprodukter
kommer så långt bort från nyttafrekvensen
som möjligt, liksom att endast mycket höga
övertoner med motsvarande små amplituder faller inom det nyttiga frekvensområdet

ETSAR

Super-VFO för 144-146 MHz

BLANDARUTGÄNGENs FREKVENsSPEKTRUM

U

f 1 =8MHz
passbandkurva
sidfrekvens

nyttasignal

f,df.
144 MHz

f2 -· f1
128 MHz

a) Förenklad framställning med VFO:n avstämd till 8 MHz

u

9,5 MHz

136MHz

126,5 MHz·--·----1 ffi--·-,.145,5 MHz

........... . .L111.

b) Förenklad framställning med VFO:n avstämd på 9,5 MHz

u

8MHz
16 MHz
24 MHz

l / )2

MHz osv

136 MHz
144 MHz

128 MHz

\

ll

1--11--"--..--A--.li--.S..-"-.L..-JL.........L

.....

27 2 MHz o s v

c) Ingångssignalens övertoner med VFO:n avstämd på 8 MHz

u

136

BMHz

16MHz
/24 MHz

/32MHz

l

osv

(136-BH1flz
( 136-16} M~z\

(136-2'dM!1zL

~1Hz

/

(136•8lt11fz
( 136+ 16) MHz

/(136+24lMHz

l

osv

( 2'72--B l ~1H7
(272+16lMHz\
\

272MHz
( 272+ Bl MHz

j(272+16lMHz

l

osv

l.LL--L---   .~~ aI...Jl,I......JII.-.L..-A(..I...JLI-'''---- f

d Blandningsprodukter från blandning av övertoner

Bild II 3-88 Frekvensspektrum från en
113-71

KRETS R

113-72

3.9 Modulatorer
Allmänt

När en signal (bärvågen) påverkas så att
den överför informationen i en annan signal,
sägs bärvågen bli modulerad. Detta förlopp
kallas modulation. Vad som då händer behandlas främst i kapitel 1, avsnitt 1.8, med
tillämpningar i kap. 5 och delvis i kap. 4 .
En anordning för modulation kallas för
modulator. En modulator kan ingå som en
funktion i sändare, mottagare m.fl. system.
Beroende på modulationsmetoden används
olika kombinationer av delkretsar som tillsammans utgör modulatorn.
l detta avsnitt ges exempel på några
vanliga modulatorer för sändare.

Amplitudmodulatorer

Med en amplitudmodulator påverkas bärvågens amplitud proportionellt med den modulerande signalens amplitud.
Vid sändningsslaget A 1A är amplituden
på den modulerande signalen antingen maximal eller ingen. Då består modulatorn av en
nycklingskrets, som påverkar t. ex. ett drivsteg i sändaren så att bärvågen släpps fram
helt eller inte alls.
Vid sändningsslaget A3E har den modulerande signalens amplitud ett analogt förlopp, t.ex. tal, med vilket bärvågens amplitud moduleras. Här beskrivs amplitudmodulation i en förstärkare med ett katodkopplat
elektronrör. En emitterkopplad transistorförstärkare kan moduleras på ett liknande
sätt. l båda fallen moduleras förstärkarens
arbetsspänning (anodspänning resp. kollektorspänning) med den modulerande signalens. Det som då händer är att två signaler
blandas på ett sätt som beskrivs i avsnitt 1 .8
med tillämpning påA3E.I vila är då bärvågsamplituden halva den möjliga inom arbetskurvans linjära del. Vid modulation kommer
bärvågens amplitud att variera mellan noll
och den möjliga amplituden.
Bild II 3-89
Bilden visar ett sändars lutsteg med en triod.
l serie med tilledningen för anodspänningen
finns sekundärlindningen av en modulationstransformator för LF-signalen.

Den LF-förstärkare som driver transformatorn måste för 100 $\circ$/o modulationsgrad
kunna avge bärvågens halva effekt. Eftersom uteffekten från en fullt utmodulerad
A3E-sändare är 150 $\circ$/o av den i vila, måste
slutsteget dimensioneras därefter. Utöver
den egna signalspänningen måste modulationstransformatorn även klara slutstegets
arbetsspänn ing.
Om som på bilden anodspänningen i ett
förstärkarsteg amplitudmoduleras, kan förstårkarsteget arbeta olinjärt, t.ex i klass C.
Varje följande förstärkarsteg måste däremot
arbeta linjärt, t.ex. i klass A.
På grund av den låga verkningsgraden
och det stora bandbreddsbehovet, används
i dagens amatörradiosändare knappast
"äkta" amplitudmodulering, d.v.s. A3E.I stället används i läget "AM" nästan alltid H3E,
d.v.s. enkelt sidband med full eller reducerad bärvåg (se nästa stycke). Trots det lägre
effektbehovet p.g.a. endast ett sidband och
ev. reducerad bärvågsamplitud kan av dimensioneringsskäl ändå inte de flesta H3Esändare avge sin fulla effekt kontinuerligt!
Som redan sagts i avsnitt 1.8, är det onödigt
sända ut två sidband, eftersom båda innehåller samma information. Det räcker med
ett sidband. Bärvågen innehåller inte någon
information. Den kan därför undertryckas
redan i sändaren för att ersättas i mottagaren. Därmed uppstår sändningsslaget J3E.

till
slutsleg

~--o+Ua

la

---1

lllg reaklans
för tonfrekvens

Dr

Bild II 3-89 A3E-modulator

ua

113-73

KR

I

--l
t

DSB-Signal

u

~6ft,, ~ftn,Molftftn..

Vllt! Wjj y VUllll" "t{lfl[V vIl"

/

•

t

Alstring av en DSB-signal (amplitudmodulering, dubbelt sidband med undertryckt bärvåg
i en ringblandare (balanserad modulator)
/

/u~
b''""'"d

modulew

~-----l

~/ ~SBSSBkristallfilter

Ul

lfr

'

fT ·fLF

:

fT · fLF

r---lL. 1

f

Framfiltrering av övre sidbandet

Bild II 3-90 Alstring av J3E (SSB)

113-74

/

DSBSi nal~Signa~

t

KRETSAR
Vid sändningsslaget J3E (SSB) sänds
således endast ett sidband. Det andra sidbandet och bärvågen undertrycks, vilket kan
göras på flera sätt. Numera är den s.k.
filtermetoden allra vanligast och den enda
som behandlas här.
Bild II 3-90
Med filtermetoden blandas HF- och LFsignalerna i en balanserad blandare där de
undertrycks medan blandningsprodukterna
med deras summa- och skillnadsfrekvenser
blir kvar, d.v.s. det övre och nedre sid bandet.
För att undertrycka det ena sidbandet
före sändningen, följs blandaren av ett bandpassfilter med bandbredd och frekvensläge
för avsett sidband. Den signal som sänds ut
innehåller på så sätt endast ett sidband
(Single Side Band).
Valet mellan USB och LSB kan göras på
två sätt. Antingen genom att välja mellan ett
separat passbandfilter för respektive sidband eller genom att använda ett enda filter
och flytta H F-signalen från ena sidan till den
andra av det filtret (se bild Il 1-28 i avsnitt
1.8).
En J3E-modulator enligt filtermetoden
består således av en balanserad blandareofta en s.k. ringblandare (se bild Il 3-87 i
avsnitt 3.8) samt ett bandpassfilter.
För att SSB-signalen skall få avsedd sändarfrekvens kan ytterligare frekvensblandning behövas (se kapitel 5).

Vinkelmodulation
Vinkelmodulation är samlingsnamnet för frekvensmodulation (FM) och fasmodulation
(PM).
Frekvensmodulation
Vid sändningsslaget F3E (även kallat FM)
varierar bärvågens frekvens i takt med den
modulerande signalens amplitud. Bärvågen
kommer på så sätt att pendla omkring en
nominell frekvens, d.v.s. vara frekvensmodulerad. Bärvågsamplituden ändras däremot
inte vid frekvensmodulation.
Likspänningsnivåer kan således överföras eftersom en frekvensawikelse (deviation) i bärvågen endast påverkas av den
modulerande signalens amplitud.
Vid F3E påverkas resonansfrekvensen i
den svängningskrets i oscillatorn som bestämmer dess arbetsfrekvens. Det görs enklast genom att tillföra en kondensator med
variabelt kapacitansvärde, en s.k. varicap
(se avsnitt 2.5).
Bild II 3-91
Bilden visar en LC-svängningskrets där
det ingår en varicapsom styrs av en likspänning med en överlagrad modulerande LFsignal. En likspänning tjänar som en ställbar
förspänning till varicap. På så sätt kan man
påverka arbetsfrekvensen. Med den överlagrade LF-signalen påverkas arbetsfrekvensen i takt med signalamplituden.

till oscillatorn

Alstring av FM:
Oscillatorn moduleras

Bild II 3-91 Alstring av F3E (FM)

113-75

Fasmodulation
Vid sändningsslaget G3E (även kallat PM)
varierar bärvågens fasläge i förhållande ,till

en referens, som är en omodulerad bärvåg.
Bärvågens amplitud ändras däremot inte.
Fasändringen -deviationen -är direkt proportionell till hur snabbt fasläget ändras och
till den totala fasändringen. Hastigheten på
fasändringen är direkt proportionell till frekvensen på den modulerande signalen och till
dess amplitud.
Det betyder att deviationen vid fasmodulation ökar både med amplituden och frekvensen på den modulerande signalen. Ändringar i likspänningsnivåer kan därför överföras endast om en fasreferens används.
Fasmodulation kan alstras t.ex. genom
att påverka resonansfrekvensen i en svängningskrets någonstans efteroscillatorn, d.v.s.
där oscillatorfrekvensen inte påverkas. Denna svängningskrets har i viloläge samma
resonansfrekvens som oscillatorn. När kretsen bringas ur resonans genom modulation
-samtidigt som kretsen påtrycks oscillatorsignalen -så uppstår i kretsen omväxlande
en induktiv och kapacitiv reaktans - detta
inom tiden förvarje halv period. Reaktansen
skapar därvid den fasförskjutning som innebär fasmodulation. Se även avsnitt 3.1, bilderna Il 3-18 och -19
Bild 113-92
Liksom vid frekvensmodulation kan t. ex.
en varicapanvändas för att med en modulerande signal påverka resonansfrekvensen i
en krets.

Från oscillatorn

J-l

~---

Alstring av PM:
Kretsens reaktans
moduleras

Bild 113-92

113-76

avG3E (PM)

~~?~:' ----~
fYYY"\r -J:

ID~- l~ ö, , , ~,"'

rrC>--:LJ stamnrngsspannrngen
,.

~ ~-J


