\chapter{MÄTNING}

I forskning, utveckling och produktion är mätning en hörnpelare i verksamheten. Även
inom mättekniken sker en snabb utveckling och digitaltekniken kommer alltmer till
användning, men grunderna för mätning är desamma. I detta kapitel behandlas de viktigaste
mättekniska begreppen som radioamatörer kan behöva känna till.

\section{Att mäta}

Mäta likspänning
Vid spänningsmätning bestämmer man potentialskillnaden -spänningen - mellan två
punkter. Om det finns en spänning, så flyter
en motsvarande (mät)ström genom instrumentetinstrumentet presenterar mätströmmen som spänning.
Mätströmmen påverkar emellertid spänningsfördelningen i kretsen och då uppstår
ett mätfel, vilket inte framgår av det visade
mätvärdet. Med kännedom om kretsens och
instrumentets data kan man dock beräkna
mätfelet En voltmeter skall ha hög inre
resistans för att mätfelet skall bli litet.
Endast vid mycket noggrann mätning
kan man behöva räkna om det visade mätvärdet med hänsyn till voltmeterns inre resistans och förkopplingsresistansen - om
en sådan används).
På grund av den höga inre resistansen är
en voltmeter endast lämpad för spänningsmätning -INTE för direkt strömmätning f
Utöka mätområdet för en voltmeter
Bild II 3-1
Med hjälp av förkopplingsresistor i serie
med voltmetern kan man mäta högre spänning än den som voltmetern är gjord för.
Spänningen fördelas då proportionellt mellan förkopplingsresistorns resistans och instrumentets inre resistans.
När förkopplingsresistor används måste
mätvärdet räknas om med en skalfaktor eller
en skala med motsvarande gradering användas. En voltmeter med valbar förkopplingsresistor kan därför har flera skalor. l
digitala voltmetrar anpassas "skalan" oftast
automatiskt.

Mäta likström
Vid strömmätning bestämmer man strömstyrkan i en gren av en elektriskt strömkrets.
Amperemetern skall kopplas i serie med den
aktuella strömgrenen. Det visade mätvärdet
motsvarar strömstyrkan. Amperemeterns
inre resistans adderas emellertid till resistansen i strömgrenen och då uppstår ett
mätfeL En amperemeter skall ha låg inre
resistans för att mätfelet skall bli litet.
Endast vid mycket noggrann mätning kan
man behöva räkna om det visade mätvärdet
med hänsyn till amperemeterns inre resistans och resistansen i strömshunten-om en
sådan används.
På grund av den låga inre resistansen
skall en amperemeter ALDRIG användas för
spänningsmätning. Då förstörs den!
Utöka mätområdet för en amperemeter
Bild Ii 3-2
Med en strömshunt (en resister parallellt)
över amperemetern kan man mäta högre
ström än den som amperemetern är gjord
för.
Shunten dimensioneras så att större delen av strömmen leds förbi amperemetern.
Kvar är den mätström som behövs för att
amperemetern skall göra fullt utslag.
Mätströmmen fördelar sig omvänt proportionellt till instrumentets och shuntens
resistanser.
När en strömshunt används måste mätvärdet räknas om med en skalfaktor eller en
skala med motsvarande gradering användas. En amperemeter med valbar shuntresister kan därför har flera skalor. l digitala
amperemetrar anpassas "skalan" oftast automatiskt.

Mäta växelspänning och växelström
Grunderna för mätning av växelspänning
och växelström är samma som för likspänning och likström, men att bl.a. en instrumentlikriktare oftast behövs.
Beroende på frekvensen i strömkretsen
och vilket slags värde man vill mäta, används olika instrument.

118-1

ÄTNIN
Olika typer av instrument ger olika
ter, men också begränsningar.
Mjukjärnsinstrument utan ii!rl"i!r+.-,,..,.... kan
mäta växelströmmar ner till c:a 50 mA och
upp till c:a1 O A. Frekvensen får dock inte
vara högre än c:a 100 Hz.
Vridspoleinstrument används dels direkt
för likströmsmätning och dels med likriktare
även för växelströmsmätning.
Vridspoleinstrument med likriktare används ofta för frekvenser upp till c:a i O kHz
och strömmar nertill 0.1 mA. Noggrannheten
är sällan bättre än 1.5\(\circ\)/o av fullt utslag.
Beroende på funktionsprincipen kan det
skilja på hur instrument mäter, vilket nödvändigtvis inte är detsamma som hur mätvärdet
presenteras.
Mjukjärnsinstrument mäter effektiwärdet
av en växelström medan ett vridspoleinstrument med likriktare mäter likriktade medelvärdet. Som exempel kan skalan i ett instrument med likriktare även graderas för effektiwärdet för sinusformade förlopp.
För mätning av växelström används vanligen instrument med likriktare, men för HF
även instrument med termokors, vilka bygger på termogalvanisk spänning mellan metaller.

Frekvensens inverkan
Frekvensen på den mätta signalen inverkar
mer eller mindre på mätresultatet. Till en del
beror det på den instrumenttyp, som används. En faktor är instrumentets gränsfrekvens, d.v.s. hur högt i frekvens som instrumentet fortfarande är rimligt rättvisande. Detta
kallas instrumentets bandbredd, vilken bör
vara dokumenterad.
Vågformens inverkan
Även formen på den signal som mäts inverkar på mätresultatet och det är viktigt att veta
för vilken vågform som instrumentet presenterar mätvärdet. Det vanligaste är att vågen
förutsätts vara sinusformad, vilket ofta inte är
fallet i praktiken. Det innebär att fel värde
presenteras om vågformen är en annan än
den förutsatta.

Mäta resistans

Mätning av resistans är enklast att göra på en
fristående komponent, medan man vid mät-

118-2

ning
en resister i en strömkrets också
måste ta hänsyn till att andra komponenter i
kretsen kan påverka mätresultatet.
Resistans kan mätas på flera sätt. Det
grundläggande är att mäta strömmen genom resistorn och spänningen över den och
sedan beräkna resistansen med Ohms lag.
Därutöverfinns direktvisande instrument
för resistansmätning- s.k. ohm-metrar Sådana instrument innehållervanligen en egen
strömkälla i form av ett batteri.
1-tt,::.trttnl"'n"!lal"' vid likväxelström
(medel-, effektiv- och toppvärden)

Vid likström:

P= U ·l [W] (watt),
d.v.s. Joules lag

Vid sinusformad växelström:
medelvärde

p
 O.B·Ue/
medelR

effektiwärde

p

toppvärde

P,

Såndareffekt

eff

= ueff2
R

umax2
PEP=~

En sändares effekt kan mätas på olika sätt.
Förr, då radioamatören hade små möjligheter att mäta uteffekt, så var det naturligt att
föreskrifterna angav en mätmetod baserad
på ineffekt, vilket var enkelt och rättvisande
förtelegrafi m.fl. sändningsslag med bärvåg.
Även för SSB fick effektmätning göras så,
trots att resultatet var långt ifrån rättvisande.
Numera är instrument som mäter uteffekt
mer tillgängliga för radioamatören - även
toppvärdeskännande sådana för mätning av
p.e.p. Mot den bakgrunden anges nu (i 997)
i föreskrifterna såndareffekten som uteffekt.
Därvid måste även p.e.p. avses, fastän det
inte uttryckligen uttalas.
För N-licensen föreskrivs fortfarande att
uteffekten uttrycks i e.r.p. N-amatören kan
dock endast i undantagsfall mäta e.r.p. utan
den måste beräknas. Bakgrunden till detta
unika myndighetsbeslut var att se som en
anpassning till gällande regler för begränsning av radiostrålning.
Observera, att radioamatören måste beakta EMC-Iagen. Se vidare kapitel 9.

TNING
Metoder för mätning av sändareffekt

Tidigare har avhandlats effektberäkning i
allmänhet. Här nedan kommenteras mätning av såndareffekt i synnerhet.
Ett tillförlitligt sätt att mäta såndareffekt är
att ansluta sändaren till en konstlast med
samma resistans som sändarens utgångsimpedans och mäta spänningen över lasten
med ett osci!loscop med tillräcklig bandbredd. Då kan man se och mäta HF-spänningens topp-toppvärde och samtidigt se
signalens vågform.
Med spänningen och konstlastens impedans (resistans) bekanta så kan uteffekten
beräknas enligt formlerna på förra sidan
Den största HF-amplitud som uppstår
momentant vid modulering motsvarar PEPeffekten (PEP= Peak Enveiope Power).
En mindre exakt metod att mäta HFspänning är med voltmeter med likriktare.
Utifrån den uppmätta spänningen kan man
beräkna effekten över en belastning. På
grund av instrumentets tröghet visas emellertid bara ett "utjämnat" toppvärde, vilket
inte är det faktiska värde som instrumentet
"känner". Jämför med oscilloscopet som inte
har denna visningströg het.
Bild II 8-1
Bilden visar en voltmeter med likriktare,
som kopplats till en sändare över spänningsdelare. Två alternativa delare visas; den ena
består av resistorer och den andra av kondensatorer.
Den resistiva delaren är bättre i den meningen att den är frekvensoberoende och
inte belastar sändaren kapacitivt. Dessutom
dämpas övertoner som bildas vid likriktningen. l den kapacitiva delaren kan övertoner
passera lättare.
Denna mätmetod är noggrann bara när
impedansen är lika i sändaren, kabeln till
lasten och själva lasten. Lasten kan vara en
konstlast, en antenn etc och skall ha ett känt
värde för att effekten skall kunna beräknas.
Ett sätt att skaffa underlag för beräkning
av PEP-effekten är att mäta HF-strömmen
med ett termokorsinstrument och spänningen med en toppvärdesvisande voltmeter.
Utifrån dessa värden beräknar man effekten. Denna metod är dock inte så vanlig.

Direktvisande effektmetrar

Bild 118-6
Många föredrar direktvisande effektmätare.
En HF-voltmeter kan givetvis graderas för
att visa effekt i stället för spänning, men då
med den viktiga förutsättningen att impedansen måste ha en fastställt värde.
Om man avläser effekten genom en 75
Q-kabel på ett instrument för 50 Q, så är det
verkliga värdet ett annat än den avlästa.
De effektmetrar som förekommer i SVFinstrument är egentligen voltmetrar, men
med skalan graderad i effekt.

~Last
Sändare~

l

r~n
I
v

Bild II 8-1 Mätning av sändareffekt

118-3

ÄTNIN
Mäta ståendevågförhållande - SVF
När t. ex. en antennledning ansluts till en
antenn och deras impedanser inte är lika, så
kommer en del av inmatade effekten i ledningen att reflekteras tillbaka från antennen.
Det uppstår då en stående våg i ledningen. Förhållandet mellan inmatad och
reflekterad effekt uttrycks som ett ståendevågförhållande SVF (eng. SWR).
Med en SVF-meter som sätts in mellan
effektkälla och ledning kan man mäta hur
stor effekt som matas in i ledningen och hur
stor effekt som vänder tillbaka från slutet av
ledningen.
SVF-värdet kan bestämmas på något av
följande sätt:
• Man mäter framåt- respektive bakåtgående effekt var för sig med en riktningskänslig effektmete r. Man beräknar därefter SVF eller tar fram det ur ett diagram.
• Man använder ett instrument som beräknar eller visar SVF på något sätt.
studera vågformen
Vågformen för snabba växelströmsförlopp
studeras bäst med oscilloskop.
Mäta frekvens
Frekvensmätning gör man bäst med en s.k.
frekvensräknare, som är ett digitalt instrument. Man kan också använda en s.k. absorbtionsvågmeter, som är mycket enkel
och inte alls så exakt. Vid frekvensmätning
ansluter man instrumentet till mätobjektet
med en svag elektrisk eller magnetisk koppling.
Mäta resonansfrekvens
Mäta resonansfrekvensen för en passiv
svängningskrets gör man enklast med en
s.k. dip-meter. Även mer exakta metoder
finns.

118-4

Mätfel
Mätinstrument indelas i noggrannhetsklasser
efter största tillåtna felvisning. Klasserna är
0.1, 0.2, 0.5, 1.0, i .5, 2.5 och 5.0 varvid
klassen anges på instrumentet. Som exempel får ett instrument i klass 2.5 ha ett tillåtet
mätfel av\(\pm\) 2.5 o/o av fullt utslag.
Mätresultatet bestäms av flera faktorer;
dels av instrumentets s.k.mätonoggrannhet,
dels av hur mätvärdet presenteras och slutligen av hur noga användaren läser av.
Vid analog visning presenteras mätvärdet med en visare mot en graderad skala
med en viss upplösning. Visaren kan vara
mekanisk eller optisk (ljusspalt). Vid snabba
mätvärdesändringar är instrumentets mekaniska tröghet en faktor att ta hänsyn till.
Vid digital visning presenteras mätvärdet
med siffror eller som längden på en pelare.
Det är förledande att se digital visning med
siffror som mer exakt än analog, men det är
inte alls säkert. Utöver instrumentets mätonoggrannhet, bestäms nämligen noggrannheten av hur många siffror som mätresultatet presenteras i.
En oberäknelig källa till mätfel är elektromagnetiska fält från apparater i närheten.
En ofta förbisedd felkälla är temperaturen i mätobjektet och/eller i instrumentet, det
kan vara av inkopplingstiden m.m ..
Visningströgheten är inget mätfel i sig
men kan till nackdel vid snabba förlopp.
Trögheten förekommersåväl vid analog som
digital visning. l det förra fallet är masströghet i instrumentets rörliga delar orsaken och
i det andra fallet är orsaken klackfrekvensen
för instrumentets mikroprocessor.

ÄTNIN
8.2 Mätinstrument

Presentation av mätvärden

Bild 118-2
Mätvärden kan presenteras på olika sätt. De
vanligaste sätten är optiska och då med
digital eller analog visning. Mätresultat kan
även överföras till datorerförvidare bearbetning och visning.

Multimeter

Bild II 8-2
Flera mätfunktioner kan utföras med samma
basinstrument Genom omkoppling mellan
olika tillsatser väljer man mätfunktion och
mätområde. Instrumentskalan utformas så
att olika slags mätvärden kan avläsas.
Kombinationer med elektroniska förstärkare
och digital visning etc. är nu vanligt.

Vridspoleinstrument

0

v·6"·n

.... "

A

-

O

+

MULTIMETER O

·D·.

.D:
Trigger

@

AY

in

Instrument med analog visning

A

.:Q:. Jololli3J.Jsl
DIGITALMETER

0

Bild II 8-3
Vridspoleinstrument kan bara användas för
likströmsmätning, eftersom visarutslaget
beror av strömriktningen. Instrumentet har
låg effektförbrukning och stor noggrannhet.
Visningen är vanligen linjär, men kan göras
annorlunda.
Funktion: En spole är upplagrad i fältet av
en hästskomagnet När den ström, som skall
mätas, passerar genom den vridbara spolen
så alstras ett magnetfält även i denna. De två
magnetfälten påverkarvarandra så att spolen
vrider sig. Spolen förses med en visare och
en returfjäder. Ju större ström det flyter genom spolen desto större blir visarutslaget

Volt
Ampere
Ohm

Instrument med digital visning

Bild II 8-2 Presentation av mätvärden

MODELL

----

Hästskomagnet

Vridspole
Visare\

Bild II 8-3 Vridspoleinstrument

118-5

MÄTNING

Spole

MODELL

Bild II 8-4 Mjukjärnsinstrument

Mjukjärnsinstrument

Bild II 8-4
Mjukjärnsinstrument kan användas för mätning av såväl lik- som växelström. Vid växelströmsmätning kommer effektiwärdet att
visas, oavsett strömmens kurvform.
Detta instrument har en relativt dålig precision, men är användbart för enklare ändamål. Det ersätts dock efter hand med billiga
digital instrument.
Funktion: l ett mjukjärnsinstrument sitter
två järnstycken B placerade. Det ena järnstycket sitter vridbart upphängt och är försett
med en returfjäder och en visare. Det andra
järnstycket sitterfast upphängt. Järnstyckena
magnetiseras av fältet i spolen. P.g .a. polariseringen kommer de alltid att stöta bort
varandra, oavsett strömriktningen i spolen.
Bortstötningskraften är ej proportionell mot
strömmen och skalan blir således olinjär.

Konstlast

Bild II 8-5
En konstlast (dummy load) bör ingå i varje
amatörradiostation. Vid mätning och inställning av t. ex. modulation och uteffekt, är det
lämpligt att belasta sändaren med dess nominella utgångsimpedans. För att då undvika att energi strålas ut bör en väl skärmad
konstlast användas.
l moderna amatörradiosändare med koaxialkabel utgång är utgångsimpedansen 50
n. Konstlasten skall då vara en 50 n res istor
utan reaktiva egenskaper. Den kan bestå av
en eller flera sammankopplade resistorer.
Sändareffekten skall kunna tas upp utan
att resistansen förändras nämnvärt. Det är
viktigt att resistorerna kyls effektivt med luft
eller vätska i ett kär! med tillräckligt utrymme, även när vätskan expanderar av värmen. Vätskan får inte vara lättantändlig eller
miljöfarlig. T. ex. är oljormed PCBförbjudna!

S!
r---------,

L  jO.Q.Sl

Bild fl 8-5 Konstlast

118-6

--j

IN

TX
Bild II 8-6 Fältstyrkemätare

Fältstyrkemätare
Bild II 8-6

styrkan av elektromagnetiska fält kan bestämmas med fältstyrkemätare.
En fältstyrkemätare är en högfrekvensdetektor, vars utspänning visas med ett instrument med skala. Den selektiva
kretsen kan bestå enbart av den ....,,u··+r>...,"'r~ ....
antennen, men även av ytterligare selektiva
kretsar. Instrumentet visar endast relativa
värden och används t. ex. för att bestämma
strålningsegenskaperna i sändarantenner
och för antennjustering. Mätresultatet påverkas även av utstrålning från andra sändare inom mätarens bandbredd. Bilden visar
en sändare och en fältstyrkemätare. Dessutom två enkla fältstyrkemätare.

Kalibreringsoscillator
Bild II 8-7

En kalibreringsoscillator används för attfrekvenskalibrera andra apparaters inställningsskalor. Den är kristallstyrd och avger särskilt
precisa och frekvensstabila signaler.
Oscillatorsignalen förvrängs avsiktligt, så
att det utöver grundfrekvensen även skapas
harmoniska övertoner. En
oscillator med t. ex. grundfrekvensen 25
avger på så sätt även frekvenserna 50
75 kHz, i 00 kHz, i 25 kHz o.s.v .. Man får
således en "kalibreringsfrekvens"
25kHz.
Detta övertonsspektrum kan sträcka
flera 100 MHz upp. Man "nollsvävar"
apparat mot närmaste kalibreringsfrekvens
och kan kalibrera t. ex. VFO-skalan.
Användningsområden:
• Kalibrering av mottagare och
• Gradering av nya skalor o.s.v.

trafikmottagare har VFO med
L C-krets och ofta en inbyggd kalibreringsoscillator. En kalibreringsoscillator kan i sin tur
behöva kalibreras. Det enklaste sättet är då,
att jämföra frekvensen på en känd rundradiosändare på mellanvåg med kalibreringsoscillatorn. Dagens mottagare och sändare
har syntesoscillator och då behövs normalt
ingen kalibreringsoscillator.

Bild 118-7 Kalibreringsoscillator i mottagare

Brusmätbrygga

Bild Ii 8-8
Brusmätbryggan används vid mätning i antennsystem. Den består av en brusgenerator och en Wheatstone-brygga för mätning
av resistans och reaktans.

Bild II 8-8 Brusmätbrygga
118-7

PT

MÄTNIN

f~

a---------.------n
~----------------------,
l
l

l
l
l
l

l

[[I[]J

~tonn

l

l

l
l

l

LJ --- - --- JJ
)l

)l

Bild II 8-9 SVF-meter, princip och inkoppling
Till bryggan ansluts en antenn som mätobjekt och en mottagare som nollindikeringsinstrument för brussignalen. Mottagaren
ställs in på den frekvens där mätvärden
önskas. Bruset hörs svagast när bryggan är
in justerad. Man kan då avläsas mätvärdena
för R och X. Mäter man vid flera frekvenser,
kan t.ex. ett impedansdiagram upprättas.
ståendevågmeter (SVFameter)
Bild 118-9
När en transmissionsledning eller apparat
ansluts till en annan med awikande impedans, kommer HF-energi att reflekteras i
övergången.
Med ståendevåg-förhållande (SVF) eller
standing W ave Ratio (SWR)) menas förhållandet mellan den effekt som flyter framåt
respektive bakåt i en transmissionsledning.

Användningområden för SVF-meter:
• Mätning av framåtgående effekt.
e Mätning av bakåtgående effekt.
e Bestämning av SVF.
• Bestämning av resulterande, relativ effekt.
Anmärkning: Vid bestämning av absolut
effekt måste anslutningsimpedansen vara
lika i instrument och transmissionsledning.
SVF-metern är ett av de mest användbara instrumenten vid HF-mätningar. En
SVF-meter kan ha separata instrument för
fram- respektive backeffekt eller ett gemensamt.
SVF-metern kan vara ständigt inkopplad
t.ex. mellan sändare och antenn, men skall
då kunna tåla effektutvecklingen. En SVFmeter kan alstra övertoner, vilka kan medföra störningar. Orsaken är olinjäriteten
halvledardioderna i instrumentet.

~--------------------,

l

i

~

l

i

t~;~-=t--~-J
Bild/18-10 Frekvensräknare

118-8

Bild II 8-11 Absorbtionsvågmeter

MÄTNIN
Frekvensräknare
Bild II 8-1 O
Frekvensräknaren, som är ett digitalt instrument, används för att bestämma oscillatorfrekvensen i sändare, mottagare m.m.
l frekvensräknaren räknas antalet svängningar i den aktuella inkommande signalen
under en bestämd tidsenhet. Först förstärks
signalen i en analog förstärkare och omvandlas till kantvågspulser. En elektronisk
"kontakt", en s. k. gate, släpper därefter den
behandlade ingångssignalen vidare till en
digital räknare under en viss tid. Detta sker
med stor precision och i ett periodiskt förlopp. Antalet pulser räknas under genomsläppsperioden. Resultatet motsvarar insignalens frekvens.
Resultatet visas som siffror i ett fönster.
Noggrannheten i den s.k. tidbasen erhålls
med en kristallstyrd oscillator, vars frekvens
delas ner till önskat värde.
Absorbtionsvågmeter
Bild II 8-11
Absorbtionsvågmetern används för att bestämma en oscillators arbetsfrekvens. Den
består av en resonanskrets med variabel
frekvens, som kan avläsas, och ett mätinstrument som resonansindikator.
Vågmetern kopplas induktivttill den krets,
vars frekvens skall bestämmas. När frekvensen i kretsen och vågmetern stämmer
överens, ger resonansindikatorn utslag. Frekvensen avläses då på vågmeterns skala.
Anmärkning: Frekvensmätning på en
passiv svängningskrets kan inte göras med
detta instrument, vilket däremot går med en
dip-meter. Principen för en absorbtionsvågmeter är annorlunda än den för en dip-meter,
men i regel kan en dip-meter också användas som absorbtionsvågmeter.

Bild II 8-12 Dip-meter

fj:obje~
Bild II 8-13 Mätning med dip-meter
Dip~meter

Bild II 8-12 och -i 3
Dipmetern är i princip en oscillator med
variabel frekvens och utbytbara induktorer
för olika frekvensområden.
Den används för att bestämma resonansfrekvensen på passiva och aktiva svängningskretsar samt vid bestämning av induktanser och kapacitanser.
Noggrannheten är c:a 3 \(\circ\)/o.
Funktion: Instrumentet avger alternativt
reagerarför en H F-signal med viss frekvens.
Resonansfrekvensen i dip-meterns svängningskrets är steglöst variabel och frekvensvärdet kan avläsas på en skala.

Y-förstärkare

X-ingång

Bild II 8-14 Oscilloscop

l
118-9

MÄTNIN
Vid mätning av resonansfrekvensen i en
passiv svängningskrets kopplas dip-meterns
induktor induktivt till kretsen. När resonansfrekvensen i kretsen och dip-metern överensstämmer, ändras belastningen i d lp-meterns svängningskrets varvid instrumentet
uppvisaren strömminskning-en "di p". Frekvensen avläses då på skalskivan.
Vid mätning på en aktiv svängningskrets,
d.v.s. som drivs av någon H F-källa, uppstår
i stället en strömökning vid resonans vilket
också visas på instrumentet.
Induktansen i en svängningskrets kan
bestämmas med dip-metern, om kapacitansenär bekant. På motsvarande sätt kan en
obekant kapacitans bestämmas om induktansen i svängningskretsen är bekant.
Namnet grid-dipmeter kommer från
elektronrörsepoken. Ändringar i gallerströmmen (g rid current) i ett oscillatorkopplat
elektronrör används som indikation på att en
svängningskrets är i resonans. Då minskar
gallerströmmen - det blir en "ström-dip".
Numera används en transistor i stället för
röret och instrumentet benämns dip-meter.

Oscilloskop
Bild II 8-14
Oscilloskopet är ett mycket användbart instrument. Mycket snabba förlopp kan med
fördel studeras på en oscilloskopskärm.
Spänningsförlopp kan visas som funktion av tiden. Tillsammans med andra instrument kan frekvenskaraktäristiken i filter,
modulationskvalitet o.s.v. åskådliggöras.
Oscilloskopet består av ett katodstrålerör, där styrningen av katodstrålen sker med
hjälp av X- och Y-förstärkare och en s.k.
triggerförstärkare. Den signal som skall mätas ansluts vanligen till Y-förstärkaren medan
en tidbasgenerator som alstrar en sågtandformad signal ansluts till X-förstärkaren.
Bildenvisar ett blockschema på oscilloscop.

118- i o

PT
