\section{Raka mottagare}

\subsection{Mottagare med kristalldetektor}

Bild II 4-1 Detektormottagare

Bild II 4-1

Detektormottagaren består av ett mycket
litet antal komponenter. Princip och arbetssätt framgår av bilden. Samma princip används även i mer komplicerade mottagare,
mätinstrument etc. Antennkretsen består
av antenn, jordtag och däremellan en induktor (kopplingsspole), som överför energin
från antennen till en svängningskrets. Svängningskretsen används för att välja ut (selek-

tera) en bärvåg med önskad frekvens. Bärvågen kan naturligtvis inte höras, men av
kurvformen på bilden framgår att bärvågen
är amplitudmodulerad med en LF-signal.
För att återvinna LF-signalen utför man
en s.k. demodulering med hjälp av dioden.
Dioden klipper bort antingen de positiva
eller negativa halvvågorna i den mottagna
signalen, beroende på hur dioden är vändpolariserad. Kondensatorn, som är kopplad
parallellt över hörtelefonen, glättar de högfrekventa spänningstopparna till ett amplitud-

114-1

M
medelvärde (jämför med entaktsblandare i
Kapitel 3). Detta spänningsvärde varierar
på ett sätt, som motsvararden modulerande
spänning i sändaren som kommer av tal,
musik etc. Vi har nu demodulerat bärvågen,
återställt LF-signalen och kan höra den i
mottagaren.
Bild II 4-2
signalspänningen över svängningskretsen
är störst när dess resonansfrekvens och
antennströmmens frekvens är lika.
Överst i bilden ser man att mottagaren är
inställd på samma frekvens som sändare 2.
Även sändare 3 hörs eftersom bandbredden
i svängningskretsen är stor. Nederst i bilden
är svängningskretsen inställd på sändare 3,
men man hör också sändare 2 och 4.
Bandbredden i svängningskretsen blir
mindre ju mindre den belastas, d.v.s dämpas. l Bild II 4-1 består belastningen av
antennen (via kopplingsspolen), hörtelefonen och avkopplingskondensatorn (via dioden).
Mindre belastning kan åstadkommas på
två sätt; dels med "lösare" koppling mellan
antennkrets och svängningskrets och dels
med bättre impedansanpassning mellan
svängningskrets och diod. Båda sätten tillämpas i Bild II 4-3. Hur selektionen då
förbättras visas i Bild II 4-4, vilket skall
jämföras med Bild II 4-2.

demodulering

u

Bild II 4-2 Selektion i detektormottagare

u
550kHz

900kHz 1100kHz 1500kHz

Bild II 4-4 Förbättrad selektion

lågpass

volymkontroll

LFförstärkare

högtalare

selektering

demodulering

lågpass

LFförstärkare

Bild II 4-3 Detektormottagare med LF-förstärkare

114-2

högtalare

f

GARE

Bild II 4-5 Förbättrade HF-egenskaper i detektormottagare

u

membranen i hörtelefonen knäpper
(ingen ton)

/\r---------------t
Bild II 4-6 Hög HF-selektion

\subsection{Detektormottagare med förstärkare}

Bild II 4-3

Om man vill höra sändningarna över högtalare, behövs högre effekt än vad som kan
fångas upp genom antennen. För ändamålet används en LF-förstärkare, som drivs av
en annan energikälla, t. ex. ett batteri. LFförstärkaren kan även minska belastningen
på svängningskretsen.
l bilden har ett LF- lågpassfilter satts in
efter HF-avkopplingskondensatorn. Det
dämparLF-signaler med högre frekvens än
vad som behövs för god mottagning.
Mottagare med bättre HF-egenskaper
Bild II 4-5
Ett sätt att minska bandbredden i en detektormottagare är att koppla flera svängningskretsar med samma frekvens efter varandra. Den större dämpningen av fler kretsar
kan kompenseras med en HF-förstärkare.
Sådana mottagare används för speciella
ändamål, t.ex. för övervakning av en enda
frekvens. Då är svängningskretsarna fast
avstämda. Kanske utnyttjas till och med en
kvartskristall som filterförden speciella frekvensen. Se Bild II 4-6 om hög selektion.

"knack" "knack" "knack"

\

"knack"

---------

membranen släpper

Bild 114-7 CW i detektormottagare

\subsection{Detektormottagare och sändningsslag}

l huvudsak fungerar detektormottagaren endast vid amplitudmodulering. Det innebär
sändningsslagen A3E och A2A, d.v.s. amplitudmodulerad telefoni resp. tonmodulerad
telegrafi, båda med full bärvåg.
Bild II 4-7 Däremot fungerar detektormottagaren inte vid A 1A, d.v.s. telegrafi med
endast bärvåg. En omodulerad bärvåg alstrar nämligen endast en likström i en
detektormottagare. Vid nyekling hörs då
endast knäppningar i hörtelefonen vid början och slutet av teckendelarna.
Detektormottagaren fungerar inte heller
vid J3E, d.v.s. SSB och övriga sändningsslag med undertryckt bärvåg. Ljud såsom tal
förvrängs nämligen kraftigt i en J3E-signal
eftersom bärvågskomponenten saknas.
l båda ovannämnda fal kan talet återställas med tillsats av en bärvåg.
Slutligen kan sändningsslag som innebär frekvens- och fasmodulering i princip
inte dernoduleras med detektormottagare.
114-3

PT

M

~2 +D
~l
fl

f1 =1830kHz

och dess

ö"ctoo"

f2- f1 = 1 kHz

f2 - f1 ::: 1kHz

/,
[>

t
~

f 2 = 1831 kHz

VFO

blandare

A
antenn

x

t:;+

b,

u

demodulering LF-Iågpassfilter

förse lek·
ter ing

[>
LF-för·
stärkare

högtalare

~
VFO

\subsection{Mottagare med direkt frekvensblandning}

Bild II 4-8 Mottagare med direkt frekvensblandning

För att demcdulera A 1A och J3E i en rak
mottagare- detektormottagare-måste den
kompletteras med en oscillator som alstrar
en intern bärvåg. Denna blandas med den
mottagna signalen. Det uppstår då en svävningston - beat frequency. Därav namnet
Beat Frequency Oscillator- BFO.
Förfarandet har givit mottagartypen sitt
namn- direktblandad mottagare.

114-4

Bild II 4-8
Ett sätt att komplettera den raka mottagaren
med BFO framgår av bilden. När BFO kopplas till och ställs in på en frekvens tillräckligt
nära mottagningsfrekvensen så uppstår en
hörbar ton.
Demodulatordioden tillförs alltså två HFsignaler, dels den från antennen och dels
den från BFO. Dessa båda signaler blandas
i dioden och skillnadsfrekvensen är den
hörbara tonen. Övriga blandningsprodukter
dämpas av ett lågpassfilter.

A ARE
HF på blandaringången

u~

LF-skillnadsfrekvens på
blandarutgången

VFO-

CWsignal
1830 kHz
l

si nal

1lb1 kHz

l1

... f

f1 f2

UrF01is2~1
/f-

u
kHz

CWsignal 1830 kHz

j

1 kHz

-JAI.I. f

1
~~~----------f

mo

f2 f1

u

f1- f2

119 kHz
~----------~~-~f-, f
V F O-

ur

si nal
1S29 2 kHz C;W,
sognal
1830kHz

1l

f

l

0,8 kHz

1

--f

fz f1

~-L------------f

f,- f2

Bild II 4-9 Demodulering i mottagare med direkt frekvensomvandling- C W-signaler

U

u~d~rtryckt
barvag

l SB

l

1832
kHz

~---

!l

i

1835kHz

l

fl\ 1

1834 1834,7
kHz kHz

u
1kHz
300 Hz 

.. f

r-1

111

3 kHz

1:
L------!l.-A--.L..----f

Bild II 4-1 O Demodulering i mottagare med direkt frekvensomvandling - SSB-signaler
Mottagning av telegrafi (CW)
Bild II 4-9
Då BFO (VFO) är inställd på frekvensen f2
=1831 kHz och den mottagna signalen f1 har
frekvensen 1830 kHz så hörs en svävningston med frekvensen 1000 Hz.
Samma resultat fås om BFO ställs in på
frekvensen f2 = i 829 kHz. Med BFO på

frekvensen f2 = 1830 kHz hörs ingenting av
signalen f 1 = 1830 kHz från sändaren.
Frekvensskillnaden är noll Hz.
De flesta föredrar en ton med frekvensen
c:a 800Hz för mottagning av telegrafi. BFOfrekvensen skulle i så fall ställas in på 1830.8
eller i 829.2 kHz om f 1 vore en telegrafisändning.

114-5

Selektering av blandningsprodukterna

[>
LF-lågpass
3kHz

förselektering
av ca 300kHz
bandbredd

Val av mottagarfrekvens

VFO

U HF
VFO-

frekvens

.H,

l r--4s
CW-Signal
1830 kHz

1829,2

1
,J

.

rll

~il

u

SSB-Signal
1835 .-·

lågpassfilterkurva

O - 3 kHz

1832 kHz

 L,

35 kHz

l

·f

1632 1834 1634,7
kHz kHz kHz

l
l

I....L-A.I..'-.L...ll.------ f

800 Hz 2,8 kHz

Vid mottagning av en CW-signal tillsammans med en SSB-signal hörs båda samtidigt

u
BOO Hz

4,8 kHz

5,5kHz

VFO
Förbättring av selekteringen med ett LF-CW-fiiter

Mottagning av J3E (SSB)
När en SSB-sändare sägs arbeta t. ex. på
frekvensen 1835 kHz, så innebär det frekvensen på den bärvåg som undertryckts i
sändaren redan före utsändningen.
Vad som uppfattas av mottagarens ingångskretsar är alltså det utsända sidbandet När en SSB-signal demoduleras, så
blandas den lokala bärvågen i mottagaren
med de mottagna modulationsprodukterna.
Vid blandningen uppstår blandningsproduk-

114-6

ter som består dels av LF, dels av andra
högre frekvenser som dämpas i ett låg passfilter.
Bild II 4-1 O
l nom amatörradio används för SSB det lägre sidbandet vid frekvenser under 1O MHz.
Med en frekvens av t. ex. 1835 kHz och ett
talspektrum av 300-3000 Hz kommer det
lägre sidbandet att finnas mellan 1834.7
och 1832.0 kHz. Tre modulerande frekvenser 300, 1000 och 3000 Hz visas på bilden.

TTA ARE
Med en bärvågsfrekvens av 1835kHz motsvaras dessa modulerande frekvenser av
utfrekvenserna1834.7, 1834 och 1832kHz.
VFO ersätter SSB-sändarens bärvåg och
skall ha samma frekvens-1835kHz- för att
kunna återge 300, 1000 och 3000 Hz.

\subsection{selektionen i direktblandade mottagare}

Bild II 4-11

Direktblandade mottagare kan ses som en
typ av detektormottagare, även kallad "rak"
mottagare. Begreppet "rak" kommer av att
HF-signalen från antennen passerar genom en selektiv krets och en eventuell HFförstärkare rakt fram till detektorn, utan att
frekvensen omvandlas.
l en detektormottagare är bandbreddenoftast rätt stor. Flera sändare hörs därför
samtidigt.
P.g.a. att blandningsdioden i en direktblandad mottagare även fungerar som AMde modulator, så hörs faktiskt alla sändare
inom förkretsens bandbredd. Detta kan undvikas till en del genom att dioden, som
fungerar som entaktsblandare, byts till en
mottaktblandare eller ännu hellre till en ringblandare. Sådana blandare undertrycker ingångsfrekvenserna och släpper endast igenom blandningsprodukter. Bara den sändarsignal hörs då, vars frekvens tillsammans med VFO-frekvensen ger blandningsprodukter, som faller inom LF-filtrets passband. Mottagningsfrekvensen är VFO-frekvensen. Svängningskretsen fungerar som
en ställbar förselektor och LF-Iågpassfiltret
ger den egentliga frekvensselektionen.
Vilka HF-signaler bildar blandningsprodukter med VFO-frekvensen och vilka av
dessa passerar sedan genom lågpassfiltret
efter nedblandning till LF-nivå?
Exempel:
En CW-sändare med frekvensen 1830 kHz
tas emot genom att mottagarens VFO ställs
in på frekvensen 1829.2 kHz. Från blandarutgången kommer då en ton med frekvensen 800Hz.
Men sändaren är inte ensam på bandet.
Kommer t. ex. SSB-sändaren på 1835, som
moduleras med 300, 1000 och 3000 Hz, att
störa mottagningen? (Bild II 4-1 0).

Förkretsen i mottagaren är så bred att
denna sändning passerar. SSB-sändarens
signalfrekvenser i det utsända sidbandet är
1834.7, 1834.0 och 1832kHz. Dessa frekvenser blandas med mottagarens VFO-frekvens 1829.2 kHz och alstrar blandningsprodukterna 5.5, 4.8 och 2.8 kHz. Eftersom
lågpassfiltret i mottagarens LF-förstärkare
har bandbredden 0-3000 Hz, så kommer
endast blandningsprodukten 2.8 kHz attvara
störande. För att förbättra CW-mottagningen, så kan lågpassfiltret bytas ut mot ett
bandpassfilter, som endast släpper igenom
ett smalt frekvensområde omkring mittfrekvensen 800 Hz.

\subsection{Passband och spegelfrekvenser i direktblandare}

Bild II 4-12
l exemplet i förra stycket blev problemet
med en störande ton löst med ett bandpassfilter med annan frekvensgång.
Men vilka frekvenser kan tas emot genom
ett lågpassfilter, 0-3000 Hz, om VFO-frekvensen är t.ex. 1829.2 kHz?
Experiment:
Ändra frekvensen på en CW-sändare långsamt från 1820 till 1840 kHz.
Såndarfrekvensen i 820 kHz hörs knappast eftersom överlagringstonen har frekvensen 9.2 kHz och den dämpas kraftigt av
lågpassfiltret Först när sändarfrekvensen
är 1826.2 kHz hörs en tydlig ton med frekvensen 3000 Hz. Fortsätter man att ändra
sändarfrekvensen, så sjunker tonens frekvens för att bli noll (svävningsnoll), när såndarfrekvensen är lika med mottagarens VFOfrekvens 1829.2 kHz. Om man nu fortsätter
med att höja frekvens, så blir överlagringstonens frekvens åter högre. Vid såndarfrekvensen 1832.2 är den 3000 Hz. Vid ännu
högresändarfrekvens dämpas överlagringstonen igen av lågpassfiltret
Slutsatsen av experimentet blir följande:
Vid en direktblandande mottagare med VFOfrekvensen i 829.2 kHz och ett 3 kHz lågpassfilter blir varje sändare hörbar, som har
en sändningsfrekvens mellan 1826.2 och
i 832.2, varvid överlagringstonen har frekvenser från 3000Hz, ner genom noll och upp
till 3000 Hz igen.

114-7

TTAGARE
HF

U

LF

u

.fvFo~
(---r-----,

-3 kHz

r-----.. ,

+3 kHz

l

l
l

l.f-----1-t

l
l

.

l

l

..

:

f

~--~----------f
3kHz

1826,2 1829,2 1832,2

kHz

kHz

kHz

6kHz HF-bandbredd vid3kHz LF-bandbredd

U

l - - . -:. .f.1. s.fvFO.L.~. . !\----f

1828,5 ·-·
1828,3 kHz

1829,2

kHz

1829,9 ---

1830,1 kHz

u

,-,

l

l

l
l
l

l
l
l

l

j

l

l

l

\

~~~----------f
700-900 Hz

Mottagningsfrekvens och spegelfrekvens
med ett LF-CW-filter

l

fvFO
r----r----~
l

-

!

183 2
kHz

1835

1838

kHz

-t

kHz

Mottagningsfrekvens och spegelfrekvens
med ett LF-Iågpassfilter

Bild II 4-12 Passbandbredd och spegelfrekvenser i direktblandade mottagare
Vår mottagare har bandbredden 6 kHz.
Varje annan sändare inom denna passbandbredd kommer att höras eller-om man
så tycker- störa mottagningen.
Tillbaka till exemplet med bandpassfiltret
Vilka frekvenser kan tas emot med ett
bandpassfilter 700-900 Hz (mittfrekvens 800
Hz), om VFO-frekvensen är 1829.2 kHz?
Jo, vi kan lyssna rätt ostört till vår CWsändares 800 Hz-ton på frekvensen 1830
kHz. Ändå kan en annan sändare med
frekvensen 1828.4 kHz störa mottagningen
därför att denna är spegelfrekvens till mottagningsfrekvensen 1830 kHz. Vid VFOfrekvensen 1829.2 kHz uppstår en överlagringston, inte bara vid sändarfrekve~sen
1830kHz utan också vid 1828.2 kHz. Aven
denna andra sändarfrekvens, liksom nytto114-8

frekvensen, släpps igenom bandpassfiltret
Spegelfrekvensmottagning är en principiell nackdel i mottagare med direktblandning. Nyttafrekvens och spegelfrekvens i
det senaste exemplet ligger 1.6 kHz (2 · 800
Hz) ifrån varandra, alltså dubbla värdet av
bandpassfiltrets mittfrekvens.
Vid 888-mettagning måste naturligtvis
hela LF-området upp till 3000 Hz kunna
släppas igenom. Utöver det önskade frekvensområdet 1832-1835 kHz, kommer även
spegelfrekvenser i området 1835-1838 kHz
att kunna tas emot.
Vid en LF-bandbredd av 3 kHz har således den direktblandade mottagaren en bandbredd av 6kHz, vilket är en god avstämningsskärpa i jämförelse med den 300 kHz breda
förkretsen.

M
\subsection{För- och nackdelar med direktblandare}
Enkel uppbyggnad, men trots det en god
känslighet och hygglig avstämningsskärpa.
VFO kan även användas till att styra en
sändare.
Spegelfrekvensmottagning är tyvärr
oundviklig. Vidare kan signaler från starka
sändare stråla in i den känsliga LF.:.förstärkaren och orsaka LF-detektering, om mottagaren är otillräckligt skärmad. Förbättrad
isolering mellan antenn och VFO kan dock
fås med en HF-förstärkare.
Entakts diodblandare är olämplig i en
direktblandad mottagare. Den tar emot alla
sändare inom förkretsens passband och en
del av VFO-signalen kommer att strålas ut i
antennen. Ingen av dessa nackdelar finns i
en mottakts-eller ringblandare.

f.MF

ARE

