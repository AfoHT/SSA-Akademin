\section{Antenner för VHF/UHF/SHF}

\subsection{Allmänt}
Alla antenner fungerar efter samma principer. Principerna för kortvågsantenner kan
därför tillämpas även för antenner för högre
frekvenser. Byggmåtten på en VHF/UHFantenn är betydligt mindre än för en motsvarande KV-antenn. JämförA= c:a2 m vid 145
MHz och A= c:a 80m vid 3.5 MHz. Det är
därför möjligt att bygga riktantenner med
rimliga dimensioner för VHF/UHF, även om
flera element används.
Om man bortser från rundstrålande vertikalantenner för trafik på korta avstånd och
mobil trafik, så används riktantenner främst
p.g.a. den större räckvidden. En riktantenns
egenskaper uttrycks i första hand i storheterna strålningsvinkel, antennvinst, fram/
backförhållande och halvvärdesbredd.
Eftersom polarisationsvridning sällan förekommer vid högre frekvenser, är det viktigt att sändar- och mattagarantenner har
samma polarisationsriktning.
Horisontell polarisation anses vara bättre
lämpad för långa distanser, eftersom vågor
med horisontell polarisation böjer av bättre
över horisontella formationer (bergryggar
etc). Även passage genom skogspartiergår
bättre med horisontellt polariserade vågor.
Antenner med horisontell polarisation används därför ofta för SSB- och CW-trafik på
långa avstånd och utmed markytan. Sådan
trafik sker i allmänhet från fasta stationer.
För mobil trafik och lokal fast trafik används dock antenner med vertikal polarisation. Vertikala antenner ger de önskvärda
rundstrålande egenskaperna för mobil trafik
och är bäst lämpade att montera på fordon.

\subsection{Riktantenner}

En A/2-antenn strålar vinkelrätt ut från
antennledaren och runt omkring den.
Placeras ett reflektorelement (längd ~Al
2 +5o/o) bakom antennen på ett avstånd av
::::: A/5 så reflekteras den bakåtriktade strålningen delvis framåt. En större del av energin kommer då att samlas i en riktning. Med
ett direktorelement (längd= A/2- 5o/o) framför det strålande elementet på ett avstånd
av~ A/1 O så kommer utstrålningsvinkeln att
bli mindre.

\subsection{Yagi-antenner}

Bild II 6-20
Den typ av riktantenn, som består av en
strålare, en passiv reflektor samt ett antal
passiva direktorer, kallas Yagi-antenn.
Vagi-antennen kan utföras med olika
antal direktorelement i kombination med
olika längd.
Det finns tre sätt att optimera en riktantenn, nämligen maximal riktverkan, minimum sidlober och maximalt fram/backförhållande. Dessa egenskaper är, emellertid ej möjliga att uppnå samtidigt. Okas t. ex.
antalet element, så ökar den s.k. antennvinsten genom att öppningsvinkeln på strålningen blir mindre, men samtidigt minskar
matningsimpedansen och den användbara
bandbredden.

\subsection{Gruppantenner}

Ordnas flera riktantenner vid sidan av och/
eller över varandra så erhålls en s.k. gruppantenn. Ett sådant arrangemang av s.k.
stackade antenner ger en ännu mindre
öppningsvinkel på strålningen vertikalt och/
eller horisontellt. Därigenom erhålls ytterligare antennvinst

\subsection{Parabolantenner}
Särskilt på frekvenser i mikrovågsområdet
och högre har radiovågorna i stort sett
samma utbredningsegenskaper som ljusets.
Behöver stor riktverkan uppnås på dessa
höga frekvenser, används ofta en parabolisk yta som spegel bakom själva antennen.
Jämför med reflektorn i en ficklampa.
Den egentliga antennen (den s.k. mataren), vars strålning är riktad mot parabolen
för att reflekteras, kan vara utformad på
många sätt. Eftersom parabolens storlek
står i omvänd proportion till frekvensen, så
används av praktiska skäl inte paraboliska
reflektorer på låga frekvenser.

\subsection{Övriga antenntyper}

Rundstrålande antenner: Ground plane,
A/4-,A/2-, 5A/8-antenner m.fl.
Riktantenner: Quad-, HB9CV-, helical-,
parabol- och hornantenner m fl.

116- 15

HORISONTALDIAGRAM

l

Di pol

-----::>

---r-·s

l--r-·

Yagiantenn

S = strårare
R == reffektor

:.::.::.:;;;;;>

s

R

D= direktor

:::::::::=;>

-Il-R <;

R

-ffi-

s

-«()-

o

l

D D

:=:::::;>

o

--~-

D

VERTIKALDtAGRAM

Di pol

--t- l

Dipol i

t A över jord

s
Vagi-

antenn

Yagi i

R

1'>

D

Bild II 6-20 Strålningsdiagram för horisontell Yagi-antenn

116-16

fÅ

över jord

ANTENNSYSTEM
