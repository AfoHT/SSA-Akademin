\section{Magnetiskt fält}
\textbf{HAREC a.\ref{HAREC.a.1.4}\label{myHAREC.a.1.4}}

\subsection{Magnetism}

Enligt den romerske författaren Plinius lär, vid tiden ungefär 160 år f. K. herden Magnes
en dag ha känt hur järnstiften i sandalerna häftade vid en viss sorts sten. Det kunde ha
varit svart järnmalm, som grekerna i äldsta tider benämnde Lithos herakleia efter staden
Herakleia i Lydien, där sådan malm förekommer. Staden fick sedermera namnet Magnesia och
man kan tänka sig att stenen kom att kallas Magnetes. En hel mineralgrupp med liknande
egenskaper, såsom järn, nickel m. fl. kallas magnetiska.

Magnetism uppstår av elektriska laddningar i rörelse. Elektronernas rörelser i en atom
skapar nämligen magnetfält. Det gör att atomerna var för sig fungerar som en magnetisk
dipol - en magnet. I de flesta material är atomerna orienterade så att deras magnetiska
kraftertar ut varandra. Materialet som helhet är då omagnetiskt och utövar inga yttre
krafter. Men vid påverkan från ett yttre magnetfält kan dipolerna (atomerna) i ett
material orienteras i samma riktning och deras magnetfält kommer då att
samverka. Hela materialet blir då magnetiskt. När det yttre magnetfältet avlägsnas,
kvarstår orienteringen endast delvis- magnetisk remanens. l terrornagnetiska legeringar
kvarstår en större del av orienteringen, även om påverkan från det yttre magnetfältet har
upphört. Materialet är då permanentmagnetiskt

\subsection{Kraftfält i och omkring magneter}

Bild 111-7

Varje magnet omges av ett magnetiskt kraftfält Magnetfältets fördelning, styrka och
riktningar beskrivs som kraftlinjer med slutna kretslopp.

Utanför magneten går kraftlinjerna från nord- till sydpol och inne i magneten i motsatt
riktning. Kraftriktningen i varje punkt av fältet är den som nordändan på en kompassnål
skulle peka åt. Om man hänger upp en magnet i en tråd, så kommer den att inta
samma riktning som jordens magnetfält.

Poler med samma polaritet stöter bort varandra (repellerar).

Poler med olika polaritet dras till varandra (attraherar).

\subsection{Magnetiska fält omkring strömbanor}

Bild II 1-8

Omkring varje ledare, som det flyter en elektrisk ström igenom, alstras det ett
magnetiskt kraftfält.

Magnetiska kraftlinjerna fördelar sig koncentriskt omkring en rak ledare och vinkelrätt
mot denna.

Mellan ändarna av en ledare med bågformad utsträckning bildas kraftlinjer som verkar med
varandra.

En strömgenomfluten cylindrisk spole induktor- uppvisar samma magnetiska fältbild som en stavformad permanentmagnet

\subsection{Bestämma magnetiska fältriktningen}

Magnetfältets riktning omkring en ledare kan enkelt bestämmas med vänsterhandsregeln

När en ledare fattas med vänster hand och med tummen i strömmens riktning, så
kommer fingrarna att peka i fältriktningen.

När en ledare formas som en spole och en elektrisk ström flyter genom den, kommer
magnetfältet att ha ett utseende som liknar det omkring en permanentmagnet

En sådan spole kallas elektromagnet.

Magnetfältets riktning i en spole kan också bestämmas med vänsterhandsregeln.
När en spole fattas med vänster hand och med fingrarna i strömmens riktning, så
kommer den utsträckta tummen att peka mot spolens nordpol.

Fälten omkring alla slags magneter, såväl permanentmagnetiska som e!~ktromag­
netiska, återverkar på varandra. Aven enkla
elektriska ledare är elektromagneter.

Bild II 1-7 Kraftfält omkring magneter

Bild II 1-8 Magnetiska fält omkring strömledare

\subsection{Exempel på elektromagneter}

Bild II 1-9

\subsubsection{Elektromagnet}
Det bildas ett magnetfält genom en spole så länge som det flyter ström genom den. En
järnkärna i spolen koncentrerar fältet p.g.a. den större magnetiska ledningsförmågan.

Elektromagneter används för att sätta magnetiska material i rörelse eller hålla fast
dem.

\subsubsection{Elektrisk ringklocka}
Anordningen består av en elektromagnet och en järnplatta på en fjäder. På plattan
sitter en självbrytande kontakt samt en kläpp som kan slå på en klocka.

Kontakten åstadkommer en växelvis brytning och slutning av strömmen genom
elektromagneten. Armaturen med kläppen kommer då i svängning och slår på klockan.

\subsubsection{Telefon}
I en enkel telefon finns bl.a. en mikrofon, ett batteri och en hörtelefon.

Särskilt i äldre telefoner består mikrofonen av en kolkornskammare med ett membran.
Trycksvariationer (ljud) får membranet att vibrera, varvid resistansen genom kolkornen
varierar i motsvarande grad. Därmed varierar talströmmen genom mikrofonen.

Hörtelefonen består av en elektromagnet och ett membran av mjukjärn. Variationer i
talströmmen genom mikrofonen passerar även hörtelefonen får dess magnetfält att variera.
Hörtelefonens membran alstrar då trycksvariationer, d.v.s. ljud.

\subsubsection{Elektromagnetiskt relä}
Reläet består av en elektromagnet, en järnplatta (ankare) på en fjäder och en elektrisk
kontakt. Med en svag ström/låg spänning genom spolen i manöverkretsen, så kan
man med reläets arbetskontakt styra starkare ström/högre spänning i huvudkretsen.

Bild II 1-9 Tillämpade elektromagneter

\subsection{Magnetisk fältstyrka}

Som magnetisk fältstyrka Henry $[H]$ förstår man flödet per meter fältlinje, d.v.s.

$H=\frac{\Phi}{l} = \frac{I \cdot N}{l}$

$H [A/m]$ $I [A]$ $N [varvtal]$ $l [fältlinjelängd]$

Magnetisk fältstyrka uttrycks således som Ampere per meter flödesväg.

\subsection{Magnetisk flödestäthet}

Den magnetiska flödestätheten mäts i enheten Tesla $[T]$ (förut Gauss).

Formeltecknet är $B$.
Formeln är $B = \mu_0 \cdot HH$

Flödestäthet $B [Vs/m^2]$ Fältstyrka $H [A/m]$

$\mu_0$ är permeabilitetstalet (fältkonstanten) för den magnetiska ledningsförmåga för
luft och omagnetiska material.

För järn eller annat magnetiskt ledande material tillkommer permeabilitetstalet $\mu_r$.
Det anger hur många gånger bättre än luft etc., som materialet det leder ett magnetisk
flöde.

Formeln är $B = \mu_0 \cdot \mu_r \cdot H$

\subsection{Magnetiskt flöde}

Det magnetiska flödet är produkten avflödestätheten $B$ och tvärsnittsytan $A$ av flödesvägen, således

$\Phi = B \cdot A$
$\Phi [Weber eller Vs]$ $B [T eller Tesla]$ $A [m^2]$

\subsection{Skärmning av magnetiska fält}

I grunden finns det två slags fält, det elektriska och det magnetiska. Det finns även
elektromagnetiska fält, som är sammansatt av båda dessa. Fält kan vara permanenta eller
rörliga, varav här avses de rörliga. Ett rörligt magnetiskt fält genererar ett elektriskt
fält.
Omvänt generar ett rörligt elektriskt fält ett rörligt magnetiskt fält. Denna växelverkan
gör att fälten kan hållas igång med tillförsel av yttre energi.

Fält i rörelse alstrar elektromagnetisk strålning, som påverkarfunktioner i omgivningen.
När påverkan inte är önskvärd, måste fältet skärmas av. Ett sätt att skärma magnetiska
fält är en metallisk kapsling. Kapslingenskall vara tät och bilda en sluten magnetisk
krets. Kapslingen skall vara utförd i ett material som är en god ledare av magnetiskt
flöde.
(Jämför 1.3)
