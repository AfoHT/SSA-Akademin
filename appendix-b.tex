\chapter{Grundläggande matematik för radioamatörer}

GRUNDLÄGGANDE
Detta avsnitt omfattar några matematiska
begrepp, ekvationer och formler som kan
vara till hjälp vid studium inför radioamatörcertifikat. Svårighetsgraden spänner över
grundskolans och gymnasiets nivåer.
Genomgången av exponentiella tal och
logaritmer ligger till grund för förklaringen av
begreppen decibel och s-enhet, vilka ofta
förekommer i radiotekniska sammanhang.

Ekvationer
Ekvation är ett annat ord för likhet. Vid matematiska beräkningar ställs storheterna upp i
en eller flera ekvationer.
l en s.k. sann ekvation har resultatet av
de uppställda storheterna samma värde på
båda sidor om likhetstecknet.
Ex. 3 · 5 = 15 (3 multiplicerat med 5 är 15)
4 + 7- 1 = 1O (4 plus 7 minus 1 är i O)

~ =3
5

(15 dividerat med 5 är 3)

(Multiplikationstecknet bör skrivas som
en höjd punkt · och inte som x. Då undviks
förväxlingar med bokstaven x i ekvationer,
där okända tal betecknas med bokstäver).
För att resultatet skall bli rätt måste givna
regler alltid följas vid behandlingen av storheterna i uppställningarna.
Vid multiplikation och addition kan storheterna hanteras i godtycklig ordning, men
däremot inte vid division och subtraktion.
Resultatet blir 15, antingen vi skriver 3 • 5
eller 5 · 3.
Likaså är resultatet 8, antingen vi skriver
3 + 5 eller 5 + 3.
Däremot blir resultatet annorlunda när man
skriver

3
15
i stället för
15

5

Likaså blir resultatet annorlunda när man
skriver 15 - 5 i stället för 5 - 15
Vid division kan talen ställas upp som s.k.
bråktal. De kan skrivas på något av sätten
15:3 eller 15/3 eller

~
5

Talet före kolon, före snedstrecket respektive över bråkstrecket kallas för täljare.
Talet efter kolon, efter snedstrecket respektive under bråkstrecket kallas för nämnare.
För att tydligare beskriva allmängiltiga samband mellan storheterna i en ekvation, kan
storheterna uttryckas med bokstäver i st. f.
med siffror. En sådan ekvation kallas för
formel.
Sökta eller okända storheter brukar betecknas med bokstäver från slutet av alfabetet, t.ex. x, y eller z. Givna eller kända
storheter brukar betecknas med bokstäver
från början av alfabetet, t. ex. a, b eller c.
Antag två tal a och b, vars produkt är c.
Formeln är då

a·b=c

Sätts c= 15, så är a·b=15. Då kan a· b
vara 3 · 5 eller 5 · 3 eller 7.5 · 2 eller vilka
andra tal som helst vars produkt blir 15.
Likheten .!! =!! kan enligt de matematiska

y

b

reglerna skrivas på något av följande sätt:

b·X=a·y
b·X
a

l=!!
x a
b·X
a=
-

y

!!=l
a b
b-- a·
x

X=a·y
b

Att alla dessa sätt är varianter av en och
samma ekvation kan bevisas, genom att
multiplicera den ursprungliga likheten
b·x=a·y med b·y på båda sidor om
likhetstecknet,

x a
b·y·-=-·b·y
y b

d.v.s.

Detta visar den s.k. diagonalregeln, som
innebär korsvis uppmultiplicering av nämnarna till täljarna.
Vid multipliceringen fås samma resultat
för var och en av varianterna, vilket visar att
de är likvärdiga.

B-1

E

APPENDIX
Räkneexempel med 1 obekant storhet:
Om tredjedelen av ett tal är 8 enheter större
än femtedelen av samma tal, vilket är då
talet?
Det sökta, okända talet kallas t. ex. för x.
Tredjedelen av x är !. och femtedelen är !..

3

5

När 8 läggs till femtedelen fås tydligen två
lika tal, och en ekvation (likhet) kan skrivas

!=B+!

3

5

Vi kan multiplicera, dividera, addera eller
subtrahera godtyckligt på ena sidan om likhetstecknet om vi också gör samma operationer på den andra sidan.
Likhetsvillkoret får aldrig äventyras.
För att kunna utläsa vilket tal som motsvarar x, gäller det att få x ensamt- "fritt"på den ena sidan om likhetstecknet.
Vi multiplicerar alla termer på båda sidorna med 3 i ovanstående formel.
3 ·x= 3. 8 + 3 ·x

3
5
3·X
X=24+5

vilket kan avkortas till

Därefter multipliceras termerna på båda
sidorna med 5.

3·X·5
5·X=5·24+-5
5·X= 120+3·X

d.v.s.

Båda sidor om likhetstecknet minskas
därefter med 3 · x
således 5·x-3·x= 120+3· x-3· x
Multiplikationstecknet brukar inte skrivas
ut, varken mellan tal och bokstäver eller
mellan bokstavsgrupper. Alltså

5x-3x= 120+3x-3x
5x-3x= 2x och 3x-3x= O
Kvar blir då 2x = i 20,
där x är detsamma som 1 · x eller 1x.

8-2

Den sist erhållna ekvationen divideras med
2 på båda sidor om likhetstecknet

-2x = -120
2

2

vilket ger x = 60

Det sökta talet är alltså 60
Kontroll:

60
60
=8+
3
5

20 = 8 + 12; 20 = 20 vilket skulle bevisas.
l det första exemplet använde vi diagonalregeln. De två exemplen visar, att det går att
göra omflyttningar när man löser en ekvation. Ett tal med positivt eller negativt förtecken, och som står på ena sidan om likhetstecknet, kan t.ex. "flyttas" över till andra
sidan om likhetstecknet, om förtecknet byts
till det motsatta.
5x = 120 + 3x kan också skrivas

+5x= +120+ 3x
5x-3x= 120
5x- 3x-120 =O
Kom ihåg: 5x-x-x är samma som
5x-x-x eller 5x-(x+x), d.v.s. 3x
Räkneexempel med 2 obekanta storheter
{ekvationssystem):
Endast en obekant storhet har behandlats i
föregående exempel och en ekvation har
varit tillräcklig för det.
Två eller flera obekanta storheter kan
inte behandlas med bara en ekvation.
Antag, att vi skall beräkna

7x+6y=34
Det går det inte att lösa denna ekvation
entydigt, eftersom x och y kan ha många
olika olika värden, som uppfyller ekvationens villkor- satisfierar den.
Men när ännu en ekvation ställs upp, blir
det möjligt att göra en entydig lösning.
Således

!:

{;::~~=~:

eller

G::~:=~~

Nu passar endast ett och samma x- respektive y-värde in i båda ekvationerna.

APPENDIX
Om x "löses" genom att de båda
ekvationerna skrivs om fås:
34-

1.

X=---"-

7

29-9y

x

2.

5

Vi kan nu göra en ekvation 3 där det bara
finns en obekant, y, som är lätt att beräkna.
3.

34-6y- 29-9y

7

-

eller

5

170- 30y= 203- 63y
33y= 33 d v. s. y= 1

eller

Allmänt gäller att det behövs minst lika
många ekvationer som antalet obekanta storheter.
Exempel:
Vi vet, att ytan i en rektangel är produkten av
dess längd och bredd.
Om en husgrund är 1Ometer lång och har
en yta av 50m 2 , så får vi bredden b genom
att dividera ytan med längden,
50

b= 10 = 5 Bredden är således 5 meter.
Om ytan av ett hus är 300 m2 och bredden är
en tredjedel av längden, vilken bredd och
längd har då huset?
Antag, att längden är x meter. Bredden är då
en tredjedels x och vi får alltså ekvationen

x

x · x kan också skrivas >t vilket uttalas
"kvadraten på x" eller "x upphöjt till 2".
x· x· x kan också skrivas x' vilket uttalas
"kuben på x" eller "x upphöjt till 3".
När vi som i ovanstående exempel har
>(- = 900 och vill veta värdet på x, måste vi
"dra kvadratroten ur" 900.
Detta skrivs x= ,)900 = 30
Ett tal kan även vara negativt, men det
behövde vi inte beakta i detta exempel.
Annars skriver man x= $\pm$,)900 = $\pm$30
Potenser, digniteter

Värdet på y sätts in i ekvationerna 1. och
2., varefter även värdet på x beräknas.
Pröva själv! Svaren är y= 1 och x= 4

X·3=300

B

X·X=300·3, d.v.s. >(- =900

Hur stort är då x?
Vi prövar med olika tal och först med x=
2~, men 20 · 20 = 400 vilket är ett för lågt
varde. Sedan prövar vi med x = 40, men 40
· 40 = 1600 vilket är ett för högt värde. Sätt x
= 30. Eftersom 30 · 30 = 900, så är det sökta
talet x = 30.
30
Huset är 30 meter långt och
=1
meter brett.
3

o

Produkten av två eller flera exakt lika stora
2
faktorer kallas potens. l uttrycket x kallas
faktorn x för bas. Det antal gånger, som
faktorn ingår i produkten, kallas för exponent. Om exponenten är ett positivt helt tal
kallas produkten av faktorerna för dignitet.
Uttrycket X 2 är t. ex. 2:a digniteten av x.
Ett annat exempel är 5 · 5 · 5 = 125
Faktorn är 5 och produkten i 25 är 3:e digniteten av 5.
Det är opraktiskt att skriva många faktorer efter varandra. Man skriver därför faktorn
en gång och exponenten med en liten siffra
till höger ovanför faktorn.
Produkten 5 · 5 · 5 kan i stället skrivas 53 •
Basen är 5 och exponenten är 3. Digniteten
utläses 5 upphöjt till 3.
i O · 1O skrivs 1 02 och läses i O upphöjt till 2
5
2 · 2 · 2 · 2 · 2 skrivs 2 och läses 2 upphöjt
till 5
Om vi går över till bokstavsbeteckningar
gäller allmänt att

an= a· a· a ... n gånger= a upphöjt till n
Faktorn a kallas potensens bas och faktorernas antal kailas potensens exponent.
5
Om vi nu skriver 2 · 2 · 2 · 2 · 2 som 2

'

hur skrivs då

?

2·2·2·2·2 .

Vi kan s~riva ; men det är mer praktiskt
2
att skriva 2
5
Minustecknet anger att 2 står i nämnaren, alltså under bråkstrecket
B-3

EPT

APPENDIX

1

På samma sätt kan vi skriva
1

2- i stället för

i

2

5 i stället för

2

2- i stället för

2~

$\pm$

o.s.v.

6

10 anger att 1O skall multipliceras med sig
självt 6 gånger, d.v.s. resultatet är 1 miljon.
6
10- anger på samma sätt i miljondel
Hur beräknas uttrycket a 3 . a 2 ?
Eftersom a 3 =a· a· a och a 2 =a· a
är tydligen a 3 • a 2 =a· a· a· a· a= a 5
Produkten av två digniteter med samma
bas är lika med basen upphöjd till summan av exponenterna.
Allmänt uttrycks detta
På samma sätt beräknas am
an
Exempel:

a·a·a·a·a
2
----=a·a=a
a· a· a

således

När potenser med samma bas skall divideras med varandra, fås resultatet genom att den gemensamma basen "upphöjs till" skillnaden mellan exponenterna,
d.v.s.
Är n större än m får exponenten negativt
tecken t.ex. för m = 5 och n = 7

as

-=a

-2

al

0

Alla tal upphöjt till noll blir= 1 t.ex. a = i
Med m = n i den föregående formeln får vi

an
an

-=1
an

Vi får också -

an

8-4

=a

n-n

=a

o

r

r

Att upphöja en produkt till en potens görs
så, att var och en av faktorerna upphöjs
till potensen, varefter resultaten multipliceras med varandra.
3

am= a 5 =a· a· a· a· a
an = a 3 = a. a. a

2

r

r

(O.OOOOOi ).

1

T.ex. 10$\circ$ = 1 ingår i serien 10- , i0$\circ$, 10 , 10
.... ,vilket är ett annat sätt att skriva O, 1, 1, 1O,
100 .... etc
Uttrycket a. bn betyder att a skall multipliceras med "b upphöjt till n".
Uttrycket (a· b betyder att a och b skall
multipliceras med varandra n gånger:
(a. b = (a . b) . (a. b) . .. o. s. v. n gånger.
l det senare fallet kan parenteserna slopas,
utan att resultatet förändras:
(a. b = a. b. a. b . . . o. s. v. n gånger.
Samlar vi alla a respektive alla b var för sig
fås an . bn = (a. b
Skilj noga mellan abn = a. bn
och
(abr = (a·br

3

3

Exempel1 . (4 · 5) = 4 · 5 = 64 · 125 =8000
2
2
2
2
Exempel 2. (3a) = 3 · a = 9a
På samma sätt kan ett bråk upphöjas till en
potens genom att upphöja täljaren och
nämnaren
a a a
an
b =b·b·b ... o.s. v. n gånger= bn

a)n
(
2)

(3

3

2 2 2- 2

3

8

-3'3'333 27
-

-

Rötter
Roten ur ett tal är den faktor, vars kvadrat är
talet.
Tidigare behandlade vi uttrycket i = 900 för
att få fram värdet på x. Vi "drog kvadratroten
ur 900".
Tecknet
kallas rottecken.

r-

x= ~900 skall egentligen skrivas !\ /900,
men tvåan brukar uteslutas när det gäller
kvadratroten. l övriga fall är det nödvändigt
att skriva ut rottermen, t. ex. ~1 00 (uttalas
som 3:e roten ur) eller ~ (uttalas som
6:e roten ur).
Kom ihåg följande allmänna regler

~=-fä·{b

och

~={a
~b {b

APPENDIX
Den första regeln förenklar dragning av roten ur stora tal,
~ 1225 = ~ 25. 49 = {25. {49 = 5. 7 = 35
Ett större tal kan alltså delas i flera mindre tal,
vars respektive rotvärden är lättare att få
fram. Rotvärdena kan erhållas ur matematiska tabeller eller miniräknare.
Roten ur tal kan blir ändlös, t. ex.
{2. = 1. 414.... och {3 = 1. 732....

logaritmer

Beräkningar kan göras enklare med användning av logaritmer.
Först studerar vi följande tabell över digniteter av talet 2,
4
9
t.ex. 2 = 16 och 2 = 512.
Produkten eller kvoten av tal kan beräknas
med addition respektive subtraktion sedan
talen omvandlats till exponentiella tal med
samma bas.
Exempel
4
9
4 9
13
1) 16. 512 = 2 .2 =2 + = 2 = 8192
2) 2048 =~=211-6 =25 =32
64
26
Här är sambandet mellan exponent och dignitet för basen 2:
Expo- DigniExpo- Digninent
tet
nent
tet

1
2
3
4
5
6
7

8
9

2

4
8

4

16 (=2 )
32
6
64 (=2 )
128
256
9
512 (=2 )

10
11
12
13
14
15
16
17
18

1 024
11
2 048 (=2 )
4 096
13
8 192 (=2 )
16 384
32 768
65 536
131 072
262 144

En sådan tabell har emellertid begränsad
användbarhet vid behandling av godtyckliga
taluppställningar. Begreppet logaritm är däremot mera användbart.

Med logaritmen för ett tal menas den
exponent, som basen skall upphöjas till,
för att potensens värde skall bli talet.

B

Exempel
l ekvationen 2x = 31 säger man att x är
logaritmen för talet 31 i det logaritmsystem,
vars bas är 2.
Detta skrivs x= 2!og 31 och läses x= tvålogaritmen för talet 31 .
2
Kvadraten på talet 1Oär 100, d.v.s. 10 =
100. Talet 2 är alltså den exponent som talet
1Oskall upphöjas med för att digniteten skall
bli 100. Således 1Olog 100 = 2
Vid omvandling mellan decimala tal och
deras logaritmer används s.k. logaritmtabeller eller miniräknare (inte de allra enklaste). För överslagsberäkningar används
även diagram och skalor (t. ex. räknestickan).
Så här räknar man med logaritmer:
När decimala tal skall multipliceras med varandra, omvandlar man dem först till logaritmer. Man adderar dessa och återvandlar
resultatet till decimala tal igen.
När decimala tal skall divideras med varandra, omvandlar man dem först till logaritmer. Man subtraherardessaoch återvandlar
resultatet till decimala tal igen.
Exempel
Talen 100, 100, 100,2 och 2 skall multipliceras med varandra.
Det decimala förfarandet är:
6
100. 100. 100 . 2. 2 = 4000000 = 4. 10
Förfarandet med logaritmer är att manomvandlar talen till deras respektive 1Ologaritm, vilken är 1Olog x, varefter logaritmerna adderas. Dessa räkneoperationer kan
göras t.ex. med en miniräknare. Då fås
2 + 2 + 2 + 0.301 03 + 0.30103 = 6.60206 som
är summan av logaritmerna för talen.
För att uttrycka svaret som ett decimalt
tal omvandlas logaritmen till antilogaritm,
6 60206
6
~ 4 · 10
vilken är 1Ox= 10 .
(samma somvid det decimala förfarandet).
Skulle talen ha dividerats så skulle deras
respektive logaritmer ha subtraherats från
varandra i stället.
Här är sambandet mellan en serie decimala tal och deras 1O-logaritmer (1 Olog x).

B-5

APPENDIX
Antilogaritmen
för tal med
10-bas och
exponenten
x d.v.s. (i Ox)
1.00
1.25
i.6
2.0
2.5
3.2
4
5
6
7
8
9
iO
20
30
50
iOO
500
1 000
5 000
10 000
100 000
1 000 000

8-6

~©

Dignitet

Logaritmen
för 1Olog x
(avrundade
tal)

1.00 . 10$\circ$
1.25 . 10$\circ$
1.6. i 0$\circ$
2.0. 10$\circ$
2.5. 10$\circ$
3.2. 10$\circ$
4. 10$\circ$
5. 10$\circ$
6. 10$\circ$
7. i0$\circ$
8. i0$\circ$
9. 10$\circ$
1
i .i0
1
2. i0
1
3. 10
1
5. i0
2
i .i0
2
5. i0
3
i . 10
3
5. 10
4
i . 10
5
1 . 10
6
1 . 10

0.00
0.097 ~O. i O
0.204 ~ 0.20
0.301 ~ 0.30
0.398 ~ 0.40
0.505 ~ 0.50
0.602 ~ 0.60
0.699 ~ 0.70
0.778 ~ 0.80
~ 0.85
0.903 ~ 0.90
~ 0.95
i .00
1.301 ~ 1.30
1.477 ~ 1.50
1.699 ~i .70
2.00
~ 2.70
3.00
~3.70

4.00
5.00
6.00
