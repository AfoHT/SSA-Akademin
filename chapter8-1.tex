\emph{I forskning, utveckling och produktion är mätning en hörnpelare
  i verksamheten. Även inom mättekniken sker en snabb utveckling och
  digitaltekniken kommer alltmer till användning, men grunderna för
  mätning är desamma. I detta kapitel behandlas de viktigaste
  mättekniska begreppen som radioamatörer kan behöva känna till.}

\section{Att mäta}
\textbf{
HAREC a.\ref{HAREC.a.8.1.1.1}\label{myHAREC.a.8.1.1.1}
}

\subsection{Mäta likspänning}

Vid spänningsmätning bestämmer man potentialskillnaden - spänningen -
mellan två punkter. Om det finns en spänning, så flyter en motsvarande
(mät)ström genom instrumentetinstrumentet presenterar mätströmmen som
spänning.

Mätströmmen påverkar emellertid spänningsfördelningen i kretsen och då
uppstår ett mätfel, vilket inte framgår av det visade mätvärdet. Med
kännedom om kretsens och instrumentets data kan man dock beräkna
mätfelet. En voltmeter skall ha hög inre resistans för att mätfelet
skall bli litet.

Endast vid mycket noggrann mätning kan man behöva räkna om det visade
mätvärdet med hänsyn till voltmeterns inre resistans och
förkopplingsresistansen - om en sådan används).

\emph{På grund av den höga inre resistansen är en voltmeter endast
  lämpad för spänningsmätning - INTE för direkt strömmätning!}

\subsubsection{Utöka mätområdet för en voltmeter}

Bild II 3-1

Med hjälp av förkopplingsresistor i serie med voltmetern kan man mäta
högre spänning än den som voltmetern är gjord för.  Spänningen
fördelas då proportionellt mellan förkopplingsresistorns resistans och
instrumentets inre resistans.

När förkopplingsresistor används måste mätvärdet räknas om med en
skalfaktor eller en skala med motsvarande gradering användas. En
voltmeter med valbar förkopplingsresistor kan därför har flera
skalor. I digitala voltmetrar anpassas ``skalan'' oftast automatiskt.

\subsection{Mäta likström}

Vid strömmätning bestämmer man strömstyrkan i en gren av en elektriskt
strömkrets.  Amperemetern skall kopplas i serie med den aktuella
strömgrenen.  Det visade mätvärdet motsvarar strömstyrkan.
Amperemeterns inre resistans adderas emellertid till resistansen i
strömgrenen och då uppstår ett mätfel.  En amperemeter skall ha låg
inre resistans för att mätfelet skall bli litet.

Endast vid mycket noggrann mätning kan man behöva räkna om det visade
mätvärdet med hänsyn till amperemeterns inre resistans och resistansen
i strömshunten-om en sådan används.

\emph{På grund av den låga inre resistansen skall en amperemeter
  ALDRIG användas för spänningsmätning. Då förstörs den!}

\subsubsection{Utöka mätområdet för en amperemeter}

Bild II 3-2

Med en strömshunt (en resister parallellt) över amperemetern kan man
mäta högre ström än den som amperemetern är gjord för.

Shunten dimensioneras så att större delen av strömmen leds förbi
amperemetern.  Kvar är den mätström som behövs för att amperemetern
skall göra fullt utslag.

Mätströmmen fördelar sig omvänt proportionellt till instrumentets och
shuntens resistanser.

När en strömshunt används måste mätvärdet räknas om med en skalfaktor
eller en skala med motsvarande gradering användas. En amperemeter med
valbar shuntresister kan därför har flera skalor. I digitala
amperemetrar anpassas ``skalan'' oftast automatiskt.

\subsection{Mäta växelspänning och växelström}

Grunderna för mätning av växelspänning och växelström är samma som för
likspänning och likström, men att bl.a. en instrumentlikriktare oftast
behövs.

Beroende på frekvensen i strömkretsen och vilket slags värde man vill
mäta, används olika instrument.

Olika typer av instrument ger olika möjligheter, men också
begränsningar.

\emph{Mjukjärnsinstrument} utan likriktare kan mäta växelströmmar ner
till c:a 50 mA och upp till c:a 10 A. Frekvensen får dock inte vara
högre än c:a 100 Hz.

\emph{Vridspoleinstrument} används dels direkt för likströmsmätning
och dels med likriktare även för växelströmsmätning.

Vridspoleinstrument med likriktare används ofta för frekvenser upp
till c:a 10 kHz och strömmar ner till 0.1 mA. Noggrannheten är sällan
bättre än 1.5\% av fullt utslag.

Beroende på funktionsprincipen kan det skilja på hur instrument mäter,
vilket nödvändigtvis inte är detsamma som hur mätvärdet presenteras.

Mjukjärnsinstrument mäter effektivvärdet av en växelström medan ett
vridspoleinstrument med likriktare mäter likriktade medelvärdet. Som
exempel kan skalan i ett instrument med likriktare även graderas för
effektivvärdet för sinusformade förlopp.

För mätning av växelström används vanligen instrument med likriktare,
men för HF även instrument med termokors, vilka bygger på
termogalvanisk spänning mellan metaller.

\subsubsection{Frekvensens inverkan}
\textbf{
HAREC a.\ref{HAREC.a.8.1.2.1}\label{myHAREC.a.8.1.2.1}
}

Frekvensen på den mätta signalen inverkar mer eller mindre på
mätresultatet. Till en del beror det på den instrumenttyp, som
används. En faktor är instrumentets gränsfrekvens, d.v.s. hur högt i
frekvens som instrumentet fortfarande är rimligt rättvisande. Detta
kallas instrumentets bandbredd, vilken bör vara dokumenterad.

\subsubsection{Vågformens inverkan}
\textbf{
HAREC a.\ref{HAREC.a.8.1.2.2}\label{myHAREC.a.8.1.2.2}
}

Även formen på den signal som mäts inverkar på mätresultatet och det
är viktigt att veta för vilken vågform som instrumentet presenterar
mätvärdet. Det vanligaste är att vågen förutsätts vara sinusformad,
vilket ofta inte är fallet i praktiken. Det innebär att fel värde
presenteras om vågformen är en annan än den förutsatta.

\subsection{Mäta resistans}
\textbf{
HAREC a.\ref{HAREC.a.8.1.3}\label{myHAREC.a.8.1.3}
}

Mätning av resistans är enklast att göra på en fristående komponent,
medan man vid mätning på en resister i en strömkrets också måste ta
hänsyn till att andra komponenter i kretsen kan påverka mätresultatet.

Resistans kan mätas på flera sätt. Det grundläggande är att mäta
strömmen genom resistorn och spänningen över den och sedan beräkna
resistansen med Ohms lag.

Därutöver finns direktvisande instrument för resistansmätning -
s.k. ohm-metrar. Sådana instrument innehållervanligen en egen
strömkälla i form av ett batteri.

$>>>>>$ TODO: Här borde vi främst beskriva den moderna DMM:en och hur man väljer en

\subsection{Mäta effekt}
\textbf{
HAREC a.\ref{HAREC.a.8.1.4}\label{myHAREC.a.8.1.4}
}

Effektformler vid lik- och växelström (medel-, effektiv- och
toppvärden)

Vid likström:

\begin{gather*}
P = U \cdot I \quad \text{[W (watt)]} \\
\text{d.v.s. Joules lag}
\end{gather*}

Vid sinusformad växelström:

\begin{align*}
  \text{medelvärde   } \quad P_{medel} &= \frac{0.8\cdot U_{eff}^2}{R} \\
  \text{effektivvärde} \quad   P_{eff} &= \frac{U_{eff}^2}{R} \\
  \text{toppvärde}     \quad   P_{PEP} &= \frac{U_{max}^2}{R} \\
\end{align*}

\subsection{Sändareffekt}

En sändares effekt kan mätas på olika sätt.  Förr, då radioamatören
hade små möjligheter att mäta uteffekt, så var det naturligt att
föreskrifterna angav en mätmetod baserad på ineffekt, vilket var
enkelt och rättvisande förtelegrafi m.fl. sändningsslag med bärvåg.
Även för SSB fick effektmätning göras så, trots att resultatet var
långt ifrån rättvisande.

Numera är instrument som mäter \emph{uteffekt} mer tillgängliga för
radioamatören - även toppvärdeskännande sådana för mätning av
p.e.p. Mot den bakgrunden anges nu (1997) i föreskrifterna
sändareffekten som uteffekt.  Därvid måste även p.e.p. avses, fastän
det inte uttryckligen uttalas.

För N-licensen föreskrivs fortfarande att uteffekten uttrycks i
e.r.p. N-amatören kan dock endast i undantagsfall mäta e.r.p. utan den
måste beräknas. Bakgrunden till detta unika myndighetsbeslut var att
se som en anpassning till gällande regler för begränsning av
radiostrålning.

$>>>>>$ TODO: Ta bort texten om N-licensen

Observera, att radioamatören måste beakta EMC-lagen. Se vidare kapitel
\ref{EMC-lagen}.

\subsection{Metoder för mätning av sändareffekt}

Tidigare har avhandlats effektberäkning i allmänhet. Här nedan
kommenteras mätning av sändareffekt i synnerhet.

Ett tillförlitligt sätt att mäta sändareffekt är att ansluta sändaren
till en konstlast med samma resistans som sändarens utgångsimpedans
och mäta spänningen över lasten med ett osci!loscop med tillräcklig
bandbredd. Då kan man se och mäta HF-spänningens topp-toppvärde och
samtidigt se signalens vågform.

Med spänningen och konstlastens impedans (resistans) bekanta så kan
uteffekten beräknas enligt formlerna på förra sidan

Den största HF-amplitud som uppstår momentant vid modulering motsvarar
PEPeffekten (PEP= Peak Enveiope Power).

En mindre exakt metod att mäta HF-spänning är med voltmeter med
likriktare.  Utifrån den uppmätta spänningen kan man beräkna effekten
över en belastning. På grund av instrumentets tröghet visas emellertid
bara ett ``utjämnat'' toppvärde, vilket inte är det faktiska värde som
instrumentet ``känner''. Jämför med oscilloskopet som inte har denna
visningströghet.

Bild II 8-1

Bilden visar en voltmeter med likriktare, som kopplats till en sändare
över spänningsdelare. Två alternativa delare visas; den ena består av
resistorer och den andra av kondensatorer.

Den resistiva delaren är bättre i den meningen att den är
frekvensoberoende och inte belastar sändaren kapacitivt. Dessutom
dämpas övertoner som bildas vid likriktningen. I den kapacitiva
delaren kan övertoner passera lättare.

Denna mätmetod är noggrann bara när impedansen är lika i sändaren,
kabeln till lasten och själva lasten. Lasten kan vara en konstlast, en
antenn etc och skall ha ett känt värde för att effekten skall kunna
beräknas.

Ett sätt att skaffa underlag för beräkning av PEP-effekten är att mäta
HF-strömmen med ett termokorsinstrument och spänningen med en
toppvärdesvisande voltmeter.  Utifrån dessa värden beräknar man
effekten. Denna metod är dock inte så vanlig.

\subsection{Direktvisande effektmetrar}

Bild II 8-6

Många föredrar direktvisande effektmätare.  En HF-voltmeter kan
givetvis graderas för att visa effekt i stället för spänning, men då
med den viktiga förutsättningen att impedansen måste ha en fastställt
värde.

Om man avläser effekten genom en 75 Ω-kabel på ett instrument för 50
Ω, så är det verkliga värdet ett annat än den avlästa.

De effektmetrar som förekommer i SVF-instrument är egentligen
voltmetrar, men med skalan graderad i effekt.

Bild II 8-1 Mätning av sändareffekt

\subsection{Mäta ståendevågförhållande - SVF}
\textbf{
HAREC a.\ref{HAREC.a.8.1.5}\label{myHAREC.a.8.1.5}
}
\label{mäta ståendevåg}

När t.ex. en antennledning ansluts till en antenn och deras impedanser
inte är lika, så kommer en del av inmatade effekten i ledningen att
reflekteras tillbaka från antennen.

Det uppstår då en stående våg i ledningen. Förhållandet mellan inmatad
och reflekterad effekt uttrycks som ett ståendevågförhållande SVF
(eng. SWR).

Med en SVF-meter som sätts in mellan effektkälla och ledning kan man
mäta hur stor effekt som matas in i ledningen och hur stor effekt som
vänder tillbaka från slutet av ledningen.

SVF-värdet kan bestämmas på något av följande sätt:
\begin{itemize}
\item Man mäter framåt- respektive bakåtgående effekt var för sig med
  en riktningskänslig effektmeter. Man beräknar därefter SVF eller
  tar fram det ur ett diagram.
\item Man använder ett instrument som beräknar eller visar SVF på
  något sätt.
\end{itemize}

\subsection{Studera vågformen}
\textbf{
HAREC a.\ref{HAREC.a.8.1.6}\label{myHAREC.a.8.1.6}
}

Vågformen för snabba växelströmsförlopp studeras bäst med oscilloskop.

\subsection{Mäta frekvens}
\textbf{
HAREC a.\ref{HAREC.a.8.1.7}\label{myHAREC.a.8.1.7}
}

Frekvensmätning gör man bäst med en s.k.  frekvensräknare, som är ett
digitalt instrument.  Man kan också använda en
s.k. absorbtionsvågmeter, som är mycket enkel och inte alls så exakt.
Vid frekvensmätning ansluter man instrumentet till mätobjektet med en
svag elektrisk eller magnetisk koppling.

\subsection{Mäta resonansfrekvens}
\textbf{
HAREC a.\ref{HAREC.a.8.1.8}\label{myHAREC.a.8.1.8}
}

Mäta resonansfrekvensen för en passiv svängningskrets gör man enklast
med en s.k. dip-meter. Även mer exakta metoder finns.

$>>>>>$ TODO: Komplettera med spec, SNA och VNA

\subsection{Mätfel}

Mätinstrument indelas i noggrannhetsklasser efter största tillåtna
felvisning. Klasserna är 0.1, 0.2, 0.5, 1.0, i .5, 2.5 och 5.0 varvid
klassen anges på instrumentet. Som exempel får ett instrument i klass
2.5 ha ett tillåtet mätfel av \(\pm\)2.5\% av fullt utslag.

Mätresultatet bestäms av flera faktorer; dels av instrumentets
s.k.mätonoggrannhet, dels av hur mätvärdet presenteras och slutligen
av hur noga användaren läser av.

Vid \emph{analog} visning presenteras mätvärdet med en visare mot en
graderad skala med en viss upplösning. Visaren kan vara mekanisk eller
optisk (ljusspalt). Vid snabba mätvärdesändringar är instrumentets
mekaniska tröghet en faktor att ta hänsyn till.

Vid \emph{digital} visning presenteras mätvärdet med siffror eller som
längden på en pelare.  Det är förledande att se digital visning med
siffror som mer exakt än analog, men det är inte alls säkert. Utöver
instrumentets mätonoggrannhet, bestäms nämligen noggrannheten av hur
många siffror som mätresultatet presenteras i.

En oberäknelig källa till mätfel är elektromagnetiska fält från
apparater i närheten.

En ofta förbisedd felkälla är temperaturen i mätobjektet och/eller i
instrumentet, det kan vara av inkopplingstiden m.m.

\emph{Visningströgheten} är inget mätfel i sig men kan till nackdel
vid snabba förlopp.  Trögheten förekommersåväl vid analog som digital
visning. I det förra fallet är masströghet i instrumentets rörliga
delar orsaken och i det andra fallet är orsaken klockfrekvensen för
instrumentets mikroprocessor.

