\section{Superheterodynmottagare}

Superheterodynprincipen ger mycket större möjligheter, när önskemålet är en högselektiv mottagare för flera olika frekvenser.
Skillnaden mellan en direktblandad mottagare och en "super" är, att blandningsprodukterna i direktblandaren blir till LF direkt, medan de i supern först bildar en
mellanfrekvenssignal MF, vilken sedan
dernoduleras och blir till LF- detekteras.
l det följande kallas superheterodynmottagaren enbart SUPER. l supern blandas de mottagna signalerna med signalen
från en VFO. Före blandningen har HFsignalerna passerat ett selektivt försteg, som
dämpar spegelfrekvenser. För att inte störa
mottagningen placeras VFO-frekvensen alltid utanför det frekvensband, där man vill ta
emot signaler.
Bild 114-13
Alla mottagna signaler blandas med VFOsignalen. Mottagningsfrekvensen är vanligen skillnaden mellan en fast s.k. mellanfrekvens MF och VFO-frekvensen. Mellanfrekvensen är egentligen mittfrekvensen i
ett fast passband skapat av ett antal filter.

=455 kHz
M F-filter

DetektorIdemodulator

j~sfilter

fast avstämt bandpassfilter
VFO

fvro :: 4055 kHz

demodulator

LF-Iågpassfilter

LF

fMF = fvFo - fM (VFO-frekv. över mottagn.frekvensen)
eller
fMF = fM - fvFo (VFO-frekv. under mottagn.frekvensen)
Hög selektivitet, enkel avstämning {jämfört med en rak mottagare)

Bild II 4-13 Superheterodynmottagaren i princip

114-9

M TTA ARE
\subsection{Dubbelsuperheterodynmottagare}
Bild 114-14
Det är svårt att bygga enkla mellanfrekvensfilter för höga frekvenser, med liten bandbredd och branta flanker. Det är fallet för en
enkelsuper för kortvåg med en enda mellanfrekvens, t.ex. 9 MHz.
En god närselektion på höga frekvenser
är endast möjlig med relativt dyrbara kristallfilter. Däremot går det att få god närselektion
med enklare medel på lägre frekvenser.
En dubbelsuper, d.v.s. en super med
dubbel frekvensomvandling, möjliggör god
både när- och förselektion. l 1 :a blandaren
blandas den mottagna signalen med signalen från en 1 :a oscillator (VFO) till en hög
mellanfrekvens, t. ex. 9 eller 10.7 MHz.
Därmed kan en god spegelfrekvensdämpning erhållas. Första M F-filtret kan göras enklare och utan den höga selektivitet
som hade behövts i en enkelsuper. 1:a MF
blir sedan blandad ytterligare en gång i 2:a
blandaren till en 2:a MF, t.ex. 455kHz. För
den andra blandningen används en fast
oscillator. Filtret i 2:a MF kan lättare utföras
med en hög selektivitet, p.g.a. den lägre
frekvensen.
Exempel:
Trots att M F-filtret inte är en enkel
svängningskrets, kan ett "Q-värde" beräknas. Vid en passbandbredd av 6kHz och en
centerfrekvens av 455 kHz kan Q-värdet
anses vara

Bild II 4-13 visar en mottagare med mellanfrekvensen 455kHz, som är vanlig i äldre
mottagare. MF-filtret kan i enklaste fall bestå av ömsesidigt magnetiskt kopplade LCsvängningskretsar. Bättre avstämningsskärpa fås med resonatorer av keramik eller
kvarts eller de är elektromekaniska.
Exempel:
En sändning på frekvensen 3600 kHz
skall tas emot. Vi ställer då in VFO-frekvensen till 4055 kHz, eftersom mellanfrekvensen är 4055 - 3600 = 455 kHz. Den
mottagna signalen hamnar då mitt i MFfiltrets passband.
Signaler på angränsande frekvenser tas
också emot och alstrar blandningsprodukter.
Med ett mellanfrekvensfilter med t. ex. 3kHz
bandbredd (453.5-456.5 kHz), kan signalfrekvenser mellan 3598.5 och 3601.5 passera genom filtret. En signal med en närliggande frekvens t. ex. 3603kHz, och blandad
med den inställda VFO-frekvensen 4055
kHz, kommer att alstra en skillnadsfrekvens
av 452 kHz. Denna signal ligger utanför
filtrets passband och kommer att dämpas
och når inte detektorn.
VFO-signalen kan givetvis läggas under
i stället för över mellanfrekvensen.
Exempel: VFO-frekvensen 3145kHz kan
också användas för mottagning av frekvensen 3600 kHz, om mellanfrekvensen är 455
kHz (3600- 455 =3145kHz). Men för att
undvika att eventuella övertoner från VFOsignalen blandas med mottagna signaler är
det lämpligt att placera VFO-frekvensen över
mottagningsfrekvensen.
Efter M F-filtren följer bl.a. detektorer för
olika sändningsslag samt LF-förstärkare.
Jämför med Bild II 4-5 och Il 4-6

Q= f,es

b

CJ

Bild II 4-14 Dubbelsuperheteodynen i princip

o

6

l ett MF-filter med centerfrekvensen 9
MHz skulle det behövas ett nära 20 gånger
högre Q-värde för samma bandbredd6kHz

I

114- 1

= 455 =76

ARE
Q= ~es

b

= 9000 = 1500
6

Ett så högt Q-värde kan endast erhållas
med kristallfilter.
För högre mottagningsfrekvenser räcker
det, på grund av filterproblematiken, oftast
inte med en dubbel frekvensomvandling,
Om man antar en dubbelsuper-mottagare
för VHF-området 144-146 MHz enligt bilden, så skulle en i :a MF med frekvensen
i 0.7 MHz inte vara tillräckligt hög. Vid en
mottagningsfrekvens av 146 MHz är nämligen spegelfrekvensen i 46 + (2 • 1O. 7) =
i 67.4 MHz, alltså endast i .15 gånger mottagningsfrekvensen. Det hade alltså varit
lämpligt med en trippelsuper, d.v.s. en trefaldig frekvensomvandling, med en i :a MF
i frekvensområdet 70 MHz.
