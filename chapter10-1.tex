\section{Människokroppen}

\subsection{Elektrisk chock}

Människokroppen är ett komplicerat elektrokemiskt system, som främst
kontrolleras av hjärnan. Musklerna styrs av svaga elektriska
strömimpulser genom nervsystemet. Främmande strömmar genom kroppen kan
störa kroppsfunktioner och kan i olyckliga fall göra stor
skada. Styrkan och frekvensen på strömmarna avgör skadans art och
omfattning.  Elektrisk chock kan döda av flera orsaker.

En orsak är att hjärtrytmen störs. Hjärtkammarflimmer och
hjärtstillestånd kan lätt uppstå. Flimmer innebär att hjärtat arbetar
okontrollerat och med kraftigt nedsatt eller helt upphävd
pumpfunktion. Hjärtstillestånd inträffar lätt av hög spänning. Av
otillräcklig blodtillförsel blir det syrebrist i hjärncellerna, som då
skadas snabbt. Medvetslöshet inträder redan efter ett fåtal sekunder.

En annan orsak är andningsstillestånd genom att andningscentum
blockeras. Det kan hända när strömmen från en högspänningskondensator
i en sändare går genom kroppen.

\subsection{Resistansen genom människokroppen}

Vid kontakt med ett strömförande föremål kommer kroppen att bli en del
av strömkretsen. Det flyter då en främmande ström genom kroppen.

Strömstyrkan följer Ohms lag och beror av strömkällans spänning och
inre resistans samt av övergångsresistansen i huden och kroppens inre
resistans.

Övergångsresistansen minskar med fuktigare hud samt med större
kontaktyta och större kontakttryck. Beröringsspänningen inverkar
också. Vid spänningar över ca 75 V minskar övergångsresistansen med
ökande spänning. Vid allvarliga förbränningar minskar
övergångsresistansen särskilt mycket.  Den totala resistansen genom
kroppen blir då nära lika med dess inre resistans - ungefär 500 Ω.

\textbf{VARNING. Experimentera inte med detta!}


\subsection{Strömmens inverkan på människan}

Strömstyrkan påverkar kroppen olika från fall till fall och det är
osäkert vilken strömstyrka som är farlig. Det finns både de som
överlevt höga strömmar och de som inte har klarat några
milliampere. Strömmar som går genom hjärta eller hjärna är särskilt
farliga.  Hjärtat sitter i strömvägen för vänster hand, så när man
arbetar med elektriska apparater under spänning, bör man för säkerhets
skull hålla vänstra handen i fickan!

Starka strömmar ger häftiga muskelkramper och/eller
brännskador. Muskelkramp kan förekomma redan vid strömmar under 10
mA. För vuxna, friska människor är det direkt farligt när strömmen
överstiger detta värde.  För unga eller sjuka redan vid lägre värden.

Men även en ``ofarlig'' ström kan trots allt vara ett indirekt
faromoment. Vid en oväntad ``stöt'' blir man rädd och gör okontrollerade
rörelser, vilket kan leda till fall eller oavsiktlig beröring av
spänningsförande föremål i närheten.

\subsection{Påverkan från elektromagnetiska fält}

Undersökningar harvisat attvistelse i starka elektromagnetiska fält
kan kan påverka människan. Personer som har varit utsatta för kraftig
exponering av fält har bl.a. klagat över svettningar och
huvudvärk. Det forskas omkring dessa fenomen.

Elektromagnetiska fält kan förorsaka fel i
elektronikutrustningar. Halvledare är särskilt känsliga för
kraftfält. Det är möjligt att känsliga instrument, hjärtstimulatorer
(pacemaker) etc. kan påverkas av högfrekventa elektromagnetiska
fältfrån radiosändare. När du använder en sändare, mobiltelefon etc.
och någon får svårigheter med hjärta eller andning så skall du
omedelbart stänga av din apparat helt! Med tiden utvecklas
störningsokänsligare elektronik, men säker mot störningar kan man
aldrig vara.

\subsection{Normer för fältstyrkor}

Det finns normer och rekommendationer för elektromagnetiska
fältstyrkor i olika syften.

Dels avses fältstyrkor som människor och djur får utsättas för, dels
fältstyrkor som olika slags apparater skall kunna fungera i respektive
själva utsänder (EMC).

Samarbete mellan länderna utvecklas inom båda dessa områden.

\subsection{Konstgjord andning}

Vid hjärtstillestånd, hjärtkammarflimmer och andningsstillestånd skall
hjärtmassage och konstgjord andning sättas in omedelbart och på ett
kunnigt sätt. Obotliga hjärnskador av syrebrist kan nämligen uppstå
inom några få minuter.

\emph{Livräddning vid elskada} är ett instruktivt häfte från
Energikontorets Förlagsservice, 101 53 Stockholm. Studium
rekommenderas.  Det är för sent när olyckan har skett. Häftet kan
beställas på fax 08 677 26 05.

