\section{Speciella mottagare - Panoramamottagare}

Bild II 4-15

I en panoramamottagare visas på en oscilloskopskärm var det finns
signaler inom ett frekvensband. En panoramamottagare är en
superheterodyn.

Mottagaroscillatorn är en VCO (spänningsstyrd oscillator). Dennas
frekvens styrs av en sågtandformad likspänning, som stiger linjärt för
att snabbt falla tillbaka och återupprepas. VCO sveper då över det
önskade frekvensbandet med ett antal gånger gånger per sekund. Med
samma sågtandspänning avlänkas strålen på skärmen utmed x-axeln. Bild
II 4-16

Den mottagna signalen dernoduleras och översätts till en likspänning
som skildrar de mottagna signalernas styrka. Med denna likspänning
avlänkas strålen på bildskärmen utmed y-axeln. Strålens avstånd från
x-axeln anger alltså den mottagna stationens styrka och strålens läge
utmed x-axeln anger var stationen ligger i det frekvensområde som
avsöks. Beroende på hur stort frekvenssving som ges VCO, så kommer ett
större eller mindre frekvensområde att avsökas och visas på
skärmen. Området kan vara så brett som ett amatörband eller mer och
ner till några få kHz.

Utöver övervakning av ett frekvensband kan en panoramamottagare
användas för studium t. ex. av signaler och sidefrekvenser som alstras
i den egna stationen. För noggranna mätningar behövs emellertid ett
hjälpmedel av högre kvalitet, kallat spektrumanalysator. En sådan
arbetar i grunden på samma sätt som en panoramamottagare.

Bild II 4-16

En panoramamottagare kan anslutas till en mottagare för att studera
signalerna inom MF-passbandet. Då är mottagningsfrekvensen i
bildskärmens mitt. stationerna under och över i frekvens visas till
vänster respektive höger om den egna frekvensen.

Vid ändrad mottagningsfrekvens blir denna fortfarande kvar mitt på
skärmen.

Bild II 4-17 Signal- och svepspänningar

Bild II 4-15 Panoramamottagare

Bild II 4-16 Anslutning av panoramamottagare till stationsmottagare
