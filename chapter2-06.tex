\chapter{ANTENNSYSTEM}

Aldrig så förnämliga radioapparater kommer inte till sin fulla rätt utan ett effektivt
antennsystem. Det är en huvudförutsättning för framgångsrik radiokommunikation.
Antennen omsätter elektrisk energi från
sändaren till elektromagnetiska fält som strålas ut, d. v. s. radiovågor.
Vid mottagning fångar antennen upp
radiovågorna och omsätter dem till elektriska signaler som förs till mottagaren.
Antennsystemet består av den egentliga
antennen och transmissionsledningen mellan denna och sändaren respektive mottagaren. l antennsystemet ingår även impedansanpassningar, antennkopplare m. m.

Antenner- allmänt
Våghastighet
l vakuum breder elektromagnetiska vågor
ut sig med hastig heten c 0 , vilken mest kallas
ljushastigheten.
C0 :::< 300 ·1 Os [m/s]

l andra media än vakuum har samma
vågor utbredningshastigheten c
Formeln är då

c=~

[m/s]
!lo. s,
där !lo är relativa permeabilitetskonstanten
och Er är relativa dielektricitetskonstanten för
det medium som vågorna passerar igenom.
För enkelhetens skull sätts här !lo och er till
1 , alltså c 0 = c.
Sambandet mellan våghastigheten i vakuum, frekvensen och våglängden är förenklat
c= A. f
c [m/s] f[Hz] A [m]
och våglängden således A=

!j

[m]

Antennlängd
Elektriska längden
Längden för en resonant, ideal antenn som
är en våglängd lång kan beräknas med
ovanstående formel. Vi kallar denna längd
för den elektriska längden. Således le= A.

Elektriska längden (le) för en halwågsantenn (A/2) är hälften av den elektriska
längden för en helvågsantenn (A):

l =~

[m]
2f
Mekaniska längden
Man skiljer på antennens elektriska och
mekaniska längd. Av flera orsaker blir den
mekaniska antennlängden (lm) för samma
frekvens kortare än den elektriska {1 6 ). Det
beror bl. a. på våghastighet och ledningsförmåga i de material som ingår samt övriga
elektriska egenskaper beroende på antennens mekaniska utförande, påverkan från
jordplan och omgivning m. m.
Ett förhållande mellan längd och tjocklek
av 10000 ger t. ex. en c:a 2 $\circ$/o mekaniskt
kortare antenn. Förhållandet 30 ger en c:a 5
o/o kortare antenn. Det första värdet kan
passa för en 2 mm tjock halvvågsantenn för
7 MHz. Det andra värdet för en 3.5 mm tjock
halvvågsantenn för 145 MHz. Diagram för
den s.k. förkortningsfaktorn finns i de flesta
ante n nhandböcker.
l följande formel har den mekaniska längden (lm) för en fritt upphängd trådantenn
valts 2 $\circ$/o kortare än den elektriska längden.
e

l
m

=~ .
2f

O 98 = 147 . 1os
.
f

[m]

Exempel: Beräkna den elektriska och mekaniska längden på en halvvågsantenn med
resonansfrekvensen f= 7 MHz.
c= A, . f
c [m/s] f [H z] A [m]
Elektriska våglängden för 7 MHz är

A =E= 300. i os
f
7 ·1 os

= 300 = 42.86

[m]

7

Antennen är en halvvågsantenn, således är
elektriska längden

1 =~ =
e
2

42 86
·

2

= 21.43

[m]

och

mekaniska längden

l

m

= 3:
2f

. 0.98 =

42 86
· · 0.98 = 21

2

[m]

116-1

ANTENNSYSTE
)..

r--2-~~,

~
..........

Ström- och spännings-''-... ...... .!)-fördelning

Bild II 6-1 Spänning och ström i en
halvvågsantenn

l mpedansfördelning

Bild II 6-2 Matningsimpedansen
i en halwågsantenn

116-2

Ström och spänning i en halvvågsantenn
När en halwågsantenn matas med HF-energi på grundfrekvensen, så uppstår en stående våg med ett typiskt utseende.
Bild II 6-1 visar att i vardera änden av
antennen uppnår spänningen U ett maximum (en spänningsbuk), l mitten uppnår
strömmen I ett maximum (en strömbuk).
Antennen strålar mest där strömbuken finns .
Tag t.ex. en 40 meter lång metalltråd
som antenn. Dess grundresonansfrekvens
är ca 3.5 MHz, men den är även i resonans
på de harmoniska övertonerna (7, 14, 21,
28 MHz o.s.v.).
Bild II 6-3 visar ström- och spänningsfördelningen på antennen vid de respektive
övertonerna.
80 m (3.5 MHz):
l matningpunkten står ett spänningsminimum (en spänningsnod) och ett strömmaximum (en strömbuk). Strömmen är hög
därför att matningspunkten har låg impedans.
Samma antenn på 40 m, 20 m, 15 m, 1O
m (7, 14, 21, 28 MHz) har ett spänningsmaximum (spänningsbuk) och ett strömminimum (strömnod) i matningspunkten, som
då har hög impedans.
Ur horisontaldiagrammet för antennen
kan utläsas att ytterligare strålningskäglor
(strålningslober) utvecklas för varje överton
i den påmatade frekvensen. Samtidigt blir
strålningen alltmertill riktad längs med antennen.
Impedansen i antennens matningspunkt
Impedansen Z för varje punkt på en antenn
kan beräknas med Ohms lag Z = U/1.
Bild II 6-2
l mitten av en halwågsantenn på grundfrekvensen är impedansen Z låg eftersom
spänningen är låg där och strömmen hög.
Ute i ändarna är däremot impedansen hög
eftersom spänningen där är hög och strömmen låg.
Impedansen i mittpunkten är 73 Q på
grundfrekvensen, när antennen mätt i våglängder befinner sig mycket högt över jordytan, d.v.s. utan nämnvärd påverkan från
omgivningen. l praktiken kan impedansen
awika mycket från detta värde.

ANTENNSYSTE

3,5MHz {80m)

2:5',

..ooO!-o........
, ......
-------......

-- ---~y?/ 7
......

7 MHz (40m)

~

o<lli-------.. . . .
'

/

Grundfrekvens =
1 :a harmoniska

l::

t

.......

/

2:a harmoniska

l::2·-'}

4:e harmoniska

l= 4
21MHz(15m)
28 MHz (10m)

6

t

6:e harmoniska

l::6·t

Ä"> /'75:'> ..,.-?S> /. .
' . . SZ7 . .,szy ,\ Z./

zs:>. . <.;.(:;;,. -

8:e harmoniska

(::8·f

~---------------40m----------------~

-ström

HORISONTALDIAGRAM FÖR EN
HARMONISKA ÖVERTONER

- - - ... spänning

t-

DIPOL VID MATNING MED

Grundfrekvens

3:e harmoniska

2:a harmoniska

Bild II 6-3 Halwågsdipol matad med harmoniska övertoner

116-3

ANTENNSYSTE
Antenn och matningskabel måste vara
impedansanpassade till varandra för att det
inte skall skall uppstå vågreflexion i anslutningen.
Märk, att halwågsantennen är i resonans inte bara på grundtonen utan även på
jämna övertoner, 2:a, 4:e etc. harmoniska,
varvid matningspunkten har hög impedans.
Vid matning med en lågohmig koaxialkabel
uppstår då en kraftig missanpassning i anslutningen mellan antenn och kabel, vilket
måste åtgärdas på något sätt. Se avsnittet
Transmissionsledningar i detta kapitel.
Matningsimpedansen i några antenner
Med W3DZZ-antennen (se nedan) löses
hjälpligt anpassningsproblemet med mittmatade partier på 2:a harmoniska övertonen, d.v.s. dubbla grundfrekvensen. På 80och 40 m-banden är antennens matningsimpedans c:a 60 .Q och på de högre banden
ca 120 n. En kompromiss är att mata denna
antenn med en 75 n-kabel för att inte få
alltför stor missanpassning på något band.
Den omvikta dipolen (folded dipole):
Matningsimpedansen är c:a 240 n. En
bandkabel med impedansen 300 n kan användas alternativt en koaxialkabel med
impedansen 50 eller 75 n över en transformator med impedansomsättningen 4:1.
Jordplanantennen (GP-antennen):
Matningsimpedansen är 30-60 n. När
jordplanets spröt inte riktas horisontellt, utan
snett nedåt, erhålls en matningsimpedans
av 50 n, vilket passar bra för en koaxialkabel med 50 .Q impedans.
Yagi- och Quad-antenner:
En anpassningsanordning för anslutning
av 50-60 n koaxialkabel ingår oftast i fabriksgjorda riktantenner. En 50-60 n koaxialkabel kan då användas direkt.

Reaktansen i en icke resonant antenn
Den elektriska svängningskretsen behandlas i kapitel 3. Där framställs svängningskretsens grundegenskaper resistans R, induktans L och kapacitans C som koncentrerade till komponenter kallade resistor, induktor respektive kondensator.
116-4

Även en enkel tråd har dessa egenskaper, men utfördelade över hela tråden. Denna
kan därför ses som ett stort antal komponenter, som tillsammans bildar en svängningskrets, vilken naturligtvis kan fungera som
antenn.
När antennen matas med växelström
med samma frekvens som antennens resonansfrekvens, så svänger antennen med de
minsta förlusterna. Resonansfallet kan i
korthet beskrivas så att den induktiva och
kapacitiva reaktansen i antennen tar ut varandra medan resistansen kvarstår.
Impedansen är vektorsumman av resistansen och de kapacitiva och induktiva reaktanserna. l resonans är antennens impedans lika med resistansen, vilket är ett
specialfall. Antennströmmen har alltid sändarens frekvens. Om sändningsfrekvensen
är en annan än antennens resonansfrekvens, så händer endera av följande:
När antennströmmen har lägre frekvens
än antennens resonansfrekvens, så blir den
resulterande reaktansen negativ (kapacitiv),
d.v.s. Xc är större än XL.
När antennströmmen har högre frekvens
än antennens resonansfrekvens, så blir den
resulterande reaktansen positiv (induktiv),
d.v.s. XL är större än Xc.

Elektrisk "förlängning" och "förkortning"
Om sändarfrekvensen, awiker mycket från
antennens resonansfrekvens, så kan
reaktansen i antennen behöva elimineras
eller åtminstone minskas för en bättre
impedansanpassning mellan antenn och
matarledning. Den enklaste åtgärden är då
att försöka ändra antennlängden.
Bild II 6-4. Om detta inte låter sig göras,
så kan man i serie med en "för kort" antenn
sätta in en induktor - en s.k. elektrisk förlängning. Om i motsatt fall antennen är "för
lång", så kan man sätta in en kondensatoren s.k. elektrisk förkortning.
l amatörradio ändras sändarfrekvensen
mycket och ofta, varför antennsystemet bör
kunna stämmas av från marken/operatörsplatsen. Då kan en antennkopplare med
nödvändiga reaktiva komponenter behövas.
Se längre fram i kapitlet.

NSYSTE
dipol (l:::

eller

f)

~~

Dipol, elektriskt förlängd

Dipol, elektriskt förkortad

förkortning av antenner

Anpassning till sändarens impedans
Ett sändars lutsteg med elektronrör är vanligen utrustat med en avstämningsanordning
vid H F-utgången. Syftet är att kunna anpassa sändarens utgångsimpedans till impedansen i antennledningen. l moderna sändare består denna anordning mycket ofta av
ett s.k. n-filter, vars utgångsimpedans kan
variera mellan c:a 30-150 n.
Ett transistoriserat slutsteg är oftast utfört för en fast utgångsimpedans av 50 n
och är alltså i behov av en avstämningsanordning, om inte antennsystemet inom vissa
gränser håller samma impedans. Toleransgränsen för felanpassning brukar vara ett
SVF av storleksordningen 2:1 innan sändarens skyddskretsar styr ner uteffekten.
Vid lika impedans i sändarutgång, matarledning och antennanslutning uppträder
ingen stående våg på matarledningen och
mesta möjliga effekt överförs från sändaren
till antennen.

h ::::

Antennens strålningsdiagram
En antenns strålningsbild beskrivs bäst i tre
dimensioner. Bild II 6-3 visar bl. a. ett horisontaldiagram för en halvvågsantenn.
Bild II 6-5 visar strålningen i vertikalplanet som funktion av antennhöjden för samma
antenn. Vertikaldiagrammet kan ha mycket
olika utseende beroende på antennens utförande, dess elektriska höjd över mark och
omgivningens elektriska egenskaper. För
att överbrygga stora avstånd, måste antennen ha en flack utstrålning relativt markplanet. En horisontelit upphängd antenn med
en längd av ')J2 har övervägande flack utstrålning när den placeras på en höjd av AJ
2, A, 3 A/2, 2 A o.s.v. över mark. När en
horisontell antenn däremot placeras /J4, 3
A/4, 5 'A/4 o.s.v. över mark, är utstrålningen
övervägande vertikal, vilket inte skall förväxlas med polarisationen, som i detta fall är
horisontell.
Samma diagram gäller både för en
sändar-och mottagaranten n. styrkan på en
utstrålad signal motsvaras av styrkan på
mottagen signal.
Antennvinst
Med antennvinst G (eng. gain) menas förhållandet mellan effekten Pt i huvudstrålningsriktningen (framriktningen för en antenn med osymmetriskt utstrålad effekt) och
effekten från en definierad referensantenn.
En referensantenn som tänks vara oändligt liten och som strålar med exakt samma
effekt Pi i alla riktningar kallas isotropisk
antenn.
En isotropisk antenn är emellertid endast
teoretisk definierbar.

=A2

Bild II 6-5 Vertikaldiagram för halvvågsantenn

116-5

ANTENNSYSTE
Med effekten Pi från den isotropiska antennen som referens blir antennvinsten

P.

G= 1O log t
[dBi]
~
En i praktiken definierbar referens är
halwågsdipolen, vars huvudstrålning ärvinkelrätt ut från dipolen och runt omkring den.
Referenseffekten är då Pd och antennvinsten

P.

G=10 logL
pd

Di pol

[dBd]

Bild II 6-6

Bild II 6-6 Antennvinst dBd i effekt
Antennvinsten kan också definieras som
förhållandet mellan den elektriska fältstyrkan
uf i huvudstrålningsriktningen och referensfältstyrkan
(dipol).
Jämfört med A./2-dipol är antennvinsten

ud

u

ud

[dBd]

Ungefärlig antennvinst för olika antenner med en isotropantenn som referens
A./2-dipol
Isotrop
Isotrop antenn
-2.1 dBd O
dBi
GP, A/4
-1.8 dBd 0.3 dBi
Dipol, A/2
O
dBd 2.1
dBi
1.2 dBd 3.3 dBi
GP, 5/8 A
Dipol, 1/1 A
2-elements yagi
2-elements quad
3-elements yagi

Riktantenn

G= 20 log-'

6 dB antennvinst motsvarar en fördubblad fältstyrka [V/m], d.v.s. 1 S-en hets ökning
vid den mottagande stationen, liksom att
6 dB antennvinst motsvarar en 4-faldigad
sändareffekt [W/m 2 ).

Bild II 6-7

1.8

5

6
8

dBd
dBd
dBd
dBd

3.9
7.1
8
10.1

dBi
dBi
dBi
dBi

Effektivt utstrålad effekt

Effektivt utstrålad effekt (ERP - effective
radiated power) är den effekt som sändarantennen strålar ut i sin bästa strålningriktning. ERP beräknas som den effekt som
tillförs själva antennen, multiplicerat med
antennvinsten relativt en halwågsdipol. Förlusterna på vägen från sändaren ut till antennen är alltså borträknad före beräkningen
av ERP.

Fram-/backförhållande (antennvinst)

Di pol

Med fram-/backförhållande (F/B) för en riktantenn menas förhållandet mellan den utstrålade effekten i framriktningen P1 och
effekten i backriktningen Pb

Riktantenn

Bild II 6-7 Antennvinst dBd i spänning
Man använder uttrycket d Bi när antennvinsten anges i förhållande till en isotrop
antenn och d Bd i förhållande till en halvvågsanten n.
Se Appendix C om decibelbegreppet
Exempel på beräkning av antennvinst
U1 = 40 J-tV Ud= 20 J-tV
G=?

u

40
G= 20 log-r = 20 log-=
ud
20
=20 log 2=20·0.3=6

116-6

[dBd]

P.
pb

Fl B= 10 logL

Di pol

[dB]

Bild 6-8

Riktantenn

Bild II 6-8 F/B-förhållande i effekt
Fram/backförhållandet kan också definieras som förhållandet mellan elektriska
fältstyrkan uf i framriktningen och referensfältstyrkan ub i backriktningen

ANTE N
u
ub

Fl B= 20 log-'

Di pol

[dB]

Bild 6-9

Riktantenn

Bild II 6-9 F/B-förhållande i spänning

Exempel1
Ut= 40 ~-LV

Ub = 4 ~-LV

F l B= 20 log

u, = 20
ub

F/B= ?
log

40

=
4
=20 log 10=20·1=20 [dB]

F/B = 20 dB betyder att fältstyrkan Ut i
huvudriktningen är 1O gånger så hög som
referensfältstyrkan Ub.
Exempel2
Ut= 15 ~-LV Ub = 15 ~-LV

F/B= ?
15
Fl B= 20 log-' = 20 log-=
15

u
ub

=20 log 1=20·0=0

F/B= O dB betyder att Ut= Ub, d.v.s. att
fältstyrkorna i fram- och backriktning är lika
stora, vilket inträffar för en dipol.

Halvvärdesbredd
studera diagrammet för den horisontella
strålningen från en riktantenn.
Antennen avger sin största utstrålade
effekt Pt i huvudriktningen. Effekten avtar
utanför huvudriktningen. Fältstyrkan Ut förhåller sig på liknande sätt.
Med effekthalvvärdesbredd menas den
vinkel inom vilken nyttaeffekten är minst
hälften så stor som i huvudriktningen.
Bild II 6- i O
p
Observera, att
motsvarar ~

[T

d

2 u,

( ~ 0,7 Ut motsvarande 3 dB).
Med spänningshalvvärdesbredd menas
den vinkel inom vilken spänningen (fältstyrkan) är minst hälften så stor som den
största nyttaspänningen Ut. Spänningshalvvärdesbredden på en di pol är ungefär 90$\circ$.
Bild 116-10

[dB]

..öppningsvinkel (X

Umax

u
Effekthalwärde

,Öppningsvinkel

\

\

/

/

\c-6d8
0,5
J

Spänningshalwärde

l

/j;

Umax

u

l

Bild II 6-1 O Halvvärdesbredder

116-7

ANTE N
Vågpolarisation
Se även i kapitlen 1 och 7. Här nämns något
om polarisation vad gäller radioantenner.
En elektromagnetisk våg är sammansatt
av ett magnetiskt och ett elektriskt fält, vinkelrätt orienterade mot varandra.
Polariseringsriktningen för en elektromagnetisk våg definieras som den riktning
som dess elektriska fält har;
vertikalt elektriskt fält- vertikal polarisation,
horisontellt elektriskt fält- horisontell polarisation.
Polarisationsriktningen på de utsända
radiovågorna beror i främst på sändarantennens utförande.

Polarisation på HF- Kortvåg
För bästa mottagning skall användas en
antenn för samma polarisationsriktning som
i den infallande vågen. Vilken polarisation
man väljer är av mindre betydelse än att den
börvara lika både i sändar-och mottagarantennen. På kortvåg är det nödvändigtvis inte
samma riktning som den från sändarantennen, eftersom de utsända vågorna oftast
har reflekterats i jonosfären. Det kan då
uppstå en polarisationsvridning som inte
kan förutses. Att då kunna växla mellan
mottagaranten ner med olika polarisation kan
vara en fördel. Riktantenner för kortvåg monteras nästan alltid med horisontella element
-horisontell polarisation.

116-8

Polarisation på VHFIUHF!SHF

l dessa högre frekvensområden tilläm-

pas både horisontell, vertikal och cirkulär
polarisation.
Polarisationsriktningen ändras inte spontant under överföringen så länge som vågorna inte reflekterats på vägen. Jämför
med sändningar från rymdsatelliter då två
program sänds på samma frekvens, men
med olika polarisation. satelliten får då inte
ändra läge i förhållande till jorden.
För cirkulärt polariserade antenner, där
polarisationen vrider sig omkring utbredningsaxeln, gäller att överföringen är bäst,
när vridningens riktning är lika både i sändar-och mottagarantennen.
Bild II 6-11

Lämpligt antennarrangemang

Olämpligt antennarrangemang

Bild II 6-11 Inverkan av polarisation

NSYSTEM
Antenner för kortvåg
Mittmatad halvvågsantenn

Se föregående avsnitt

Ändmatad halvvågsantenn

Utstrålningen från en halvvågsantenn är i
princip lika hur den än matas. En än d matat
halvvågsantenn fungerar m.a.p. strålningsriktningar på samma sätt som en mittmatat
Vid längre antenner blir strålningskaraktären däremot en annan.
Skillnaden mellan änd- och mittmatade
halvvågsdipoler är att anslutningsimpedansen är mycket högre i ändarna än i
mitten. För att mata antennen längst ut i ena
änden behövs en transmissionsledning med
hög impedans, varvid ledningens ena part
ansluts till antennen och den andra parten
lämnas fri. En sådan anordning kallas zeppantenn och användes först i luftskepp, s.k.
zeppelinare.

Omvikt dipol (folded dipole)

Bild II 6-12
En omvikt di pol kan ses som två eller flera
parallella element, som är sammankopplade i ändarna. Mittpunkten på ett av elementen är ansluten till antennledningen.
Matningsimpedansen för en omvikt /J2dipol med två element är c:a fyra gånger
högre än den för en enkel dipol, d.v.s. 200
- 300 Q. Den omvikta dipolen, som endast
fungerar på grundfrekvensen och på dess
udda övertoner, är relativt bredbandig. Matningsimpedansen kan ändras med sinsemellan olika diametrar på de ingående elementen samt med antalet parallellkopplade
element.

Bild II 6-12 Omvikt dipol

Jordplanantenn

Bild II 6-13
Jordplanantennen eller GP-antennen (GP
av ground plane) består av en lodrätstrålare
som den ena polen och flera sammankopplade A./4-radialer eller markplanet som den
andra polen.
GP-antennen är rundstrålande och har
vertikal polarisering. Dess relativt flacka utstrålning, i jämförelse med en horisontell
antenn, gör den lämpad för långa distanser.
Av mekaniska skäl används den mest på
högre frekvenser (14 MHz och högre).
Med horisontella radialer som jordplan
är matningsimpedansen c:a 35 Q. För att få
god impedansanpassning, t.ex. till en 50 Q
koaxialkabel som matarledning, görs radialerna sluttande nedåt i en lämplig vinkel.
Koaxialkabelns innerledare ansluts till
antennen och kabelskärmen till radialerna.
Om antennen placeras omedelbart ovan
markytan, kan marken användas som jordplan, särskilt om dess elektriska ledningsförmåga är god.
Bild II 6-14
Om antennelementet inte har en elektrisk längd av ')J4, kan längden anpassas
elektriskt på liknande sätt som beskrivits
tidigare i detta kapitel för dipolantenner.

Bild II 6-13 GP-antenn

116-9

GP med
seriekondensator

l >

GPmed
toppkapacitans

t

Bild II 6-14 GP-antenner med elektrisk längdanpassning

Flerbands GP-antenner

Antennen fungerar som 'A/4 GP-antenn
åtminstone på de lägsta banden. Den mekaniska längden på en flerbands GP för
kortvåg blir kort, 4 6.5 meter, vilket på de
lägre banden innebär dålig verkningsgrad
och liten bandbredd. Jämför med SVF-kurvorna på bilden. Flerbands G P-antenner för
upp till sju kortvågsband tillverkas.

En GP-antenn kan fås att fungera på flera
band genom inbyggnad av en spärrkrets i
antennelementetför tillkommande band och
av jordplansradialer med anpassad längd
eller med spärrkretsar även i jordplanet för
de banden.
Bild II 6-15

a

SVF

SVF

'

l

1\ J

3.5

- 1 - - - --

-r---··- --- -·· - -

MHz

3

'l'..

5
1.u

.B

.1

svF

SVF
3

.6

28

2 .4

.6

.8 29 .2 .4

MHz

1

-

l

.05

J
MHz

.10

SVF
J

GP
( 80 /40 l 20 /15 /10m )

Bild II 6-15 SVF-kurvor för flerbands GP-antenn
116-10

1.5
1.2
1.0

i""'oo.

14

-

.2

...,..,.,.
.3

MHz

Det finns flera principer för denna
antenntyp. l den typ som visas här
används spärrkretsar.

Flerbands halvvågsantenner
Bild II 6-i6

En vanligt förekommande flerbandsantenn
är W3DZZ-antennen (namnet efter konstruktörens anropssignal). Den är en oftast horisontellt upphängd dipolantenn för 80, 40,
20, i 5 och i O m-banden.
W3DZZ-antennen är c:a 33.6 meter lång
och har två spärrkretsar, symmetriskt utplacerade omkring matningspunkten. Matningen sker med koaxialkabel och balun.

Antennen har en matnings impedans av
c:a 50 Q på 80- och 40-metersbanden På de
högre banden är anpassningen inte optimal
- matningsimpedansen stiger där upp till c:a
120 Q. Många använder bl. a. av den anledningen inte W3DZZ-antennen på höga kortvågsband utan föredrar där en flerbandig
GP-antenn eller en riktantenn (Yagi, quad
m.fl.).

Fysiska data
6,7tm

Praktiskt utförande
Balun1:1

spärrkrets

l

spärrkrets

om möjligt 6 mtr
lodrätt nedåt

koaxialkabel

Strömfördelning

80m

40m

./---o------------~~
~~~~

Bild II 6-16 W3DZZ-antennen

116-11

ANTENNSYSTE
W3DZZ-antennens arbetssätt:
80m-bandet
Hela antennen fungerar som en A/2dipol med resonansfrekvensen 3.7 MHz.
Den mekaniska längden är 2 · 16.8 meter
och förlängs elektriskt med induktanserna
i spärrkretsarna, vilka f.ö. är ur resonans
på detta band.
40 m-bandet
Spärrkretsarna är i resonans och "kopplar bort" antenndelen utanför dem. Delen
där innanför fungerar som en A/2-dipol
med resonansfrekvensen 7.05 MHz.
20m-bandet
Hela antennen fungerar som 3A/2-dipol
med resonansfrekvensen 14.1 MHz.
15m-bandet
Hela antennen fungerar som 5A/2-dipol
med resonansfrekvensen 21.2 MHz.
10m-bandet
Hela antennen fungerar som 7A/2-dipol
med resonansfrekvensen 28.4 MHz.

Riktantenner för kortvåg

Riktbar dipol-antenn
Bild II 6-17
En dipolantenn av måttlig mekanisk storlek
kan göras vridbar så att utstrålningen kan
riktas. Men eftersom en ensam di pol strålar
i många riktningar, låtvara mestvinkelrätt ut
från antennen, så är energin i flesta riktningarna att ses som "förlorad". När ett passivt
antennelement - en reflektor - placeras
bakom det aktiva elementet kan emellertid
bakåtstrålningen delvis vändas framåt och
man får i stället en viss riktverkan. För att det
skall fungera skall de båda elementen ha en
viss inbördes längd och ett visst avstånd
emellan.
Vagi-antenner
Bild II 6-18
Med ytterligare passiva antennelement
- s.k. direktorer- framför det aktiva elementet, blir riktverkan ännu bättre. Reflektorn är
alltid elektriskt längre än strålaren och
direktorerna alltid elektriskt kortare och allt
kortare ju längre bort från strålaren. En
sådan antenn kallas Yagi-antenn, efter sin
japanske upphovsman. Den är ursprungligen avsedd för ett enda frekvensband, en
s.k. monoband-beam.

116- 12

Om alla element förses med lämpliga
spärrkretsar, med W3DZZ-antennen som
förebild, fås en riktantenn som är användbar
på flera frekvensband, en s.k. multibandbeam. De vanligaste flerbandsantennerna
är konstruerade för 20 m-, 15m- och 1Ombanden och har två till tre element.
Bilden visar riktbara multibandantenner
med 2, 3 resp 5 element samt deras strålningsdiagram i horisontalplanet.
Matningen sker oftast med en koaxialkabel med 50 n karakteristisk impedans. Eftersom matningsimpedansen för själva riktantenn nästan aldrig är 50 n, så behövs
oftast en impedansanpassning mellan antenn och kabel.
Cubical Quadaantenner
Bild II 6-19
Cubical quad-antennen är en kvadratisk
helvågsstrålare med en sidlängd av /J4,
d.v.s. totalt 1 A.
En 2-elements quad-antenn består av
en strålare och en reflektor på ett inbördes
avstånd av 0.15 - 0.2 A. Det finns även 3och 4-elements quad-konstruktioner med
beaktansvärda dimensioner. Antennen görs
lämpligen vridbar och bör monteras åtminstone 3/4 "A över mark.
Matningsimpedansen är 50-70 n, beroende på elementavståndet Matningen sker
oftast med en koaxialkabel. Beroende på
hur matningspunkten placeras är det möjligt
att välja mellan kan horisontell eller vertikal
polarisering.
Det finns två utföranden av quad-antenner, det ena med en bärande bom med
spridare för att bära upp antennelementen
och det andra med bara spridare från ett
centralt fäste, den s.k. spider quad (spindel).
Quad-antennen byggs för vanligen för
20 m-, 15 m- och 1O m-banden. Spiderprincipen är att föredra vid flerbandsutförande, eftersom ett optimalt elementavstånd kan väljas för varje band utan att
spridarantalet behöver ökas.
Genom den flacka strålningsvinkeln är
quad-antennen en utmärkt DX-antenn. En
två-elements quad anses kunna ge ett resultat som en 3-elements Yagi-antenn.

NSYSTEM

Bild II 6-17 Riktbar dipol-antenn

----

~

/

Bild II 6-18 Yagi-antenner
116- 13

EPT

NSYSTE

1 - band Boom - Ouad

3- band Boom· Ouad

1 - band Spider - Ouad

3 - band Spider - Ouad

Bild II 6-19 Cubical Quad-antenner
För kortvågsbruk finns många antenntyper, såsom longwire-, zepp-, windom-, romb-, delta
loop-, quad laop-antenner etc. För mer information hänvisas till antennlitteratur.

116-14

ANTENNSYSTEM
Antenner för VHF/UHF/SHF

Allmänt
Alla antenner fungerar efter samma principer. Principerna för kortvågsantenner kan
därför tillämpas även för antenner för högre
frekvenser. Byggmåtten på en VHF/UHFantenn är betydligt mindre än för en motsvarande KV-antenn. JämförA= c:a2 m vid 145
MHz och A= c:a 80m vid 3.5 MHz. Det är
därför möjligt att bygga riktantenner med
rimliga dimensioner för VHF/UHF, även om
flera element används.
Om man bortser från rundstrålande vertikalantenner för trafik på korta avstånd och
mobil trafik, så används riktantenner främst
p.g.a. den större räckvidden. En riktantenns
egenskaper uttrycks i första hand i storheterna strålningsvinkel, antennvinst, fram/
backförhållande och halvvärdesbredd.
Eftersom polarisationsvridning sällan förekommer vid högre frekvenser, är det viktigt att sändar- och mattagarantenner har
samma polarisationsriktning.
Horisontell polarisation anses vara bättre
lämpad för långa distanser, eftersom vågor
med horisontell polarisation böjer av bättre
över horisontella formationer (bergryggar
etc). Även passage genom skogspartiergår
bättre med horisontellt polariserade vågor.
Antenner med horisontell polarisation används därför ofta för SSB- och CW-trafik på
långa avstånd och utmed markytan. Sådan
trafik sker i allmänhet från fasta stationer.
För mobil trafik och lokal fast trafik används dock antenner med vertikal polarisation. Vertikala antenner ger de önskvärda
rundstrålande egenskaperna för mobil trafik
och är bäst lämpade att montera på fordon.

Riktantenner

En A/2-antenn strålar vinkelrätt ut från
antennledaren och runt omkring den.
Placeras ett reflektorelement (längd ~Al
2 +5o/o) bakom antennen på ett avstånd av
::::: A/5 så reflekteras den bakåtriktade strålningen delvis framåt. En större del av energin kommer då att samlas i en riktning. Med
ett direktorelement (längd= A/2- 5o/o) framför det strålande elementet på ett avstånd
av~ A/1 O så kommer utstrålningsvinkeln att
bli mindre.

Vagi-antenner
Bild II 6-20
Den typ av riktantenn, som består av en
strålare, en passiv reflektor samt ett antal
passiva direktorer, kallas Yagi-antenn.
Vagi-antennen kan utföras med olika
antal direktorelement i kombination med
olika längd.
Det finns tre sätt att optimera en riktantenn, nämligen maximal riktverkan, minimum sidlober och maximalt fram/backförhållande. Dessa egenskaper är, emellertid ej möjliga att uppnå samtidigt. Okas t. ex.
antalet element, så ökar den s.k. antennvinsten genom att öppningsvinkeln på strålningen blir mindre, men samtidigt minskar
matningsimpedansen och den användbara
bandbredden.
Gruppantenner
Ordnas flera riktantenner vid sidan av och/
eller över varandra så erhålls en s.k. gruppantenn. Ett sådant arrangemang av s.k.
stackade antenner ger en ännu mindre
öppningsvinkel på strålningen vertikalt och/
eller horisontellt. Därigenom erhålls ytterligare antennvinst
Parabolantenner
Särskilt på frekvenser i mikrovågsområdet
och högre har radiovågorna i stort sett
samma utbredningsegenskaper som ljusets.
Behöver stor riktverkan uppnås på dessa
höga frekvenser, används ofta en parabolisk yta som spegel bakom själva antennen.
Jämför med reflektorn i en ficklampa.
Den egentliga antennen (den s.k. mataren), vars strålning är riktad mot parabolen
för att reflekteras, kan vara utformad på
många sätt. Eftersom parabolens storlek
står i omvänd proportion till frekvensen, så
används av praktiska skäl inte paraboliska
reflektorer på låga frekvenser.
Övriga antenntyper
Rundstrålande antenner: Ground plane,
A/4-,A/2-, 5A/8-antenner m.fl.
Riktantenner: Quad-, HB9CV-, helical-,
parabol- och hornantenner m fl.

116- 15

HORISONTALDIAGRAM

l

Di pol

-----::>

---r-·s

l--r-·

Yagiantenn

S = strårare
R == reffektor

:.::.::.:;;;;;>

s

R

D= direktor

:::::::::=;>

-Il-R <;

R

-ffi-

s

-«()-

o

l

D D

:=:::::;>

o

--~-

D

VERTIKALDtAGRAM

Di pol

--t- l

Dipol i

t A över jord

s
Vagi-

antenn

Yagi i

R

1'>

D

Bild II 6-20 Strålningsdiagram för horisontell Yagi-antenn

116-16

fÅ

över jord

ANTENNSYSTEM
Transmissionsledningar
En matarledning skall med så små förluster
som möjligt överföra den högfrekventa energin från sändaren fram till sändarantennen.
Omvänt skall den energi som fångats upp
av mottagarantennen transporteras till mottagaren med så små förluster som möjligt.

Bild II 6-22
Om matarledningen i stället är A/2 lång, så
uppstår i stället en spänningsnod och en
strömbuk i nedre änden av ledningen, vilket
innebär att matarledningen skall strömkopplas till sändaren.

Avstämd matarledning

Spännings- och strömkoppling
Bild II 6-21
En A,/2-dipol kopplas till sändarutgången via
en A,/4 matarledning. För tydlighetens skull
visas ledningen som en bandkabeL
Vid sändning uppstår en stående våg på
matarledningen och på dipolen. Även matarledningen svänger med och är avstämd
till resonans - därav namnet avstämd matarledning.
Vi följer ström- och spänningsfördelningen bakåt från dipolen till sändaren och finner
följande:
l vardera änden av A/2-dipolen uppträder en spänningsbuk (streckade linjer) och
i mitten av dipolen uppträder en strömbuk
(heldragna linjer). Den stående vågen, med
strömbuken på dipolens mitt, fortsätter ner
på A/4-matarledningen. l nedre änden av
matarledningen vid sändarutgången hardet
uppstått en strömnod och en spänningsbuk,
vilket innebär att matarledningen skall
spänningskopplas till sändaren.

r··-··------ . . ·2A

··-

··1

-.,.....:::""1~~--=::·~ ~--···---~--l

<llilliiioo.o........

. "if

Spänningskoppling

Bild If 6-21 Spänningskopplad Y2-dipol

Ä

2

n

l

Strömkoppling

Bild II 6-22 Strömkopplad }J2-dipol
Ström- och spänningsfördelningen kan
ritas upp för en A--di pol resp A/2-dipol i kombination med matarledningar med längderna
n · A-14 (med n = 1, 2, 3, ..... ). Med hjälp av
teckningen kan man avgöra om ström- eller
spänningskoppling måste användas.
Bild II 6-23
En A/2-dipol för 80 m-bandet ansluts till
en avstämd matarledning med längden A/2
=40 m.
Önskar man använda denna di pol för 80
m-bandet på 40-, 20- och 1O m-banden
måste en s.k. antennkopplare anslutas mellan sändaren och matarledningen. Kopplaren har alltid strömmatad ingång och valmöjlighet för ström- resp spänningsmatad
utgång. Se om antennkopplare sist i detta
kapitel.

116-17

ANTENNSYSTEM
f

=

f::::: 7 MHz

3,5 MHz

!--·-

--.....!

l

r·····

-~=40m

Å= 40m

2

n

Strömkoppling

J
Spänningskoppling

Bild II 6-23 Samma A/2-dipo/ på grundfrekvensen respektive 1:a övertonen
Oavstämd matarledning

Begreppet "oavstämd" syftar på ledningslängden, som under vissa bestämda förutsättningar kan vara godtyckligt lång. l motsats till den avstämda matarledningen behöver ledningslängden på en oavstämd matarledning inte stå i förhållande till våglängden A.. Som matarledning kan användas en
koaxialkabel eller öppen transmissionsledning.
Fördelar: Enkel uppbyggnad, mindre kritisk kabelfäring och längden kan väljas godtyckligt.
Nackdelar: Sändaren, matarledningen
och antennen måste alltid vara impedansanpassade till varandra. Dessutom måste
antenn- och kabelströmmarna balanseras. l
det följande visas hur dessa krav kan uppfyllas.
Som matarledning upp till mikrovågsområdet är koaxialkabeln vanligast.

116-18

Koaxialkabel

Bild II 6-24
Koaxialkabelns uppbyggnad framgår av bilden. l en koaxialkabel bildas ett radiellt
elektriskt fält mellan mittledaren och insidan
av ytterledaren. Av strömmen bildas också
ett magnetiskt koncentriskt fält mellan inner- och ytterledare n. Resultatet blir ett elektromagnetiskt fält, som breder ut sig i kabeln
som en TEM-våg (TE-våg = transversell
elektrisk, TM-våg = transversell magnetisk
och TEM-våg = transversell elektromagnetisk våg).
Koaxialkabeln består av en isolerad innerledare omgiven av en ytterledare, vars insida är kabelns andra strömledare. Ytterledaren förhindrar dessutom H F-utstrålning
och inkommande störningar. l motsats till
den symmetriskt uppbyggda bandkabeln,
tillhör koaxialkabeln de osymmetriska ledningarna.
Vanliga karakteristiska impedanser
koaxialkabel är 50 och 75 .Q.

ANTENNSYSTEM

skärm

isolerande hölje---·-·-····

En koaxialkabel har hastighetsfaktorn
11-{B, där c är den relativa dielektricitetskonstanten i isolationsskiktet Ett vanligt förekommande isolationsmaterial i koaxialkablar är polyetylen med dielektricitetskonstanten c = 2.25.
Hastighetsfaktorn v (velocity factor) blir
då

Bild II 6-24 Koaxialkabel

Bandkabel

Bild 6-25
Som framgår av bilden består bandkabeln
av två parallella ledare med samma dimensioner. Kabelns isolering håller samtidigt
ledaravståndet rätt. l ett kraftigare utförande
övergår denna ledningstyp till att bestå av
ett ledarpar med isolerade spridare på jämna
avstånd. Den kommer att likna en stege det ursprungliga utförandet på en matarledning.
Vanliga karakteristiska impedanser
bandkabel är 75 och 300 n.

Bild II 6-25 Bandkabel

Vågledare

Inom mikrovågsområdet är den vanligaste
typen av matarledning s.k. vågledare som
saknar mittledare. l en vågledare däremot,
matas energin fram enbart som speciella
elektriska och magnetiska fält (TEM) i möns ...
ter som kallas moder.

Hastighetsfaktor

Vid bestämning av den mekaniska längden
på en matarledning måste hänsyn tas till att
våghastigheten längs ledningen är lägre än
ljushastigheten. Man talarom en hastighetsfaktor relativt ljushastigheten. Hastighetsfaktorn beror på ledningens utförande och
ingående material.

1

1

1

V= -{B= ..j 2. 25 = tS = 0.666

1 meter av en sådan koaxialkabel är
1/0.666 =i .333 meter för en H F-signal.
Även bandkablar har naturligtvis en
hastighetsfaktor, vanligen 0.7- 0.85.

Karaktäristisk impedans Z i ledningar

Antag att en H F-sändare har kopplats till en
oändligt lång ledning. Om man undersöker
kvoten mellan spänning och ström på godtyckliga ställen utmed ledningen, så kommer man att finna samma kvot överallt.
Denna konstant uttrycks i ohm, om spänning och ström uttrycks i volt respektive
ampere. Konstanten kallas vågimpedans
eller karaktäristisk impedans.
Oändligt långa ledningar är ju orealistiska och då kan man i stället bestämma vågimpedansen genom ledningens geometriska uppbygg nad, dielektricitetskonstant
och dess induktivitet och kapacitet per längdenhet.
Exempel:
Vi undersöker elektriska karakteristika i
en kabel av typ RG 213/U.
På en provbit med längden 1 meter mäter vi en kapacitans av 97 p F mellan inneroch ytterledaren. När kabelns ena ände
kortsluts mäter vi en induktans av 262 nH.
Den uppmätta kapacitansen och induktansen bestämmer kabelns karaktäristiska
impedans Z, också kallat våg motstånd, som
är oberoende av ledningens längd.
Med ovanstående uppmätta värden blir
impedansen:

Z=\ 

z=

L [H]

262000 ·10-"
97 ·10-12

C [F]

Z [O]

=~ 262ooo =52
97

0

116-19

ANTENNSYSTE
Den karaktäristiska impedansen för en
matarledning, bestäms av dimensionerna i
ledningen och av isolationsmaterialets dielektricitetskonstant.
För en bandkabel är

Z= 276 ·log 2a

Fr

[Q]

d

[a = centrumavståndet mellan ledarna i
mm]
[d = ledardiametern i mm]
[er= dielektricitetskonstanten, överslags
värde 1.5]
[er för luft = 1.0]
För en koaxialkabel är
138
D
Z=-·log[Q]

Fr

d

[D = ytterledarens innerdiameter i mm]
[d = innerledarens ytterdiameter i mm]

Punkterna för maxima och minima beror
av belastning relativt vågresistansen och av
frekvensen.
stäende vågor uppträder inte bara i
antennkablar utan även i fasta material (trådar o. dyl.), i luft (ljud), i ljus (t.ex. laser), i
elektromagnetiska fält o.s.v.
Bild II 6-26
Spänningen utmed kabeln varierar regelbundet mellan
u max = uf + ub och umin = uf - ub
ståendevågförhållande (SVF)
(även SWR = standing Wave Ratio).
Med ståendevågförhållandet SVF menas
förhållandet mellan
umax och umin eller mellan !max och 'min

SVF = Umax
umin

Data, impedansdiagram och formler för
beräkning av transmissionsledningar finns
bl.a. i antennhandböcker.

eller

SVF=

=

U,+ Ub

u,-ub

/max
/min

Stående vågor
Både när sändarens och matarledningens
anslutningsimpedans är olika liksom när
matarledningens och antennens anslutningsimpedans är olika, så uppstår s.k. missanpassning som hindrar energitransporten.
Antag att matarkabelns och antennens
anslutningsimpedans är olika. En del av H Fenergin kommer då att strålas ut från antennen, men resten reflekteras tillbaka i matarledningen. På kabeln finns alltså en framåtgående våg mot antennen och samtidigt en
reflekterad våg tillbaka mot sändaren.
Den spänning och ström som man då
kan mäta var som helst på kabeln, är den
algebraiska summan av amplituden hos den
framåtgående och den reflekterade vågen.
Flyttar vi mätpunkten stegvis utmed kabeln, så kommer spänningen och strömmen
att stiga och sjunka på ett regelbundet sätt.
Den tillbakagående vågens spänning Ub
och den framåtgående vågens spänning Ut
överlagras på varandra. Kvoten för ström
och spänning är därmed inte konstant utmed matarledningen, utan får ett vågformat
förlopp- en stående våg.
116-20

ståendevågförhållandet SVF kan även
anges med hjälp av impedanserna i matarledningen (Z) och i antennens matningspunkt (Za)·

SVF = !

där Z > Za

~a

där Za > Z.

za

SVF =

eller

ståendevågmätning beskrivs i Kapitel 8.
Bild II 6-27
Bilden visar SVF-problemet enkelt sett
och vad en SVF-meter visar beroende på
var den kopplas in i kedjan sändare-ledning-antennkopplare-ledning-antenn.
Vid ett högre SVF-tal än 2:1 till 3:1 vid
sändarutgången, bör en antennkopplaresättas in efter sändaren för att skydda den från
(överhettning och) överslag. Antennkopplare har även andra benämningart.ex. matchbox, antennavstämningsenhet o.s.v. Bäst
är att göra sådana impedansanpassningar i
alla led, att antennkopplaren blir onödig.

NSYSTEM

Bild II 6-26 Ståendevåg på ledning

Pilarna visar HF-energins riktning
Sändare
Utgångson

SVFmeter

Koaxialkabel 50Q

Antenn
Koaxialkabel 50Q

Ingång SOn

SVF=1

HF-energin från sändaren

reflekteras delvis

men sänd s återigen till antennen
Sändare

Antenn

SVFmeter

Utgång
-:t-SOn

Ingång
-:t-SOn

Sändaren är efterinställd,
SVF
utgången är inte längre 50 Q,
>1
den inkommande reflekterade HFenergin sänds tillbaka till antennen
. f ron san d aren
HF-energrn
o

..

Sändare,\

9VIS7r

re fl e kteras d l.

SVFmeter

Utgång SOQ

Sändaren är inte efterinstätld,
den reflekterade HF-energin
stannar i sändarens slutsteg
och värmer det ytterligare

/

Antenn
Ingång
-:t-SOn

SVF>1

Antenn
Ingång
valfri

SVF=1

~

SVF>1

Lsönder den reflekterade energin
från antennen tillbaka igen

Bild II 6-27 SVF-problemet enkelt sett

efter DJ3XV. cq-DL 5/1978

116- 21

ANTENNSYSTEM
Effektförluster
l varje matarledning uppstår förluster, dels
av resistansen i ledarna och dels i isolationsmaterialet (dielektrikum) mellan ledarna samt i någon mån av fältutstrålning från
dem. De mest påtagliga effektförlusterna i
en ledning beror av förlusterna per längdenhet och därmed även av längden. Vidare
beror förlusterna av ståendevågförhållandet på ledningen på grund av dålig impedansanpassning.
Ett högt SVF-förhållande ger större ledningsförluster eftersom den reflekterade effekten då pendlar fler gånger på ledningen.
Den reflekterade effekt som återvänder till
ledningens inända är mindre när ledningen
har stora förluster än om den inte hade det.
Det innebär att det verkliga SVF-förhållandet i ledningens utände är större än vad som
syns på ett instrument i inänden.
Bal u ner- Balansering- Transformering
Balansering
Man skiljer mellan symmetriska ledningar
(bandkabel m.fl.) och osymmetriska (koaxialkabel), där dessutom den ena ledaren (skärmen) ofta är jordad.
På samma sätt finns det symmetriska
antenner (dipol, W3DZZ m .fl.) och osymmetriska (ground plane, Marconi m.fl.).
Vill man ansluta en symmetrisk (mittmatad) antenn till en osymmetrisk ledning
(koaxialkabel), så måste en strömbalansering göras i övergången. Om inte, så kommer matarledningen att stråla, vilket kan
medföra störningar på radio och TV. Utan
balansering kommer dessutom dipolens
strålningsbild inte att vara symmetrisk.
En balansering måste också göras i övergången mellan en bandkabel (symmetrisk)
och sändaren när den har anslutning för
koaxialkabel (osymmetrisk),
Balansering av impedans och därmed
ström sker med en anordning kallad BAL UN
(av de engelska orden BALanced-UNbalanced).
Bal uner kan utföras på flera sätt. Grundläggande har balunen lika in- och utgångsimpedans,
Exempel:
Ringkärnebalun 1 :1 för balansering.
Koaxialledare anordnad som balun 1:1.

116-22

Transformering
l samband med balanseringen kan en
impedanstransformering behövas och det
finns baluner (transformatorer) som både
balanserar och transformerar impedanser.
Bild II 6-28
Bilden visar en transformator med osymmetrisk ingång och symmetrisk utgång. Om
båda lindningarnas varvtal är lika så sker
ingen impedanstransformering. Om förhållandet mellan varvtalen är 1 :2 så blir förhållandet mellan impedanserna 1 :4. Se vidare
i Kapitel1.
Bilden visar också att matarledningens
impedans Z transformeras om så att den blir
lika antennens anslutningsimpedans Ra.
Denna transformering kan ske induktivt eller kapacitivt.
Exempel:
Ringkärnebalun 1 :4.
Koaxialledare anordnad som bal un 1:4.

BALANSERING

TRANSFORMERING

0118Ra
gz
~~: !Ra
Qz

i::

!Ra

Bild II 6-28 Balansering - Transtorrnerin

ANTENNSYSTE
Ringkärnebalun

Bild II 6-29
Ringkärnebalunen är en form av transformator. l den finns en ringkärna av hårt
sammanpressat järnpulver av en legering,
som tillsammans med lindningarnas utförande gör att frekvensbandbredden blir stor.
"1 = varvtal, primär
n2 = varvtal, sekundär

l den mellersta figuren är den översta
delen av matningskabeln en A./4 lång parallellsvängningskrets tillsammans med parallellt ansluten ledare (i detta fall en koaxialkabel som kortslutits i båda ändar). Den
nedrersta högra figuren i bilden visar den
kortslutna A./4-ledningen i tre varianter. l
samtliga fall uppstår HF-mässigt en strömbalancerande effekt mellan dipolhalvorna.
Dessutom hindras även antennströmmar från att komma ner på utsidan av matningskabelns skärm.

Osymmetrisk

ingång

Bild II 6-29 Ringkärnebalun

Koaxialledare som balun
Bild II 6-30
Balansering kan även göras med ett ett
koaxialkabelarrangemang, som i så fall är
starkt frekvensberoende. Bilden visar tre
utföranden, som alla arbetar enligt principen för en matarledning med en elektrisk
längd av A./4 och kortsluten i ena änden.
Den mekaniska längden är k· A/4, varvid
k är hastighetsfaktorn för våghastigheten i
kabeln. För de vanligaste koaxialkablarna
RG58 och RG213 är k= ca 0.66.
A./4-ledningen i den översta figuren fungerar som en parallellsvängningskrets med
mycket hög impedans Z i den öppna övre
änden.

----- --- - -·· . .

1·'

.A.
i ·--

Bild II 6-30 Koaxialledare som balun

116-23

ANTE N
- anpassning

)(

Delta - anpassning

Il
/il,r

!""------

anpassning

2Å

--------4

l= le~

l

i

Bild II 6-31 Sätt att ansluta en matningsledning

Sätt au ansluta en matningsledning

Bild II 6-31
T-, delta- och gamma-anpassning
Funtion: En mittmatad halvvågsdipol har
i fria rymden en impedans av c:a 73 n.
Flyttas matningspunkten bort från mitten, åt det ena eller andra hållet, så är
impedansen högre än i mitten.
Det finns alltid två symmetriskt liggande
punkter på antennen där impedansen är
precis lika stor.
T-, delta- och gamma-anpassning är användbar när matarkabelns
är högre än antennens mittpunktsimpedans. Matningsledningen kan anslutas till de punkter
på antennen som har samma impedans
som matarledningen. T-anpassning används för symmetriska matarledningar, gamma-anpassning för osymmetriska ledningar
och delta-anpassning för båda ledningstyperna.

A/4 -anpassningsledning - stub
Uppbyggnad: Antennen ansluts till en ')J
2 anpassningsledning och matarledningen
i sin tur till anpassningsledningen.
Funktion: Anpassningsledningen består
av en öppen 'A/4-matarledning. Den har
teoretiskt impedansen Z= Oi den ände som
är ansluten till antennen och Z = = i den
116-24

-Anpassningsledning

andra. Utmed anpassningsledningen finns
alltid en impedans som är lika matarledningens impedans.

},./2-fasningsledning
Bild II 6-32
Funktion: När t.ex. en omvikt dipol med
matningsimpedansen 240 n skall anslutas
till en 50 n-kabel, behövs en impedanstransformering med förhållandet 4:1. En 'A/2
lång fasningsledning kan användas fördetta
ändamål. Fasningsledningen har dessutom
en strömbalanserande verkan.
Observera: Med en 'A/2-fasningsledning
enligt bilden kan impedanstransformering
endast göras i förhållandet 4:1.

i- stinga
Bild II 6-32 )./2-fasningsledning

ANTENNSYSTEM
Transmissionsledningen
En transmissionsledning för radiofrekvent energi består av två elektriska ledare.
Den enklaste formen av en sådan ledning är
tvåparallella ledare. En annan form av transmissionsledning är koaxialkabeln, där den
ena ledaren löper inuti den andra.
Försök: Koppla en parallelledning till utgången på en VHF-sändare - t.ex. med
induktiv koppling. Ge ledningen passande
längd och mata ut högfrekvent energi på
ledningen. Nu kan fördelningen mellan spänning och ström på olika punkter utmed ledningen undersökas. När det finns en spänning mellan de två ledarna i ledningen alstras det ett elektriskt fält mellan dem.
Eftersom en glimlampa lyser när den
omges av ett elektriskt fält kan den användas som en enkel spänningsindikator.
När en elektriskt ledande krets - en induktionsslinga - omges av ett varierande
magnetiskt fält alstras det en ström i slingan.
Med en glödlampa inkopplad i slingan kan
den användas som en enkel strömindikator.
Öppen transmissionsledning
Bild II 6-33
Håll glimlampan nära intill ledningen.
Glimlampan tänds med jämna mellanrum
när den flyttas utmed ledningen.
När i stället en induktionsslinga med glödlampa hålls nära intill ledningen, kommer
glödlampan att lysa mitt emellan de ställen
där glimlampan lyser. Där glimlampan tänder har det bildats spänningsmaximum och
där glödlampan lyser har det bildats strömmaximum. Det har bildats en stående våg
på ledningen.
Bilden visar ström- och spänningsfördelningen för en öppen transmissionsledning med längden i = n · IJ4 med udda n =
1' 3, 5, ......
För bilden har valts n = 5.
Utmed ledningen uppstår omväxlande
elektriska och magnetiska fält allt efter som
svängningen fortsätter. Med en serie om
fyra figurer visas förloppet av en svängning,
en period. skillnaderna i den elektriska fältstyrkan framställs som olika långa fältlinjer.
Observera fältlinjernas riktning.

Skillnaderna i den magnetiska fältstyrkan
kan också utläsas ur bilderna i form av
antalet symboler "
resp " x ". Båda
tecknen betecknar elektromagnetiskt fält, "
i riktning ut ur papperet och " x " in i
papperet. För tydlighetens skull skildras
endast den elektromagnetiska fältstyrkan
mellan ledarna och inte utanför ledarparet
G

G

"

"

Kortsluten transmissionsledning
Bild II 6-34
På bilden visas såväl ström- och spänningsförhållandena som fältlinjeförloppen
på en avstämd, kortsluten transmissionsledning med längden l = IJ4 med jämna n =
2, 4, 6, 8, ... För bilden har valts n = 6.

A./4-ledning som svängningskrets
Bild II 6-35
Bilden visar ström- och spänningsfördelningen för en öppen resp. en kortsluten
transmissionsledning med längden l= IJ4.
Den öppna A./4-ledningen har en strömbuk i ingångsänden. En sådan ledning måste
således strömkopplas, d.v.s. den anslutande impedansen måste vara låg.
Den kortslutna A./4-ledningen har en
spänningsbuk i ingångsänden. En sådan
ledning måste spänningskopplas, d.v.s. den
anslutande impedansen måste vara hög.
En öppen A/4-ledning kan ses som en
seriekopplad LC-krets. När ledningen är i
resonans flyter en hög ström i ingången,
medan spänningen där är låg.
En kortsluten A./4-ledning kan ses som
en parallellkopplad LC-krets. När ledningen
är i resonans är spänningen hög över ingången, medan strömmen där är låg.

116-25

UPPBYGGNAD OCH INKOPPLING

SPANNINGS- OCH STRÖMFÖRDELNING

(l :::5· t A)

+++
Spänningsförlopp
(visat med glimlampa)

-ffi-

Strömförlopp
(visat med glödlampa
och slinqa

c®=J
FÖRLOPPEN FÖR DE ELEKTRISKA OCH MAGNETISKA FAL TEN

t=

o

' Elektriska fältlinjer

Magnetiska fältlinjer

Elektriska fältlinjer

Magnetiska fältlinjer

Bild II 6-33 Förlopp i öppen :1/4 transmissionsledning

116-26

UPPBYGGNAD OCH INKOPPLING

SPÄNNINGS- OCH STRÖMFÖRDELNING

( ( ::::

6.

t A)
Spänningsförlopp
{visat med glimlampa)

Strömförlopp
(visat med glödlampa
och slinga)

c®=l

FÖRLOPPEN FÖR DE ELEKTRISKA OCH MAGNETISKA FÄLTEN

+++

+++
Elektriska fältlinjer

Magnetiska fältlinjer

Elektriska fältlinjer

+++

+++

Magnetiska fältlinjer

Bild II 6-34 Förlopp i kortsluten A/4 transmissionsledning

116-27

ANTENNSYSTEM
STRÖM- OCH SPÄNNINGSFÖRDELNING

iT

a - öppen ledning

R=O

R::c:lO

'::---:::;:::::::>
1
u--

'X:;{

u
(l)

+---+----- f

fr

(l)xx
u

b- kortsluten ledning

+---1-----

FRÄN SERIESVÄNGNINGSKRETS Tl LL ÖPPEN

I:I
T

J

N'

t-

LEDNING

:::L,

::::=::::>
o

lfY'!

i

FRÄN PARALlEllSVÄNGNINGSKRETS Till KORTSlUTEN

1.

+

f·- LEDNING

...,  .Ä,

T

Bild II 6-35 JJ4 transmissionsledning som svängningskrets

116-28

J

--...-.j

f

NSYSTEM
Antennkopplare
Bild II 6-36
Bilden visar en antennkopplare för bandkabel av olika längder. storleken på kondensatorerna: C1 = C2 = 500 pF, C3 = 300pF
Avstämning vid spänningskoppling
c1 och c2 helt invridna eller kortslutna,
C 3 avstäms för resonanstillstånd
(parallellresonans).
Avstämning vid strömkoppling
helt utvriden,
C 1 och C2 avstäms för resonanstillstånd
(serieresonans), med maximal och likaström
i båda ledarna.
Matarledningen kan förlängas elektriskt
med induktanser när den är för kort för att
kunna avstämmas.

c3

Märk, att en antennkopplare mycket väl
även kan utformas för koaxialkabelutgång.

Antennkopplare för en
parallell-ledning med längden

l : : n·

:A.

4

För- och nackdelar med avstämd matarledning
När en matarledning är rätt avstämd
transporterar den energi utan att stråla själv.
När dipolen kopplas till en avstämd matarledning, kan den med hjälp av en antennkopplare arbeta på flera amatörradioband. Detta är en anledning till varför en
avstämd matarledning gärna används för
portabla installationer (t.ex. för field days).
lnjusteringen mot sändaren blir enklare.
Inom amatörradion används numera nästan uteslutande koaxialkabel som matarledning i st.f. bandkabeL Detta är av flera
skäl:
• En bandkabel måste hängas upp så fritt
som möjligt och den får inte komma för nära
murutsprång, takrännor o.s.v. Vidare måste
den isoleras väl vid genomföringar i väggar.
• De flesta såndaramatörer har inte plats
med långa matarledningar (n· /../4 med n=
1' 2, 3, ... ).
• Vid tvära bockar på ledningen kan det upp
stå oönskad utstrålning och därmed risk för
störningar på radio och TV m. m.

AAA.A.J

Q
,......,

Antennkopplare för en
parallell-ledning med längden

l.

* n ·-f:

Bild II 6-36 Antennkopplare

116-29

ANTENNSYSTEM

116-30

~©~

EPT

