\section{Förstärkare}
\textbf{HAREC a.\ref{HAREC.a.3.4}\label{myHAREC.a.3.4}}

\subsection{Allmänt}

Bild II 3-40

Elektronrör och transistorer är de aktiva komponenter som används i
oräkneliga elektroniska kopplingar för alstring av signaler, för
förstärkning och blandning av signaler, för multiplicering av
signalfrekvenser etc.

Transistorn presenteras i avsnitt 2.6 och elektronröret i avsnitt 2.7.

Först förekom endast elektronrör. Dessa har emellertid på några få
decennier nästan helt ersatts av transistorer. Elektronrör används
dock fortfarande, särskilt i effektförstärkare för sändare. Det finns
därför skäl att här behandla såväl elektronrör som transistorer.

\subsection{Huvudegenskaper hos förstärkare}
\subsubsection{LF- och HF-förstärkare}
\textbf{HAREC a.\ref{HAREC.a.3.4.1}, a.\ref{HAREC.a.3.4.2}, a.\ref{HAREC.a.3.4.3}\label{myHAREC.a.3.4.1}\label{myHAREC.a.3.4.2}\label{myHAREC.a.3.4.3}}

Bild II 3-41

Med \emph{LF-förstärkare} menas förstärkare som arbetar med signaler i
det lägre frekvensområdet, typiskt upp till c:a 100
kHz. LF-förstärkare är mycket vanliga såväl i mottagare som
sändare. Utöver de aktiva komponenterna (transistorer, elektronrör )
är kondensatorer och resistorer de viktigaste passiva.

Med \emph{HF-förstärkare} menas förstärkare som arbetar med signaler
med högre frekvenser än dem i LF-området. Även HF-förstärkare är
mycket vanliga såväl i mottagare som sändare. De används t.ex. i
mottagarnas ingångs- och mellanfrekvenssteg, liksom i sändarnas
oscillatorer, signalberedningssteg och slutsteg.  Utöver de
komponenter, som även finns i LF-förstärkare, används kombinationer av
frekvensberoende komponenter såsom induktorer och kondensatorer.

\emph{Förstärkning}
Med förstärkning avses här kvoten av amplituden i utgående och
inkommande signal, varvid frekvensgången har inverkan.

\emph{Frekvensgång}
Förstärkare arbetar endast inom ett visst frekvensområde, vilket kan
skilja från fall till fall.

\emph{Bandbredd}

Det frekvensområde där förstärkaren arbetar med fulla data kallas
bandbredd. Bandgränserna uttrycks som en nedre och övre gränsfrekvens,
där signalnivån avviker från ett givet värde, vanligen med högst 3 dB.

För LF-förstärkare för amatörradiobruk är kravet på bandbredd litet;
inom ett band av 300 Hz till 3 kHz uppnås godtagbar
återgivningskvalitet för tal. Bandbredden bestäms främst av
kondensatorer i kretsen avsedda för överföring och avkoppling.

HF-förstärkare används för signaler med hög frekvens, typiskt 100kHz
och däröver.  Det finns s.k. bredbandig a förstärkare för ett stort
frekvensområde, men även avstämda förstärkare för smala frekvensband.

Bild II 3-40 Från elektronrör till transistor

Bild II 3-41 Principen för förstärkare med elektronrör respektive transistor

\subsection{Grundkopplingar för förstärkarsteg}
\textbf{HAREC a.\ref{HAREC.a.2.6.4.1}, a.\ref{HAREC.a.2.6.4.2}, a.\ref{HAREC.a.2.6.4.3}, a.\ref{HAREC.a.2.6.4.4}\label{myHAREC.a.2.6.4.1}\label{myHAREC.a.2.6.4.2}\label{myHAREC.a.2.6.4.3}\label{myHAREC.a.2.6.4.4}}

Bild II 3-42

I det föregående har redan visats att en av polerna i ingången
respektive utgången i en förstärkare är gemensam. I ovanstående bild
är rörförstärkarens katod den gemensamma polen - därav namnet
katodkoppling.  På liknande sätt är NPN-transistorns emitter gemensam
- därav namnet emitterkoppling.

På ett liknande sätt kan någon annan pol vara gemensam. Man får då i
stället en baskoppling eller kollektorkoppling.

Beroende av kopplingsätt fås olika egenskaper. På nästa sida visas tre
olika grundkopplingar för ett elektronrör (triod) respektive en
NPN-transistor.

I praktiken känns en grundkoppling igen på vilken elektrod som är
avkopplad till 0-potential över en kondensator.

\emph{Emitterkoppling} används för LF och HF när hög förstärkning
eftersträvas. Eftersom effektförstärkningen är produkten av spännings-
och strömförstärkningen, så fås en effektförstärkning av mellan 200
till 50000 gånger. Nackdelen med denna koppling är den ibland låga
ingångsimpedansen och den relativt låga gränsfrekvensen.

\emph{Baskoppling} använd som HF-förstärkare på grund av sin höga
gränsfrekvens och goda isolation mellan in- och utgång.

\emph{Kollektorkoppling} används när hög ingångsimpedans och
utgångsimpedans önskas. Denna koppling har emellertid ingen
spänningsförstärkning, men kan användas för s.k. impedansomvandling.

\begin{table*}[!h]
\caption{Grundkopplingarnas typiska egenskaper vid NPN-transistor}
  \begin{tabular}{l|l|l|l}
    \bf Egenskap & \bf Emitterkoppling & \bf Baskoppling & \bf Kollektorkoppling \\
    \(Z_{in}\) & medel \quad 1 kΩ & liten \quad 50 Ω & stor \quad 100 kΩ \\
    \(Z_{ut}\) & medel \quad 10 kΩ & stor \quad 100 kΩ & liten \quad 50 kΩ \\
    Strömförstärkning & stor \quad 100 ggr & < 1 \quad 0.9 ggr & stor \quad 100 ggr \\
    Spänningsförstärkning & stor \quad 100 ggr & stor \quad 100 ggr & < 1 \quad 0.99 ggr \\
    Effektförstärkning & mycket stor \quad 10000 ggr & stor \quad 100 ggr & stor \quad 100 ggr \\
    Fasläge & motfas \quad 180° & medfas \quad 0° & medfas 0° \\
  \end{tabular}
\end{table*}

Bild II 3-42 Grundkopplingar för elektronrör och NPN-transistor


%%


\subsection{Stabilisering av arbetspunkten}

För att en förstärkare skall kunna arbeta på avsett sätt måste
arbetspunkten, d.v.s.  arbetsströmmens vilavärde ställas rätt.

Det gör man genom att placera en förspänning över den styrande
elektroden i elektronröret eller transistorn i fråga.

I en katodkopplad rörsförstärkare innebär det att styrgallret skall
ges en viss negativ spänning i förhållande till katoden. Det kan man
göra t.ex. med en separat spänningskälla eller vanligare med en
avkopplad res istor mellan katod och minuspolen (jord).

I en emitterkopplad transistorförstärkare innebär det att basen skall
ges en viss positiv spänning i förhållande till emittern. Det kan man
göra t.ex. med en separat spänningskälla eller vanligare med en
avkopplad resisto r mellan emittern ochminuspolen samt en resistiv
spänningdelare mellan plus- och minuspolen.

\subsection{Klass A-, B- och C-förstärkare}
\textbf{HAREC a.\ref{HAREC.a.3.4.4}\label{myHAREC.a.3.4.4}}

\subsubsection{Arbetspunkt}

\emph{Arbetspunkten} för förstärkare väljs olika, beroende på önskat
arbetssätt. En olämpligt vald arbetspunkt resulterar i förvrängning av
utsignalens form i förhållande till insignalens form,
s.k. \emph{distorsion}. Distorsion uppstår även vid överstyrning,
d.v.s. när insignalens amplitud är för stor för att kunna återges med
oförändrad form, även om arbetspunkten är rätt vald.

Med avseende på arbetspunktens läge klassas därför förstärkare på sätt
som framgår av följande diagram för elektronrör. En emitterjordad
NPN-transistor får anses mest motsvara elektronrörkopplingen här
nedan.  Anodströmmen \(I_a\) motsvaras då närmast av kollektorströmmen
\(I_C\) och styrgallerspänningen \(U_{gi}\) av spänningen
\(U_{BE}\). Den stora skillnaden är att styrgallerspänningen i dessa
fall alltid är negativ medan bas/emitterspänningen är
positiv. Styrspänningens relativa läge (arbetspunkten) mellan olika
arbetsklasser är emellertid lika.

\subsubsection{Klass A}

Bild II 3-44

Klass A är ett arbetssätt i linjära LF- och HF-förstärkarsteg, t.ex. i
mottagare samt AM- och SSB-modulerade sändare. Vilovärdet på strömmen
i huvudkretsen, den s.k. arbetspunkten, placeras i mitten på den
rakaste delen av styrkaraktäristikan (\(I=0.5\cdot I_{max}\)).  Därmed
fås låg distorsion. Verkningsgraden är upp till 50\%.

\subsubsection{Klass AB}

Klass AB är ett godtagbart linjärt arbetssätt för AM-
resp. SSB-modulering, men med en lägre viloström. Arbetspunkten ligger
mellan den för klass A och B. Ett linjärt arbetssätt enligt klass A är
visserligen önskvärt vid SSB, men verkningsgraden är lägre. Klass AB
är en kompromiss med bättre verkningsgrad utan en alltför stor
distorsion.

\subsubsection{Klass B}

Bild II 3-45

Klass B är ett olinjärt arbetssätt med en låg vilaström i förhållande
till \(I_{max}\) d.v.s. arbetspunkten ligger i nederkant av
styrkaraktäristikans nedre krökta del. Verkningsgraden är upp till
67\%. Trots det används klass B i linjära LF-och H F-förstärkarsteg
t.ex. i SSB-sändare.

Om klass B skulle tillämpas i ett slutsteg med endast ett rör eller en
transistor skulle större delen av uteffekten förloras i splatter,
d.v.s. som förvrängda signaler långt vid sidan om den egentliga
nyttosignalen. Ett sätt att undvika det är att använda en avstämd
utgångskrets med högt Q-värde. Linjär förstärkning kan också erhållas
med två mottaktkopplade rör eller transistorer i klass B.
Utgångskretsen behöver då inte vara avstämd av linjäritetsskäl.

\subsubsection{Klass C}

Bild 113-46

Klass C används i HF-förstärkarsteg i FM-, CW- och
AM-sändare. Arbetssättet är kraftigt olinjärt. Vilaströmmen är noll,
d.v.s. arbetspunkten ligger på den negativa delen av
styrkaraktäristikan. Endast toppen av den ena halvvågen av insignalen
återges och i starkt förvrängd form. Verkningsgraden är upp till
80\%. Övertonerna dämpas av sväng- ningskretsen. En resonanskrets med
högt Q-värde behövs som utgångskrets varvid amplituddistorsion inte
framstår som besvärande vid CW och FM. Med hjälp av en utgångskrets
kan frekvensmultiplicering utföras med förstärkare i klass C.  (På
följande tre bilder är \(I_R\)=anodviloström).

Bild If 3-44 Förstärkare i klass A

Bild II 3-45 Förstärkare i klass B

Bild II 3-46 Förstärkare i klass C

\subsection{Frekvensmultiplicering}

Bild II 3-47

Frekvensmultiplicering kan användas för att skapa en högre frekvens än
den som avges av oscillatorn. Oscillatorn följs då av ett eller flera
frekvensmultiplicerande förstärkarsteg som arbetar i klass C.

I utgången av ett frekvensmultiplicerande steg måste finnas en
svängningskrets, som är avstämd till önskad frekvens, d.v.s. överton
av insignalen. Denna överton förstärks i efterföljande förstärkarsteg,
vilket också kan vara frekvensmultiplicerande.

Ju högre multiplikationsfaktorn är, desto högre förspänning krävs för
att svängningskretsen i utgången skall svänga obehindrat.  Med hög
multipliceringsfaktor i ett enda steg dämpas signalen då så mycket att
en hög förstärkning behövs i efterföljande steg. I praktiken anordnas
därför en kedja av frekvensdubblande och frekvenstripplanda Den totala
multipliceringsfaktorn är faktorerna för vartdera steget multiplicerat
med varandra.

Som exempel visar bilden blockschemat för en VHF-sändare med
oscillatorkristaller i 8 MHz-området. Som räkneövning kan andra
kristallfrekvenser sättas in för beräkning av den slutliga
sändningsfrekvensen. I frekvensmultiplicerande sändare kan även
slutsteget arbeta i klass C, vilket är vanligt i sändare för telegrafi
eller FM-telefoni. För att då förhindra utsändning av alla de
övertoner som alstras i förstärkarkedjan, så förses slutstegets utgång
med en svängningskrets som är avstämd till sändningsfrekvensen.
Övertonsdämpningen kan förbättras ytterligare med ettefterföljande
lågpassfilter. Overtoner för 144 MHz är 288 MHz, 432 MHz o.s.v.

Frekvensmultiplicering behöver nödvändigtvis inte göras med ett
förstärkarsteg i klass C. En diod har nämligen olinjär karaktäristik
och därmed alstras det övertoner i de strömmar som passerar genom
den. En av dessa övertoner kan filtreras fram och
förstärkas. T.ex. finns det frekvenstripplingssteg byggda kring en
speciell typ av kapacitansdiod - varaktordiod. Vanliga frekvensområden
för s.k. varaktortripplare är 144/432 MHz och 432/1296 MHz.

Såväl signalen från en kristalloscillator som den från en VFO kan
multipliceras till en högre frekvens.

Förr täckte VFO i amatörradiosändarna oftast frekvensområdet 3.5-3.8
MHz. Med en så vald VFO-frekvens kunde alla upplåtnafrekvensband för
amatörradio nås med frekvensmultiplicering. De ursprungliga
amatörradiobanden i KV-området ligger fortfarande harmoniskt
relaterade av detta skäl.

Således
\begin{align*}
  &2 \cdot 3.5 = 7\ \text{MHz} \\
  &2 \cdot 2 \cdot 3.5 = 14\ \text{MHz} \\
  &2 \cdot 2 \cdot 2 \cdot 3.5 = 21\ \text{MHz} \\
  &2 \cdot 2 \cdot 2 \cdot 2 \cdot 3.5 = 28\ \text{MHz} \\
\end{align*}

Bild 113-47 Frekvensmultipliceringskedja

Vid frekvensmultiplicering flerfaldigas inte bara oscillatorfrekvensen
utan även variationerna i den. Om t.ex. VFO-frekvensen i området 3.5
MHz ändras med 50 Hz, så ändras utfrekvensen i området 28 MHz med \(2
\cdot 2 \cdot 2 \cdot 50 = 400\) Hz. Alla frekvenser i signalen
multipliceras på detta sätt. Amplitudmodulerad telefoni kan därför
inte överföras genom en frekvensmultipliceringskedja utan att talet
förvrängs.

\subsection{Sändarslutsteg}

\subsubsection{Slutsteg med en transistor}

Bild II 3-48

Transistorslutsteg för HF byggs vanligen emitterkopplade p.g.a. den
högre effektförstärkningen.

Bilden visar ett sådant förstärkarsteg.  Kollektorbelastningen består
av en svängningskrets. För att anpassa transistorns kollektorimpedans
till svängningskretsens impedans, har kollektorn anslutits till ett
uttag på svängningskretsens spole.

Drossel \(Dr\) och kondensator \(C\) fungerar som en HF-mässig
avkoppling av strömförsörjningen. Uteffekten tas ut från
svängningskretsen över en kopplingslindning med samma impedans som
belastningen. För linjär återgivning krävs drift i klass A eller
möjligen klass AB.

Bild II 3-48 Slutsteg med en transistor

\subsubsection{Slutsteg med två transistorer}

Bild II 3-49

Ett mottaktkopplat (eng. push-pull) förstärkarsteg i klass B har god
verkningsgrad samtidigt som det är nöjaktigt linjärt för SSB i
amatörradio. I ett slutsteg med endast en transistor skulle denna
behöva klara fyra gånger så stor förlusteffekt

P.g.a. de låga impedansvärdena i transistoriserade förstärkarsteg
används transformatorer, vilka inte är frekvensselektiva och därför
inte dämpar övertoner. Med mottaktkopplingen alstras dock inte jämna
övertoner. För övertonsdämpning används fast avstämda bandpassfilter,
ofta ett per frekvensband, mellan drivsteg och slutsteg samt mellan
slutsteg och antenn.

För noggrann anpassning till antennen behövs en antennkopplare -
s.k. matchbox - med ett π-, T- eller L-kopplat LC-filter.

Att ett slutsteg är ``bredbandsavstämt'' är således en fråga om
definitioner.

\subsubsection{Högeffekts/utsteg med en tetrod}

Bild II 3-50

Bilden visar ett effektslutsteg för HF med ett elektronrör, en
s.k. tetrod, i katodkoppling.  Det kan även vara en triod eller en
pentod.

Med LC-kretsen i styrgallerkretsen filtreras (selekteras) önskade
signalfrekvens ut ur signalerna från föregående steg.

Drosslarna \(Dr\) spärrar HF och kondensatorerna \(C_1\), \(C_2\) och
\(C_3\) kortsluter (avkopplar) HF till jord. Allt för att hindra HF
att komma in i kraftaggregatet.

Bild II 3-49

HF-förstärkare kan råka i oönskad självsvängning. Orsakerna kan vara
många, bl.a.  dålig avkoppling av matningsspänningar, induktiv
och/eller kapacitiv återkoppling i kretsarna m.m.

Återkopplingsvägar både före och efter röret kan bilda oavsiktliga
svängningskretsar, som genererar självsvängning, ofta på mycket höga
frekvenser t.ex. i VHF-området. Sådana s.k. parasitsvängningar kan
stoppas/dämpas med UHF-drosslar (UHF Dr) omedelbart intill
röranslutningarna.

En åtgärd mot självsvängning i elektronrör är en motkopplingsväg från
anod till styrgaller över en trimningsbar
s.k. neutraliseringskondensator CN. Slutstegets utgångskrets kan
utformas på olika sätt. Bilden visar ett numera vanligt sätt, det
s.k. n-filtret (utläses pi-), som fungerar som
\begin{itemize}
\item en svängningskrets som är avstämd till sändningsfrekvensen,
\item ett övertonsdämpande lågpassfilter,
\item anpassning mellan rörets utgångsimpedans och antenntilledningens impedans.
\end{itemize}

Bild II 3-50 Högeffekts/utsteg med en tetrod
I
Bild II 3-51 Högeffekts/utsteg med två trioder

\subsection{Högeffekts/utsteg med två gallerjordade trioder (elektronrör)}

Bild II 3-51

Gallerjordad koppling innebär att elektronrörets styrgaller ligger på
HF-mässig nollpotential medan styrsignalen matas in på
katoden. likspänningen mellan katod och styrgaller väljs så att rörets
arbetspunkt blir den avsedda.

Gallerjordad koppling passar särskilt för slutsteg med höga effekter,
men fordrar en högre styreffekt än andra kopplingar. I gengäld
``överförs'' styreffekten till utgången via röret och ingår där i
uteffekten. I gallerjordad koppling är kapacitansen låg mellan katod
och anod, d.v.s.mellan in- och utgång. Därmed är risken för
självsvängning betydligt mindre än i ett katodjordat steg.

Uteffekten kan ökas genom att parallellkoppla två eller flera rör, som
då skall ha så lika data som möjligt. Uteffekten står i direkt
proportion till antalet rör.

Flera parallellkopplade rör medför emellertid ökade totala
rörkapacitanser, ökade kapacifanser i kopplingsledningarna m.m.,
vilket är till nackdel vid höga frekvenser.

Ett enda slutrör för hela effekten är emellertid dyrare än flera små
med tillsammans jämförbar effekt. Mottaktkoppling av två rör
(eng. ``push-pull'') i st.f. parallellkoppling har en fördel i högre
förstärkning, men nackdelar i mer komplicerad bandomkoppling av
svängningskretsar m.m. I moderna rörutrustade slutsteg för amatörradio
förekommer därför endast ett slutrör eller flera
parallellkopplade. Utgångskretsen är i regel ett nfilter med manuell
eller automatisk avstämning.

\subsection{Slutsteg med elektronrör jämfört med transistoriserade slutsteg}

Ett slutsteg med transistorer är kompakt och skaktåligt och använder
bara klenspänningar. Det är därför särskilt vällämpat för portabelt
och mobilt bruk.

Men transistorer är känsliga för överbelastning. Redan ytterst
kortvarig överbelastning eller överspänning kan förstöra dem.
Transistorer är också känsliga för termisk överbelastning. Särskilt
vid höga effekter i trånga utrymmen är det nödvändigt med god kylning,
eventuellt med fläkt.

Ett slutsteg med elektronrör är inte så skaksäkert, men är mycket
okänsligare i övriga avseenden. En nackdel är att det behövs extra
effekt för uppvärmning av rörens katoder samt höga anodspänningar, som
är farliga vid ovarsamhet. P.g.a. behovet av flera olika spänningar är
även strömförsörjningen för ett slutsteg med elektronrör mer
komplicerad och omfångsrik.

\subsection{Bestämning av PEP-effekten}

Bild II 3-52

Moduleringsspänningens topp-toppvärde \(U_{SS}\) mäts lämpligen med
ett oscilloscope. För korrekt belastning vid mätningen används en
konstlast.

Med topp-toppvärdet känt kan man med följande formler beräkna
toppvärdet (amplituden

\begin{align*}
  \text{toppvärdet (amplituden)} \quad
  U_S & = \frac{U_{SS}}{2}
  \quad \text{och} \\
  \text{effektivvärdet} \quad
  U_{eff} & = \frac{U_S}{\sqrt{2}}
\end{align*}

Effekten vid moduleringstopparna, s.k.  PEP (Peak Envelope Power), kan
beräknas med följande formler

\begin{align*}
  P_{PEP} &= \frac{U_{eff}^2}{R} \quad \text{respektive} \\
  P_{PEP} &= \frac{U_{SS}^2}{8R}
\end{align*}

Bild II 3-52 Bestämning av PEP-effekten

\subsection{Linjäritetskontroll vid SSB}

Bild II 3-53

Linjäriteten i en SSB-sändare kan kontrolleras med ett
oscilloscop. Sändaren moduleras då med två övertonsfria toner.

Slutsteget bör först belastas med konstlast upp till max tillåten
effekt. Resultatet jämförs därefter med antennen som last.

\subsubsection{Linjäritetens betydelse i förstärkare}

Bild II 3-54

Förstärkningen bör ske med god verkningsgrad och minsta möjliga
förvrängning, så att det alstras ett minimum av oönskade frekvenser
inom minsta möjliga bandbredd.

Linjär förstärkning innebär att den är lika över hela det aktuella
frekvensområdet Frekvensgången måste därför vara så rak som
möjligt. Med tilltagande olinjäritettillkommer nämligen allt fler
oönskade frekvenser.

Det uppstår blandningsprodukter av högre ordning vid olinjär
förstärkning. Genom förvrängning p.g.a. olinjär förstärkning uppstår
ömsesidiga summa- och skillnadsfrekvenser av de modulerande
frekvenserna.

Varje sådan blandningsprodukt blandar sig additivt och subtraktivt med
grundfrekvenserna till ytterligare blandningsprodukter av näst högre
ordning.

Dessa är:
\begin{itemize}
\item blandningsprodukter i LF-området och deras övertoner, vilka
  undertrycks i efterföljande HF-krets,

\item grundfrekvenserna och deras harmoniska övertoner, som alla ner
  till 1 :a harmoniska dämpas kraftigt av efterföljande H F-krets,

\item alla summa-och skillnadsfrekvenser av de förstnämnda frekvenserna.
\end{itemize}

Bild II 3-54 Linjäritetens betydelse

I området för nyttofrekvenserna kallas dessa produkter för
intermodulationsprodukter och ger talförvrängning.

Utanför nyttafrekvenserna uppfattas intermodulationsfrekvenserna som
störningar och kallas splatter. På grund av det lilla
frekvensavståndet till nyttasignalen kan den intermodulation, som
alstrats i slutsteget inte filtreras bort i efterhand.

Vid linjär drift uppträder grundfrekvensernas övertoner och
intermodulationsfrekvenser endast svagt inom och utom
överföringsbandet och kommer knappast att uppfattas som inkräktande på
annan radiotrafik.  De svaga övertonerna kommer också att dämpas
tillräckligt i π-filtret och eventuella ytterligare övertonsfilter.


\subsection{Utstyrningskontroll av slutsteg}

Slutstegets linjära utstyrningsområde överskrids, om ingångssignalens
amplitud blirför stor. Då ökar utgångssignalens amplitud inte mycket
mer, men utgångssignalens toppar blir tillplattade (s.k. klippta). Det
betyder att slutsteget är överstyrt.

Vid överstyrning uppstår signalförvrängningar, som medför
intermodulation, förvrängt tal, splatterstörningar och övertoner.  Den
extra effektökning som uppnås med överstyrning förbrukas i stort sett
till signalförvrängning och kommer inte nyttasignalen till
godo. Överstyrning skall därför undvikas.

Drivstegets uteffekt får inte vara så stor att slutsteget blir
överstyrt. Ett slutsteg med jordad katod blir fullt utstyrt redan vid
en driveffekt av ett fåtal watt. Ar uteffekten från drivsteget större,
än vad som behövs för full utstyrning av slutsteget, och driveffekten
inte kan regleras ner, så måste en dämpsats kopplas in mellan drivsteg
och slutsteg. En sådan dämpsats kan behöva ta upp en betydande effekt,
från en vanligt förekommande amatöradiosändare upp till 100 watt PEP.

Ett slutsteg med jordat galler fordrar en större driveffekt, varvid
risken för överstyrning i slutsteget är något mindre och de
förebyggande åtgärderna inte så omfattande.

Linjära slutsteg innehåller oftast en funktion kallad ALC (Automatic
Load Control), som kontinuerligt känner av driveffektens inverkan på
slutsteget När driveffekten blir för hög, alstras en kontrollspänning
i proportion till utstyrningsgraden. ALG-spänning återförs till
drivsteget och reglerar dess uteffekt så att överstyrning av
slutsteget inte sker. I transistoriserade slutsteg skapas
ALC-spänningen genom likriktning av slutstegets utspänning. I
rörslutsteg börjar styrgallret dra ström, när styrgallerspänningen
blir positiv i signaltopparna, vilket används för att styra
ALC-spänningen. När ALC-regleringen sätter in, är överstyrningen
således redan ett faktum. Överstyrning kan ske både på LF- och HF-nivå.

En orsak till övermodulering är för stor amplitud på den modulerande
signalen. Detta kan bl.a. avhjälpas med inställning av
mikrofonförstärkaren och riktig mikrofonhantering.
