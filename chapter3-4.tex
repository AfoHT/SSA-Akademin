\section{Förstärkare}

\subsection{Allmänt}

Bild Ii 3-40

Elektronrör och transistorer är de aktiva komponenter som används i oräkneliga elektroniska kopplingar för alstring av signaler, för
förstärkning och blandning av signaler, för
multiplicering av signalfrekvenser etc.
Transistorn presenteras i avsnitt 2.6 och
elektronröret i avsnitt 2.7.
Först förekom endast elektronrör. Dessa
har emellertid på några få decennier nästan
helt ersatts av transistorer. Elektronrör används dock fortfarande, särskilt i effektförstärkare för sändare. Det finns därför skäl att
här behandla såväl elektronrör som transistorer.

Förstärkning
Med förstärkning avses här kvoten av amplituden i utgående och inkommande signal,
varvid frekvensgången har inverkan.
Frekvensgång
Förstärkare arbetar endast inom ett visst
frekvensområde, vilket kan skilja från fall till
fall.
Bandbredd
Det frekvensområde där förstärkaren arbetar med fulla data kallas bandbredd. Bandgränserna uttrycks som en nedre och övre
gränsfrekvens, där signalnivån avviker från
ett givet värde, vanligen med högst 3 dB.
För LF-förstärkare för amatörradiobruk
är kravet på bandbredd litet; inom ett band
av 300 Hz till 3 kHz uppnås godtagbar
återgivningskvalitet för taL Bandbredden
bestäms främst av kondensatorer i kretsen
avsedda för överföring och avkoppling.
HF-förstärkare används för signaler med
hög frekvens, typiskt 100kHz och däröver.
Det finns s.k. bredbandig a förstärkare för ett
stort frekvensområde, men även avstämda
förstärkare för smala frekvensband.

\subsection{Huvudegenskaper hos förstärkare}
LF- och HF-förstärkare
Bild II 3-41
Med LF-förstärkare menas förstärkare som
arbetar med signaler i det lägre frekvensområdet, typiskt upp till c:a i 00 kHz. LF-förstärkare är mycket vanliga såväl i mottagare
som sändare. Utöver de aktiva komponenterna (transistorer, elektronrör ) är kondensatorer och resistorer de viktigaste passiva.

Med H F-förstärkare menas förstärkare som
arbetar med signaler med högre frekvenser
än dem i LF-området. Även HF-förstärkare
är mycket vanliga såväl i mottagare som
sändare. De används t.ex. i mottagarnas
ingångs- och mellanfrekvenssteg, liksom i
sändarnas oscillatorer, signalberedningssteg
och slutsteg.
Utöver de komponenter, som även finns
i LF-förstärkare, används kombinationer av
frekvensberoende komponenter såsom
induktorer och kondensatorer.

Anod

·~·

Galler

.p

Katod

n

Kollektor
Bas
Emitter

npn
Bild II 3-40 Från elektronrör till transistor

113-29

KRETSAR

o[]

o[]
Katod kopp l i ng

Emitterkoppling

Bild II 3-41 Principen för förstärkare med elektronrör respektive transistor

\subsection{Grundkopplingar för förstärkarsteg}
Bild 113-42

l det föregående har redan visats att en av
polerna i ingången respektive utgången i en
förstärkare är gemensam. l ovanstående
bild är rörförstärkarens katod den gemensamma polen -därav namnet katodkoppling.
På liknande sätt är N PN-transistorns emitter
gemensam -därav namnet emitterkoppling.
På ett liknande sätt kan någon annan pol
vara gemensam. Man får då i stället en
baskoppling eller kollektorkoppling.
Beroende av kopplingsätt fås olika egenskaper. På nästa sida visas tre olika grundkopplingar för ett elektronrör (triod) respektive en NPN-transistor.
l praktiken känns en grundkoppling igen
på vilken elektrod som är avkopplad till Opotential över en kondensator.

Emitterkoppling används för LF och HF
när hög förstärkning eftersträvas. Eftersom
effektförstärkningen är produkten av
spännings- och strömförstärkningen, så fås
en effektförstärkning av mellan 200 till50000
gånger. Nackdelen med denna koppling är
den ibland låga ingångsimpedansen och
den relativt låga gränsfrekvensen.
Baskoppling använd som H F-förstärkare pågund av sin höga gränsfrekvens och
goda isolation mellan in- och utgång.
Kollektorkoppling används när hög ingångsimpedans och utgångsimpedans önskas. Denna koppling har emellertid ingen
spänningsförstärkning, men kan användas
för s.k. impedansomvandling.

Grundkopplingarnas typiska egenskaper vid NPN-transistor
Egenskap

Emitterkoppling

Baskoppling

Z in

medel

1 KQ

liten

son

stor

100 kn

Z ut

medel

10 kQ

stor

100 kQ

liten

50 kQ

ström-

stor

100 ggr

<1

0.9 ggr

stor

100 ggr

spänning-

stor

100 ggr

stor

100 ggr

<1

0.99 ggr

effekt-

mycket stor 10000 ggr

stor

100 ggr

stor

100 ggr

motfas

medfas

oo

medfas

oo

Kollektorkoppling

Förstärkning

Fasläge

113-30

180°

ETSAR

Ut
In

In

Katodkoppling

~

In

!

l

j

Emitterkoppling

l
In

Ut

Ut

Gallerkopp li ng

j

Ut

In

Ut

Baskoppling

~

In

!

In,

In

Anod kopp! in g

Kollektorkoppling

Bild II 3-42 Grundkopplingar för elektronrör och NPN-transistor

113-31

KR

EPT

\subsection{Stabilisering av arbetspunkten}

För att en förstärkare skall kunna arbeta på
avsett sätt måste arbetspunkten, d.v.s.
arbetsströmmens vilavärde ställas rätt.
Det gör man genom att placera en förspänning över den styrande elektroden i
elektronröret eller transistorn i fråga.
l en katodkopplad rörsförstärkare innebär det att styrgallret skall ges en viss negativ spänning i förhållande till katoden. Det
kan man göra t.ex. med en separat spänningskälla eller vanligare med en avkopplad
res istor mellan katod och minuspolen (jord).
l en emitterkopplad transistorförstärkare
innebär det att basen skall ges en viss positiv
spänning i förhållande till emittern. Det kan
man göra t.ex. med en separat spänningskälla eller vanligare med en avkopplad resisto r mellan emittern ochminuspolen samt en
resistiv spänningdelare mellan plus- och
minuspolen.

\subsection{Klass A-, B- och C-förstärkare}
Arbetspunkt
Arbetspunkten för förstärkare väljs olika,
beroende på önskat arbetssätt. En olämpligt
vald arbetspunkt resulterar i förvrängning av
utsignalens form i förhållande till insignalens
form, s.k.distorsion. Distorsion uppstår även
vid överstyrning, d.v.s. när insignalens amplitud är för stor för att kunna återges med
oförändrad form, även om arbetspunkten är
rätt vald.
Med avseende på arbetspunktens läge
klassas därför förstärkare på sätt som framgår av följande diagram för elektronrör. En
emitterjordad NPN-transistor får anses mest
motsvara elektronrörkopplingen här nedan.
Anodströmmen la motsvaras då närmast av
kollektorströmmen le och styrgallerspänningen U~ 1 av spänningen UsE· Den stora skillnaden år att styrgallerspänningen i dessa fall
alltid är negativ medan bas/emitterspänningen är positiv. styrspänningens relativa läge
(arbetspunkten) mellan olika arbetsklasser
är emellertid lika.

113-32

KRETSAR
Klass A
Bild II 3-44
Klass A är ett arbetssätt i linjära LF- och HFförstärkarsteg, t.ex. i mottagare samt AMoch SSB-modulerade sändare. Vilavärdet
på strömmen i huvudkretsen, den s.k. arbetspunkten, placeras i mitten på den rakaste delen av styrkaraktäristikan (1=0.5 lmax).
Därmed fås låg distorsion. Verkningsgraden
är upp till 50 °/o.

ningskretsen. En resonanskrets med högt
Q-värde behövs som utgångskrets varvid
amplituddistorsion inte framstår som besvärande vid CW och FM. Med hjälp av en
utgångskrets kan frekvensmultiplicering utföras med förstärkare i klass C.
(På följande tre bilder är IR=anodviloström).

Klass AB
Klass AB är ett godtagbart linjärt arbetssätt
för AM- resp. SSB-modulering, men med en
lägre viloström. Arbetspunkten ligger mellan
den för klass A och B. Ett linjärt arbetssätt
enligt klass A är visserligen önskvärt vid
SSB, men verkningsgraden är lägre. Klass
AB är en kompromiss med bättre verkningsgrad utan en alltför stor distorsion.
Klass B
Bild II 3-45
Klass B är ett olinjärt arbetssätt med en låg
vilaström i förhållande till !max• d.v.s. arbetspunkten ligger i nederkant av styrkaraktäristikans nedre krökta del. Verkningsgraden
är upp till 67o/o. Trots det används klass 8 i
linjäraLF-och H F-förstärkarsteg t. ex. i SSBsändare.
Om klass B skulle tillämpas i ett slutsteg
med endast ett rör eller en transistor skulle
större delen av uteffekten förloras i splatter,
d.v.s. som förvrängda signaler långt vid sidan om den egentliga nyttosignalen. Ett sätt
att undvika det är att använda en avstämd
utgångskrets med högt Q-värde. Linjär förstärkning kan också erhållas med två mottaktkopplade rör eller transistorer i klass B.
Utgångskretsen behöver då inte vara avstämd av linjäritetsskäl.
Klass C
Bild 113-46
Klass C används i HF-förstärkarsteg i FM-,
CW- och AM-sändare. Arbetssättet är kraftigt olinjärt. Vilaströmmen är noll, d.v.s. arbetspunkten ligger på den negativa delen av
styrkaraktäristikan. Endast toppen av den
ena halvvågen av insignalen återges och i
starkt förvrängd form. Verkningsgraden är
upp till80°/o. Övertonerna dämpas av sväng-

~järt

l

Förvrängning
(distorsion)

genom fel arbetspunkt

Bild If 3-44 Förstärkare i klass A

113-33

KRETSAR

Bild II 3-45 Förstärkare i klass B

\subsection{Frekvensmultiplicering}
Bild 113-47

Frekvensmultiplicering kan användas för att
skapa en högre frekvens än den som avges
av oscillatorn. Oscillatorn följs då av ett eller
flera frekvensmultiplicerande förstärkarsteg
som arbetar i klass C.
l utgången av ett frekvensmultiplicerande steg måste finnas en svängningskrets,
som är avstämd till önskad frekvens, d.v.s.
överton av insignalen. Denna överton förstärks i efterföljande förstärkarsteg, vilket
också kan vara frekvensmultiplicerande.
Ju högre multiplikationsfaktorn är, desto
högre förspänning krävs för att svängningskretsen i utgången skall svänga obehindrat.
Med hög multipliceringsfaktor i ett enda steg
dämpas signalen då så mycket att en hög
förstärkning behövs i efterföljande steg. l
praktiken anordnas därför en kedja av frekvensdubblande och frekvenstripplanda
Den totala multipliceringsfaktorn är faktorerna för vartdera steget multiplicerat med varandra.
Som exempel visar bilden blockschemat
för en VHF-sändare med oscillatorkristaller i
8 MHz-området. Som räkneövning kan andra kristallfrekvenser sättas in för beräkning
av den slutliga sändningsfrekvensen. l frekvensmultiplicerande sändare kan även slutsteget arbeta i klass C, vilket är vanligt i
sändare förtelegrafi eller FM-telefoni. För att
då förhindra utsändning av alla de övertoner

113-34

Bild II 3-46 Förstärkare i klass C
som alstras i förstärkarkedjan, så förses
slutstegets utgång med en svängningskrets
som är avstämd till sändningsfrekvensen.
Övertonsdämpningen kan förbättras ytterligare med ettefterföljande lågpassfilter. Overtoner för 144 MHz är 288 MHz, 432 MHz
o.s.v.
Frekvensmultiplicering behöver nödvändigtvis inte göras med ett förstärkarsteg i
klass C. En diod har nämligen olinjär karaktäristik och därmed alstras det övertoner i de
strömmar som passerar genom den. En av
dessa övertoner kan filtreras fram och förstärkas. T.ex. finns det frekvenstripplingssteg byggda kring en speciell typ av kapacitansdiod- varaktordiod. Vanliga frekvensområden för s.k. varaktortripplare är 144/
432 MHz och 432/1296 MHz.
Såväl signalen från en kristalloscillator
som den från en VFO kan multipliceras till en
högre frekvens.
Förr täckte VFO i amatörradiosändarna
oftast frekvensområdet 3.5-3.8 MHz. Med
en så vald VFO-frekvens kunde alla upplåtnafrekvensband för amatörradio nås med
frekvensmultiplicering. De ursprungliga amatörradiobanden i KV-området ligger fortfarande harmoniskt relaterade av detta skäl.
Således
2 • 3.5 = 7 MHz
2 · 2 · 3.5 = 14 MHz
2 • 3 · 3.5 = 21 MHz
2 · 2 · 2 · 3.5 = 28 MHz

KRETSAR

PT
-~ 1%1

T

CD

r·---{I>J---~--{g
Olinjär förstärkare

~····
f

\

f"' " 24 HHz

= 8 MHz

!f

8 MHz

+övertoner

~---·----

·-·lliJ--..--------..·--· ---

r

------------o

.,..............

n = 3 · 3 · 2 = 18

fs : : n. fa
fs = 18 · fa

[ii

fq = 8, ... MHz

l

f :.: ..., ... MHz

f = B, ... MHz

= 24MHz

24 MHz

.
f re kvenstnpp 1are

[o

DJtAAAMAft

!Lfl[ .....

f= ... , ... MHz

f = .. , ... MHz

f = ... , ... MHz

Fyll i frekvensvärdena för styrkristallen och beräkna resterande frekvensvärde
i multiplikatorkedjan

Bild 113-47 Frekvensmultipliceringskedja
Vid frekvensmultiplicering flerfaldigas inte
bara oscillatorfrekvensen utan även variationerna i den. Om t.ex. VFO-frekvensen i
området 3.5 MHz ändras med 50 Hz, så
ändras utfrekvensen i området 28 MHz med
2 · 2 • 2 = 400 Hz. Alla frekvenser i signalen
multipliceras på detta sätt. Amplitudmodulerad telefoni kan därför inte överföras genom
en frekvensmultipliceringskedja utan att talet förvrängs.

Se5.3

113-35

KRETSAR
\subsection{Sändarslutsteg}
Slutsteg med en transistor
Bild 113-48

Transistorslutsteg för HF byggs vanligen
emitterkopplade p.g.a. den högre effektförstärkningen.
Bilden visar ett sådant förstärkarsteg.
Kollektorbelastningen består av en svängningskrets. För att anpassa transistorns kollektorimpedans till svängningskretsens impedans, har kollektorn anslutits till ett uttag på
svängningskretsens spole.
Orossel Dr och kondensator C fungerar
som en HF-mässig avkoppling av strömförsörjningen.
Uteffekten tas ut från svängningskretsen
över en kopplingslindning med samma impedans som belastningen.
För linjär återgivning krävs drift i klass A
eller möjligen klass AB.
Or* +U

Bild II 3-48 S!utsteg med en transistor

slutsteg med två transistorer
Bild 113-49
Ett mottaktkopplat (e ng. pus h-pull) förstärkarsteg i klass B har god verkningsgrad
samtidigt som det är nöjaktigt linjärt för SSB
i amatörradio. l ett slutsteg med endast en
transistor skulle denna behöva klara fyra
gånger så stor förlusteffekt
P.g.a. de låga impedansvärdena i transistoriserade förstärkarsteg används transformatorer, vilka inte är frekvensselektiva
och därför inte dämpar övertoner. Med mottaktkopplingen alstras dock inte jämna övertoner. För övertonsdämpning används fast
avstämda bandpassfilter, ofta ett per frekvensband, mellan drivsteg och slutsteg samt
mellan slutsteg och antenn.
För noggrann anpassning till antennen
behövs en antennkopplare - s.k. matchbox
- med ett n-, T- eller L-kopplat LC-filter.
Att ett slutsteg är "bredbandsavstämt" är
således en fråga om definitioner.
Högeffekts/utsteg med en tetrod
Bild II 3-50
Bilden visar ett effektslutsteg för HF med ett
elektronrör, en s.k. tetrod, i katodkoppling.
Det kan även vara en triod eller en pentod.
Med LC-kretsen i styrgallerkretsen filtreras (selekteras) önskade signalfrekvens ut
ur signalerna från föregående steg.
Orossiarna Dr spärrar HF och kondensatorerna C1 , C2 och C3 kortsluter (avkopplar)
HF till jord. Allt för att hindra HF att komma in
i kraftaggregatet

bredband
U2

Bild 113-49

113-36

omkopplingsbart för
vart och ett av banden

KRETSAR
HF-förstärkare kan råka i oönskad självsvängning. Orsakerna kan vara många, bl. a.
dålig avkoppling av matningsspänningar, induktiv och/eller kapacitiv återkoppling i kretsarna m.m.
Återkopplingsvägar både före och efter
röret kan bilda oavsiktliga svängningskretsar, som genererar självsvängning, ofta på
mycket höga frekvenser t.ex. i VHF-området. Sådana s.k. parasitsvängningar kan
stoppas/dämpas med UHF-drosslar (UHF
Dr) omedelbart intill röranslutningarna.

En åtgärd mot självsvängning i elektronrör är en motkopplingsväg från anod till styrgaller över en trimningsbar s.k. neutraliseringskondensator CN.
slutstegets utgångskrets kan utformas
på olika sätt. Bilden visar ett numera vanligt
sätt, det s.k. n-filtret (utläses pi-), som fungerar som
• en svängningskrets som är avstämd till
sändningsfrekvensen,
• ett övertonsdämpande lågpassfilter,
• anpassning mellan rörets utgångsimpedans och antenntilledningens impedans.

UKV-drossel

Cp

Bild II 3-50 Högeffekts/utsteg med en tetrod

- Ug1

I

I I

I
Bild II 3-51 Högeffekts/utsteg med två trioder

113-37

KRETSAR
Högeffekts/utsteg med två gallerjordade trioder (elektronrör)
Bild II 3-5i
Gallerjordad koppling innebär att elektronrörets styrgaller ligger på HF-mässig nollpotential medan styrsignalen matas in på
katoden. likspänningen mellan katod och
styrgaller väljs så att rörets arbetspunkt blir
den avsedda.
Gallerjordad koppling passar särskilt för
slutsteg med höga effekter, men fordrar en
högre styreffekt än andra kopplingar. l gengäld "överförs" styreffekten till utgången via
röret och ingår där i uteffekten. l gallerjordad
koppling är kapacitansen låg mellan katod
och anod, d.v.s.mellan in- och utgång. Därmed är risken för självsvängning betydligt
mindre än i ett katodjordat steg.
Uteffekten kan ökas genom att parallellkoppla två eller flera rör, som då skall ha så
lika data som möjligt. Uteffekten står i direkt
proportion till antalet rör.
Flera parallellkopplade rör medför emellertid ökade totala rörkapacitanser, ökade
kapacifanser i kopplingsledningarna m.m.,
vilket är till nackdel vid höga frekvenser.
Ett enda slutrör för hela effekten är emellertid dyrare än flera små med tillsammans
jämförbar effekt. Mottaktkoppling av två rör
(eng. "push-pull") i st.f. parallellkoppling har
en fördel i högre förstärkning, men nackdelar i mer komplicerad bandomkoppling av
svängningskretsar m.m. l moderna rörutrustade slutsteg för amatörradio förekommer
därför endast ett slutrör eller flera parallellkopplade. Utgångskretsen är i regel ett nfilter med manuell eller automatisk avstämning.

\subsection{Slutsteg med elektronrör jämfört med transistoriserade slutsteg}

Ett slutsteg med transistorer är kompakt och
skaktåligt och använder bara klenspänningar. Det är därför särskilt vällämpat för portabelt och mobilt bruk.
Men transistorer är känsliga för överbelastning. Redan ytterst kortvarig överbelastning eller överspänning kan förstöra dem.
Transistorer är också känsliga för termisk
överbelastning. Särskilt vid höga effekter i
trånga utrymmen är det nödvändigt med god
kylning, eventuellt med fläkt.

113-38

Ett slutsteg med elektronrör är inte så
skaksäkert, men är mycket okänsligare i
övriga avseenden. En nackdel är att det
behövs extra effekt för uppvärmning av rörens katoder samt höga anodspänningar,
som är farliga vid ovarsamhet. P.g.a. behovet av flera olika spänningar är även strömförsörjningen för ett slutsteg med elektronrör
mer komplicerad och omfångsrik.

\subsection{Bestämning av PEP-effekten}
Bild 113-52
Moduleringsspänningens topp-toppvärde
Uss mäts lämpligen med ett oscilloscope.
För korrekt belastning vid mätningen används en konstlast
Med topp-toppvärdet känt kan man med
följande formler beräkna
toppvärdet (amplituden

U - Uss

effektiwärdet

Us
Uett= {2.

s- 2

och

Effekten vid moduleringstopparna, s.k.
PEP (Peak Envelope Power), kan beräknas
med följande formler

ue~
k.
P,PEP =
R respe t1ve
P,
PEP=

Us~
SR

PA

os c i lloscope

Bild II 3-52 Bestämning av PEP-effekten

KRETSAR
\subsection{Linjäritetskontroll vid SSB}

Bild II 3-53
Linjäriteten i en SSB-sändare kan kontrolleras med ett oscilloscop. Sändaren moduleras då med två övertonsfria toner.

slutsteget bör först belastas med konstlast
upp till max tillåten effekt. Resultatet jämförs
därefter med antennen som last.

SSB-TVATONSSl GNAL
Oscillogram

u

Spektrum

Spektrumanalys

(tid/spänn i ngsdiagra m l
normal SSB-bandbredd

-----t

l.lL 

f

Idealisk linjär förstärkning (klass A- utan överstyrning)

Nästan linjär förstärkning (klass AB- utan överstyrning)

Olinjär förstärkning - för låg vilaström

överstyrning (klippning)
SSB-SIGNAL VID TAL ("aaah")

Linjär förstärkning

Överstyrning (klippning)

Bild II 3-53 Unjäritetskontro/1 vid SSB
113-39

PT

KRETSAR
Linjäritetens betydelse i förstärkare
Bild II 3-54
Förstärkningen bör ske med god verkningsgrad och minsta möjliga förvrängning, så att
det alstras ett minimum av oönskade frekvenser inom minsta möjliga bandbredd.
Linjär förstärkning innebär att den är lika
över hela det aktuella frekvensområdet Frekvensgången måste därför vara så rak som
möjligt. Med tilltagande olinjäritettillkommer
nämligen allt fler oönskade frekvenser.
Det uppstår blandningsprodukter av högre ordning vid olinjär förstärkning. Genom
förvrängning p.g.a. olinjär förstärkning uppstår ömsesidiga summa- och skillnadsfrekvenser av de modulerande frekvenserna.

Varje sådan blandningsprodukt blandar
sig additivt och subtraktivt med grundfrekvenserna till ytterligare blandningsprodukter
av näst högre ordning.
Dessa är:
0
blandningsprodukter i LF-området och
deras övertoner, vilka undertrycks i efterföljande HF-krets,
e
grundfrekvenserna och deras harmoniska
övertoner, som alla ner till 1 :a harmoniska dämpas kraftigt av efterföljande
H F-krets,
0
alla summa-och skillnadsfrekvenser av
de förstnämnda frekvenserna.

l ngångssignal

l  l 
Utgångssignal vid linjär förstärkning

Utgångssignal vid olinjär förstärkning

Olinjär karaktäristik

Linjär karaktäristik

la

Ug

Bild II 3-54 Linjäritetens betydelse

113-40

Ug

KRETSAR
l området för nyttafrekvenserna kallas
dessa produkter för intermodulationsprodukter och ger talförvrängning.
Utanför nyttafrekvenserna uppfattas
intermodulationsfrekvenserna som störningar och kallas splatter. På grund av det
lilla frekvensavståndet till nyttasignalen kan
den intermodulation, som alstrats i slutsteget
inte filtreras bort i efterhand.
Vid linjär drift uppträder grundfrekvensernas övertoner och intermodulationsfrekvenser endast svagt inom och utom överföringsbandet och kommer knappast att uppfattas som inkräktande på annan radiotrafik.
De svaga övertonerna kommer också att
dämpas tillräckligt i n-filtret och eventuella
ytterligare övertonsfilter.

tion till utstyrningsgraden. ALG-spänning
återförs till drivsteget och reglerar dess uteffekt så att överstyrning av slutsteget inte
sker. l transistoriserade slutsteg skapas ALGspänningen genom likriktning av slutstegets
utspänning. l rörslutsteg börjar styrgallret
dra ström, när styrgallerspänningen blir positiv i signaltopparna, vilket används för att
styra ALG-spänningen. När ALG-regleringen
sätter in, är överstyrningen således redan ett
faktum. Överstyrning kan ske både på LFoch HF-nivå.
En orsak till övermodulering är för stor
amplitud på den modulerande signalen. Detta
kan bl.a. avhjälpas med inställning av
mikrofonförstärkaren och riktig mikrofonhantering.

\subsection{Utstyrningskontroll av slutsteg}

slutstegets linjära utstyrningsområde överskrids, om ingångssignalens amplitud blirför
stor. Då ökar utgångssignalens amplitud inte
mycket mer, men utgångssignalens toppar
blir tillplattade (s.k. klippta). Det betyder att
slutsteget är överstyrt.
Vid överstyrning uppstår signalförvrängningar, som medför intermodulation, förvrängt tal, splatterstörningar och övertoner.
Den extra effektökning som uppnås med
överstyrning förbrukas i stort sett till signalförvrängning och kommer inte nyttasignalen
till godo. Överstyrning skall därför undvikas.
Drivstegets uteffekt får inte vara så stor
att slutsteget blir överstyrt. Ett slutsteg med
jordad katod blir fullt utstyrt redan vid en
driveffekt av ett fåtal watt. Ar uteffekten från
drivsteget större, än vad som behövs för full
utstyrning av slutsteget, och driveffekten inte
kan regleras ner, så måste en dämpsats
kopplas in mellan drivsteg och slutsteg. En
sådan dämpsats kan behöva ta upp en betydande effekt, från en vanligt förekommande
amatöradiosändare upp till1 00 watt PEP.
Ett slutsteg med jordat galler fordrar en
större driveffekt, varvid risken för överstyrning i slutsteget är något mindre och de
förebyggande åtgärderna inte så omfattande.
Linjära slutsteg innehåller oftast en funktion kallad ALG (Automatic Load Gontrol),
som kontinuerligt känner av driveffektens
inverkan på slutsteget När driveffekten blir
för hög, alstras en kontrollspänning i propor113-41

KRETSAR

113-42

KRETSA
