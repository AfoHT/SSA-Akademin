\section{Transistorn}
\textbf{HAREC a.\ref{HAREC.a.2.6}\label{myHAREC.a.2.6}}

Allmänt

En transistor består av skikt av halvledarelement som sammanfogats. Vanligt är två Nskikt och ett mellanliggande P-skikt (NPNtransistor) eller två P-skikt och ett mellanliggande N-skikt (PNP-transistor). Skikten är
försedda med anslutningar.

B~

...-----Kollektor
(C)
r

n
p

~

+ + ~-------Bas (B)

+ +
~

n
G

l... Emitter (E)

E

NPN

PNP

FET

Bild II 2-16 Schemasymboler

n

c

n

C

Vanliga transistortyper
N PN-transistorer (bipolära)
PNP-transistorer (bipolära)
FET-transistorer (fälteffekt-)

NPN-transistorer

Halvledarskikten kallas
E emitter
B bas
C kollektor

E BC

Spärrzonerna
Bild II 2-17 överst
Mellan skikten B och E respektive mellan B
och C bildas zoner, vars ledningsförmåga
kan styras elektriskt över anslutningarna.
Bild II 2-17 mitten
Spänningskällan U8 E
Mellan bas och e mitter finns en diodsträcka.
När en positiv spänning läggs på basen och
en negativ spänning på emittern, så polariseras diodsträckans spärrzon i passriktningen. Spärrzonen upplöses då och det flyter en
s.k. basström 16 •

p~B
+ +
n

+

E

Bild II 2-17 Skikten i en bipolär transistor

112-23

K MP NENTER
Bild II 2-17 nederst
Spänningskällan UeE

När en positiv spänning läggs på kollektorn
och en negativ spänning läggs på emittern,
så polariseras diodsträckan i spärriktningen.
Spärrzonen förstärks då och det flyter ingen
ström.
Bild 112-18
Inverkan av både U8 Eoch UeE

Två spänningskällor U8 Eoch UcE ansluts till
en emitterkopplad NPN-transistor.
Ur den starkt dopade emitterzonen strömmar elektronerna in i den svagt dopade
baszonen (spänning: U8 E). De flesta elektronerna blir emellertid inte kvar i basen. De
stöter igenom det tunna basskiktet och når
fram till kollektorskiktet med spänningen U E.
Det flyter en kollektorström.
c
För strömmer l (emitterström), 18 (basström) och le (kollektorström) gäller:
lE= Is+ le där 18 <<le

(<<mycket mindre än)

Kollektorströmmen le kan styras med basspänningen U8 E.
En liten ändring i basspänningen ger stor
förstärkande verkan i kollektorströmmen.

n

c

p + + B
+ +

n

h
FE

hFE
IIie

!ll 8

=IIie
/:lf
B

strömförstärkningsfaktorn
ändringen i kollektorströmmen
ändringen i basströmmen

PNP-transistorer
Ersätter man de två N-skikten i en NPNtransistor med P-skikt och P-skiktet med ett
N-skikt så erhåller man en PNP-transistor.
Uppbyggnad, koppling och användning
av en PNP-transistor motsvarar i övrigt den
för en NPN-transistor. Spänningskällorna
måste emellertid ha motsatt polaritet.

R

Is

R

UcE

E
..,..
lE

la<< lE

lE

=Is+ le

Bild II 2-18 Emitterkopplad transistor

112-24

Förstärkningsfaktor
Om strömmen i ingångskretsen för en transistor ändras, så kan strömmen i utgångskretsen ändras mer. Det blir då en förstärkning.
Av sambandet le= f(/8 ) framgår strömförstärkningsfaktorn ~ eller hFE' som är kvoten av ändringen i utgångsströmmen och i
ingångsströmmen i transistorns aktiva (linjära) område.
Bild II 2-19
För emitterkoppling gäller:

UsE

UcE

l c (mA)

100

5

10

UcE ::0 V

UcE = 5 V

UsE {mV)

Bild II 2-19 Karaktäristika för transistor BC 107

112-25

K
Fälteffekttransistorer

Allmänt
Fälteffekttransistorer (förkortat FET) har
mycket hög ingångsimpedans och styrströmmen blir därför mycket svag. Man säger
därför att en FET är spänningsstyrd.
Även NPN- och PNP-transistorer- kallade bipolära transistorer - styrs med spänning, men dessa typer har en relativt lågt
ingångsimpedans och därför högre styrströ m.
Man säger därför att de är strömstyrda.

D+

sBild II 2-20 Schemasymbol för en FET

Bild 112-21 Skikten i en N-kanal FET

Fälteffekttransistorn har tre anslutningar
(elektroder)
S source (katod)
D drain (anod)
G gate (grind, styre)
Fälteffekttransistorns uppbyggnad
Bild II 2-21
Bilden visar ett N-ledande skikt (även kallat
N-kanal) med elektroderna S och D anslutna
till respektive ändar av skiktet. N-kanalen
passerar mellan två P-ledande skikt förbundna med styrelektraden G.
När en spärrspänning läggs mellan G
och S, så breder spärrskikten ut sig och Nkanalen blir trängre. Läggs en negativ spänning på S och en positiv spänning på D, så
kommer det att flyta en ström i N-kanalen.
Strömmens styrka kan påverkas med spänningen på G.
En liten spänningsändring llU medför
stor ändring av strömmen lll i tf-kanalen.
Detta innebär förstärkning. os

Bild II 2-22
l en MOS-FET är G-elektroden isolerad med
ett kiseloxidskikt Funktionssättet är samma
som för en FET. Drain-strömmen kan ökas
eller minskas med hjälp av en positiv respektive negativ spänning på G.

112-26

Bild 112-22 Skikten i en N-kanal MOS-FET
Resistansen mellan gate och source
För att erhålla en förstärkning med en FETtransistor, sätter man in en resister R0 i
drain-strömkretsen. Över resistorn uppstår
då spänningsändringar i proportion med
strömändringarna.
För att fastställa vilaströmmen och därmed arbetspunkten för samma transistor
sätter man in en resister R i source-strömkretsen. storleken på soufce-resistorn ger
sig av önskad gate-förspänning -U 88 .
 -UGs
R s-

lo

K MP NENTER
Sambandet drain-ström och spänning
Bild 112-23
För att beskriva en FET använder man sig
av karaktäristiska kurvor. Vi har redan presenterat bipolära transistorers in- och utgångsegenskaper i kuNform. Eftersom ingångsströmmen (gateströmmen) i en FET
är praktiskt taget noll, så är en sådan kuNa
utan praktisk mening. l stället framställer
man grafiskt sammanhanget mellan styrspänningen UGs och utgångsströmmen (drainströmmen 10 ). Eftersom det finns N-kanal
FET och P-kanal FET så skiljer polariteten
på UGs för dessa båda typer.

Bild II 2-23 Karaktäristikför N-kanal FET
