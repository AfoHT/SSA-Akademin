\chapter*{INLEDNING}

\Huge{VAD, HUR, VAR?}\normalsize

\section*{VAD behöver en radioamatör kunna?}

CEPT är ett samarbetsorgan mellan europeiska länders teleadministrationer (myndigheter). En av dem är svenska Post- och telestyrelsen - PTS.

Dessa administrationer har antagit rekommendationer om sinsemellan harmoniserade krav på radioamatörers kompetens.


Sverige har antagit CEPT-rekommendationen T/R 61-02. Vid genomförandet av
kompetensprov ska de i den rekommendationen angivna kraven särskilt beaktas.

\begin{rev-omarbetas}

För den som godkänts i ett sådant prov utfärdas ett harmoniserat radioamatörcertifikat
(HAREC). Rekommendationen anger kompetensnivåerna HAREC A och HAREC B. De svenska
certifikatsklasserna CEPT 1 respektive CEPT 2 motsvarar dessa nivåer.

\begin{itemize}
\item Kompetenskraven för klassen CEPT 1
(HAREC nivå A) omfattar

Ämnesgrupp I - Färdighet i morsesignalering,

Ämnesgrupp II - Kunskaper i radioteknik,

Ämnesgrupp III - Kännedom om reglemente och trafikmetoder.

\item Kompetenskraven för klassen CEPT 2
(HAREC nivå B) omfattar endast

Ämnesgrupp II- Kunskaper i radioteknik,

Amnesgrupp III- Kännedom om reglemente och trafikmetoder.

Eftersom certifikat av denna klass ej dokumenterar föreskriven färdighet i
morsesignalering, är de lägre frekvensbanden för amatörradio ej tillgängliga för klassen.
\end{itemize}

\end{rev-omarbetas}

\subsection*{HUR blir man radioamatör?}

\begin{rev-omarbetas}

För att få inneha och använda amatörradiosändare måste man ha Post- och telestyrelsens
amatörradiotillstånd (licens). Det får man ansöka om efter att ha avlagt certifikatsprov
i avsedd klass med godkännande.

Till tillståndet knyts en internationellt unik anropssignal.

\end{rev-omarbetas}

\subsection*{VAR hålls det certifikatskurser?}

Vissa amatörradioklubbar, militära förband,
FRO-förbund och andra sammanslutningar
håller certifikatskurser. Det går också att
studera på egen hand.

\subsection*{VILKA läromedel behöver man?}

\begin{rev-omarbetas}

Denna bok omfattar teorin för certifikatsklasserna CEPT 1 och CEPT 2.

För ämnesgrupp l behövs även någon separat kurs i praktisk morsesignalering.
Sådana finns på ljudband eller datadiskett

För ämnesgrupp III behövs även gällande lagar, föreskrifter och anvisningar inom
området.

Alla dessa läromedel kan köpas från SSA.

\end{rev-omarbetas}
