\documentclass[a4paper]{article}
\usepackage[swedish]{babel}
\usepackage[utf8]{inputenc}

\setlength{\parindent}{0pt}

\title{Diskussionsunderlag för omdisponering av kursbok för amatörradio}
\author{Hans SM3UTY}
\date{Mellandagarna 2016}

\begin{document}

\maketitle

\begin{abstract}
\noindent
  Här är mitt förslag till ny disposition av kursboken för
  amatör\-radio\-certifikat, ämnat som underlag för diskussioner.

  Det övergripande målet och syftet med den föreslagna förändring är
  att motivera läsaren att ta certet.

  Jag har med flit inte tagit med alla detaljer kring vilka kapitel
  och deras inbördes ordning, eftersom syftet med det här dokumentet
  är att vara underlag för vidare diskussioner.

  I vårt arbete att uppdatera SSAs utbildningsmaterial kommer även
  andra böcker tas fram (trafikbok, referensbok), dessa diskuterar jag
  inte i det här dokumentet.
\end{abstract}

\section{Inledning}
\subsection{Vad är amatörradio?}
Här visar vi upp vilken fantastisk hobby amatörradio är, med
inspirerande foton och beskrivningar av vad man kan göra som
radioamatör, t.ex:
\begin{itemize}
\item DX-ing på kortvåg, contesting
\item Köra digitalmoder, morsetelegrafi
\item Köra portabel kortvåg i skogen
\item VHF/UHF - FM-repeater, tropo, $E_s$, $M_S$, NAC ``tisdagstesterna'', m.m.
\item Satelliter, APRS
\item Bygga själv - egna konstruktioner, byggsatser
\item Bygga antenner och antennsystem
\item Ragchewing, ``gubbringar''
\item Sambandsuppdrag
\end{itemize}

Mål: Ge en bra överblick av bredden av amatörradiohobbyn, och skapa en
känsla av \emph{vad kul det vore att ha amatörradiocertifikat!}

Syfte: Ge läsaren en anledning att bli radioamatör.

\subsection{Hur man blir radioamatör}
Här beskriver vi hur man rent praktiskt blir radioamatör, dels rent
kunskaps\-mässigt, och hur certifikatsprovet går till, hur man får
certet, osv.

Mål: Läsaren förstår hur man tar certifikatet, och känner att det är
möjligt.

Syfte: Få läsaren att bestämma sej för att bli radioamatör.

\section{Exempel på några amatörradiostationer}
Handgriplig information (foton!) om hur några ``typiska''
amatörradiostationer är uppbyggda, och förklarar de grundläggande
begreppen och vad de olika stationerna är ämnade för.

Exempel:
\begin{itemize}
\item Handapparat, repeatertrafik
\item Portabel kortvågsstation
\item Kortvåg med trådantenn eller beamar för DX-ing
\item VHF/UHF med ``blindkäpp'' på radhustaket
\item VHF/UHF med riktantenner.
\end{itemize}

Här är det nog smart om vi återkopplar till vad vi beskrivit i
inledningen. Skillnaden mot inledningen är att vi beskriver mer
detaljer kring kablage, jordning, kontakter, strömförsörjning, etc.

Mål: Läsaren ska få en bild av hur en amatörradiostation kan se ut,
och varför det finns så många olika sätt att bygga en
amatörradiostation.

Syfte: Bygga förståelse och motivation.

\section{Hur man bygger en amatörradiostation}
Vi går igenom några schematiska stationer, ``tänket'' bakom, och för-
och nackdelar med olika sätt att bygga stationen.

Läsaren introduceras till begrepp som:
\begin{itemize}
\item Antennsystem
\item Blixtskydd, skyddsjordning, HF-jord, isolation
\item Matarledningar, baluner
\item EMC, EMF
\item Sändare, mottagare, slutsteg
\end{itemize}

Vi pratar endast kortfattat om varje ``grunka'', men hänvisar till
senare kapitel i boken, så att den som är särskilt intresserad av
någon viss del direkt kan hoppa dit och läsa om sitt favoritämne.

Mål: Läsaren själv ska förstå \emph{varför} de ska lära sej allt som
boken innehåller.

Syfte: Bygga kunskap och motivation. Uppfylla HAREC.

\section{Radiosystem}
Vi går igenom radioteknik på ``apparatnivå'', och förklarar begrepp som:
\begin{itemize}
\item Frekvens, våglängd
\item Radiohorisont, vågutbredning
\item Antenner, länkbudget
\item Effekt, effektförlust, decibel
\item Transmissionsledningar, baluner, impedansmatchning, SWR
\item Sändare, mottagare, transceiver
\item Strömförsörjning
\item Jordning, blixtskydd
\end{itemize}

Mål: Läsaren ska kunna planera sin egen framtida station.

Syfte: Motivera att läsa vidare. Uppfylla HAREC.

\section{Radioteknik}
\label{radioteknik}
Här ``lyfter vi på locket'' och går igenom varje apparat mer i detalj.

Först introduceras grundläggande begrepp som \emph{oscillator, blandare,
filter, bandbredd , modulation, övertoner, distorion, m.fl.}

Därefter bygger vi ihop dessa block till mottagare, sändare,
transceiver, slutsteg, osv.

Vi kan återanvända en hel del av Wibergs material, och komplettera med
det vi känner kan förbättras.

Mål: Läsaren har god koll på hur de olika apparaterna fungerar, och
kan själv leta information om de områden som är extra intressanta.

Syfte: Bygga kunskap. Uppfylla HAREC.

\section{Antenner}
Vi kan återanvända Wibergs material mer eller mindre rakt av.

Mål: Uppfylla HAREC.
Syfte: Bygga kunskap. Uppfylla HAREC.

\section{Likströmslära}
Vi kan återanvända en hel del av Wibergs material, och komplettera med
det vi känner kan förbättras.

Mål: Uppfylla HAREC.
Syfte: Bygga kunskap. Uppfylla HAREC.

\section{Växelströmslära}
Vi kan återanvända Wibergs material mer eller mindre rakt av.

Mål: Uppfylla HAREC.
Syfte: Bygga kunskap. Uppfylla HAREC.

\section{Komponenter}
Här går vi ner på komponentnivå, dvs resistor, kondensator, spole,
diod, osv.

Vi kan återanvända Wibergs material mer eller mindre rakt av.

Mål: Uppfylla HAREC.
Syfte: Bygga kunskap. Uppfylla HAREC.

\section{Kretsar}
Här sätter vi ihop komponenterna till de radiobyggblock som beskrivits
i avsnitt \ref{radioteknik}.

Exempel: vi beskriver hur en oscillator fungerar i detalj.

Mål: Uppfylla HAREC.
Syfte: Bygga kunskap. Uppfylla HAREC.

\section{Mätning}
Vi kan återanvända en hel del av Wibergs material, och komplettera med
det vi känner kan förbättras.

Mål: Uppfylla HAREC.
Syfte: Bygga kunskap. Uppfylla HAREC.

\section{Elsäkerhet}
Vi kan återanvända Wibergs material mer eller mindre rakt av.

Mål: Uppfylla HAREC.
Syfte: Bygga kunskap. Uppfylla HAREC.

\section{Vågutbredning}
Vi kan återanvända Wibergs material mer eller mindre rakt av.

Mål: Uppfylla HAREC.
Syfte: Bygga kunskap. Uppfylla HAREC.

\section{Trafikmetoder}
Vi kan återanvända en hel del av Wibergs material, och komplettera med
det vi känner kan förbättras.

Mål: Uppfylla HAREC.
Syfte: Bygga kunskap. Uppfylla HAREC.

\section{Regler \& lagar}
Vi kan återanvända en hel del av Wibergs material, och komplettera med
det vi känner kan förbättras.

Mål: Uppfylla HAREC.
Syfte: Bygga kunskap. Uppfylla HAREC.

\end{document}
