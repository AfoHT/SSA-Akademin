\chapter{EMC}

Endast en del av frekvensspektrum av elektromagnetiska vågor används för radiosändningar. Samtidigt används detta utrymme av
allt fler intressenter och för allt fler ändamål.
Samhället blir alltmer tekniskt avancerat och
elektroniktätheten tilltar kraftigt. Den ökande mängden och komplexiteten hos apparaterna kräver därför regler, som styr både
utförande och användning med rimligt bibehållen säkerhet och funktion.

Störningar och störkänslighet
Om EMC-Iagen
Det kan inte längre ges enkla svar på vad
som är att vara störd och att störa. Internationella och nationella väl preciserade regler
för radio- och teletekniskt samexistens är
numera helt nödvändiga.
Samlingsbegreppet är Electromagnetic
Compatibility (EMC), d.v.s. en apparats förmåga att fungera tillfredsställande i sin elektromagnetiska omgivning utan att alstra elektromagnetiska störningar som överstiger en
nivå, som tillåter radio- och teleutrustning
och andra apparater att fungera som avsett.
Vidare skall apparater ha en sådan tillräcklig
inbyggd tålighet mot elektromagnetiska störningar, att de kan fungera som avsett.
Till skydd för liv, personlig säkerhet och
hälsa samt kommunikationer och näringsverksamheter har därför Lag om elektromagnetisk kompatibilitet införts. Denna lag
är anpassad efter av EG/EES utfärdade
direktiv angående bl.a. radiostörningar.
Förordning om elektromagnetisk kompatibilitet definierar nyckelbegreppen; apparater, EMC, elektromagnetiskstörning och
tålighet. Elsäkerhetsverket är ansvarig myndighet, med rätt att utfärda föreskrifter om
bl.a. skyddskraven, kontroll och märkning
samt om vissa undantag. Sådana föreskrifter är bl.a. ELSÄK-FS.

EPT

Post och telestyrelsens föreskrifter om innehav och användning av amatörradioanläggningar m.m. hänvisar till den överordnade Lag om radiokommunikation där följande finns om Åtgärder mot störningar:

Ur radiolagen

"Om en radiosändare stör användningen av
en annan radioanläggning skall den som har
tillstånd att inneha och använda radiosändaren ombesörja att störningen upphör eller i
görligaste mån minskas. Motsvarande skyldighet gäller för innehavare av radiomottagare som stör användningen av en annan
radiomottagare.
Den myndighet som regeringen bestämmerfårförelägga den som enligtförstastycket
är skyldig att vidta åtgärder mot en störning
att fullgöra denna skyldighet. Ett sådant föreläggande får förenas med vite."
" Elektriska eller elektroniska anläggningar
som, utan att vara radioanläggningar, är
avsedda att alstra radiofrekvent energi för
kommunikationsändamål eller industriellt,
vetenskapligt, medicinskt eller något liknande ändamål, får användas endast i enlighet
med föreskrifter som meddelas av regeringen eller den myndighet som regeringen bestämmer.
Den myndighet som regeringen bestämmer får meddela de förelägganden och förbud som behövs i ett enskilt fall för att föreskrifterna i första stycket skall följas. Ett
sådant föreläggande får förenas med vite.
Regeringen får meddela föreskrifter om
förbud mot att inneha elektriska eller elektroniska anläggningar som inte omfattas av
första stycket och som, utan att vara radioanläggningar, är avsedda att sända radiovågor."
l radiolagen definieras bl. a. radioanläggningsom en anordning för radiokommunikation eller radiobestämning genom sändning
(radiosändare) eller mottagning (radiomottagare) av radiovågor.

119-1

Utstrålning från amatörradiosändare
Vad som sägs i radiolagen innebär att
såndareffekten alltid skall anpassas så att
styrkan av utstrålade fält inte förorsakar störningar. Den enligt amatörradioföreskrifterna
högsta tillåtna effekten kan alltså inte användas hinders/öst. l samma paragrafstår också
att PTS i tillståndet kan besluta om andra
effektgränser, om det finns särskilda skäl.
Om störningarna inte kan avhjälpas kan
PTS komma att anvisa om restriktioner (begränsningar i sändningstillståndet), det kan
vara sändningsförbud under vissa tider, på
vissa frekvenser, över viss såndareffekt etc.

PM vid störningsproblem
• Störningar är alltid förenade med obehag
och ställer grannsämjan på prov. Håll Dig
väl med dem som bor i omgivningen!
• Om det väcks klagomål på Dig om störningar, skall Du först kontrollera Din egen
sändare och antennanläggning.
• Be därefter att få undersöka antennanläggning och apparater hos den som besväras av störningar.
• Om Du ser en lösning, berätta om vad
som kan göras. Kom överens om vad
som får göras. Ändra då inte något inne i
apparater, men provagärna ut yttre, kompletterande filter etc.
• Om det inte går att komma till rätta med
störningarna bör de som levererat och
installerat anläggningen anlitas.
• Störningsanmälan kan även ske till PTS
närmaste tillsynsområde.
Arbeta aktivt med avstörning
• Låna hem en av SSA:s avstörningslådor
och försök att finna en lösning. l lådan
finns ett sortiment av frekvensfilter för
avstörning,
• Undvik att störa i onödan. Sänk sändareffekten och begränsa sändningstiden
under utprovningen av en lösning.
•
•

Lyckas Du inte själv med att störa av
Ta gärna hjälp av en radioamatör med
erfarenhet av avstörning eller
Anlita annan sakkunnig hjälp.

119-2

9a 1 Störningar i elektronik

Liksom att radiomottagning kan "störas" av
sändningar som inte är av intresse, så kan
störningar i form av radiovågor från olika
slags elektrisk utrustning försvåra mottagning eller andra funktioner.
Utstrålning från t. ex. datorer, kabel-TV,
hushållsmaskiner, tändgnistorfrån oljebrännare, bilar och mopeder etc. är radiovågor.
Elektriska apparater kan alltså både störa
och störas genom radiovågor, även om de
inte är definierade som radioanläggning,
d.v.s. radiosändare och/eller radiomottagare.
Störningar som uppstår av elektromagnetiska fält kallas Electromagnetic lnterference - EMI. Känsligheten för sådana störningar kallas för Electromagnetic Susceptibility- EMS.
Äldre benämningar på störningar är t. ex.
BCI (broadcasting interference) och TVI (television interference). Dessa återfinns dock
inte i nu gällande terminologi.
Blockering
l de flesta radiomottagare finns en automatisk förstärkningsreglering. Om insignalerna
blir för starka, så räcker regleringen inte till.
Då överstyrs förstärkarstegen så att de arbetar olinjärt. Detta kallas blockering.
Ett sätt att undvika blockering är att koppla en dämpsats - attenuator- till mottagaringången. En sådan sänker dock signalstyrkan över hela frekvensområdet, inte bara för
en viss signalfrekvens.
Interferens
När den önskade signalen störs av en annan frekvensnära signal, kallas det interference. l mottagaringången finns frekvensfilter, som undertrycker ej önskade signaler,
om de inte ligger allför nära. Om ingången
inte är tillräckligt selektiv, kan det behövas
en tillsats som förbättrar selektiviteten.
lntermodulation (se även kapitel 4)
Blandningsprodukter av signaler i en mottagare eller sändare kallas för intermodulation.
lFmdetektering
HF-signaler kan komma in genom in- och
utgångarna för LF samt genom nätkabein.

EMC
Dessutom förekommer direktinstrålning
av radiovågor genom apparathölj et, om detta
inte har tillräckligt avskärmande verkan.
LF-detektering uppstår när HF-signaler
dernoduleras i diodsträckor i den störda ap~
paratens komponenter. Detta sker oavsett
vilken frekvens som sändaren eller mottagaren är inställd på.
LF-detektering uppstår särskilt vid AMeller SSB-modulerade sändningar samt av
transienter vid bärvågsnyekling av sändare.
Ofta är det inte möjligt att förhindra LFdetektering utan ingrepp i den störda apparaten. Ingrepp får endast göras av fackman.

\section{Störningsorsaker}

Störningar från sändare
HF-förstärkare, t.ex. i sändarslutsteg, kan
komma i oönskad självsvängning, vilket kan
uppstå av flera orsaker; det kan vara bristande avkoppling av matningsspänningar, induktiv och/eller kapacitiv återkoppling etc ..
Effektuförstärkare kan även överstyras.
Då uppstår intermodulation och övertoner
som strålas ut på oönskade frekvenser.
l många fall kan störningar undvikas med
en eller flera av följande åtgärder:
• Undvik att överstyra sändarslutsteget
(kontrolleras t.ex. med ALG-mätaren).
• Förse sändarutgången med lågpassfilter.
På så sätt undertrycks övertoner.
• Anpassa sändarens och antennanläggningens impedanser till varandra. Stäm
av sändarens re-filter och/eller en separat
antennanpassningsenhet rätt. En felinställd sändare kan medföra oavsiktligt
utstrålning.
• Koppla in balanseringsnät (bal un) mellan
osymmetriska antenntilledningar (koaxialkablar) och symmetriska antenner.
• Placera antennen högt och fritt och så
långt från personer och störningskänslig
utrustning som möjligt. Fältstyrkan är
nämligen högst närmast antennen.
• Undvik direkt HF-instrålning på belysningsnätet genom att använda nätfilter.
• Använd "mjuk" nyekling av bärvågen (avrundade telegrafitecken). Vid hård nyekling alstras övertoner i form av knäppar
som hörs långt vid sidan av sändningsfrekvensen.

Störningar på radiomottagning
l regel uppstår störningar på radiomottagning
först när utstrålade signaler uppnått en viss
styrka- immunitetsnivån för HF.
Man kan tala om tre slags HF-immunitet
hos mottagare;
• mot signaler genom antenningången,
• mot signaler genom övriga anslutna ledningar, t.ex. högtalar- och nätledningar,
• mot elektriska och/eller magnetiska fält
som strålar direkt in i apparaten
l de båda första fallen kan det hjälpa med
komplettering med hög- och/eller lågpassfilter och skärmströmsfilter.
Instrålningsstörningar är svårast att avhjälpa och fordrar ingrepp i mottagaren, vilket bör överlåtas till en fackman med tillgång
till tillverkarens serviceinstruktioner.
Störningar på TV-mottagning
Störningar från radiosändare kan yttra sig
t.ex. på följande sätt:
• Vid sändning av amplitudmodulerade signaler, t.ex. AM och SSB, uppstår ljudförvrängning i ljudkanalen samt ränder
m.m. i bilden,
• Vid sändning av FM och CW uppstår ljudstörningar samt kontrastvariationer, interferensmönster (moire-effekter) m.m. i
bilden.

Störningar i TV som orsakas av sändare
på lägre frekvenser kan i många fall avhjälpas med frekvensfilter. Det kan t. ex. uppstå
TV-störningar, när en amatörradiostation
sänder på 21 MHz-bandet. Dess 3:e överton
hamnar då på TV-kanal E-4 (61 - 68 MHz).
Ett lågpassfilter efter en KV-sändare kan
t.ex. dimensioneras att endast släppa igenom signaler under c:a 35 MHz.
Ett högpassfilter före en TV-mottagarekan t.ex. dimensioneras att endast släppa
igenom signaler med frekvenser över c:a 35
MHz.
Om inte mottagning i TV-band l (40- 68
MHz) är av intresse, så kan ett högpassfilter
med en gränsfrekvens av ca 160 MHz sättas
in. Det dämpar då oönskad utstrålning från
sändare i KV- och VHF-området, d.v.s. upp
t.o.m. 144-146 MHz amatörband. Däremot
släpps TV-band III (174- 230 MHz) och TVbanden IV och V igenom (470- 890 MHZ).
119-3

E
Ytterligare avstörningsmedel kan sättas
in om det uppstår störningar av amatörradiosändningar. Det kan vara skärmströmsfilter
på antennkablar, bandspärrar samt sug- och
spärrkretsar avstämda till störfrekvensen,
bandpassfilter avstämt till nyttofrekvensen.
Ett vanligt störningsfall är att en dåligt
skärmad och bredbandig antennförstärkare
blir överstyrd av starka sändare.

Det kan förekomma kraftiga spänningstransienter (spänningsstötar) på belysningsnätet. Dessa translenter kan leda till felfunktioner i anslutna apparater. För att förebygga
sådana fel kan man koppla in ett överspänningsfilter, som kan vara separat eller sammanbyggt med nätfiltret

Störningar på LF-apparater
Störningar av HF-instrålning i ljudbandspelare, LF-förstärkare, telefonapparater etc. kan
ofta stoppas med avkopplingskondensatorer och HF-drosslar. Moderna avstörningsdrassiar innehåller oftast något ferritmaterial
i form av rör, stavar eller ringar.

\section{Avstörningsmetoder}

Allmänt
För att prova ut ett filter, som bäst löser ett
visst radiostörningsproblem, kan man behöva tillgång till ett filtersortiment
Som exempel nämns bl. a. filter i SSA:s
avstörningslådor.
Nätfilter
Nätledningar kan fungera som antenn. l
såndarfallet kan HF-signaler komma ut i
elnätet genom nätledningen och störa andra
apparater både genom direktanslutning och
strålning. l mottagarfallet kan HF-signaler
uppfångas av nätledningen, ledas in i apparaterna och LF-detekteras där. För att förhindra sådana störningar behövs ett nätfilter.
Nätfiltret skall vara dimensionerat för den
nätström, som apparaten är avsäkrad för och
bör anslutas så nära apparaten som möjligt.
Om filtret inte kan placeras där, kan det vara
nödvändigt att även skärma nätledningen
mellan filtret och apparaten och jorda skärmen.
Om ledningen förses med t. ex. en serieinduktans -en drassel -så dämpas HFsignalerna. En drassel kan man göra t.ex.
genom att linda upp några varv av nätsladden
närmast apparaten på torcider eller en eller
flera sammanlagda ferritstavar. l svåra fall
kan det behövas ett bredbandigt nätfilter,
liknande det på bild Il 9-1.

119-4

lågpassfilter
Lågpassfilter släpper igenom signaler med
frekvenser under filtrets gränsfrekvens.
Ett lågpassfilter med lämpligt vald gränsfrekvens dämpar t. ex. övertonsutstrålningen
från en sändare, vars såndarfrekvens ligger
under filtrets gränsfrekvens medan övertonerna ligger över dess gränsfrekvens.
Övertoner kan dämpas med lågpassfilter. En överton är i detta sammanhanhang
en multipel av sändningsfrekvensen (grundtonen) exempelvis för 3.5 MHz
grundtonen = (1 :a harmoniska) 3.5 MHz,
1 :a överton = (2:a harmoniska) 7.0 MHz,
2:a överton = (3:e harmoniska) 10.5 MHz
o.s.v.
Viktigt för avsedd filterverkan är, att filtret
ansluts med korrekt impedansanpassning
och med kortast möjliga ledningar. Detta
gäller f.ö. alla filter.
Utstrålning utanför sändningsslagets tillåtna bandbredd anses som "icke önskad
utstrålning". Vidare gäller att sådan utstrålning från amatörradiosändare skall hållas så
låg som dagens amatörradioteknik medger.
Bild 119-2 visar principen för lågpassfiltret TP
30 för kortvåg, med gränsfrekvensen 36
MHz, att kopplas mellan sändaren och antennledningen. Med denna gränsfrekvens
dämpas övertoner från sändare så att risken
för TV-störningar minskar.

r---------------------------l koaxial-

koaxial- 1
kabel från~
sändaren 1

1

r:- kabel till
l

l
l

l

:
l

L---------------

Högpassfilter

l

Om en störande signal råkar finnas inom
passbandet för mottagaren kan man undertrycka - "spärra" - den signalen med ett
spärr- eller sugfilter. Vilket man väljer är inte
kritiskt.
Den störande signalen kan "spärras" med
en parallellresonanskrets i serie med
mottagaringången (Bild 9-5). Kretsen består av en induktans och en kapacitans.

.

koax1al-

:l

Bild II 9-2

l
---------J

Högpassfilter släpper igenom signaler med
frekvenser över filtrets gränsfrekvens.
Bild II 9-3 visar principen för högpassfiltret HP 40-S med gränsfrekvensen 47 MHz,
att kopplas in mellan antennledningen och
en mottagare för VHF eller högre frekvenser.
Störningar kommer inte alltid "utifrån".
De kan t.ex. alstras i bredbandiga antennförstärkare, vilka lätt överstyrs av alla slags
signaler från ett stort frekvensområde. Man
kan då koppla in ett högpassfilter före bredbandsförstärkaren, men en bättre lösning är
att byta till en väl skärmad passbands- eller
ännu hellre kanalförstärkare.
Koaxialkablar med täta skärmar och rätt
monterade anslutningskontakter är också
viktigt för en lyckad avstörning.

Spärrfilter och sugkretsar

antennen

:

Lågpassfilter
försändare

Om man använder enstub som resonanskrets- t. ex. en koaxialkabel- så skall den ha
längden fi./4 och vara "kortsluten" eller ha
längden A./2 och vara "öppen".
Man kan även kortsluta - "suga bort" den störande signalen med en serieresonanskrets parallellt över mottagaringången
(Bild 9-6). Om man då använder en stub, så
skall den ha längden ').)4 och vara "öppen"
eller ha längden ').)2 och vara "kortsluten".
Den störande signalen kan undertryckas
ytterligare med fler stubar, som ordnas som
i Bild 9-6. Filtret består då av öppna ').)4stubar, som utgör avgreningar från antennkabeln med ett avstånd av ').)4.
(Om stubarna i detta filter kortsluts, så
bildas ett bandpassfilter i stället).
Exempel på kommersieila spärrfilter är SF
145-S för 144 MHz och SF 435-S, för 435
MHz amatörband. De är avsedda att kopplas in före mottagare som störs av amatörradiosändningar.
SF 145-S spärrar amatörbandet 144 148 MHz och släpper igenom banden O-120
och 174- 870 MHz.
SF 435-S spärrar amatörbandet 430 440 M Hz och släpper igenom O- 350 och 470
-870 MHz.

r------------------------------, koaxial-

..~:-,

~~~:~~:~ ~~~'fTiT

l: l
i

1

11!11 ~~t~~~~ren
!

'--------------- --- --------------------.J
'

l

Bild II 9-3
HögpassfilterförVHF!UHF-mottagare

119-5

EMC

Z=~

Parallellresonanskrets

Kortsluten A/4-ledning

Öppen :A./2-Iedning

Serieresonanskrets

Öppen Ä/4-ledning

l Kortsluten A/2-ledning

z= o

Bild II 9-4 Ingångsimpedansen i resonanskretsar

från
antennen

till
mottagaren

från
antennen

till
mottagaren

Bild II 9-5 Spärrfilter för mottagare

Sugkre~

~x

T

Bild II 9-6 sugkretsar för mottagare

119-6

EMC

PT
Nät- och skärmströmfilter för mottagning
Bild II 9-7
Utsidan av antennkabelns skärm kan också
fungera som antenn. Särskilt i skärmskaNar
kan HF-strömmar läcka ut och in. De kan då
passera förbi eventuella antennförstärkare,
filter etc. och orsaka störningar.
l enkla fall kan yttre skärmströmmar stoppas med att linda upp kabeln några vaN på
ferritstavareller genom en storferritring som
på bilden. En nätkabel, s.k. sladdställ, får
inte kapas och skaNas.

Högtalarledningsfilter (EM 502-B)
Bild II 9-9
HF-instrålning på högtalarledningar kan ha
en störande påverkan. Detta kan undvikas
genom koppla in HF-drosslar i ledningarna.
Dessa d rosslar bör vara skärmade så att de
inte verkar som antenner istället.
l enklare fall kan det räcka med att byta till
skärmade högtalarkablar eller att linda upp
en sträcka av ledningarna på en ferritkärna.

Phono .. ingångsfilter (TBA 302)
Bild II 9-8
Störande påverkan från radiosändningar kan
uppstå om anslutningsledningarna till phono-ingången i LF-förstärkare är dåligt skärmade och avkopplade. Sådana störningar
kan avhjälpas med ett filter.

Ferritstav

Ferritring

'·
J,

Bild 119-7 Nät- och skärmströmfilter

o

från
mottagaren
o

l

r;n- l

rLt..

I

J

•l']
or

~

·r

o

till
högtalaren
o

Bild II 9-9 Högtalarledningsfilter

Avkoppling av HF-signaler
Med avkoppling av en signal menas att
den avleds från en signalväg till en
annan. Vid avstörning avkopplas vanligen den störande signalen till systemjord.
störimmuniteten i mottagare kan
alltså förbättras genom att LF-ingångarna H F-avkopplas med kondensatorer
och/eller HF-spärras med drosslar.
l svåra störningsfall kan det också
bli nödvändigt med H F-avskärmning av
LF-ingångsstegen, liksom med ytterligare avstörningsfilter inne i förstärkaren.
Sådana åtgärder innebär emellertid
att konstruktionsändringar har gjorts.
Apparatens elsäkerhetsmärkningar är
då ogiltiga.

r----------------,

l

l

phono

förstärkare

l

l

l

ll

r

1

~~~--~--~~~

l

--------.J

Bild II 9-8 Phonoingångsfilter

Bild II 9-1 Oa H F-avkopplat styrgaller

119-7

EM

Bild II 9-1 Ob H F-avkopplad bas på tre sätt

Dr

hårda CW-tecken

-llllllllllllllllllll-lllllllllllllllllllllllllllllllllllllllllllllllllllllllll-.

Bild II 9-11 Parasitfilter i H F-förstärkare
Bild 119-10a-b visar några sätt att avkoppla
en oönskad signal från styrgallret i ett elektronrör respektive från basen i en transistor.

Parasitfil ter
Bild Ii 9-11

Förstärkarsteg kan råka i självsvängning,
ofta på frekvenser i VHF/UHF-området. Ett
sätt att stoppa det är med s.k. parasitfilter.

mjuka CW-tecken

-I I I I I /J/ I i~-,i!I JI I I I I I I I I I I I I I I I I I I I I/ /I f!r-

Nyekling stilter

Bild II 9-12
När en bärvåg nyck! as, så bildas övertoner.
Blandningsprodukter av övertonerna och
bärvågen hörs som knäppar på omkringliggande frekvenser. Märk att övertoner uppstår vid all bärvågsnyekling - inte bara vid
morsetelegrafering!
När övergångstiden är kort (hård nyckling), så bildas fler övertoner än när den är
längre (mjuk nyckling). Knäpparna kan till en
del dämpas med ett nycklingsfilter där dels
insvängningsförloppet bromsas med en d rossel i serie med nycklingskontakten och dels
ursvängningsförloppet med en seriekrets av
en resister och en kondensator, kopplade
parallellt över nycklingskontakten.

119-8

Bild II 9-12 Nycklingsfilter

skärmning

HF-energi kan i olyckliga fall även stråla ut
genom sändarens hölje och in genom andra
apparaters hölje. Det medför att apparaternas skärmningar och jordning måste förbättras. Följ då elsäkerhetsbestämmelserna!
Se även kapitel 1.3, 1 .4 och 1O.
