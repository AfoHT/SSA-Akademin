\section{Egenskaper i mottagare}

\subsection{Selektivitet}

Med selektivitet menas en mottagares förmåga att skilja ut önskade signaler och
undertrycka övriga. Summariskt beskrivet
kallas avståndet mellan yttergränserna för
det önskade frekvensområdet för bandbredd. När det gäller superheterodynmotta.;
gare finns två selektivitetsbegrepp.
Det ena är förselekteringen för att dämpa
de spegelfrekvenser som uppstår i samband med blandning av mottagna signaler
och oscillatorfrekvenser i mottagaren.
Det är selektiviteten i en superheterodynmottagares MF- steg för att utskilja den
önskade signalen efter blandningsförloppen.

\subsection{Spegelfrekvensproblemet vid mottagning}
Bild II 4-22
Exempel:
En sändning på 3600kHz skall tas emot och
VFO-frekvensen är 4055 kHz. Mellanfrekvensfiltret undertrycker sändningar på så
närliggande frekvenser som t. ex. 3603 och
3597kHz. Denna egenskap kallas för närselektion.
Men tyvärr kan en sändning på så avlägsen frekvens som 451 O kHz ändå störa
mottagningen, den goda närselektionen till
trots. Avståndet mellan 451 O kHz och vår
mottagningsfrekvens3600kHz är 91OkHz.
Frekvensen 451 O kHz och VFO-signalen
bildar också en blandningsprodukt, som har
frekvensen 455 kHz. Vid en VFO-frekvens
av 4055 kHz och en mottagningsfrekvens
av 3600kHz benämns451OkHz som spegeifrekvensen. Avståndet mellan spegelfrekvens och mottagningsfrekvens är dubbla
värdet av mellanfrekvensen -i detta exempel 2 ·455kHz= 91OkHz.
Signaler på mottagningsfrekvensen och
spegelfrekvensen alstrar båda blandningsprodukter med VFO-frekvensen, som har
mellanfrekvensens värde. Mellanfrekvensfiltret kan därför inte undertrycka en främmande signal på spegelfrekvensen.
Bild II 4-23
Däremot kan en mottagaringång med
förselektering undertrycka den. En selektiv
krets före blandaren släpper igenom ett smalt
frekvensband med mittfrekvensen 3600kHz,

114-17

M

AR
/pegelfrekvens

u

fsp ::::

närselektion

fM

+

t

2·

/mellanfrekvens

fMF

"- mottagningsfrekvens

l

fvFO

IM

fsp

tfMF ~fMFj

3600
kHz

Mottagning av önskad sändare:

fvFo- fE

:::

Mottagning av ej önskad sändare: fsp - fvFo :::

4055
kHz

4510
kHz

4055 ---3600kHz = 455 kHz
4510 --4055kHz = 455kHz

Bild II 4-22 Enkelsuper med låg MF och ingen förselektion

u

MF =455kHz

Bild II 4-23 Enkelsuper med låg MF och med förselektion
men dämpar t. ex. frekvensen 451 O
kHz p.g.a. den stora frekvensskillnaden. En förselektion har alltså tillförts
som komplement till den närselektion
som erhålls med mellanfrekvensfiltret
Bild 114-24
Ju längre ifrån varandra nyttofrekvens och spegelfrekvens ligger,
desto bättre är förselektionen. Med en
mellanfrekvens av 455 kHz är alltså
detta avstånd 91 O kHz. l långvågsoch mellanvågsområdet är det tillräckligt för att man med enkla medel skall
kunna skapa tillräckligt selektiva filter.
Exempel:
Vid den högsta mottagningsfrekvensen på mellanvåg 1605 kHz är
spegelfrekvensen 2515 kHz, som ligger 1.57 gånger högre i frekvens och
med ett avstånd av 91OkHz. l kortvågsområdet dämpas inte en spegelfrekvens på avståndet91OkHz tillräckligt
kraftigt. Vid den högsta mottagnings-

114- 18

u

MF =900kHz

Bild II 4-24 Enkelsuper med hög MF och
med förselektion
frekvensen på kortvåg 30 MHz ligger nämligen
spegelfrekvensen 30.91 OMHz endast 1.03 gånger
högre i frekvens. Med antagandet, att förselektionskretsen har ett Q-värde av 30, blir bandbredden 53.5 kHz vid frekvensen 1605 kHz.
Med samma Q-värde blir bandbredden 1000
kHz vid frekvensen 30 MHz, vilket innebär att
förkretsen inte längre kan dämpa så närliggande
spegelfrekvenser på ett effektivt sätt.

ARE
l mottagare för högre frekvenser används
därför högre mellanfrekvens för att öka avståndet till spegelfrekvensen. l moderna
kortvågsmottagare är det vanligt med en
mellanfrekvens av 9 MHz eller högre. Vid en
mottagningsfrekvens av 30 MHz och en
mellanfrekvens av 9 MHz är spegelfrekvensen 48 MHz, vilket är 1.6 gånger mottagningsfrekvensen. Detta möjliggör förselektionsfilter med tillräcklig dämpning av spegelfrekvensen.
Bild II 4-25
Bilden visar hur när- och förselektion kompletterar varandra i ett frekvensspektrum.
Märk, att passbandbredden b i förselektionskretsen anger avståndet mellan de frekvenser där signalamplituden dämpats till 70
0
/o av toppvärdet. l exemplet här ovan har
antagits att förkretsen för kortvågsmottagning har samma Q-värde som förkretsen för
mellanvågsmottagning.

Vid högre frekvenser, i VHF- och UHFområdet, kan inte önskat Q-värde erhållas i
sådana kretsar som användsiKV-området
och lägre. Andra lösningar blir då nödvändiga, t.ex. kavitetsfilter och helixfilter.
MF-bandbredd vid AM (A3E)
Bild II 4-26
En amplitudmodulerad signals frekvensspektrum består av bärvågen och två sidfrekvenser - eller sidband om sidfrekvenserna är många.
Bandbredden i MF-kretsarnamåstevara
minst så stor att sidfrekvenserna längst bort
från bärvågen kan passera. Dessa frekvenser motsvarar de högsta modulerande tonerna. Vid rundradiosändningar på mellanvåg utsänds alla frekvenser upp till 4.5 kHz.
Detta motsvarar en bandbredd av 9 kHz.
För enbart talöverföring är en bandbredd av
6 kHz tillräcklig, vilket motsvarar en LFgränsfrekvens av 3 kHz.

u
HF

FÖRSELEKTERING:

Undertrycker spegelfrekvenser

önskad sändare
flera andra
annan sändare på
spegelfrekvensen

0.7

~~~~~~~ww~~~~~----f

l fM :
r--b---G<>j

u

MF-fittrets passband

1r

MF

NÄRSELEkTERING:

Undertrycker angränsande sändare

Bild II 4-25 Samtidig för- och närselektion i superheterodynmottagare

114- 19

M
Ett för smalt MF-filter skär bort de yttre
delarna av sidbanden. LF-signalerna kommer då att förlora de höga tonerna (diskanten). Om däremot filtret är för brett, kommer
närliggande utsändningar också att höras.
l vissa mottagare kan MF-bandbredden
anpassas till förhållandena. Det är alltså en
fråga om en kompromiss mellan bättre ljudkvalitet och mindre störd mottagning.

u

MF-bandbredd vid SSB (J3E)
Bild 114-27
Mellanfrekvensfiltret för SSB-mottagning
skall endast släppa igenom ett av de två
sidbanden, vars bredd är skillnaden mellan
högsta och lägsta överförda LF-frekvens.
Inom amatörradio är detta3kHz- 0.3 kHz =
2.7 kHz, alltså något mindre än hälften av
bandbredden vid AM.

MF

LF

u
fr

fLF

=

(fr t fLF ) -

fLF

E

fr -

fr och

( fr - f L F)

fL F
~r--~--~--~-------

8999
kHz

9 000

f

~~----------------~f

kHz

fL F

= 9001 -·· 9000kHz = 1 kHz

fL F

= 9000

····8999kHz

= 1 kHz

A3E- demodulering i frekvensspaktrat

u

u

MF
M F -filterkurva

MF

inställd sändare
med bärvåg och sidband

riktig MF-bandbredd

M F-bandbredden för smal, delar av
sidband bortklippta. Diskanten borta

b = 2 · fLFmax
b = 2 · 3 kHz = 6 kHz

u

MF

u

MF

Störning från angränsande sändare.
Undertryckning genom smalt MFfilter på bekostnad av diskant

För stor M F-bandbredd, alltför
flack filterkurva, störningar från
angränsande kanaler

MF-bandbredd vid A3E

Bild II 4-26 MF-bandbredd vid AM (A3E)
114-20

EPT

~©~

M TTA ARE

Ett alltför brett MF-filter skulle också
släppa igenom oönskaqe signaler från angränsande frekvenser. A andra sidan skulle
ett för smalt M F-filter skära bort signaler i det
önskade frekvensregistet och försvåra mottagningen. Smala filter kan å andra sidan
utnyttjas för att dämpa signaler, t. ex. från en
för nära liggande sändare eller som har för
stor bandbredd.

U

MF

När närliggande sändare stör mottagningen
ges följande möjligheter:
Att göra en liten snedavstämning, uppåt
eller nedåt i frekvens. Därigenom ändras frekvensläget på det mottagna talet, men vid små frekvensawikelser blir
förvrängningen liten. Läsligheten blir
sämre, men mottagningen på det hela
taget bättre.

U

MF-filterkurva
l

rätt MF-bandbredd
b = fLFmax - fLFmin
b = 3 --·- 0,3 kHz = 2,7 kHz

u

För stor MF-bandbredd, alltför
flack filterkurva, störningar från
angränsande kanaler

u

Minskning av grannstörning genom :
snedavstämning

Störning från grannkanalssändare

U

MF

u

den undertryckta
bärvågen

\l
l

l

normal BFO-frekvens

~~--~~==~~-----f

MF-skift

Avklippning av de lägre frekvenserna
genom förskjutning av 2:a MF-filterkurvan (passbands-tuning)

Bild fl 4-27 MF-bandbredd och passband-tuning vid SSB (J3E)

114-21

•

•

Vidare måste VFO, 1 :a BFO och 2:a BFO
kunna ställas in var för sig. Frekvensläget
på MF l och/eller MF Il kan då förskjutas
över respektive filters passband, oberoende av varandra. Därigenom uppstår
skenbart effekten att filterkurvorna skjuts
emot varandra. Samma effekt skulle fås
om kristallfiltren gick att avstämma, vilket
ju inte är möjligt.

M F-skift. Som just beskrivits kan en liten
snedavstämning göras. l vissa mottagare
är det ordnat så att också BFO-frekvensen kan förskjutas så att frekvensläget på
talet blir återställt igen. Därmed blir MFpassbandet skenbartbart förflyttat uppåt
eller nedåt i frekvens (M F-skift, IF-shift).
Det verkliga frekvensläget mellan nytto-:
signal och BFO behålls. l alla händelser
blirbasen ellerdiskanten på nyttasignalen
avskuren, beroende på var denna ligger
i frekvens.
Passband-tuning. Om det finns störande
sändare både över och under i frekvens,
går det inte att skära bort störningarna
med ett enkelt M F-skift, eftersom antingen den ena eller den andra störande
sändaren ändåskulle höras. Fördetfallet
erbjuder några moderna mottagare möjligheten att flytta MF-passbandets övre
och undre frekvensgräns oberoende av
varandra (bandpass tuning m.m.). Detta
förutsätter, att mottagaren är en trippelsuper med branta filter i varje MF-steg.

u

vid CW (A tA)
Bild II 4-28
En
har som bekant inte bandbredden non Hz, utan det handlar i grunden
om en amplitudmodulerad signal. Vid en
nycklingshastighet av 60 tecken per minut
är bandbredden c:a i 00 Hz och vid i 20
per minut den dubbla, c:a 200 Hz.
l vissa mottagare används ett SSB-filter
även för mottagning av CW. En vanlig bandbredd på ett SSB-filter är 2.7 kHz och då
kommer även stationer på närliggande frekvenser att höras. Låt vara att de flesta av
dessa stationer hörs med olika frekvens.

u

MF

LF

BFO
CW-Signa!

8

U

~~~

2

9

"-"'------f
~~~ b ca 100-200Hz

MF

""---------f

800Hz

U

MF
BFO .

f
CW-mottagning med SSB-filter

Bild II 4-28 Olika MF-bandbreder vid CW (A

114-22

CW-mottagning med 250 Hz CW-filter

PT
Fler än 20 CW-stationer får plats inom en
bandbredd motsvarande en SSB-kanal. Den
mänskliga hjärnan, kan med någon övning
koncentrera sig på en av dessa signaler
medan övriga uppfattas som störande.
Det tidigare nämnda LF-bandpassfiltret
skulle emellertid åstadkomma en bättre selektion och bekvämare avlyssning. Men om
en annan station inom passbandet är mycket
starkare än den station som är av intresse,
då blir MF-förstärkaren antingen överstyrd
av den starkare signalen eller AGC reglerar
ner förstärkningen så att den svagare signalen inte längre kan höras trots det smala LFfiltret. selektionen i en mottagare bör därför
sitta "så långt fram som möjligt". l det skildrade exemplet skulle ett smalt filter i MF vara
till bättre nytta vid CW-mottagning. Bandbredden på ett sådant filter är 250- 500 Hz,
således endast något bredare än CW-signalen.
Medettännu smalare CW-filterkan, p.g.a.
bristande frekvensstabilitet hos sändare och/
eller mottagare, svårigheter uppstå att finna
den önskade signalen. Välutrustade mottagare har passband-tuning även för CW,
steglös bandbreddsreglering eller stegvis
valbara filterbandbredder. Då kan mottagaren ställas in på den önskade signalen med
en stor bandbredd som därefter minskas.
För mottagning av RTTY (radiofjärrskrift)
med 170 Hz skift mellan de två frekvenserna, kan ett 500 Hz-filter användas. Smalare filter går däremot inte så bra.

Bandbredd vid FM (F3E)
En FM-sändare med frekvensdeviationen
~~ax och högsta modulerande LF-moduleringsfrekvensen fLFmax har bandbredden

b= 2( ~~ax + (Fmax) •

Inom amatörradio är det brukligt med en
maximal deviation av 3 kHz och en övre
gränsfrekvens av 3kHz, vilket motsvarar en
bandbredd av 12 kHz.
Fullgod mottagning är möjlig endast om
M F-filtren i mottagaren har minst den bandbredd, som sändaren har. Men vid för stor
mottagarbandbredd kan även stationer på
närliggande frekvenser uppfattas. Sedan
1996 är det av IARU Region 1 rekommenderade kanalavståndet 12.5 kHz vid FM-trafik
på VHF- och UHF-amatörradiobanden.

TTAGARE
Det är vanligare med för stor deviation på
FM-sändaren än att mottagaren är alltför
smaL En för stor deviation, avsaknad av
deviationsbegränsare och för hög LF-gränsfrekvens medför en onödigt stor bandbredd
på sändaren. Motstationen får då mottagningssvårigheter och stationer på angränsande kanaler blir också störda.
Det blir allt vanligare med 12.5 kHz kanalavstånd även för repeatrar, varför det är
viktigt att alla sändare är rätt inställda.

\subsection{Signalkänslighet och brus}

Om man ställer in mottagaren på en ledig
frekvens, så hör man vid full förstärkning ett
brus likt det från ett vattenfall.
Bruset kommer från de svaga växelspänningar som uppstår när laddningsbärarna
rör sig genom de material som strömkretsen
består av. Beroende av bruskällan sträcker
sig frekvensspektrum från noll till nära nog
oändligt. På grund av egenskaperna skiljer
man mellan en rad specifika bruskällor:
• Resistorbrus, även kallat "vitt brus", som
uppstår i resistiva komponenter. Bruset
sträcker sig över hela det mätbara frekvensområdet varvid energifördelningen
är lika över hela området,
• Kretsbrus, som uppstår i resistanser i
svängningskretsar i resonans,
• Antennbrus, som är sammansatt av
bruset från antennens strålnings- och
förlustrasistanser samt av det galaktiska
brus som antennen tagit emot,
• Transistorbrus uppstår av laddningsbärarnas rörelser i halvledarmateriaL
Det bildas en sammanlagd brusspänning
som kan bestämmas. Man talar om ett brustal, som är ett mått på mottagningssystemets
egenbrus. Detta skall ställas mot styrkan på
den mottagna signalen. Man talar om ett
förhållande mellan signaleffekt och bruseffekt. Det finns flera metoder att mäta och
uttrycka detta förhållande som kallas S/N
(signal to noise ratio). För att uppfatta den
information som kommer ur en mottagares
LF-utgång måste nyttasignalen vara ett antal gånger starkare än bruset. Den lägre
gränsen för att uppfatta tal i kortvågsmottagare är ett brusavstånd i storleksordningen 1O dB.

114-23

M

ARE

l en broschyr på en kortvågsmottagare
kan man t. ex. läsa "Sensitivity SSB, CW:
lessthan 0.25 J! V for 1O dB SIN ... "
Termen S/N betyder Signai/Noise, d.v.s.
styrkeförhållandet signal/brus uttryckt i dB.
Det innebär att en signal kan läsas vid 25J.tV
signalnivå och ett S/N av mindre än 1O dB.
Utöver brusnivån i mottagaren spelar också
distorsionen en roll.

S+N+D
N
[dB]

Signalbrusförhållande
där S=Signalnivå
N=Brusnivå
D=Distorsionsnivå

UlllllJIDllll~N
3

f kHz

Spektrum

Bild II 4-29 SIN-värde

l en broschyr på en VHF-mottagare kan
man t. ex. läsa "Sensitivity FM: Lessthan
0.18 J.tV for 12 dB SINAD ... "
Termen SINAD betyder Signal, Noise
and Distorsion. Vid denna definition tar man
även hänsyn till distorsionsprodukter som
orsakas av den modulerande signalen.
SINAD= S+N+D
N+ D

[dB]

Nivå
(dB)

Bild 114-30 SINAD-värde

114-24

\subsection{Intermodulation, korsmodulation}

Utöver att en bra modern mottagare bör ha
tillräcklig frekvensstabilitet, känslighet och
selektivitet bör den även ha goda s. k. storsignalegenskaper.
Med storsignalegenskaper menar man
hur bra en relativt svag nyttasignal på mottagaringången motstår påverkan av starka
frekvensnära signaler med hög fältstyrka.
Störningar av detta slag uppstår genom icke
linjära förlopp i komponenter i mottagarens
ingångssteg, varvid mottagna signaler med
stor amplitud blir förvrängda.
Korsmodulation och intermodulation är
två begrepp som är förknippade med storsignalegenskaperna. Båda kan visserligen
definieras och bestämmas entydigt, men de
förväxlas ändå ofta.

\subsubsection{Korsmodulation}
Med korsmodulation menas, att den inkommande nyttasignalen amplitudmoduleras
med modulationsprodukter från en annan
frekvensnära amplitudmodulerad signal,
varvid korsmodulationen uppstår i olinjära
komponenter i mottagaringången (försteg,
blandare). När man med mottagaren i AMläge ställt in den på någon bärvåg så hörs
också andra starka, frekvensnära stationer.
Det måste alltså alltid finnas en nyttasignal på den inställda frekvensen för att det
skall uppstå korsmodulation. När nyttasignalen försvinner så försvinner även korsmodulationen.

\subsection{Intermodulation}
Vid s.k. intermodulation blandas två starka
inkommande signaler i olinjära komponenter
i mottagaringången. Deras blandningsprodukter faller på mottagningsfrekvensen
så att den störs, vare sig det finns en nyttosignal där eller inte.

Frekvensstab i l it et
Se kapitel 4, Oscillatorer


