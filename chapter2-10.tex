\chapter{FARLIGA STRÖMMEN}

\section{Människokroppen}

Elektrisk chock

Strömmens inverkan på människan

Människokroppen är ett komplicerat elektrokemiskt system, som främst kontrolleras av
hjärnan. Musklerna styrs av svaga elektriska
strömimpulser genom nervsystemet. Främmande strömmar genom kroppen kan störa
kroppsfunktioner och kan i olyckliga fall göra
stor skada. Styrkan och frekvensen på strömmarna avgör skadans art och omfattning.
Elektrisk chock kan döda av flera orsaker.
En orsak är att hjärtrytmen störs. Hjärtkammarflimmer och hjärtstillestånd kan lätt
uppstå. Flimmer innebär att hjärtat arbetar
okontrollerat och med kraftigt nedsatt eller
helt upphävd pumpfunktion. Hjärtstillestånd
inträffar lätt av hög spänning. Av otillräcklig
blodtillförsel blir det syrebrist i hjärncellerna,
som då skadas snabbt. Medvetslöshet inträder redan efter ett fåtal sekunder.
En annan orsak är andningsstillestånd
genom att andningscentum blockeras. Det
kan hända när strömmen från en högspänningskondensator i en sändare går genom
kroppen.

Strömstyrkan påverkar kroppen olika från
fall till fall och det är osäkert vilken strömstyrka som är farlig. Det finns både de som
överlevt höga strömmar och de som inte har
klarat några milliampere. Strömmar som går
genom hjärta eller hjärna är särskilt farliga.
Hjärtat sitter i strömvägen för vänster hand,
så när man arbetar med elektriska apparater
under spänning, bör man för säkerhets skull
hålla vänstra handen i fickan l
starka strömmar ger häftiga muskelkramper och/eller brännskador. Muskelkramp kan
förekomma redan vid strömmar under 1O
mA. För vuxna, friska människor är det direkt
farligt när strömmen överstiger detta värde.
För unga eller sjuka redan vid lägre värden.
Men även en "ofarlig" ström kan trots allt
vara ett indirekt faromoment. Vid en oväntad
"stöt" blir man rädd och gör okontrollerade
rörelser, vilket kan leda till fall eller oavsiktlig
beröring av spänningsförande föremål i närheten.

Resistansen genom människokroppen

Påverkan från elektromagnetiska fält

Vid kontakt med ett strömförande föremål
kommer kroppen att bli en del av strömkretsen. Det flyter då en främmande ström genom kroppen.
Strömstyrkan följer Ohms lag och beror
av strömkällans spänning och inre resistans
samt av övergångsresistansen i huden och
kroppens inre resistans.
Övergångsresistansen minskar med fuktigare hud samt med större kontaktyta och
större kontakttryck. Beröringsspänningen inverkar också. Vid spänningar över ca 75 V
minskar övergångsresistansen med ökande
spänning. Vid allvarliga förbränningar minskar övergångsresistansen särskilt mycket.
Den totala resistansen genom kroppen blir
då nära lika med dess inre resistans- ungefär 500 n.
VARNING. Experimentera inte med detta!

Undersökningar harvisat attvistelse i starka
elektromagnetiska fält kan kan påverka människan. Personer som har varit utsatta för
kraftig exponering av fält har bl.a. klagat över
svettningar och huvudvärk. Det forskas omkring dessa fenomen.
Elektromagnetiska fält kan förorsaka fel
i elektronikutrustningar. Halvledare är särskilt känsliga för kraftfält. Det är möjligt att
känsliga instrument, hjärtstimulatorer (pacemaker) etc. kan påverkas av högfrekventa
elektromagnetiska fältfrån radiosändare. När
du använder en sändare, mobiltelefon etc.
och någon får svårigheter med hjärta eller
andning så skall du omedelbart stänga av
din apparat helt! Med tiden utvecklas störningsokänsligare elektronik, men säker mot
störningar kan man aldrig vara.

111 o-1

FAR Ll

STR

EN

Normer för fältstyrkor

Det finns normer och rekommendationer för
elektromagnetiska fältstyrkor i olika syften.
Dels avses fältstyrkor som människor
och djur får utsättas för, dels fältstyrkor som
olika slags apparater skall kunna fungera i
respektive själva utsänder (EMC).
Samarbete mellan länderna utvecklas
inom båda dessa områden.

Konstgjord andning

Vid hjärtstillestånd, hjärtkammarflimmer och
andningsstillestånd skall hjärtmassage och
konstgjord andning sättas in omedelbart och
på ett kunnigt sätt. Obotliga hjärnskador av
syrebrist kan nämligen uppstå inom några få
minuter.
Livräddning vid e/skada är ett instruktivt
häfte från Energikontorets Förlagsservice,
101 53 Stockholm. studium rekommenderas.
Det är för sent när olyckan har skett. Häftet
kan beställas på fax 08 677 26 05.

\section{Allmänna elnätet}

Elektrisk energi levereras till förbrukarna över
transformatorstationerdär högspänning först
transformeras till lågspänning. Från transformatorstationerna förgrenas lågspänningsnätet till serviceskåp ute i kvarter och byar.
l Sverige är fördelningstransformatorns
sekundärlindningar oftast sammankopplade
till ett Y (s.k. Y- eller stjärnkoppling) där
mittpunkten är jordad.
De i Sverige vanligast förekommande 3fas lågspänningsnäten har huvudspänningen
400 V (tidigare 380V) och fasspänningen
230 V (tidigare 220 V). Spänningen mellan
fasledarna kallas för huvudspänning och
spänningen mellan respektive fasledare och
nollledaren kallas för fasspänning.
Bruksföremålen i huset ansluts oftast 1fasigt, d.v.s. mellan någon av fasledarnaoch
nolledaren. Någorlunda lika belastning mellan faserna är önskvärd. Mer effektkrävande
apparater som el-pannor och spisar ansluts
därför till alla tre faserna (3-fasigt). Amatörradioutrustningar ansluts oftast 1-fasigt.

Strömbrytare

Kraftförsörjningen av radiostationens apparater bör ske över en gemensam huvudströmbrytare, som lätt kan nås. En indikatorlampa får gärna markera att den brytaren är
tillslagen och att stationen är under spänning.
Informera familjen och övriga i din omgivning om hur den brytaren fungerar. Det är en
säkerhetsåtgärd om något skulle hända.
Apparaternas nätströmbrytare skall vara
utförda för den aktuella arbetsspänningen
och ha ett godkänt utförande.
Vid 1-fassystem skall nätströmbrytaren i
apparaterna vara 2-polig och bryta fas- och
N-ledare, men aldrig PE-Iedaren.
Vid 3-fassystem skall nätströmbrytaren
vara 3-polig och bryta fasledarna, men aldrig
N-ledare och PE-Iedare.

Kom ihåg, att behörig installatör skall
anlitas vid ingrepp i fasta installationer.

1110-2

FARLI
liten terminologi vid elinstallationer
•
•

•
•

•
•
•

Gruppcentral
Den säkringscentral som följer efter elmätaren, t.ex. i villor och lägenheter.
Gruppledningar
Ledningar efter en gruppcentral, d.v.s.
ledningar till belysning, el-spisar, uttag
m.m.
Fasledare
En ledare som för fasspänning.
Nolledare (N-ledare)
En ledare som är ansluten till elnätets s.k.
nollpunkt (nollskena) och som normalt
inte skall föra spänning till jord.
Skyddsledare (PE-Iedare)
De ledare i kablar och sladdar, som är
speciellt avsedda för skyddsjordning.
Bruksföremål
Ett i princip flyttbart elanslutet föremål,
t.ex. handverktyg och radioapparater.
Förstärkt isolering
Vissa bruksföremål tillverkas med en så
god isolering att de inte behöver skyddsjordas. Så isolerade får anslutningsledningen förses med en speciell stickpropp,
som passar i vägguttag, såväl med som
utan jorddon. Sådana bruksföremål är
märkta med symbolen IQI och får inte
ändras så att de kan skyddsjordas.

Färgkoder för fasa, noll- och skydds ledare.
Isoleringsmaterialet omkring gruppledarna i
fasta elinstallationer har färger som fyller en
viktig funktion. Dessa färger får därför aldrig
förväxlas.
Fasledaren har i regel svart färg. N-ledaren (nollan) har b lå färg.
Det är till fas- och N-ledarna i vägguttagen, som man kopplar apparaterna för att få
ström. Helst skall uttagen också ha jorddon
d.v.s. en extra kontakt- ett s.k. jordningsbleck. Detta bleck är anslutet till PE-ledaren
(skyddsjorden), som är färgad randigt gult/
grönt.
En gullgrön ledare är alltid en skyddsjordledare och får endast användas för det.
l äldre installationer kan emellertid skyddsledarens isolering vara t.ex. röd.

STR

MEN

Uttag och stickproppar med jorddon
Jorddonet ger förbindelse med elsystemets
skyddsjord (PE).
Det är rummets utförande, som avgör om
vägg- och lamputtagen där skall ha uttag
med jorddon. Bostadsrum är klassade som
inte särskilt riskfyllda och har därför tidigare
inte försetts medlamp-och vägguttag med
jorddon. Vid nybyggnation är emellertid numera alla uttag försedda med jorddon!
Kök och tvättstugor med ledande plåtbänkar, vattenkranar o.s.v. anses som riskfyllda rum och måste ha uttag med jorddon.
Samma gäller källare och liknande andra
rum med ledande golv, väggar och inredningar.
Det är tillrådligt att installera uttag med
jorddon för radiostationen. Observera då, att
alla uttag i det rummet skall ha jorddon!
skyddsjordning
Att jorda är det vanliga uttrycket för att ansluta ett föremål till ett jordtag. Metallhöljen
på apparater kan av olika anledningar bli
spänningsförande och är då en elsäkerhetsrisk. För att säkert ha nollpotential på höljena
kan de kopplas till jordskenan via PE-Iedaren- d.v.s. skyddsjordas. När man ansluter
apparathöljet till jorddonet, kommer
säkringen att bryta strömtillförseln om det
blir isolationsfel mellan en strömförande del
och höljet. PE-Iedaren får därför aldrig brytas!
Om skyddsjordning finns särskilda föreskrifter. Om du inte är säker på hur skyddsjordning skall utföras, fråga en behörig installatör.

Jordfelsbrytare
Jordfelsbrytare kallas en brytare som automatiskt bryter spänningen, när det uppstår
överledning till skyddsjordade detaljer- s.k.
jordfeL Brytaren mäter felströmmen och bryter spänningen innan strömmen uppnår ett
farligt värde, t. ex. 1O mA. Jordfelsbrytare får
inte ersätta skyddsjordning, men kan under
särskilda förutsättningar komplettera
skyddsjordningen som en extra säkerhetsåtgärd. Låt installera jordfelsbrytare l

111 o- 3

F RLI ASTR MM
Särjordning

Särjordning är ett uttryck för att jorda apparater till en separat jordpunkt, Det görs via
separat jordlina till ett jordtag, d.v.s. jordplåt
eller jordspett Särjordning skall ske på rätt
sätt eftersom det avsedda skyddet annars
kan bli en fara.
Särjordning får ske endast om skyddsjordning till PEN också har gjorts. Om du har
planer på särjordning, fråga en behörig installatör

Jordning av antennsystem

l brist på annan jordpunkt är det frestande att
ansluta antennjordledaren till PE-Iedarens
anslutningsbleck i vägguttaget med förhoppning att på så sätt få ett bättre HF-jordplan för
antennen. Detta är emellertid ett dåligt exempel påsärjordning, som både kan innebära säkerhetsrisker och medföra störningsproblem.

Snabba och tröga säkringar

Det finns snabba och tröga säkringar. Snabba
säkringar är det som normalt används. Tröga
säkringar för samma strömstyrka kan behövas för apparater som har speciellt hög startström, t.ex. stora nättransformatorer med
toroidkärna. Säkringarna skall kunna bryta
tillräcklig hög spänning, annars blir det en
kvarstående ljusbåge i dem vid säkringsbrott. Använd säkringar med rätta strömvärden. Det är förbjudet att laga säkringar,
vilket naturligtvis kan orsaka både brand och
andra faror.

\section{Faror}

Överhettning

Elektricitet kan lätt vålla både personskador
och materiella skador. Det är viktigt att veta
hur skador kan undvikas. Elektrisk utrustning skall vara beröringsskyddad med fullgod kapsling. Samtidigt får värmen inne i
kapslingen inte bli så hög att det innebär
brandrisk. Spontana fel kan trots allt uppstå.
Isolationsfel medför risk vid beröring och
brand kan utvecklas snabbt. När utrustning
under spänning lämnas obevakad, skall det
ske med särskild aktsamhet.
Hur elektriska apparater och anläggningar får utföras, regleras av lagar och föreskrifter. Elektriska apparater skall uppfylla
vissa krav för att få marknadsföras och användas. Utförande och ursprung skall vara
dokumenterat på föreskrivet sätt.
Även självbyggda apparater skall uppfylla kraven på elsäkerhet- d.v.s. säkerhet
mot e Ichock och brand - och byggaren bär
ensam ansvaret för att utförandet och
hanterandet av apparaterna är betryggande.
Den som bygger och använder en elektrisk apparat bör därför ha nödvändiga kunskaper om elsäkerhet.

Höga spänningar
Ingrepp i elektriska apparater under spän-

ning innebär personfara. Öppna aldrig en
apparat om spänningen är tillslagen. Vid
ingrepp t.ex. i sändare, mottagare och
kraftförsörjningsaggregat är det lätt att utsätta sig för höga likspänningar. l sändare
med elektronrör förekommer spänningar i
storleksordningen hundratals till tusentals
volt. Så är det också i bildskärmar.
Observera att även apparater som drivs
med batteri eller ackumulatorer kan innehålla kretsar som omvandlar den låga spänningen till direkt livsfarlig hög spänning. Exempel på det är likspänningsomvandlare.

Höga strömmar

Höga strömmar ger häftiga muskelkramper
och brännskador. Man vet att det skiljer
mellan skador av lik- respektive växelström.
Lågfrekvent växelström ger upphov till
muskelkramper, som kan göra det omöjligt
att släppa det strömförande föremålet.

111 o- 4

MMEN
Högfrekvent växelström i MHz-området
värmer upp kroppsvävnaderna, snarare än
att förorsaka muskel reaktioner.
Likström påverkar kroppen annorlunda
än växelström. Genom det elektriska motståndet i kroppens vävnader och vätskor
utvecklas det värme. Detta kan leda till brännskador både på huden och inne i kroppen.
Om likströmmen pulserar uppstår dessutom
muskelreaktioner på liknande sätt som vid
växelström.
Höga spänningar är alltid farliga. Det är
däremot inte så känt att även låga spänningar kan vara det. Ackumulatorer och anslutna apparater kan ge ifrån sig höga strömmar även om spänningen är låg. Oavsiktliga
strömvägar t.ex. kortslutning genom en
klocka eller fingerring kan medföra allvarliga
brännskador.
Antenner
Placera helst antennerna utom räckhåll för
obehöriga. På sändarantenner kan det nämligen uppstå höga H F-spänningar redan vid
låg sändareffekt. HF bränns vid beröring och
en reflexrörelse gör det lätt att tappa balansen och falla. Sätt gärna upp skyltar på eller
invid antennerna, med varning för högfrekvent spänning samt uppgift om ägarens
namn, adress och telefonnummer.
Antenner får inte korsa eller placeras
nära högspännings-, lågspännings- eller
telefonlinjer. Det är en olycksrisk om antenner och kraft- eller teleledningar av någon
anledning slår ihop. Det är också en olycksrisk om antenner faller ner över dessa ledningar.
Endast efter tillstånd från berörd myndighet
och/eller linjeägare får man dra ledningar av
något slag över väg eller offentlig plats även antenner.
Höga likspänningarfrån sändaren får inte
komma ut i antennen. Se till att antennernas
matarledningar är kopplade till god likströmsjord via HF-drosslar eller försedda med
överspänningsavledare. Som extra säkerhetsåtgärd bör sändaren anslutas till antennledningen över en stor kondensator.
Undvik att beröra antenner utan att de
jordats, särskilt vid vistelse på tak eller i träd.

Under åskväder, snöfall, regn eller dimma
då laddade partiklar är i rörelse, kan antennerna laddas upp till höga statiska spänningar. Arbetar man då med antennen kan man
överraskas av en elektrisk stöt. Det är då lätt
hänt att tappa taget och falla ner.
Ae~sti«:U'le!lnn·Hl i kondensatorer
Kondensatorer kan behålla en betydande
restladdning under många timmar sedan
kraften brutits.
Koppla urladdningsresistorer (bleeder)
över filterkondensatorer, så att de laddas
ur när matningen stängs av. Av säkerhetsskäl skall urladdningsresistorerna tåla
fyra gånger så stor effekt som de själva
förbrukar under drift.
• Varning: När du laddar ur en kondensakortslut den inte! Använd en resistor!
@l

säkerhetsåtgärder
Transformator med förstärkt säkerhet
• Om du är osäker på det elsäkerhetsmässiga utförandet på en apparat, t.ex. en
gammal sändare, använd då en skiljetransformator (fulltransformator) - helst
av klass Il (extraisolerad).
Vid reparation skall utrustningen vara spänningslös. Före arbetet skall du
e
Stänga av utrustningens nätströmbrytare,
• Dra ur stickproppen ur vägguttaget (dubbel säkerhet),
Om trimning eller felsökning måste ske under spänning skall följande iakttas:
• Arbeta inte med anläggningen när du är
trött eller omotiverad. Då är du minst
vaksam mot olyckor.
Se till att du inte får ström genom kropArbeta helst bara med höger hand
och håll den andra borta från den utrustning som du arbetar med. Stoppa gärna
den fria handen i fickan!
Ha inga hörtelefoner på huvudet. Använd
högtalare om du trimmar med hörseln.
®
Helst bör någon finnas i närheten när du
arbetar i ap pararter under spänning. Visa
var nätströmbrytaren sitter. Se gärna till
att han/hon kan elolycksfallshjälp.
@

@

111 o- 5

FARLI A STR MMEN
Vid arbete med ackumulatorbatterier
• Trots att spänningen är låg kan ackumulatorbatterier lämna mycket höga strömmarvid kortslutning. Tag därför av fingerringar, armbandsur m.m. Använd isolerade verktyg vid arbete med batteripoiskor.
• Akta dig för elektrolyten i ackumulatorbatterierna- den är starkt frätande.
Varning för explosionsrisk av knallgas
och syrastänk i ögonen.

PT
\section{Åska}
Faror

Vid åska utvecklas det mycket starka,
elektromagnetiska fält, som breder ut sig
och alstrar mycket korta spänningsstötar i
alla metallföremål, t.ex. i antenner. stötarna
vandrar genom kablarna in i apparaterna. Är
stötspänningen tillräckligt hög, kommer saker i strömvägen att förstöras på något sätt.
Förbränning och nersmältning är vanligt.
Men om åskurladdningen sker på långt håll,
kan stötspänningen bli så låg att man någorlunda kan undgå skada på apparater och
hus. Om åskurladdningen däremot sker
mycket nära antennen eller som direkt nerslag, då uppstår definitivt stora skador.

Skydd och jordning

Antenner och antennkablar kan man aldrig
skydda mot åsknerslag. De är ju till sin natur
en slags åskledare. Det man kan försöka att
göra är att leda en eventuell åskurladdning i
ett antennsystem bort från hus och människor. Observera, att man inte får "haka på"
husets ordinarie åskledare. Då gäller inte
husförsäkringen.
Antennkabel n, som fungerar som en (för
klent dimensionerad) åskledare, skall naturligtvis INTE l ONÖDAN dras in i huset utan
kortaste vägen utanför huset till en avgrening.
Från avgreningen fortsätter dels kabeln
in till apparaterna genom ett överspänningsskydd och dels en jordlina kortaste vägen
ner till jordtaget över en gniststräcka. Det
bästa sättet att skydda apparaterna mot
åska är fortfarande att koppla bort dem helt
från antennkabeln och vägguttag.
Om man bor i ett hyreshus är det tyvärr
oftast svårt att få vidta åtgärder som dem
härovan. Då får man nöja sig med att koppla
bortantennledningarna från apparaternaoch
lägga dem väl åt sidan - gärna utanför
husväggen.
Som permanent, men otillräckligt skydd
kan man förse de olika anslutningsställena
med lämpliga överspänningsskydd.
Att hoppas på skydd mot åsknerslag genom jordning i elsyste m et är naturligtvis helt
befängt!

1110-6
