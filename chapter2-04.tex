\chapter{MOTTAGARE}
Energin i de elektromagnetiska magnetfält,
som omger oss, alstrar högfrekventa strömmar i alla metaflföremål. För att effektivt
fånga upp dessa fält används antenner.
Fastän energin i fälten kan få en lampa
att lysa om sändarantennen är tillräckligt
nära, så går det ändå inte att uppfatta den
information som fälten också kan innehålla.
För det behövs en radiomottagare för att
dels förstärka de oftast mycket svaga signalerna och dels uttyda informationen i dem.

Lyssna på amplitudmodulerade rundradiosändningar på mellanvåg kan man enklast
göra med hjälp av en detektormottagare
Speciellt under dygnets mörka timmar
vintertid kan man höra utländska sändare
med denna enkla mottagare, låt vara att det
hörs mycket svagt. l detektormottagaren
omvandlas fältens energi till elektricitet och
sedan till ljud. Så länge som ingen förstärkare används, förbrukas ingen annan energi
än den som fångas ur fälten -radiovågorna.

\

\

\

\ \
l

\

\
l

!

antenn selektering demodulering

tF-återgivning

Bild II 4-1 Detektormottagare

Raka mottagare
Mottagare med kristalldetektor
Bild II 4-1
Detektormottagaren består av ett mycket
litet antal komponenter. Princip och arbetssätt framgår av bilden. Samma princip används även i mer komplicerade mottagare,
mätinstrument etc. Antennkretsen består
av antenn, jordtag och däremellan en induktor (kopplingsspole), som överför energin
från antennen till en svängningskrets. Svängningskretsen används för att välja ut (selek-

tera) en bärvåg med önskad frekvens. Bärvågen kan naturligtvis inte höras, men av
kurvformen på bilden framgår att bärvågen
är amplitudmodulerad med en LF-signal.
För att återvinna LF-signalen utför man
en s.k. demodulering med hjälp av dioden.
Dioden klipper bort antingen de positiva
eller negativa halvvågorna i den mottagna
signalen, beroende på hur dioden är vändpolariserad. Kondensatorn, som är kopplad
parallellt över hörtelefonen, glättar de högfrekventa spänningstopparna till ett amplitud-

114-1

M
medelvärde (jämför med entaktsblandare i
Kapitel 3). Detta spänningsvärde varierar
på ett sätt, som motsvararden modulerande
spänning i sändaren som kommer av tal,
musik etc. Vi har nu demodulerat bärvågen,
återställt LF-signalen och kan höra den i
mottagaren.
Bild II 4-2
signalspänningen över svängningskretsen
är störst när dess resonansfrekvens och
antennströmmens frekvens är lika.
Överst i bilden ser man att mottagaren är
inställd på samma frekvens som sändare 2.
Även sändare 3 hörs eftersom bandbredden
i svängningskretsen är stor. Nederst i bilden
är svängningskretsen inställd på sändare 3,
men man hör också sändare 2 och 4.
Bandbredden i svängningskretsen blir
mindre ju mindre den belastas, d.v.s dämpas. l Bild II 4-1 består belastningen av
antennen (via kopplingsspolen), hörtelefonen och avkopplingskondensatorn (via dioden).
Mindre belastning kan åstadkommas på
två sätt; dels med "lösare" koppling mellan
antennkrets och svängningskrets och dels
med bättre impedansanpassning mellan
svängningskrets och diod. Båda sätten tillämpas i Bild II 4-3. Hur selektionen då
förbättras visas i Bild II 4-4, vilket skall
jämföras med Bild II 4-2.

demodulering

u

Bild II 4-2 Selektion i detektormottagare

u
550kHz

900kHz 1100kHz 1500kHz

Bild II 4-4 Förbättrad selektion

lågpass

volymkontroll

LFförstärkare

högtalare

}\
selektering

demodulering

lågpass

LFförstärkare

Bild II 4-3 Detektormottagare med LF-förstärkare

114-2

högtalare

f

GARE

Bild II 4-5 Förbättrade HF-egenskaper i detektormottagare

u

membranen i hörtelefonen knäpper
(ingen ton)

/\r---------------t
Bild II 4-6 Hög HF-selektion

Detektormottagare med förstärkare
Bild II 4-3
Om man vill höra sändningarna över högtalare, behövs högre effekt än vad som kan
fångas upp genom antennen. För ändamålet används en LF-förstärkare, som drivs av
en annan energikälla, t. ex. ett batteri. LFförstärkaren kan även minska belastningen
på svängningskretsen.
l bilden har ett LF- lågpassfilter satts in
efter HF-avkopplingskondensatorn. Det
dämparLF-signaler med högre frekvens än
vad som behövs för god mottagning.
Mottagare med bättre HF-egenskaper
Bild II 4-5
Ett sätt att minska bandbredden i en detektormottagare är att koppla flera svängningskretsar med samma frekvens efter varandra. Den större dämpningen av fler kretsar
kan kompenseras med en HF-förstärkare.
Sådana mottagare används för speciella
ändamål, t.ex. för övervakning av en enda
frekvens. Då är svängningskretsarna fast
avstämda. Kanske utnyttjas till och med en
kvartskristall som filterförden speciella frekvensen. Se Bild II 4-6 om hög selektion.

"knack" "knack" "knack"

\

"knack"

---------

membranen släpper

Bild 114-7 CW i detektormottagare

Detektormottagare och sändningsslag
l huvudsak fungerar detektormottagaren endast vid amplitudmodulering. Det innebär
sändningsslagen A3E och A2A, d.v.s. amplitudmodulerad telefoni resp. tonmodulerad
telegrafi, båda med full bärvåg.
Bild II 4-7 Däremot fungerar detektormottagaren inte vid A 1A, d.v.s. telegrafi med
endast bärvåg. En omodulerad bärvåg alstrar nämligen endast en likström i en
detektormottagare. Vid nyekling hörs då
endast knäppningar i hörtelefonen vid början och slutet av teckendelarna.
Detektormottagaren fungerar inte heller
vid J3E, d.v.s. SSB och övriga sändningsslag med undertryckt bärvåg. Ljud såsom tal
förvrängs nämligen kraftigt i en J3E-signal
eftersom bärvågskomponenten saknas.
l båda ovannämnda fal kan talet återställas med tillsats av en bärvåg.
Slutligen kan sändningsslag som innebär frekvens- och fasmodulering i princip
inte dernoduleras med detektormottagare.
114-3

PT

M

~2 +D
~l
fl

f1 =1830kHz

och dess

ö"ctoo"

f2- f1 = 1 kHz

f2 - f1 ::: 1kHz

/,
[>

t
~

f 2 = 1831 kHz

VFO

blandare

A
antenn

x

t:;+

b,

u

demodulering LF-Iågpassfilter

förse lek·
ter ing

[>
LF-för·
stärkare

högtalare

~
VFO

Bild II 4-8 Mottagare med direkt frekvensblandning
Mottagare med direkt frekvensblandning
För att demcdulera A 1A och J3E i en rak
mottagare- detektormottagare-måste den
kompletteras med en oscillator som alstrar
en intern bärvåg. Denna blandas med den
mottagna signalen. Det uppstår då en svävningston - beat frequency. Därav namnet
Beat Frequency Oscillator- BFO.
Förfarandet har givit mottagartypen sitt
namn- direktblandad mottagare.

114-4

Bild II 4-8
Ett sätt att komplettera den raka mottagaren
med BFO framgår av bilden. När BFO kopplas till och ställs in på en frekvens tillräckligt
nära mottagningsfrekvensen så uppstår en
hörbar ton.
Demodulatordioden tillförs alltså två HFsignaler, dels den från antennen och dels
den från BFO. Dessa båda signaler blandas
i dioden och skillnadsfrekvensen är den
hörbara tonen. Övriga blandningsprodukter
dämpas av ett lågpassfilter.

A ARE
HF på blandaringången

u~

LF-skillnadsfrekvens på
blandarutgången

VFO-

CWsignal
1830 kHz
l

si nal

1lb1 kHz

l1

... f

f1 f2

UrF01is2~1
/f-

u
kHz

CWsignal 1830 kHz

j

1 kHz

-JAI.I. f

1
~~~----------f

mo

f2 f1

u

f1- f2

119 kHz
~----------~~-~f-, f
V F O-

ur

si nal
1S29 2 kHz C;W,
sognal
1830kHz

1l

f

l

0,8 kHz

1

--f

fz f1

~-L------------f

f,- f2

Bild II 4-9 Demodulering i mottagare med direkt frekvensomvandling- C W-signaler

U

u~d~rtryckt
barvag

l SB

l

1832
kHz

~---

!l

i

1835kHz

l

fl\ 1

1834 1834,7
kHz kHz

u
1kHz
300 Hz 

.. f

r-1

111

3 kHz

1:
L------!l.-A--.L..----f

Bild II 4-1 O Demodulering i mottagare med direkt frekvensomvandling - SSB-signaler
Mottagning av telegrafi (CW)
Bild II 4-9
Då BFO (VFO) är inställd på frekvensen f2
=1831 kHz och den mottagna signalen f1 har
frekvensen 1830 kHz så hörs en svävningston med frekvensen 1000 Hz.
Samma resultat fås om BFO ställs in på
frekvensen f2 = i 829 kHz. Med BFO på

frekvensen f2 = 1830 kHz hörs ingenting av
signalen f 1 = 1830 kHz från sändaren.
Frekvensskillnaden är noll Hz.
De flesta föredrar en ton med frekvensen
c:a 800Hz för mottagning av telegrafi. BFOfrekvensen skulle i så fall ställas in på 1830.8
eller i 829.2 kHz om f 1 vore en telegrafisändning.

114-5

Selektering av blandningsprodukterna

[>
LF-lågpass
3kHz

förselektering
av ca 300kHz
bandbredd

Val av mottagarfrekvens

VFO

U HF
VFO-

frekvens

.H,

l r--4s
CW-Signal
1830 kHz

1829,2

1
,J

.

rll

~il

u

SSB-Signal
1835 .-·

lågpassfilterkurva

O - 3 kHz

1832 kHz

 L,

35 kHz

l

·f

1632 1834 1634,7
kHz kHz kHz

l
l

I....L-A.I..'-.L...ll.------ f

800 Hz 2,8 kHz

Vid mottagning av en CW-signal tillsammans med en SSB-signal hörs båda samtidigt

u
BOO Hz

4,8 kHz

5,5kHz

VFO
Förbättring av selekteringen med ett LF-CW-fiiter

Mottagning av J3E (SSB)
När en SSB-sändare sägs arbeta t. ex. på
frekvensen 1835 kHz, så innebär det frekvensen på den bärvåg som undertryckts i
sändaren redan före utsändningen.
Vad som uppfattas av mottagarens ingångskretsar är alltså det utsända sidbandet När en SSB-signal demoduleras, så
blandas den lokala bärvågen i mottagaren
med de mottagna modulationsprodukterna.
Vid blandningen uppstår blandningsproduk-

114-6

ter som består dels av LF, dels av andra
högre frekvenser som dämpas i ett låg passfilter.
Bild II 4-1 O
l nom amatörradio används för SSB det lägre sidbandet vid frekvenser under 1O MHz.
Med en frekvens av t. ex. 1835 kHz och ett
talspektrum av 300-3000 Hz kommer det
lägre sidbandet att finnas mellan 1834.7
och 1832.0 kHz. Tre modulerande frekvenser 300, 1000 och 3000 Hz visas på bilden.

TTA ARE
Med en bärvågsfrekvens av 1835kHz motsvaras dessa modulerande frekvenser av
utfrekvenserna1834.7, 1834 och 1832kHz.
VFO ersätter SSB-sändarens bärvåg och
skall ha samma frekvens-1835kHz- för att
kunna återge 300, 1000 och 3000 Hz.

selektionen i direktblandade mottagare

Bild II 4-11
Direktblandade mottagare kan ses som en
typ av detektormottagare, även kallad "rak"
mottagare. Begreppet "rak" kommer av att
HF-signalen från antennen passerar genom en selektiv krets och en eventuell HFförstärkare rakt fram till detektorn, utan att
frekvensen omvandlas.
l en detektormottagare är bandbreddenoftast rätt stor. Flera sändare hörs därför
samtidigt.
P.g.a. att blandningsdioden i en direktblandad mottagare även fungerar som AMde modulator, så hörs faktiskt alla sändare
inom förkretsens bandbredd. Detta kan undvikas till en del genom att dioden, som
fungerar som entaktsblandare, byts till en
mottaktblandare eller ännu hellre till en ringblandare. Sådana blandare undertrycker ingångsfrekvenserna och släpper endast igenom blandningsprodukter. Bara den sändarsignal hörs då, vars frekvens tillsammans med VFO-frekvensen ger blandningsprodukter, som faller inom LF-filtrets passband. Mottagningsfrekvensen är VFO-frekvensen. Svängningskretsen fungerar som
en ställbar förselektor och LF-Iågpassfiltret
ger den egentliga frekvensselektionen.
Vilka HF-signaler bildar blandningsprodukter med VFO-frekvensen och vilka av
dessa passerar sedan genom lågpassfiltret
efter nedblandning till LF-nivå?
Exempel:
En CW-sändare med frekvensen 1830 kHz
tas emot genom att mottagarens VFO ställs
in på frekvensen 1829.2 kHz. Från blandarutgången kommer då en ton med frekvensen 800Hz.
Men sändaren är inte ensam på bandet.
Kommer t. ex. SSB-sändaren på 1835, som
moduleras med 300, 1000 och 3000 Hz, att
störa mottagningen? (Bild II 4-1 0).

Förkretsen i mottagaren är så bred att
denna sändning passerar. SSB-sändarens
signalfrekvenser i det utsända sidbandet är
1834.7, 1834.0 och 1832kHz. Dessa frekvenser blandas med mottagarens VFO-frekvens 1829.2 kHz och alstrar blandningsprodukterna 5.5, 4.8 och 2.8 kHz. Eftersom
lågpassfiltret i mottagarens LF-förstärkare
har bandbredden 0-3000 Hz, så kommer
endast blandningsprodukten 2.8 kHz attvara
störande. För att förbättra CW-mottagningen, så kan lågpassfiltret bytas ut mot ett
bandpassfilter, som endast släpper igenom
ett smalt frekvensområde omkring mittfrekvensen 800 Hz.

Passband och spegelfrekvenser i direktblandare

Bild II 4-12
l exemplet i förra stycket blev problemet
med en störande ton löst med ett bandpassfilter med annan frekvensgång.
Men vilka frekvenser kan tas emot genom
ett lågpassfilter, 0-3000 Hz, om VFO-frekvensen är t.ex. 1829.2 kHz?
Experiment:
Ändra frekvensen på en CW-sändare långsamt från 1820 till 1840 kHz.
Såndarfrekvensen i 820 kHz hörs knappast eftersom överlagringstonen har frekvensen 9.2 kHz och den dämpas kraftigt av
lågpassfiltret Först när sändarfrekvensen
är 1826.2 kHz hörs en tydlig ton med frekvensen 3000 Hz. Fortsätter man att ändra
sändarfrekvensen, så sjunker tonens frekvens för att bli noll (svävningsnoll), när såndarfrekvensen är lika med mottagarens VFOfrekvens 1829.2 kHz. Om man nu fortsätter
med att höja frekvens, så blir överlagringstonens frekvens åter högre. Vid såndarfrekvensen 1832.2 är den 3000 Hz. Vid ännu
högresändarfrekvens dämpas överlagringstonen igen av lågpassfiltret
Slutsatsen av experimentet blir följande:
Vid en direktblandande mottagare med VFOfrekvensen i 829.2 kHz och ett 3 kHz lågpassfilter blir varje sändare hörbar, som har
en sändningsfrekvens mellan 1826.2 och
i 832.2, varvid överlagringstonen har frekvenser från 3000Hz, ner genom noll och upp
till 3000 Hz igen.

114-7

TTAGARE
HF

U

LF

u

.fvFo~
(---r-----,

-3 kHz

r-----.. ,

+3 kHz

l

l
l

l.f-----1-t

l
l

.

l

l

..

:

f

~--~----------f
3kHz

1826,2 1829,2 1832,2

kHz

kHz

kHz

6kHz HF-bandbredd vid3kHz LF-bandbredd

U

l - - . -:. .f.1. s.fvFO.L.~. . !\----f

1828,5 ·-·
1828,3 kHz

1829,2

kHz

1829,9 ---

1830,1 kHz

u

,-,

l

l

l
l
l

l
l
l

l

j

l

l

l

\

~~~----------f
700-900 Hz

Mottagningsfrekvens och spegelfrekvens
med ett LF-CW-filter

l

fvFO
r----r----~
l

-

!

183 2
kHz

1835

1838

kHz

-t

kHz

Mottagningsfrekvens och spegelfrekvens
med ett LF-Iågpassfilter

Bild II 4-12 Passbandbredd och spegelfrekvenser i direktblandade mottagare
Vår mottagare har bandbredden 6 kHz.
Varje annan sändare inom denna passbandbredd kommer att höras eller-om man
så tycker- störa mottagningen.
Tillbaka till exemplet med bandpassfiltret
Vilka frekvenser kan tas emot med ett
bandpassfilter 700-900 Hz (mittfrekvens 800
Hz), om VFO-frekvensen är 1829.2 kHz?
Jo, vi kan lyssna rätt ostört till vår CWsändares 800 Hz-ton på frekvensen 1830
kHz. Ändå kan en annan sändare med
frekvensen 1828.4 kHz störa mottagningen
därför att denna är spegelfrekvens till mottagningsfrekvensen 1830 kHz. Vid VFOfrekvensen 1829.2 kHz uppstår en överlagringston, inte bara vid sändarfrekve~sen
1830kHz utan också vid 1828.2 kHz. Aven
denna andra sändarfrekvens, liksom nytto114-8

frekvensen, släpps igenom bandpassfiltret
Spegelfrekvensmottagning är en principiell nackdel i mottagare med direktblandning. Nyttafrekvens och spegelfrekvens i
det senaste exemplet ligger 1.6 kHz (2 · 800
Hz) ifrån varandra, alltså dubbla värdet av
bandpassfiltrets mittfrekvens.
Vid 888-mettagning måste naturligtvis
hela LF-området upp till 3000 Hz kunna
släppas igenom. Utöver det önskade frekvensområdet 1832-1835 kHz, kommer även
spegelfrekvenser i området 1835-1838 kHz
att kunna tas emot.
Vid en LF-bandbredd av 3 kHz har således den direktblandade mottagaren en bandbredd av 6kHz, vilket är en god avstämningsskärpa i jämförelse med den 300 kHz breda
förkretsen.

M
För- och nackdelar med direktblandare
Enkel uppbyggnad, men trots det en god
känslighet och hygglig avstämningsskärpa.
VFO kan även användas till att styra en
sändare.
Spegelfrekvensmottagning är tyvärr
oundviklig. Vidare kan signaler från starka
sändare stråla in i den känsliga LF.:.förstärkaren och orsaka LF-detektering, om mottagaren är otillräckligt skärmad. Förbättrad
isolering mellan antenn och VFO kan dock
fås med en HF-förstärkare.
Entakts diodblandare är olämplig i en
direktblandad mottagare. Den tar emot alla
sändare inom förkretsens passband och en
del av VFO-signalen kommer att strålas ut i
antennen. Ingen av dessa nackdelar finns i
en mottakts-eller ringblandare.

f.MF

ARE

Superheterodyn mottagare
Superheterodynprincipen ger mycket större möjligheter, när önskemålet är en högselektiv mottagare för flera olika frekvenser.
Skillnaden mellan en direktblandad mottagare och en "super" är, att blandningsprodukterna i direktblandaren blir till LF direkt, medan de i supern först bildar en
mellanfrekvenssignal MF, vilken sedan
dernoduleras och blir till LF- detekteras.
l det följande kallas superheterodynmottagaren enbart SUPER. l supern blandas de mottagna signalerna med signalen
från en VFO. Före blandningen har HFsignalerna passerat ett selektivt försteg, som
dämpar spegelfrekvenser. För att inte störa
mottagningen placeras VFO-frekvensen alltid utanför det frekvensband, där man vill ta
emot signaler.
Bild 114-13
Alla mottagna signaler blandas med VFOsignalen. Mottagningsfrekvensen är vanligen skillnaden mellan en fast s.k. mellanfrekvens MF och VFO-frekvensen. Mellanfrekvensen är egentligen mittfrekvensen i
ett fast passband skapat av ett antal filter.

=455 kHz
M F-filter

DetektorIdemodulator

j~sfilter

fast avstämt bandpassfilter
VFO

fvro :: 4055 kHz

demodulator

LF-Iågpassfilter

LF

fMF = fvFo - fM (VFO-frekv. över mottagn.frekvensen)
eller
fMF = fM - fvFo (VFO-frekv. under mottagn.frekvensen)
Hög selektivitet, enkel avstämning {jämfört med en rak mottagare)

Bild II 4-13 Superheterodynmottagaren i princip

114-9

M TTA ARE
Dubbelsuperheterodynmottagare
Bild 114-14
Det är svårt att bygga enkla mellanfrekvensfilter för höga frekvenser, med liten bandbredd och branta flanker. Det är fallet för en
enkelsuper för kortvåg med en enda mellanfrekvens, t.ex. 9 MHz.
En god närselektion på höga frekvenser
är endast möjlig med relativt dyrbara kristallfilter. Däremot går det att få god närselektion
med enklare medel på lägre frekvenser.
En dubbelsuper, d.v.s. en super med
dubbel frekvensomvandling, möjliggör god
både när- och förselektion. l 1 :a blandaren
blandas den mottagna signalen med signalen från en 1 :a oscillator (VFO} till en hög
mellanfrekvens, t. ex. 9 eller 10.7 MHz.
Därmed kan en god spegelfrekvensdämpning erhållas. Första M F-filtret kan göras enklare och utan den höga selektivitet
som hade behövts i en enkelsuper. 1:a MF
blir sedan blandad ytterligare en gång i 2:a
blandaren till en 2:a MF, t.ex. 455kHz. För
den andra blandningen används en fast
oscillator. Filtret i 2:a MF kan lättare utföras
med en hög selektivitet, p.g.a. den lägre
frekvensen.
Exempel:
Trots att M F-filtret inte är en enkel
svängningskrets, kan ett "Q-värde" beräknas. Vid en passbandbredd av 6kHz och en
centerfrekvens av 455 kHz kan Q-värdet
anses vara

Bild II 4-13 visar en mottagare med mellanfrekvensen 455kHz, som är vanlig i äldre
mottagare. MF-filtret kan i enklaste fall bestå av ömsesidigt magnetiskt kopplade LCsvängningskretsar. Bättre avstämningsskärpa fås med resonatorer av keramik eller
kvarts eller de är elektromekaniska.
Exempel:
En sändning på frekvensen 3600 kHz
skall tas emot. Vi ställer då in VFO-frekvensen till 4055 kHz, eftersom mellanfrekvensen är 4055 - 3600 = 455 kHz. Den
mottagna signalen hamnar då mitt i MFfiltrets passband.
Signaler på angränsande frekvenser tas
också emot och alstrar blandningsprodukter.
Med ett mellanfrekvensfilter med t. ex. 3kHz
bandbredd (453.5-456.5 kHz), kan signalfrekvenser mellan 3598.5 och 3601.5 passera genom filtret. En signal med en närliggande frekvens t. ex. 3603kHz, och blandad
med den inställda VFO-frekvensen 4055
kHz, kommer att alstra en skillnadsfrekvens
av 452 kHz. Denna signal ligger utanför
filtrets passband och kommer att dämpas
och når inte detektorn.
VFO-signalen kan givetvis läggas under
i stället för över mellanfrekvensen.
Exempel: VFO-frekvensen 3145kHz kan
också användas för mottagning av frekvensen 3600 kHz, om mellanfrekvensen är 455
kHz (3600- 455 =3145kHz). Men för att
undvika att eventuella övertoner från VFOsignalen blandas med mottagna signaler är
det lämpligt att placera VFO-frekvensen över
mottagningsfrekvensen.
Efter M F-filtren följer bl.a. detektorer för
olika sändningsslag samt LF-förstärkare.
Jämför med Bild II 4-5 och Il 4-6

Q= f,es

b

CJ

Bild II 4-14 Dubbelsuperheteodynen i princip

o

6

l ett MF-filter med centerfrekvensen 9
MHz skulle det behövas ett nära 20 gånger
högre Q-värde för samma bandbredd6kHz

I

114- 1

= 455 =76

ARE
Q= ~es

b

= 9000 = 1500
6

Ett så högt Q-värde kan endast erhållas
med kristallfilter.
För högre mottagningsfrekvenser räcker
det, på grund av filterproblematiken, oftast
inte med en dubbel frekvensomvandling,
Om man antar en dubbelsuper-mottagare
för VHF-området 144-146 MHz enligt bilden, så skulle en i :a MF med frekvensen
i 0.7 MHz inte vara tillräckligt hög. Vid en
mottagningsfrekvens av 146 MHz är nämligen spegelfrekvensen i 46 + (2 • 1O. 7) =
i 67.4 MHz, alltså endast i .15 gånger mottagningsfrekvensen. Det hade alltså varit
lämpligt med en trippelsuper, d.v.s. en trefaldig frekvensomvandling, med en i :a MF
i frekvensområdet 70 MHz.

Jämförelse mellan supern och detektormottagaren
Principen för detektormottagaren är enkel. l

en sådan sker allt från antenn till dernodulering på samma frekvens, d.v.s. mottagningsfrekvensen. Signalen går utan frekvensomvandlinng rakt igenom mottagaren. Nackdelen är att det kan uppstå oönskade självsvängningar på grund av den höga förstärkningen i LF-förstärkaren. Vidare är det obekvämt att ställa infrekvensen om det finns
flera förselektionskretsar. Med ett kristallfilter som är en bättre selekteringskrets kan å
andra sidan mottagning endast ske på en
fast frekvens. Detektormottagare byggs inte
annat än för specialändamål eller i enkla
utföranden för t. ex. radiopejlorientering och
byggsatser.
En utveckling av detektormottagaren är
den direktblandade mottagaren, vilken
ler en uppgift i vissa enklare sammanhang.
Denna mottagartyp är liksom supern avstämbar med en VFO.
selektionen i den direktblandade mottagaren sker, i motsats till detektormottagaren
inte i förkretsen utan i ett LF-filter. En nackdel är fortfarande den oundvikligaspegelfrekvensmottagningen. Vidare kan HF utstrålas från VFO vid ett olämpligt val av blandarprincip. Principen med direktblandning används emellertid som demodu!eringsmetod
t.ex. i SSB-mottagare.

Superheterodynmottagaren är avstämningsbar på ett enkelt sätt med en VFO.
selektionen görs i den fast avstämda MFdelen. Spegelfrekvensdämpning görs med
förselektion i kombination med en lämpligt
vald mellanfrekvens.
En nackdel med en superheterodyn är
att den är mer komplicerad. Vidare kan även
i supern HF utstrålas från VFO om olämplig
blandarprincip väljs.
Men med en dubbelsuper kan spegelfrekvensmottagning lättare undvikas p.g.a.
en hög 1 :a MF samtidigt som en låg 2:a MF
medger en bättre närselektivitet
Fortfarande är risken för oönskade blandningsprodukter stor vid olämpligt valda oscillatorfrekvenser.
Fastän komplexiteten är relativt stor redan i en dubbelsuper så är den ännu större
i en trippelsuper.

Speciella mottagare
Panoramamottagare

Bild II 4-15
l en panoramamottagare visas på en oscilloskopskärm var det finns signaler inom ett
frekvensband. En panoramamottagare är
en superheterodyn. Mottagaroscillatorn är
en VCO (spänningsstyrd oscillator). Dennas frekvens styrs av en sågtandformad
likspänning, som stiger linjärt för att snabbt
falla tillbaka och återupprepas. VCO sveper
då över det önskade frekvensbandet med
ett antal gånger gånger per sekund. Med
samma sågtandspänning avlänkas strålen
på skärmen utmed x-axeln. Bild II 4-16
Den mottagna signalen dernoduleras och
översätts till en likspänning som skildrar de
mottagna signalernas styrka. Med denna
likspänning avlänkas strålen på bildskärmen utmed y-axeln. Strålens avstånd från
x-axeln anger alltså den mottagna stationens styrka och strålens läge utmed x-axeln
anger var stationen ligger i det frekvensområde som avsöks. Beroende på hur stort
frekvenssving som ges VCO, så kommer ett
större eller mindre frekvensområde att
avsökas och visas på skärmen. Området
kan vara så brett som ett amatörband eller
mer och ner till några få kHz.
Utöver övervakning av ett frekvensband
kan en panoramamottagare användas för

114-1 i

M TTA AR
studium t. ex. av signaler och sidefrekvenser
som alstras i den egna stationen. För noggranna mätningar behövs emellertid ett hjälpmedel av högre kvalitet, kallat spektrumanalysator. En sådan arbetar i grunden på
samma sätt som en panoramamottagare.

lil!ll

f

Frekvensspektrum

Bild II 4-16
En panoramamottagare kan anslutas till en
mottagare för att studera signalerna inom
MF-passbandet. Då är mottagningsfrekvensen i bildskärmens mitt. stationerna under
och över i frekvens visas till vänster respektive höger om den egna frekvensen.
Vid ändrad mottagningsfrekvens blir
denna fortfarande kvar mitt på skärmen.

sågtandsformad avstämningsspänning

Bild II 4-17 Signal- och svepspänningar

Bild II 4-15 Panoramamottagare

·-·-·-·-·-·-·-·-·-·-·-·!
[>

!
l
i

i
i

·-·-·-·-·-·-·-·:t.atio~~~tta~re ·-·-·j

9 MHz och
grannfrekvenser

Bild II 4-16 Anslutning av panoramamottagare till stationsmottagare

114-12

E
Mottagningskonvertern

Bild II 4-18
Konverter betyder i detta sammanhang frekvensomvandlare. När det är önskvärt att
flytta över alla signalerna inom ett helt frekvensområde till ett annat, så används en
mottag ningskonverter där frekvensblandning och frekvensfilter används.
Konvertern fungerar som tillsats före en
mottagare för att denna även skall kunna
användas inom ett annat frekvensområde. l
en konverter är oscillatorfrekvensen fast,
medan avsökningen av frekvensområdet
görs med VFO i mottagaren. Mellanfrekvensfiltret i mottagaren är så brett som hela
det frekvensområde som tas emot av konvertern och avsöks med mottagaren.
Exempel: l en KV -mottagare för området
28-30 MHz vill man även kunna lyssna i
området 432-434 MHz (UHF). Den i konvertern mottagna UHF-signalen förstärks för
att sedan blandas med 404 MHz, en frekvens som uppmultiplicerats från en kristalloscillator (CO) i konvertern. De blandningsprodukter som filtreras fram kommer att
ligga inom området 28-30 MHz oc.~ kan
alltså avlyssnas i KV-mottagaren. Ovriga
blandningsprodukter blir undertryckta i KVmottagarens ingångskretsar.
Blandningsfrekvensen 404 MHz i konvertern är beräknad på följande sätt:

Mittfrekvensen i UHF-bandet är
(432 + 434)/2 = 433 MHz = f 1 •
Mittfrekvensen i KV-mottagarens frekvensband är (28 + 30)/2 = 29 MHz.
Med vilken frekvens f2 måste 433 MHz
blandas för att erhålla en blandningsprodukt
av 29 MHz? 29 MHz är mindre än f 1 , alltså
kan endast skillnadsfrekvensen komma i
fråga. (Vid summafrekvens skulle blandningsfrekvensen bli högre än 433 MHz).
Vid användning av skillnadsfrekvensen
ges två möjligheter:
för f2 - f1 = f2 - 433 = 29 MHz är f2 = 462 MHz
för f 1 - f2 = 433 -f2 = 29 MHz är f2 = 404 MHz
Vi bestämmer oss för alternativet 404
MHz av ett speciellt skäl. Här motsvaras den
högsta UH F-frekvensen 434 MHz av 434 404 = 30 MHz och den lägsta UH F-frekvensen 432 MHz av 432-404 = 28 MHz. På så
sätt kan kHz-graderingen på KV-mottagarens skala användas direkt utan omräkning.
Fördelen med en konverter är att kostnaden för en sådan är låg jämfört med den
för en komplett mottagare för ett tilkommande
band. Förutsättningen är att en mottagare
redan finns.
Nackdelen är att mottagaren inte samtidigt kan användas för sin ordinarie funktion.

UH F-antenn

f1 432 - 434 MHz
432 MHz

eller f1 430 L------l

(> t - - - - - -

UH F-försteg

CJ

~

eller

f2 = 404MHz
f2 =
~1Hz

CJ

/I I~

44,889 MHz

4 .•... MHz

Bild II 4-18 Mottagningskonverter UHF till KV

114- 13

M TTA
Transvartern
Bild II 4-19
En transverter (transmitter-converter), är
en kombinerad frekvensomvandlare för både
sändning och mottagning. Den förflyttarbåde
mottagnings- och sändningssignaler mellan två frekvensområden.
Transvertern är ett bra exempel på hur
samma teknik kan användas både i mottagare och sändare. Om t.ex. en KV-transceiver redan finns, kan både mottagning
och sändning ordnas även på andra band
med en transverter som tillsats.

efter kristalloscillatorn CO kan användas för
sändning och mottagning.
Fördelen med en transverter är att kostnaden för en sådan är låg jämfört med den
för en komplett transeeiver även för det
tillkommande bandet. Förutsättningen är att
en transeeiver för något band redan finns.
Nackdelen är att den befintliga transeeivern inte samtidigt kan användas på några
andra frekvenser än de som används för
tillfället.

Exempel
En konverterförflyttar de mottagna UH Fsignalerna till kortvågsområdet Som huvudmottagare används en KV -transceiver i
mottagningsläge. Konvertern kan utökas till
att även fungera vid sändning och kallas då
transverter. Med KV -transceivern i sändningsläge flyttas dess signaler till UH F-området genom blandning i transvertern av
KV-signalen och en multiplicerad signal från
en lokaloscillator (LO). Den önskade blandningsprodukten i UHF-områdetfiltreras fram
och förstärks i efterföljande driv- och slutsteg. Samma frekvensmultipliceringskedja
UH F-antenn

Sändarblandare

s

Bandfilter

Drivsteg

-

f1 28-30 MHz

UH F-försteg

t
T 44,889MHz
D

CO

3

t2=

Bild 114-19 Transverter mellan UHF och KV
114-14

PA

404MHz

ARE

M T
Automatiskt förstärkningsreglering
(AGC) i mottagare
För att mottagaren skall fungera bra för
såväl mycket svaga som för mycket starka
in-signaler behövs en förstärkningsreglering
i signalvägen genom mottagaren. signalspänningen på mottagaringången kan vara
från delar av en mikrovolt upp till över 100
millivolt - ett spänningsförhållande på
1:100000. Det motsvarar mer än nio senheter, vilket är ett mått på signalstyrkan
(Appendix D).

Vid mottagning av en stark signal är det
inte tillräckligt med att bara minska LFförstärkningen. Förstärkarstegen i HF- och
M F-delen blir ändå överstyrda av den starka
insignalen och utsignalen förvrängs om inget ytterligare görs. Det är därför nödvändigt
att minska förstärkningen även i HF- och
MF-förstärkarstegen, ju mer desto starkare
insignalen är. Som hjälpmedel finns oftast
ett reglage för H F-förstärkningen (RF gain),
och därutöver en automatisk förstärkningsreglering- AGC (Automatic Gain Contro l).

AGC

Automatic

Gain

Contro l

Förstärkningsreglering i A3E - mottagare
till de reglerade förstärkarstegen

från sista

R

:n

till MF-förstärkare

fö'"''k"':sJ
lJ

u
LF-signal

överlagrad likspänning
= reglerspänning

Bild II 4-20 AGG vid AM-mottagning med superheterodynmottagare

114-15

M
En mottagare med god reglering kan
arbeta med signalstyrkor mellan mikrovolt
och volt. Beroende hur den mottagna signalen är modulerad (sändningsslaget), sker
AGC på olika sätt.
Både vid AM och SSB finns informationen i sidbanden. HF- och M F-stegen måste
därför arbeta i det linjära området och de får
inte överstyras. Förstärkningen i mottagaren måste alltså regleras med hänsyn till
detta.

AGG vid AM (A3E)
Bild II 4-20
Den likspänning som uppstårvid dernoduleringen av MF-signalen i en AM-mottagare
används till förstärkningsreglering - AGC.
Den LF-spänning som är överlagrad på
likspänningen undertrycks i ett RG-Iågpassfilter. Likspänningen över kondensatorn följer variationerna i den mottagna signalens
styrka med en tidskonstant av ca 0.1 sekunder. Likspänningen blir alltså inte påverkad
av de betydligt snabbare spänningsändringarna som kommer av moduleringen.
En stark bärvågssignal alstrar en hög
likspänning och en svag signal en låg likspänning, oberoende av moduleringen.
Denna likspänning återförs till de framförliggande HF- och MF-förstärkarstegen, vilka
är gjorda så att en hög reglerspänning sänker förstärkningen, medan en låg spänning
tillåter en hög förstärkning.
På så sätt kommer signalstyrkan efter de
reglerade stegen att hållas konstant samtidigt som mottagarens ingång inte överstyrs.

Den likspänning som filtrerats fram från
detektorn kallas reglerspänning eller AGGspänning. Diodens polarisering är inte viktig
för att få ut LF vid demoduleringen, men
däremot för att få rätt polaritet på AGGspänningen. i de flesta mottagare används
negativ AGG-spänning.

AGG vid SSB (J3E)
Bild II 4-21
l de flesta utföranden lämnar produktdetektorn en växelspänning utan överlagrad likspänning. Reglerspänningen alstras därför
genom likriktning av MF-spänningen med
hjälp av en separat demoduleringsdiod eller
genom likriktning av LF-växelspänningen.
Vid SSB alstras det ju ingen MF-spänning undertalpauserna, eftersom ingen bärvåg tas emot då. Tidskonstanten på lågpassfiltret för reglerspänningen måste därför vara längre än vid AM, d.v.s. 0.5 till 2
sekunder. En alltför snabb tillbakagång i
reglerspänningen p.g.a. en för kort tidskonstant skulle leda till mer mottagningsbrus i
tal pauserna. l moderna mottagare finns det
ofta en omkopplare för olika tidskonstanter.
Bild II 4-21
AGG vid CW (A 1A)
Metoden för att alstra AGG-spänning är
samma vid CW och SSB.

AGG vid FM (F3E)
FM-mottagare brukar inte regleras av den
anledningen att det vid FM inte finns någon
information i signalamplituden, utan finns i
stället i frekvensvariationerna i signalen.

.............

AGC
'l> O,Ss

Bild II 4-21 AGG vid SSB- och C W-mottagning med superheterodynmottagare

114-16

Helt avsiktligt läggs därför förstärkningen i
mottagaren så, att en sinussignal blir en
kantvåg p.g.a. överstyrning i förstärkarstegen. Ett eller flera sådana amplitudbegränsande steg, även kallat "limiter", placeras
före demoduleringssteget. Störningar av
amplitudvariationer kommer då att klippas
bort och inte störa mottagningen.
Störande signaler inom nyttabandbredden har dock ingen större inverkan så länge
som den önskade signalens styrka är en
halv s-enhet större än den störande signalens styrka. Likaså försvinner det störande
bruset vid mottagning av en FM-sändare
mycket snabbt över denna signal nivå. Amplitudmodulerade störningar, som t. ex. de från
tändgnistor i förbränningsmotorer, har liten
påverkan vid tillräckligt stark nyttasignaL

Signalstyrkemätare (Smmeter)
AGG-spänningen i en mottag re för AM, CW
och SSB kan även styra en S-meter, som
ger besked om hur stark signalen in i mottagaren är. (Se Appendix D)
Brusspärr
l en FM-mottagare hörs bara brus när det
inte kommer in en tillräckligt stark signal.
Bruset är genomträngande eftersom FMmottagare arbetar med hög förstärkning. En
brusspärr (eng. squelch) är en anordning
som stoppar signalerna till LF-förstärkaren
när signalerna ej uppnår en viss nivå. På så
sätt slipper man att höra på bruset. l mottagare förflera sändningsslag och därför även
AGC kan denna funktion styra brusspärren,
men i en ren FM-mottagare arbetar MFförstärkarna utan AGC. l det fallet behövs
någon annan anordning för att skilja mellan
en modulerad signal och brus. Ofta finns ett
reglage (squelch) för hur stark signal.en
skall vara innan spärren öppnar.

Med selektivitet menas en mottagares förmåga att skilja ut önskade signaler och
undertrycka övriga. Summariskt beskrivet
kallas avståndet mellan yttergränserna för
det önskade frekvensområdet för bandbredd. När det gäller superheterodynmotta.;
gare finns två selektivitetsbegrepp.
Det ena är förselekteringen för att dämpa
de spegelfrekvenser som uppstår i samband med blandning av mottagna signaler
och oscillatorfrekvenser i mottagaren.
Det är selektiviteten i en superheterodynmottagares MF- steg för att utskilja den
önskade signalen efter blandningsförloppen.

Spegelfrekvensproblemet vid mottagning
Bild II 4-22
Exempel:
En sändning på 3600kHz skall tas emot och
VFO-frekvensen är 4055 kHz. Mellanfrekvensfiltret undertrycker sändningar på så
närliggande frekvenser som t. ex. 3603 och
3597kHz. Denna egenskap kallas för närselektion.
Men tyvärr kan en sändning på så avlägsen frekvens som 451 O kHz ändå störa
mottagningen, den goda närselektionen till
trots. Avståndet mellan 451 O kHz och vår
mottagningsfrekvens3600kHz är 91OkHz.
Frekvensen 451 O kHz och VFO-signalen
bildar också en blandningsprodukt, som har
frekvensen 455 kHz. Vid en VFO-frekvens
av 4055 kHz och en mottagningsfrekvens
av 3600kHz benämns451OkHz som spegeifrekvensen. Avståndet mellan spegelfrekvens och mottagningsfrekvens är dubbla
värdet av mellanfrekvensen -i detta exempel 2 ·455kHz= 91OkHz.
Signaler på mottagningsfrekvensen och
spegelfrekvensen alstrar båda blandningsprodukter med VFO-frekvensen, som har
mellanfrekvensens värde. Mellanfrekvensfiltret kan därför inte undertrycka en främmande signal på spegelfrekvensen.
Bild II 4-23
Däremot kan en mottagaringång med
förselektering undertrycka den. En selektiv
krets före blandaren släpper igenom ett smalt
frekvensband med mittfrekvensen 3600kHz,

114-17

M

AR
/pegelfrekvens

u

fsp ::::

närselektion

fM

+

t

2·

/mellanfrekvens

fMF

"- mottagningsfrekvens

l

fvFO

IM

fsp

tfMF ~fMFj

3600
kHz

Mottagning av önskad sändare:

fvFo- fE

:::

Mottagning av ej önskad sändare: fsp - fvFo :::

4055
kHz

4510
kHz

4055 ---3600kHz = 455 kHz
4510 --4055kHz = 455kHz

Bild II 4-22 Enkelsuper med låg MF och ingen förselektion

u

MF =455kHz

Bild II 4-23 Enkelsuper med låg MF och med förselektion
men dämpar t. ex. frekvensen 451 O
kHz p.g.a. den stora frekvensskillnaden. En förselektion har alltså tillförts
som komplement till den närselektion
som erhålls med mellanfrekvensfiltret
Bild 114-24
Ju längre ifrån varandra nyttofrekvens och spegelfrekvens ligger,
desto bättre är förselektionen. Med en
mellanfrekvens av 455 kHz är alltså
detta avstånd 91 O kHz. l långvågsoch mellanvågsområdet är det tillräckligt för att man med enkla medel skall
kunna skapa tillräckligt selektiva filter.
Exempel:
Vid den högsta mottagningsfrekvensen på mellanvåg 1605 kHz är
spegelfrekvensen 2515 kHz, som ligger 1.57 gånger högre i frekvens och
med ett avstånd av 91OkHz. l kortvågsområdet dämpas inte en spegelfrekvens på avståndet91OkHz tillräckligt
kraftigt. Vid den högsta mottagnings-

114- 18

u

MF =900kHz

Bild II 4-24 Enkelsuper med hög MF och
med förselektion
frekvensen på kortvåg 30 MHz ligger nämligen
spegelfrekvensen 30.91 OMHz endast 1.03 gånger
högre i frekvens. Med antagandet, att förselektionskretsen har ett Q-värde av 30, blir bandbredden 53.5 kHz vid frekvensen 1605 kHz.
Med samma Q-värde blir bandbredden 1000
kHz vid frekvensen 30 MHz, vilket innebär att
förkretsen inte längre kan dämpa så närliggande
spegelfrekvenser på ett effektivt sätt.

ARE
l mottagare för högre frekvenser används
därför högre mellanfrekvens för att öka avståndet till spegelfrekvensen. l moderna
kortvågsmottagare är det vanligt med en
mellanfrekvens av 9 MHz eller högre. Vid en
mottagningsfrekvens av 30 MHz och en
mellanfrekvens av 9 MHz är spegelfrekvensen 48 MHz, vilket är 1.6 gånger mottagningsfrekvensen. Detta möjliggör förselektionsfilter med tillräcklig dämpning av spegelfrekvensen.
Bild II 4-25
Bilden visar hur när- och förselektion kompletterar varandra i ett frekvensspektrum.
Märk, att passbandbredden b i förselektionskretsen anger avståndet mellan de frekvenser där signalamplituden dämpats till 70
0
/o av toppvärdet. l exemplet här ovan har
antagits att förkretsen för kortvågsmottagning har samma Q-värde som förkretsen för
mellanvågsmottagning.

Vid högre frekvenser, i VHF- och UHFområdet, kan inte önskat Q-värde erhållas i
sådana kretsar som användsiKV-området
och lägre. Andra lösningar blir då nödvändiga, t.ex. kavitetsfilter och helixfilter.
MF-bandbredd vid AM (A3E)
Bild II 4-26
En amplitudmodulerad signals frekvensspektrum består av bärvågen och två sidfrekvenser - eller sidband om sidfrekvenserna är många.
Bandbredden i MF-kretsarnamåstevara
minst så stor att sidfrekvenserna längst bort
från bärvågen kan passera. Dessa frekvenser motsvarar de högsta modulerande tonerna. Vid rundradiosändningar på mellanvåg utsänds alla frekvenser upp till 4.5 kHz.
Detta motsvarar en bandbredd av 9 kHz.
För enbart talöverföring är en bandbredd av
6 kHz tillräcklig, vilket motsvarar en LFgränsfrekvens av 3 kHz.

u
HF

FÖRSELEKTERING:

Undertrycker spegelfrekvenser

önskad sändare
flera andra
annan sändare på
spegelfrekvensen

0.7

~~~~~~~ww~~~~~----f

l fM :
r--b---G<>j

u

MF-fittrets passband

1r

MF

NÄRSELEkTERING:

Undertrycker angränsande sändare

Bild II 4-25 Samtidig för- och närselektion i superheterodynmottagare

114- 19

M
Ett för smalt MF-filter skär bort de yttre
delarna av sidbanden. LF-signalerna kommer då att förlora de höga tonerna (diskanten). Om däremot filtret är för brett, kommer
närliggande utsändningar också att höras.
l vissa mottagare kan MF-bandbredden
anpassas till förhållandena. Det är alltså en
fråga om en kompromiss mellan bättre ljudkvalitet och mindre störd mottagning.

u

MF-bandbredd vid SSB (J3E)
Bild 114-27
Mellanfrekvensfiltret för SSB-mottagning
skall endast släppa igenom ett av de två
sidbanden, vars bredd är skillnaden mellan
högsta och lägsta överförda LF-frekvens.
Inom amatörradio är detta3kHz- 0.3 kHz =
2.7 kHz, alltså något mindre än hälften av
bandbredden vid AM.

MF

LF

u
fr

fLF

=

(fr t fLF ) -

fLF

E

fr -

fr och

( fr - f L F)

fL F
~r--~--~--~-------

8999
kHz

9 000

f

~~----------------~f

kHz

fL F

= 9001 -·· 9000kHz = 1 kHz

fL F

= 9000

····8999kHz

= 1 kHz

A3E- demodulering i frekvensspaktrat

u

u

MF
M F -filterkurva

MF

inställd sändare
med bärvåg och sidband

riktig MF-bandbredd

M F-bandbredden för smal, delar av
sidband bortklippta. Diskanten borta

b = 2 · fLFmax
b = 2 · 3 kHz = 6 kHz

u

MF

u

MF

Störning från angränsande sändare.
Undertryckning genom smalt MFfilter på bekostnad av diskant

För stor M F-bandbredd, alltför
flack filterkurva, störningar från
angränsande kanaler

MF-bandbredd vid A3E

Bild II 4-26 MF-bandbredd vid AM (A3E)
114-20

EPT

~©~

M TTA ARE

Ett alltför brett MF-filter skulle också
släppa igenom oönskaqe signaler från angränsande frekvenser. A andra sidan skulle
ett för smalt M F-filter skära bort signaler i det
önskade frekvensregistet och försvåra mottagningen. Smala filter kan å andra sidan
utnyttjas för att dämpa signaler, t. ex. från en
för nära liggande sändare eller som har för
stor bandbredd.

U

MF

När närliggande sändare stör mottagningen
ges följande möjligheter:
Att göra en liten snedavstämning, uppåt
eller nedåt i frekvens. Därigenom ändras frekvensläget på det mottagna talet, men vid små frekvensawikelser blir
förvrängningen liten. Läsligheten blir
sämre, men mottagningen på det hela
taget bättre.

U

MF-filterkurva
l

rätt MF-bandbredd
b = fLFmax - fLFmin
b = 3 --·- 0,3 kHz = 2,7 kHz

u

För stor MF-bandbredd, alltför
flack filterkurva, störningar från
angränsande kanaler

u

Minskning av grannstörning genom :
snedavstämning

Störning från grannkanalssändare

U

MF

u

den undertryckta
bärvågen

\l
l

l

normal BFO-frekvens

~~--~~==~~-----f

MF-skift

Avklippning av de lägre frekvenserna
genom förskjutning av 2:a MF-filterkurvan (passbands-tuning)

Bild fl 4-27 MF-bandbredd och passband-tuning vid SSB (J3E)

114-21

•

•

Vidare måste VFO, 1 :a BFO och 2:a BFO
kunna ställas in var för sig. Frekvensläget
på MF l och/eller MF Il kan då förskjutas
över respektive filters passband, oberoende av varandra. Därigenom uppstår
skenbart effekten att filterkurvorna skjuts
emot varandra. Samma effekt skulle fås
om kristallfiltren gick att avstämma, vilket
ju inte är möjligt.

M F-skift. Som just beskrivits kan en liten
snedavstämning göras. l vissa mottagare
är det ordnat så att också BFO-frekvensen kan förskjutas så att frekvensläget på
talet blir återställt igen. Därmed blir MFpassbandet skenbartbart förflyttat uppåt
eller nedåt i frekvens (M F-skift, IF-shift).
Det verkliga frekvensläget mellan nytto-:
signal och BFO behålls. l alla händelser
blirbasen ellerdiskanten på nyttasignalen
avskuren, beroende på var denna ligger
i frekvens.
Passband-tuning. Om det finns störande
sändare både över och under i frekvens,
går det inte att skära bort störningarna
med ett enkelt M F-skift, eftersom antingen den ena eller den andra störande
sändaren ändåskulle höras. Fördetfallet
erbjuder några moderna mottagare möjligheten att flytta MF-passbandets övre
och undre frekvensgräns oberoende av
varandra (bandpass tuning m.m.). Detta
förutsätter, att mottagaren är en trippelsuper med branta filter i varje MF-steg.

u

vid CW (A tA)
Bild II 4-28
En
har som bekant inte bandbredden non Hz, utan det handlar i grunden
om en amplitudmodulerad signal. Vid en
nycklingshastighet av 60 tecken per minut
är bandbredden c:a i 00 Hz och vid i 20
per minut den dubbla, c:a 200 Hz.
l vissa mottagare används ett SSB-filter
även för mottagning av CW. En vanlig bandbredd på ett SSB-filter är 2.7 kHz och då
kommer även stationer på närliggande frekvenser att höras. Låt vara att de flesta av
dessa stationer hörs med olika frekvens.

u

MF

LF

BFO
CW-Signa!

8

U

~~~

2

9

"-"'------f
~~~ b ca 100-200Hz

MF

""---------f

800Hz

U

MF
BFO .

f
CW-mottagning med SSB-filter

Bild II 4-28 Olika MF-bandbreder vid CW (A

114-22

CW-mottagning med 250 Hz CW-filter

PT
Fler än 20 CW-stationer får plats inom en
bandbredd motsvarande en SSB-kanal. Den
mänskliga hjärnan, kan med någon övning
koncentrera sig på en av dessa signaler
medan övriga uppfattas som störande.
Det tidigare nämnda LF-bandpassfiltret
skulle emellertid åstadkomma en bättre selektion och bekvämare avlyssning. Men om
en annan station inom passbandet är mycket
starkare än den station som är av intresse,
då blir MF-förstärkaren antingen överstyrd
av den starkare signalen eller AGC reglerar
ner förstärkningen så att den svagare signalen inte längre kan höras trots det smala LFfiltret. selektionen i en mottagare bör därför
sitta "så långt fram som möjligt". l det skildrade exemplet skulle ett smalt filter i MF vara
till bättre nytta vid CW-mottagning. Bandbredden på ett sådant filter är 250- 500 Hz,
således endast något bredare än CW-signalen.
Medettännu smalare CW-filterkan, p.g.a.
bristande frekvensstabilitet hos sändare och/
eller mottagare, svårigheter uppstå att finna
den önskade signalen. Välutrustade mottagare har passband-tuning även för CW,
steglös bandbreddsreglering eller stegvis
valbara filterbandbredder. Då kan mottagaren ställas in på den önskade signalen med
en stor bandbredd som därefter minskas.
För mottagning av RTTY (radiofjärrskrift)
med 170 Hz skift mellan de två frekvenserna, kan ett 500 Hz-filter användas. Smalare filter går däremot inte så bra.

Bandbredd vid FM (F3E)
En FM-sändare med frekvensdeviationen
~~ax och högsta modulerande LF-moduleringsfrekvensen fLFmax har bandbredden

b= 2( ~~ax + (Fmax) •

Inom amatörradio är det brukligt med en
maximal deviation av 3 kHz och en övre
gränsfrekvens av 3kHz, vilket motsvarar en
bandbredd av 12 kHz.
Fullgod mottagning är möjlig endast om
M F-filtren i mottagaren har minst den bandbredd, som sändaren har. Men vid för stor
mottagarbandbredd kan även stationer på
närliggande frekvenser uppfattas. Sedan
1996 är det av IARU Region 1 rekommenderade kanalavståndet 12.5 kHz vid FM-trafik
på VHF- och UHF-amatörradiobanden.

TTAGARE
Det är vanligare med för stor deviation på
FM-sändaren än att mottagaren är alltför
smaL En för stor deviation, avsaknad av
deviationsbegränsare och för hög LF-gränsfrekvens medför en onödigt stor bandbredd
på sändaren. Motstationen får då mottagningssvårigheter och stationer på angränsande kanaler blir också störda.
Det blir allt vanligare med 12.5 kHz kanalavstånd även för repeatrar, varför det är
viktigt att alla sändare är rätt inställda.

signalkänslighet och brus

Om man ställer in mottagaren på en ledig
frekvens, så hör man vid full förstärkning ett
brus likt det från ett vattenfall.
Bruset kommer från de svaga växelspänningar som uppstår när laddningsbärarna
rör sig genom de material som strömkretsen
består av. Beroende av bruskällan sträcker
sig frekvensspektrum från noll till nära nog
oändligt. På grund av egenskaperna skiljer
man mellan en rad specifika bruskällor:
• Resistorbrus, även kallat "vitt brus", som
uppstår i resistiva komponenter. Bruset
sträcker sig över hela det mätbara frekvensområdet varvid energifördelningen
är lika över hela området,
• Kretsbrus, som uppstår i resistanser i
svängningskretsar i resonans,
• Antennbrus, som är sammansatt av
bruset från antennens strålnings- och
förlustrasistanser samt av det galaktiska
brus som antennen tagit emot,
• Transistorbrus uppstår av laddningsbärarnas rörelser i halvledarmateriaL
Det bildas en sammanlagd brusspänning
som kan bestämmas. Man talar om ett brustal, som är ett mått på mottagningssystemets
egenbrus. Detta skall ställas mot styrkan på
den mottagna signalen. Man talar om ett
förhållande mellan signaleffekt och bruseffekt. Det finns flera metoder att mäta och
uttrycka detta förhållande som kallas S/N
(signal to noise ratio). För att uppfatta den
information som kommer ur en mottagares
LF-utgång måste nyttasignalen vara ett antal gånger starkare än bruset. Den lägre
gränsen för att uppfatta tal i kortvågsmottagare är ett brusavstånd i storleksordningen 1O dB.

114-23

M

ARE

l en broschyr på en kortvågsmottagare
kan man t. ex. läsa "Sensitivity SSB, CW:
lessthan 0.25 J! V for 1O dB SIN ... "
Termen S/N betyder Signai/Noise, d.v.s.
styrkeförhållandet signal/brus uttryckt i dB.
Det innebär att en signal kan läsas vid 25J.tV
signalnivå och ett S/N av mindre än 1O dB.
Utöver brusnivån i mottagaren spelar också
distorsionen en roll.

S+N+D
N
[dB]

Signalbrusförhållande
där S=Signalnivå
N=Brusnivå
D=Distorsionsnivå

UlllllJIDllll~N
3

f kHz

Spektrum

Bild II 4-29 SIN-värde

l en broschyr på en VHF-mottagare kan
man t. ex. läsa "Sensitivity FM: Lessthan
0.18 J.tV for 12 dB SINAD ... "
Termen SINAD betyder Signal, Noise
and Distorsion. Vid denna definition tar man
även hänsyn till distorsionsprodukter som
orsakas av den modulerande signalen.
SINAD= S+N+D
N+ D

[dB]

Nivå
(dB)

Bild 114-30 SINAD-värde

114-24

lntermodulation, korsmodulation

Utöver att en bra modern mottagare bör ha
tillräcklig frekvensstabilitet, känslighet och
selektivitet bör den även ha goda s. k. storsignalegenskaper.
Med storsignalegenskaper menar man
hur bra en relativt svag nyttasignal på mottagaringången motstår påverkan av starka
frekvensnära signaler med hög fältstyrka.
Störningar av detta slag uppstår genom icke
linjära förlopp i komponenter i mottagarens
ingångssteg, varvid mottagna signaler med
stor amplitud blir förvrängda.
Korsmodulation och intermodulation är
två begrepp som är förknippade med storsignalegenskaperna. Båda kan visserligen
definieras och bestämmas entydigt, men de
förväxlas ändå ofta.

Korsmodulation
Med korsmodulation menas, att den inkommande nyttasignalen amplitudmoduleras
med modulationsprodukter från en annan
frekvensnära amplitudmodulerad signal,
varvid korsmodulationen uppstår i olinjära
komponenter i mottagaringången (försteg,
blandare). När man med mottagaren i AMläge ställt in den på någon bärvåg så hörs
också andra starka, frekvensnära stationer.
Det måste alltså alltid finnas en nyttasignal på den inställda frekvensen för att det
skall uppstå korsmodulation. När nyttasignalen försvinner så försvinner även korsmodulationen.
lntermodulation
Vid s.k. intermodulation blandas två starka
inkommande signaler i olinjära komponenter
i mottagaringången. Deras blandningsprodukter faller på mottagningsfrekvensen
så att den störs, vare sig det finns en nyttosignal där eller inte.

Frekvensstab i l it et
Se kapitel 4, Oscillatorer


