\emph{Endast en del av frekvensspektrum av elektromagnetiska vågor
  används för radiosändningar. Samtidigt används detta utrymme av allt
  fler intressenter och för allt fler ändamål.  Samhället blir alltmer
  tekniskt avancerat och elektroniktätheten tilltar kraftigt. Den
  ökande mängden och komplexiteten hos apparaterna kräver därför
  regler, som styr både utförande och användning med rimligt
  bibehållen säkerhet och funktion.}

\section{Störningar och störkänslighet}

\subsection{Om EMC-lagen}

Det kan inte längre ges enkla svar på vad som är att vara störd och
att störa. Internationella och nationella väl preciserade regler för
radio- och teletekniskt samexistens är numera helt nödvändiga.

Samlingsbegreppet är Electromagnetic Compatibility (EMC), d.v.s. en
apparats förmåga att fungera tillfredsställande i sin
elektromagnetiska omgivning utan att alstra elektromagnetiska
störningar som överstiger en nivå, som tillåter radio- och
teleutrustning och andra apparater att fungera som avsett.  Vidare
skall apparater ha en sådan tillräcklig inbyggd tålighet mot
elektromagnetiska störningar, att de kan fungera som avsett.

Till skydd för liv, personlig säkerhet och hälsa samt kommunikationer
och näringsverksamheter har därför Lag om elektromagnetisk
kompatibilitet införts. Denna lag är anpassad efter av EG/EES
utfärdade direktiv angående bl.a. radiostörningar.

Förordning om elektromagnetisk kompatibilitet definierar
nyckelbegreppen; apparater, EMC, elektromagnetiskstörning och
tålighet. Elsäkerhetsverket är ansvarig myndighet, med rätt att
utfärda föreskrifter om bl.a. skyddskraven, kontroll och märkning samt
om vissa undantag. Sådana föreskrifter är bl.a. ELSÄK-FS.

\emph{Post och telestyrelsens föreskrifter om innehav och användning
  av amatörradioanläggningar m.m.} hänvisar till den överordnade
\emph{Lag om radiokommunikation} där följande finns om Åtgärder mot
störningar:
% TODO Detta stämmer inte längre - PTSFS undantag från tillståndsplikt
% och Lagen om elektronisk kommunikation

\subsection{Ur radiolagen}

\begin{quote}
\emph{Om en radiosändare stör avändningen av en annan radioanläggning
  skall den som har tillstånd att inneha och använda radiosändaren
  ombesörja att störningen upphör eller i görligaste mån
  minskas. Motsvarande skyldighet gäller för innehavare av
  radiomottagare som stör användningen av en annan radiomottagare.
  Den myndighet som regeringen bestämmerfårförelägga den som
  enligtförstastycket är skyldig att vidta åtgärder mot en störning
  att fullgöra denna skyldighet. Ett sådant föreläggande får förenas
  med vite.}
\end{quote}
\begin{quote}
\emph{Elektriska eller elektroniska anläggningar som, utan att vara
  radioanläggningar, är avsedda att alstra radiofrekvent energi för
  kommunikationsändamål eller industriellt, vetenskapligt, medicinskt
  eller något liknande ändamål, får användas endast i enlighet med
  föreskrifter som meddelas av regeringen eller den myndighet som
  regeringen bestämmer.  Den myndighet som regeringen bestämmer får
  meddela de förelägganden och förbud som behövs i ett enskilt fall
  för att föreskrifterna i första stycket skall följas. Ett sådant
  föreläggande får förenas med vite.  Regeringen får meddela
  föreskrifter om förbud mot att inneha elektriska eller elektroniska
  anläggningar som inte omfattas av första stycket och som, utan att
  vara radioanläggningar, är avsedda att sända radiovågor.}
\end{quote}

I radiolagen definieras bl.a. radioanläggningsom en anordning för
radiokommunikation eller radiobestämning genom sändning (radiosändare)
eller mottagning (radiomottagare) av radiovågor.

\subsection{Utstrålning från amatörradiosändare}

\emph{Vad som sägs i radiolagen innebär att sändareffekten alltid
  skall anpassas så att styrkan av utstrålade fält inte förorsakar
  störningar. Den enligt amatörradioföreskrifterna högsta tillåtna
  effekten kan alltså inte användas hinderslöst. I samma paragrafstår
  också att PTS i tillståndet kan besluta om andra effektgränser, om
  det finns särskilda skäl.}

\emph{Om störningarna inte kan avhjälpas kan PTS komma att anvisa om
  restriktioner (begränsningar i sändningstillståndet), det kan vara
  sändningsförbud under vissa tider, på vissa frekvenser, över viss
  sändareffekt etc.}

\subsection{PM vid störningsproblem}
\begin{itemize}
\item Störningar är alltid förenade med obehag och ställer grannsämjan
  på prov. Håll Dig väl med dem som bor i omgivningen!
\item Om det väcks klagomål på Dig om störningar, skall Du först
  kontrollera Din egen sändare och antennanläggning.
\item Be därefter att få undersöka antennanläggning och apparater hos
  den som besväras av störningar.
\item Om Du ser en lösning, berätta om vad som kan göras. Kom överens
  om vad som får göras. Ändra då inte något inne i apparater, men
  prova gärna ut yttre, kompletterande filter etc.
\item Om det inte går att komma till rätta med störningarna bör de som
  levererat och installerat anläggningen anlitas.
\item Störningsanmälan kan även ske till PTS närmaste tillsynsområde.
\end{itemize}

\subsection{Arbeta aktivt med avstörning}
\begin{itemize}
\item Låna hem en av SSA:s avstörningslådor och försök att finna en
  lösning. I lådan finns ett sortiment av frekvensfilter för
  avstörning,
\item Undvik att störa i onödan. Sänk sändareffekten och begränsa
  sändningstiden under utprovningen av en lösning.
\end{itemize}

Lyckas Du inte själv med att störa av
\begin{itemize}
\item Ta gärna hjälp av en radioamatör med erfarenhet av avstörning
  eller
\item Anlita annan sakkunnig hjälp.
\end{itemize}
