\chapter{BESKRIVNINGSKOD FÖR SÄNDNINGSSLAG}

Ett resultat av World Administrative Radio
Conferece (WARC) som avhölls år 1979 var
en ny beskrivningskod för radiosändningar.
Före WARC 79 uttrycktes den nödvändiga bandbredden i termer av kHz. Till exempel användes då 0.1 för 1 00 Hz och 6000 för
6 MHz.
l dataåldern behövs ett elegantare system för taluppställningar. Det nya systemet
går ut på att uttrycka den nödvändiga bandbredden med tre siffror och en bokstav.
Bokstaven placeras på platsen för decimaltecknet och representerar enheten för bandbredd. Bokstäverna H (Hz), K (kHz), M (MHz)
och G (GHz) används, medan varken Oeller
K, M eller G får vara det första tecknet.
Numeriska värden med mer än tre signifikanta siffror rundas av till tre.
Detta benämningssystem är dock inte
utan problem. Det tar mer hänsyn till metoden hur en signal alstras, snarare än hur en
signal helt enkelt ser ut när den sänds.
Följdaktligen skall direkt modulation av
huvudbärvågen benämnas på ett sätt, medan
modulation av en underbärvåg i en enkelt
sidbandssändare med undertryckt bärvåg
skall benämnas på ett annat sätt. Om man
t. ex. nycklarett RTTY-modem med en direktskrivande fjärrskrivare och sedan bytertill en
dator för att göra samma sak, så ändras
benämningen av sändningsslaget
Bandbredd
Basbandetärdetfrekvensområde, som upptas av signaler innan de modulerar bärvågen. Signaler i basbandet ligger vanligen
mycket lägre i frekvens än bärvågen. l den
låga änden av basbandet kan frekvensen
närma sig eller vara likström (O Hz). l den
höga änden beror frekvensen på det värde
då information finns liksom att det finns underbärvågor eller andra speciella signaler
inom basbandet Det finns ett basband för
alla typer av signaler, vare sig de är analoga
eller digitala. Det skall också förstås, att
termen basband är relaterad till den modulation som avses från fall till fall.
Det kan finnas mer än ett basband i en
komplett modulationsprocess. Till exempel,
en nycklad ton som går till sändaren genom

mikrofoningången är dess analoga basband
medan nycklingspulserna till tongeneratorn
är dess digitala basband.
Sidband alstras alltid när en bärvåg moduleras. De är blandningsprodukter på båda
sidor om bärvågen, som resultat av att signaler från basbandet modulerar bärvågen
på något sätt. Det övre sidbandetkallas USB
(av upper sideband, eng.) och det undre
sidbandet LSB (av lower sideband, eng.).
l system för amplitudmodulation är bredden på sidbanden i stort lika med den högsta
frekvenskomposanten i basbandet Sidbanden är spegelbilder av varandra och innehåller exakt samma information. För att spara
bandbredd räcker det alltså med att överföra
ena sidbandet, varvid det andra sidbandet
kan undertryckas, liksom även bärvågen.
l andra modulationssystem än för amplitudmodulation däremot, kan bredden på sidbanden mycket överstiga den högsta frekvenskomposanten i basbandssignalen.
Använd bandbredd är avståndet mellan
översta och nedersta delen av ett spektrum,
där medeleffekten är lägre än 0.5°/o (-23 dB)
av den totala medeleffekten. l vissa fall kan
en annan relativ effektnivå specificeras; t ex.
i USA 0.25°/o (-26 dB) för reglering av digital
kommunikation. För amatörer är det inte
alltid lätt att bestämma den använda bandbredden. Den kan mätas med en spektrumanalysator, men ett sådant instrument är
svårtillgänglig för de flesta amatörer. Den
använda bandbredden kan även beräknas,
men det kräver matematikkunskaper i informationsteori och behandlas inte här.
Nödvändig bandbredd är den del av den
använda bandbredden, som räcker för att
säkra informationsöverföringen i den omfattning och kvalitet som krävs. Förenklade
sätt att beräkna nödvändig bandbredd vid
specifika modulationssystem finns i kapitel1.
Tilldelat frekvensband är den nödvändiga bandbredden plus två gånger den absoluta frekvenstoleransen.
Frekvenstolerans (uttryckt i del per 106 ,
procent eller i Hz) är den maximalt tillåtna
frekvensavvikelsen från den korrekta frekvensen.

E-1

(Exempel)

6M25
C3F MN
---.-- ~~~

m
l

l

1\ )

Bandbredd - - - - - - . . : .•

Huvudbärvågens modulation
N Ingen modulation
Utsändning där huvudbärvågen är linjärt modulerad
(även i fall med vinkelmodulerad underbärvåg)
A Dubbla sidband
H Enkelt sidband, full bärvåg
R Enkelt sidband, reducerad bärvåg eller bärvåg av varierande nivå
J Enkelt sidband, undertryckt bärvåg
B Sinsemellan oavhängiga sidband
c stympat sidband
Utsändning där huvudbärvågen är vinkelmodulerad
F Frekvensmodulation
G Fasmodulation
D Utsändning vars huvudbärvåg är amplitudoch vinkelmodulerad antingen samtidigt eller
i viss förutbestämd följd.
Utsändning av huvudbärvågen som tåg av pulser not.
P Omodulerade pulser
K Amplitudmodulerade pulser
L Längdmodulerade pulser
M Faslägesmodulerade pulser
Q Vinkelmodulerad bärvåg under pulsens varaktighet
V Kombination av ovanstående eller alstrat på
annatsätt
Övriga fall där utsändningens huvudbärvåg är modulerad, antingen samtidigt eller i förutbestämd följd på
två eller flera av sätten amplitud-, vinkel- eller pulsmodulering
W
övriga fall

x

not. Utsändning där huvudbärvågen är direkt

modulerad av en signal, vilken är kodad
i kvantisarad form (t.ex. pulskodmodulation) skall hänföras till amplitudeller vinkelmodulation

1

Den modulerande signalens karaktär

0 Ingen modulerande signal
En enda kanal med
1 kvantiserad eller digital information, utan användning av modulerande underbärvåg
2 kvantiserad eller digital information, med användning av modulerande underbärvåg
3 analog information
Två eller flera kanaler med
7 kvantiserad eller digital information
8 analog information
Sammansatta system av
9 en eller flera kanaler med kvantiserad ellerdigital information samt
en eller flera kanaler med analog
information
Övriga fall

x

l fråga om bassignalens karaktär skiljer
man å ena sidan på kanaler för kvantiserad eller digital information, d.v.s. där signalen växlar språngvis mellan vissa givna tillstånd, och på kanaler för analog information, där signalen kan variera kontinuerligt inom givna gränser.
Att fastställa arten av huvudbärvågens modulation kan kräva viss eftertanke. l många fall
får den information som skall överföras, modulera en underbärvåg, som i sin tur påtrycks
modulatorn för huvudbärvågen.

M l' l

f''rf
d
u t1p ex o aran e
Närmare signalbeskrivning

'

Informationens form

N Ingen överförd information
A Telegrafi
för hörselmottagning
B Telegrafi
för automatisk mottagning
C Faksimil
D Dataöverföring,
fjärrmätning,
fjärrstyrning
E Telefoni,
även ljudrundradio
F Television, video
W Kombination av
ovanstående fall
X Övriga fall

m
m

00

"<z
Jl

z

G)

:t>
"'C
"'C

m
c

z

x

00

"oc
00

)>:

z

c

z
z
G)
00
00

r

)>

G)

Telegrafisignaler är kvantiserade (till/från,
mark!space). Telefonisignaler har mestadels
varit analoga, men är allt oftare kvantiserade
(digitala). Faksimilsignaler är analoga eller
kvantiserade, beroende på om gråtoner överförs eller ej.

Obligatoriska kännetecken
för sändningsslag enligt
ITU radioreglemente (RR)

~

©

-a

-1

(Exempel)

-. l

6M25
C3F MN
---.-.....
Utsändningens bandbredd
En fullständig kodbeteckning för en radioutsändning är uppbyggd av en niostäflig teckenföljd, t. ex.
6M25 C3F MN. l teckenföljden är de första fyra
tecknen (t.ex. 6M25) bandbreddsangivelsen.
Bandbredden ska/f anges med tresiffror samt en
bokstav som decimaltecken.
Bokstaven anger även vilken enhet som bandbredden har, nämligen H för Hz, K för kHz, M för
MHz och G för GHz.
Det första tecknet får inte vara noll, K, M eller G.
Decimaltecknen används på följande sätt:
Bandbredd 0.001-999 Hz (decimaltecken H),
bandbredd 1.00-999 kHz (decimaltecken K},
bandbredd 1.00-999 MHz (decimaftecken M},
bandbredd 1.00-999 GHz (decima!tecken G).
Exempel:
O. 002 Hz skrivs H002
12.5 kHz skrivs 12K5,
O. 1 Hz
skrivs H 100 2.4kHz
skrivs 2K40,
25.3 Hz skrivs 25H3 6 kHz
skrivs 6KOO,
180. 7kHz
skrivs 181K
6.25 MHz
skrivs
6M25.

m
l

w

Det är särskift viktigt att komma ihåg bandbredden vid utsändningar nära bandgränsema. T. ex.
kommer sidbandet (USB) i en telefonisignal med
bärvågsfrekvensen 29.699, att tydligt överskrida
den övre bandgränsen för 10-metersbandet. Bandgränserna får INTE överskridas och det gäller även
sidbanden f
A v Post- och telestyrelsens föreskrifter för amatörradio framgår de tillåtna bandbreddema. Därutöver gäller att bandbredden bör hållas så smal som
möjligt med hänsyn till sändningsslaget

Obligatorisk del

Närmare beskrivning av signalen

m

m

":c<
00

Arten av mulliplex

Tvåtillståndskod med element av
N Ingen multip/ex
A Olika antal och/eller olika varaktigMultiplex med
z
het- morsetelegrafi
C Koddelning
B Samma antal och varaktighet, utan
F Frekvensdelning
C)
felkorrigering - fjärrskrift
T Tiddelning
00
C Samma antal och varaktighet, med
W Kombination av F och T
felkorrigering- fjärrskrift, AMTOR,
X Andra arter av multip/ex
paketradio m.m.
Fyratillståndskod där
D Varje tillstånd företräder ett tillstånd
Se föregående sida om
om ett antal bitar
den obligatoriska delen
00
):>:
Flertillståndskod där
av kännetecknen!
E Varje tillstånd företräder ett signalelement om ett antal bitar
F Varje tillstånd eller kombination av
tillstånd företräder ett tecken
C)
Ljud av rundradiokvalitet
00
00
G Monatoniskt
r
H stereofoniskt eller kvadrafoniskt
Ljud av kommersiell kvalitet
J Alla fall utom K och L enligt nedan
K Med användning av frekvensinversion eller banduppdelning
L Medsärskilda frekvensmodulerade signaler för styrning av den demodulerade signalens nivå
Video
M Monokrom
N Färg
Kompletterande kännetecken
Kombination av ovanstående fall

z

~

©

~

o

"oc
z
z
z
c

>

"

w
Övriga fall
x

för sändningsslag enligt
ITU radioreglemente (RR)

)'>
'1J
'1J

m
c

z

x

BESKRIVNINGSKOD

SÄNDNINGSSLAG

Exempel på fullständigt beskrivna sändningsslag
N0N
1OOH A 1A AN

Omodulerad bärvåg, ingen överförd information.
Morsetelegrafi genom nyekling av bärvåg, 125-takt, bandbredd 100 Hz, s.k. CW.
16KO F2A AN Morsetelegrafi, frekvensmodulation med nyekling av ton,
t.ex. i repeater, s.k. tontelegrafi.
254H F1 B BN Fjärrskrift genom frekvensskiftnyckling av bärvåg (FSK),
utan felkorrigering, hastighet 50 Bd, frekvensskift i 70Hz,
s.k. RTTY.
254H J2B BN Fjärrskrift genom frekvensskiftnyckling av modulerande
tonpar (AFSK), vid sändning av enkelt sidband med undertryckt bärvåg, bandbredden beroende av hastighet och
frekvenser i tonparet
Jfr 254H Fi B BN,
304H Fi B CN Fjärrskrift genom frekvensskiftnyckling av bärvåg (FSK),
med felkorrigering, hastighet 100 Bd, frekvensskift 170
Hz, t.ex. AMTOR. Jfr 254 Fi B BN.
6KOO A3E JN Telefoni, amplitudmodulation med dubblasidband och full
bärvåg, bandbredd 6kHz, s.k. AM.
2K70 J3E JN Telefoni, enkelt sidband och undertryckt bärvåg, bandbredd 2.7 kHz, s.k. SSB.
16KO F3E JN
Telefoni, frekvensmodulation, bandbredd 16 kHz, s.k.
NBFM (smalbands-FM).
2K12 F3C MN Faksimil med halvtoner (telefoto), kooperationsindex 264,
avsökningshastighet 90 linjer/minut, frekvensmodulering
med \(\pm\) 400 Hz skift.
6M25 C3F MN Television i svartvitt enligt det europeiska 625-linjerssystemet.
3KOO F3F MN Smalbandstelevision enligt amatörradiostandard, s.k. ATV.

Exempel på sändningsslag utan ITU beskrivningskod enligt ovan
Telefoni, amplitudmodulation med dubbla sidband och
reducerad bärvåg.
En enda kanal med analog information.
Sändningsslaget tillämpas i effektbesparande syfte bl.a.
för rundradiosändningar på "AM", varvid traditionella
rundradiomottagare fortfarande kan användas.

E-4
