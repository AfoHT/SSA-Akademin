\section{Modulation}
\textbf{HAREC a.\ref{HAREC.a.1.8}\label{myHAREC.a.1.8}}

\subsection{Allmänt}

Modulera (lat. modulari, rytmiskt avmäta) är att med hjälp av en oftast högfrekvent
elektrisk signal (bärvågen) överföra informationen i en lågfrekvent signal. På så sätt kan
lågfrekvens, t.ex. tal och musik, först omvandlas till en elektrisk signal, som får 
påverka (modulera) en högfrekvent elektrisk signal. Denna modulerade signal strålas ut från
antennen som ett elektromagnetiskt fält.

Den signal som innehåller informationen kallas modulerande signal eller basband eller
underbärvåg.

Den signal som informationen överförts till kallas modulerad signaleller huvudbärvåg.

\subsection{Modulationssystem}

Den största gruppen av modulationssystem är definierad med avseende på hur huvudbäNågen är
modulerad. Vanligast är då amplitud- och vinkel modulation. Av vinkelmodulation finns
främst två slag, frekvensmodulation och fasmodulation. Därutöver finns system för
pulsmodulation

\subsection{Sändningsslag}

Sätten att modulera kallas sändningsslag. Gemensamt för sändningsslagen är att en
givare-det kan vara en mikrofon, en telegrafnyckel, en fjärrskriftmaskin, en dator, en
TV-kamera o.s.v.- alstrar en analog eller digital signal. Denna styr underbärvågen så att
huvudbärvågen moduleras med den avsedda informationen och sänds ut.

Det enklaste sändningsslaget får anses vara morsetelegrafi med "nycklad bärvåg".
Då förekommer bara två tillstånd, nedtryckt och icke nedtryckt telegrafnyckel, d.v.s.
antingen bärvåg med någon varaktighet eller ingen bärvåg alls. Kombinationer av
bärvågselement med olika längd motsvarar skrivtecken.

För att återge tal, musik etc. behövs en noggrannare tillståndsstyrning av bärvågen.
Det innebär att bärvågen måste moduleras av en underbärvåg och att denna motsvarar
lufttrycksvariationerna i ljudet.

\subsection{Kännetecken för modulerade signaler}

Bild 111-22

En modulerad signal kännetecknas av dess amplitud, frekvens och fasläge.

Vid amplitudmodulation påverkas huvudbärvågens amplitud, så att den i varje tidpunkt
motsvarar den modulerande signalens variation.

Vid frekvensmodulation påverkas huvudbärvågens frekvens, så att den i varje tidpunkt
motsvarar den modulerande signalens variation.

Vid fasmodulation, som är besläktad med frekensmodulation, påverkas i ställettörfrekvensen
huvudbärvågens fasläge i förhållande till en referenssignal, så att fasläget i varje
tidpunkt motsvarar den modulerande signalens variation.

Frekvens- och fasmodulation liknar varandra och kan sammanfattas som vinkelmodulation,
eftersom fasvinkeln mellan bärvågens spänning och ström varierar i båda fallen.

Vid pulsmodulation används pulståg (korta upprepade bärvågspaket); t. ex. pulsamplitud-,
pulslängds-, pulsläges- och pulskodmodulation. Pulskodmodulation används t.ex. vid
samtidig överföring av flera telesamtal på samma linje, bärvåg etc.

\subsection{Bandbredd vid olika sändningsslag}

Varje radiosändning tar upp plats omkring den nominella bärvågsfrekvensen- tillsammans
bandbredden.

Radioamatören måste veta detta "platsbehov", främst för att inte sända utanför de
frekvensband som är tilldelade för amatörradioanvändning, men även för att kunna
umgås med annan trafik inom banden.

I alla sändningsslag ökar den använda bandbredden med ökad modulation. Eftersom största
frekvenseffektivitet alltid skall eftersträvas så upptar en sändare med kraftigare
modulation än vad som behövs för en överföring, alltid onödigt frekvensutrymme.

Bild II 1-22 Modulerade signaler

\subsection{Beskrivningskod för sändningsslagen}

Vid 1979 års radioförvaltningskonferens (WARC 79) i Geneve reviderades det internationella
radioreglementet (RR), som i huvudsak trädde i kraft 1982. Däri ingår bl. a. ett nytt
system för klassindelning och beteckning av sätten att utsända information över
radio m.m. Reglementet har reviderats senare, men i detta stycke gäller det ännu.

Indelningen i sändningsslag behövs för att känneteckna utsändningarna, t. ex. i
frekvenslistor, författningar och föreskrifter. Indelningen är också av stort värde vid
teknisk beskrivning av apparater och system för radiokommunikation.

Emellertid används av många även äldre benämningar, vilka lever kvar i litteraturen, i
märkning av manöverdonen på sändare och mottagare o.s.v ..

Dessa äldre benämningar är dock inte entydiga och skapar lätt missförstånd, varför
beskrivningskoden enligt WARC 79 bör användas för tydlighetens skull.

Här följer avkortade koder enligt WARC 79 för några av de sändningsslag, som amatörer
använder mest, samt för jämförelse även de benämningar som fortfarande används jämsides
(se vidare i Appendix E).

\begin{description}
\item[NON] Bärvåg utan modulerande signal. Ingen information.

\item[A1A] Bärvåg med dubbla sid band. En enda kanal med kvantiserad bärvåg. Ingen
modulerande underbärvåg. Telegrafi. Även kallat nycklad bärvåg (CW).

\item[A3E] Linjärt modulerad huvudbärvåg. Dubbla sidband. En enda kanal med
analog information. Telefoni.

Även kallat amplitudmodulation (AM).

\item[J3E] Linjärt modulerad huvudbärvåg. Ett sidband med undertryckt bärvåg. En
enda kanal med analog information. Telefoni.

Även kallat enkelt sidband (Single Side Band-SSB).

\item[F3E] Vinkelmodulerad bärvåg. Frekvensmodulering. En enda kanal med analog
information. Telefoni.

Även kallat frekvensmodulering (FM)

\item[G3E] Vinkelmodulerad bärvåg. Fasmodulering. En enda kanal med analog information.
Telefoni.

Även kallat fasmodulering (PM)
\end{description}

Såväl A1A, A3E som J3E är sändningsslag där amplituden moduleras. Därför är
termen amplitudmodulation inte tillräcklig för att beskriva flera likartade sändningsslag.

\subsection{Modulerande signaler}

\subsubsection{Basband}

Basband är ett frekvensområde för en modulerande signal. Det finns ett basband för
alla slags modulerande signaler, vare sig de är analoga eller digitala. Det kan finnas mer
än ett basband i en komplett modulationsprocess. Till exempel är en nycklad ton, som
går till sändaren genom mikrofoningången, dess analoga basband medan nycklingspulserna
till tongeneratorn är dess digitala basband.

Bild 111-23

Ett vanligt sätt att överföra information över radio är med telefoni, d.v.s. tal.

Frekvensområdet 300-3000 Hz räcker för god förståelighet av tal. Dels är örat känsligast
inom det området och dels finns där den mesta energin i talet.

Mikrofonen tar upp de lufttrycksvariationer, som uppstår när man talar, och omvandlar dem
till elektriska svängningar. Svängningarna varierar mellan positiva och negativa
spänningsvärden.

\subsubsection{Försök}

\begin{enumerate}
\item Anslut en mikrofon till ett oscilloskop och studera spänningsförloppen för olika slags
ljud, toner, tal o.s.v. som funktion av tiden. På bilden är dessa svängningar mycket
förenklade, t.ex. sinusformade.

\item Anslut en högtalare och ett oscilloskop till en LF-generator, vars frekvens och amplitud
kan ändras. Lyssna på ljud med låg och hög frekvens samt på svaga och starka ljud. En
baston har låg frekvens och en diskantton har hög frekvens. En svag ton har liten
amplitud och en stark ton har stor amplitud.
\end{enumerate}

\subsection{Sändningsslaget A3E (även kallat AM)}

Bild 111-24

Bilden visar frekvensspektrum av en signal vid amplitudmodulation med

\begin{enumerate}[label=\alph*.,noitemsep]
\item en sinuston,
\item en blandning av tre sinustoner,
\item ett frekvensspektrum.
\end{enumerate}

\subsubsection{Försök}

Modulera en A3E-sändare med en 3 kHzsignal. Med en mottagare utrustad med ett
smalt filter för telegrafi, kan man urskilja och påvisa bärvågen och de båda sidbanden.

\subsubsection{A3E-modulation med en ton}

Bild 111-25

En omodulerad bärvåg har konstant amplitud. En amplitudmodulerad signal är i grunden
resultatet av svävning mellan frekvenser eller av icke linjär blandning av frekvenser. När 
bärvåg och basband blandas, så är särskilt tre blandningsprodukter av intresse.

Dessa är

\begin{enumerate}[label=-,noitemsep]
\item bärvågen,
\item det lägre sidbandet (förkortat LSB) och
\item det övre sidbandet (förkortat USB).
\end{enumerate}

AM-signalen består således inte bara av bärvågsfrekvensen fHF utan även av övre
och nedre sidofrekvenser, vilka är summan och skillnaden av bärvågsfrekvensen $f_{HF}$ och
den modulerande frekvensen $f_{LF}$. Alltså $f_{HF} + f_{LF}$ (övre sidfrekvens) och
skillnadsfrekvensen $f_{HF} - f_{LF}$ (undre sidfrekvens).

Bild II 1-23 Modulerande signaler

Bild II 1-24 Sidband vid A3E-modulation

Eftersom tal inte bara omfattar en enda frekvens utan ett helt frekvensspektrum (c:a
0.3 - 3 kHz), så uppstår inte bara två sidfrekvenser utan två sidband, det lägre sidbandet
(LSB, Lower Side Band) och det övre (USB, Upper Side Band).

LF-signalens frekvens bestämmer sidfrekvensens avstånd från bärvågen. Bandbredden på en
amplitudmodulerad signal med full bärvåg och två sidband är dubbelt så stor som den högsta
modulerande LFfrekvensen:

$b= 2 \cdot f_{LFmax}$

Om de modulerande LF-frekvenserna är mellan 0.3 och 3 kHz, så blir sändningens
totala bandbredd 6 kHz.

LF-signalernas amplitud påverkar sidbandens och sidfrekvensernas amplitud. Vid
maximal modulation (100 \% modulationsgrad) varierar signalamplituden mellan noll
och dubbla värdet av det för en omodulerad bärvåg.

Som mest kan vardera sidbandet överföra en fjärdedel så mycket effekt som bärvågen, d.v.s.
en sjättedel av den totalt utsända effekten. Då avger sändaren dubbelt så stor medeleffekt
som utan modulation. Toppeffekten (PEP, Peak Envelope Power) är till och med fyra gånger
så stor.

slutförstärkaren och kraftförsörjningen måste dimensioneras för toppeffekten vid
full modulation eller att modulationsgraden anpassas så att överbelastning inte sker.

Bild II 1-25 A3E-modulation med toner med olika styrka och frekvens

\subsubsection{Fördelar med A3E-modulation}

En A3E-sändare är enkel jämfört med en J3E-sändare, vilken har en mer komplicerad
signalbehandling.

\subsubsection{Nackdelar med A3E-modulation}

Eftersom samma information finns i båda sidbanden och ingen finns i bärvågen, så sänds
effekten i bärvågen och ett av sidbanden ut till ingen nytta. I talpauser sänds endast
bärvågseffekten och till ingen nytta. Aven frekvensutrymme slösas bort. Då en annan,
alltför närliggande sändares bärvåg blandas med den egna, så alstras renstoner i
mottagarna.

\subsection{Sändningsslaget A1A (även kallat CW)}

Bild 111-26

Man kan överföra meddelanden med morsetelegrafi på olika sätt. Det enklaste sättet är att
koppla in och ur sändarens bärvåg i takt med teckendelarna i morsetecknen. Man kan kalla
det för bärvågstelegrafi. Förfarandet kallas sedan mycket länge även för CW (continous
waves), vilket egentligen anger att bärvågen svänger med konstant amplitud, om man bortser
från att den nycklas. Detta i motsats till de dämpade bärvågssvängningar som var fallet i
sedan mycket länge förbjudna s.k. gnistsändare.

Fastän en sändare "moduleras utan ton", har den en viss bandbredd. Det beror på att den
takt, som sändaren nycklas med, egentligen är en ton - låt vara med låg frekvens. Antag
att sändaren nycklas med en serie korta morsetecken. Vid telegraferingshastigheten
60 tecken/minut alstrar bärvågspulserna en kantvåg med frekvensen 5 Hz. Som tidigare
beskrivits, består en sådan kantvåg av summan av sinussignaler med frekvenserna 5 Hz,
15 Hz, 25 Hz, 35 Hz o.s.v.

Det innebär att det uppstår sidfrekvenser över och under bärvågens frekvens och med
ett avstånd till bärvågen av 5 Hz, 15 Hz, 25, 35Hz o.s.v .. Telegrafisändaren har alltså
liksom vid A3E en bandbredd, som dels står i förhållande till nycklingshastigheten och
dels till "kantigheten" på tecknen, vilket bestämmer övertonshalten i bärvågen. Vid s.k.
mjuk nyekling kan den 9:e övertonen antas vara den högsta som uppfattas av en motstation.
Med en nycklingsfrekvens av 5 Hz blir bandbredden inte större än
$2 \cdot 10 \cdot 5 = 100Hz$.

En hård (kantig) och snabb teckengivning ökar bandbredden och kan resultera i att s.k.
nycklingsknäppar kan uppfattas långt vid sidan om sändningsfrekvensen. Ju hårdare
nycklingen är, desto längre bort från bärvågsfrekvensen hörs nycklingsknäpparna. Detta
stör andra stationer.

Kännetecken för sändningsslaget A1A, telegrafi genom nycklad bärvåg:

Mycket liten bandbredd, extremt gott utnyttjande av såndareffekten, stor
överföringssäkerhet, lång räckvidd, enkla sändare.

Bild II 1-26 Amplitudmodulation med morsetecken

\subsection{Sändningsslaget J3E (även kallat SSB)}

\subsubsection{Princip}

Som sagts är det onödigt sända ut två sidband, eftersom båda innehåller samma information.

Signaler med endast ett sidband och undertryckt bärvåg kan alstras på flera sätt.
Numera är den s.k. filtermetoden i särklass vanligast och den enda som behandlas här.

Bild II i-27

Med filtermetoden blandas HF- och LFsignalerna i en speciell blandare. Där undertrycks
båda dessa signaler medan blandningsprodukterna med deras summa- och skillnadsfrekvenser
blir kvar, d.v.s. det övre och nedre sidbandet.

Utsignalen från blandaren benämns DSBsignal (Double Side Band). Till skillnad från
i A3E-signalen saknas dock bärvågen i DSBsignalen. För att även undertrycka det ena
sidbandet före sändningen, så följs blandaren av ett bandpassfilter med bandbredd
och frekvensläge för avsett sidband.

Den signal som sänds ut innehåller därför endast ett sidband (Single Side Band).

\paragraph{Exempel}

Bild II 1-28

Ett SSB-filter har ett passband av 9000.39003 kHz. Vid bärvågsfrekvensen 9000kHz
sträcker sig det övre sidbandet från 9003.39003 kHz och släpps igenom. Däremot blir
bärvågsfrekvensen undertryckt.

Det undre sidbandet 8997-8999.7 kHz faller utanför filtrets passband och blir också
undertryckt.

Skall däremot det undre sidbandet kunna passera igenom samma filter, så måste
bärvågsfrekvensen höjas med 3 kHz, alltså till 9003 kHz. Då faller det undre sidbandet,
9002.7-9000.0 kHz inom filtrets passband.

Det övre sidbandet 9003.3-9006.0 kHz faller nu utanför passbandet och blir undertryckt.

Bild II i -29

LF-signalens amplitud bestämmer amplituden på sidfrekvensen.

LF-signalens frekvens bestämmer sidfrekvensens avstånd från bärvågsfrekvensen (bärvågen
undertryckt).

Bandbredden på den utsända signalen är skillnaden mellan högsta och lägsta
modulerande frekvens i signalen:

t.ex. $b = 3kHz - 0.3 kHz = 2.7 kHz$

\subsubsection{Fördelar med J3E-modulation}
Bra verkningsgrad vid J3E-modulation jämfört med vid A3E-modulation (traditionell AM).
Effekten i det utsända sidbandet motsvarar den i ett av sidbanden vid A3E. Hela den
utsända effekten finns alltså i ett enda sidband, som överför hela informationen.

I sändningspauserna sänds ingen effekt ut. Bandbredden är mindre än hälften av den
vid A3E. Vid mottagning av en J3E-sändning (SSB) är det mindre besvär med interferenstoner
från J3E-sändningar på närliggande frekvenser, eftersom ingen bärvåg och endast ett
sidband sänds ut.

\subsubsection{Nackdelar med J3E-modulation}
J3E-modulation medför mera komplicerade apparater, både för mottagning och sändning.
En J3E-signal blir förvrängd och hörs i fel tonläge, om mottagaren inte är inställd på
exakt rätt frekvens.

Bild II 1-27 Sidband vid DSB

Bild II 1-28 Sidbandsval vid SSB

Bild II 1-29 Sidbandlägen vid SSB

\subsection{Vinkelmodulation}
Termen vinkelmodulation är samlingsnamnet för frekvensmodulation (FM) och fasmodulation
(PM). Ofta sägs utrustningar vara för frekvensmodulation, när de antingen är för frekvens-
eller fasmodulation. Det finns alltså skillnader och likheter mellan dessa system, vilka
emellertid inte är oberoende av varandra, eftersom frekvensen i en signal inte kan
varieras utan att fasen också varieras, och vice versa.

Hur effektiv kommunikationen då är, beror mest på mottagningsmetoderna. I båda fallen
uppfattas ändringar i den mottagna signalens frekvens och fasläge. Amplitudändringar
uppfattas däremot inte. De flesta störningar - särskilt pulserande sådana som från
tändningssystem - kommer att därför att skiljas bort.

För att effektivt utnyttja fördelarna med vinkelmodulation, antingen det är frekvenseller
fasmodulation, behövs tillräckligt frekvensutrymme. Det innebär att främst högre
frekvensband kommer i fråga.

\subsection{Frekvensmodulation (även kallat FM)}

Bild II 1-30 (överst och i mitten)

Vid frekvensmodulation varierar bärvågens frekvens i takt med den modulerande signalens
amplitud och polaritet. På bilden ökar bärvågens frekvens när den modulerande signalen är
positiv (första halvperioden) och minskar när den modulerande signalen är negativ (andra
halvperioden). Bilden visar att perioderna i den modulerade bärvågen tar kortare tid (har
högre frekvens), när den modulerande signalen är positiv, och mertid (har lägre frekvens)
när den modulerande signalen är negativ. Bärvågen kommer alltså att pendla omkring ett
medelvärde, d.v.s. vara frekvensmodulerad.

Frekvensawikelsen L1f (deviationen) från bärvågens vilafrekvens är vid varje tillfälle
proportionell till den modulerande signalens amplitud. Sålunda är deviationen liten när
den modulerande signalens amplitud är liten och störst när amplituden når sitt toppvärde,
antingen amplituden är positiv eller negativ. Vid en modulationsfrekvens av 300 Hz
varierar bärvågsfrekvensen 300 gånger per sekund, vid 3kHz - 3000 gånger per sekund.

Likspänningsnivåer kan överföras med FM, eftersom en motsvarande frekvensavikelse kan
framställas.

Bilden visar också vad som oftast sägs, att bärvågsamplituden inte ändras av modulationen.
Detta är emellertid bara delvis sant, eftersom såväl bärvågsamplitud som sidbandsamplitud
varierar med modulationsindex, vilket förklaras nedan.

\subsubsection{Sidbanden vid vinkelmodulation}

Vid AM produceras endast ett sidbandspar med samma inne hål!, ett över och ett under
bärvågsfrekvensen. Vid vinkelmodulation, både vid FM och PM, produceras däremot flera
sidbandspar över och under bärvågsfrekvensen. Dessa sidband uppträder på multiplerna av
varje modulerande frekvens. Vid basband med samma frekvensomfång har därför en
vinkelmodulerad signal större bandbredd än en AM-signal.

Vid vinkelmodulation beror antalet sidband på sambandet mellan den modulerande frekvensen,
frekvensdeviationen och modulationsindex.

\subsubsection{Bandbredden vid vinkelmodulation}

Bild II 1-30 (nederst)

Vi gör tankeexperimentet att en FM-sändare moduleras med en fyrkantvåg. Frekvensen
kommer då att hoppa växelvis mellan frekvenserna $f$ och $f + \Delta f$. Sättet kallas FSK
(frekvensskiftnyckling) och används t. ex. vid sändning av radiofjärrskrift (RTTY, AMTOR,
Paketradio etc.).

Vi föreställer oss två sändare, som sänder varannan gång, varav den ena sänder frekvensen
$f$ och den andra sänder $f + \Delta f$. Båda sändarnas HF-signaler kommer då att bilda
ett frekvensspektrum, som förutom $f$ och $f + \Delta f$ även innehåller sidfrekvenser.

Bredden på detta spektrum beror bl. a. på nycklingsfrekvensen. Eftersom en fyrkantvåg
innehåller summan av dess grundfrekvens och övertoner, kommer alla dessa toner att
modulera vardera sändaren. De högsta modulerande LF-frekvenserna alstrar sidfrekvenserna
längst ut från vilofrekvensen. LF-signalens frekvensspektrum påverkar alltså
HF-signalens bandbredd.

Spektrum nederst i bilden är en förenklad framställning av frekvensskiftnyckling.

Bild II 1-30 Frekvensmodulation

Vid modulation med en sinussignal istället för med en fyrkantsignal, uppstår ett
frekvensspektrum som på överst i bilden.

\paragraph{Frekvensdeviation och modulationsindex}

Bild II 1-31

Vid vinkelmodulation uppstår talrika sidefrekvenser, som beror av den modulerande
frekvensen $f_{LF}$. Amplitudfördelningen mellan sidfrekvenserna står i förhållande till
deviationen, varvid deras amplitud blir mindre desto längre bort från bärvågen de är.

I praktiken anses en sidfrekvens försumbar när dess amplitud är mindre än 1 \% av
amplituden för omodulerad bärvåg.

För beräkning av bandbredden används begreppet modulationsindex m, vilket är kvoten av
maximal deviation $\Delta f$ och högsta frekvensen $f_{LF}$.

$m = \frac{\Delta f_{max}}{f_{LFmax}}$

Inom amatörradion är det vanligt att arbeta med $\Delta f_{max} = 3 kHz$ och
$f_{LFmax} = 3 kHz$, d.v.s. $m = 1$.

Vid modulationsindex $m = 1$, gäller följande
formel för bandbredden $b$

$b = 2 \cdot ( \Delta f_{max} + f_{LFmax}) = 2 \cdot \Delta f_{max} + 2 \cdot f_{LFmax}$

Med ovan nämnda värden blir bandbredden $b = 2 \cdot (3 kHz + 3 kHz) = 12 kHz$

Bandbredden ökar således både med ökande deviation och ökande modulerande frekvens. För
att inte interferera med trafik på grannkanalerna måste såväl deviation som frekvensen på
den modulerande signalen begränsas. En deviationsbegränsare begränsar amplituden på denna
signal. Ett lågpassfilter reducerar den distorsion, som uppstår av begränsningen. Vidare
undertrycks modulerande frekvenser högre än 3 kHz, vilket är tillräckligt för överföring
av tal.

\paragraph{Jämförelse}

En VHF-rundradiosändare är tilldelad ett större frekvensutrymme och kan därför använda
mycket större bandbredd

Där är $\Delta f_{max} = 75 kHz$ och $f_{LFmax} =15 kHz$, därmed är $m = \frac{75}{15} = 5$
och $b = 2 \cdot (75 + 15) = 180 kHz$.

Som framgår av tabellen på nästa uppslag varierar bärvågens liksom sidfrekvensernas
inbördes amplitud med modulationsindex. Detta skall jämföras med AM där bärvågens amplitud
är konstant och endast sidbandens amplitud varierar.

Vid vinkelmodulation utsläcks bärvågen $A_0$ vid modulationsindex 2.404. Den blir sedan
"negativ" vid högre index, vilket betyder att den återkommer, men att dess fasläge blir
omvänt. I vinkelmodulation tas energin i sidbanden från bärvågen, vilket innebär att
den totala effekten förblir densamma oavsett modulationsindex.

\paragraph{Kännetecken för sändningsslaget F3E (FM)}

Fördelar: F3E-sändaren är enkel till sin uppbyggnad och hög överföringskvalitet
uppnås vid stor bandbredd, störningar från amplitudmodulerade signaler t. ex. tändgnistor
undertrycks i mottagaren.

Nackdelar: En relativt stor bandbredd behövs för överföring av ett basband med
stort frekvensomfång. Sändaren måste avge full effekt, även när modulation inte sker.

Bild 111-31 sidbandsspektrum vid FM-modulering med 1 sinuston

\begin{table*}[h]
\begin{center}
\begin{tabular}{ll|l|l|l|l|l|l|l|l|}
\cline{3-9}
&\multicolumn{1}{l}{}  & \multicolumn{7}{|c|}{Modulationsindex} \\ \cline{3-9}
&\multicolumn{1}{l|}{}  &   1   &   2   &    3   &    4   &    5   &    6   &    7   \\ \hline
\multicolumn{1}{|c|}{\multirow{11}{*}{\rotatebox[origin=c]{90}{Relativ amplitud på}}}&$A_0$ & 0.765 & 0.224 & -0.260 & -0.397 & -0.178 &  0.151 &  0.300 \\
\multicolumn{1}{|c|}{}&$A_1$ & 0.440 & 0.577 &  0.334 & -0.066 & -0.328 & -0.277 & -0.005 \\
\multicolumn{1}{|c|}{}&$A_2$ & 0.115 & 0.353 &  0.486 &  0.364 &  0.047 & -0.243 & -0.301 \\
\multicolumn{1}{|c|}{}&$A_3$ & 0.020 & 0.129 &  0.309 &  0.430 &  0.365 &  0.115 & -0.168 \\
\multicolumn{1}{|c|}{}&$A_4$ &       & 0.034 &  0.132 &  0.281 &  0.391 &  0.358 &  0.158 \\
\multicolumn{1}{|c|}{}&$A_5$ &       & 0.016 &  0.043 &  0.132 &  0.261 &  0.362 &  0.348 \\
\multicolumn{1}{|c|}{}&$A_6$ & \multicolumn{2}{c|}{} &  0.011 &  0.049 &  0.131 &  0.246 &  0.339 \\
\multicolumn{1}{|c|}{}&$A_7$ & \multicolumn{3}{c|}{} &  0.015 &  0.053 &  0.130 &  0.234 \\
\multicolumn{1}{|c|}{}&$A_8$ & \multicolumn{4}{c|}{}           &  0.018 &  0.057 &  0.128 \\
\multicolumn{1}{|c|}{}&$A_9$ & \multicolumn{4}{c}{Tomma fält för $A_n$ under 0.01 (1 \%)} &        &  0.021 &  0.059 \\
\multicolumn{1}{|c|}{}&$A_{10}$ & \multicolumn{5}{c}{} &  &  0.024 \\ \hline
\end{tabular}
\end{center}
\caption{Relativa amplituden på bärvåg $A_0$ och sidfrekvenser $A_1$-$A_{10}$ vid
modulationsindex 1-7 (Vid omodulerad bärvåg är modulationsindex 0. Då är
bärvågens relativa amplitud 1.0)}
\end{table*}


\subsection{Fasmodulation (även kallat PM)}

Vid fasmodulation varierar bärvågens fasläge i förhållande till ett
referensvärde. Vid PM är frekvensändringen - deviationen - direkt proportionell
till hur snabbt fasläget på den modulerande frekvensen ändras och till den
totala fasändringen. Hastigheten på fasändringen är direkt proportionell till
frekvensen på den modulerande frekvensen och till den momentana amplituden på
den modulerande signalen.

Det betyder att deviationen i PM-system ökar både med den momentana amplituden
och frekvensen på den modulerande signalen. Detta att jämföras med FM-system där
deviationen är proportionell till den momentana amplituden på den modulerande
signalen.

I PM-system uppfattar demodulatorn i mottagaren endast momentana ändringar i
bärvågsfrekvensen. Till skillnad från vid FM, så kan därför ändringar i
likspänningsnivåer överföras endast om en fasreferens används.

Med konstant amplitud på insignalen till modulatorn, så är vid PM
modulationsindex konstant oavsett modulerande frekvens, medan vid FM
modulationsindex varierar med den modulerande frekvensen.

\subsection{Frekvens- och fasmodulation jämförs}

\begin{itemize}
\item Frekvensmodulation (FM) alstras genom att sändarens oscillatorfrekvens
varieras (devieras) i takt med den modulerande signalen (t.ex. tal). Det gör man
genom att variera resonansfrekvensen i den svängningskrets som styr
oscillatorfrekvensen.

\item Fasmodulation (PM) alstras vanligen genom att efter sändarescillatorn
variera den modulerande signalens fasläge i förhållande till en omodulerad
bärvåg - s.k. fasmodulering. Det gör man genom att variera resonansfrekvensen i
en svängningskrets efter oscillatorn- d.v.s. utan att påverka
oscillatorfrekvensen.

\item I båda fallen ändrar man alltså resonansfrekvensen i en svängningskrets i
takt med frekvensen i den modulerande spänningen, men att denna krets har olika
placering i FM-sändare respektive PM-sändare.

\item I sändaren alstras det i båda fallen utfrekvensersom devierarfrån
oscillatorns vilofrekvens. Graden av deviation skiljer emellertid vid FM och PM.
Vid FM är deviationen proportionell mot amplituden på den modulerande
underbärvågen medan deviationen vid PM är proportionell mot produkten av den
modulerande underbärvågens amplitud och frekvens.

\item Den hörbara skillnaden mellan FM och PM är därför en annorlunda
frekvensgång. Vid samtidig användning av PM-sändare och FM-mottagare är det
alltså lämpligt att justera frekvensgången i PM-sändarens modulator, lämpligen
med 6 dB dämpning per oktav ökad frekvens.
\end{itemize}

\subsection{Pulsmodulation}

Pulsmodulation används mest i mikrovågområdet. Pulsmodulerade signaler sänds
vanligen som en serie korta pulser åtskilda av relativt långa pauser utan
modulering.

En typisk sändning kan bestå av pulser med en längd av 1 $\mu$s och en frekvens
av 1000 Hz. Toppeffekten på en pulssändning är därför mycket högre än dess
medeleffekt

Före WARC 79 var symbolen för all pulssändning P. Därefter används P endast för
omodulerade pulståg. Annan pulsmodulation har följande symboler

\begin{description}
\item[K] - puls-/amplitudmodulation (PAM)
\item[L] - pulsviddmodulation (PWM)
\item[M] - pulsposition/fasmodulation (PPM)
\item[Q] - vinkelmodulation under pulsen
\item[V] - kombination av dessa eller annat sätt.
\end{description}

\begin{table*}[h]
\begin{center}
\begin{tabular}{|l|l|l|l|l|}
\hline
Sändningsslag & Amplituden på & Tonhöjden på & Bandbredden b      & För stor amplitud \\
              & LF-signa!en   & LF-signalen  & förhåller sig till & på LF-signalen \\
              & påverkar      & påverkar     &                    & medför \\ \hline
A3E (AM) & amplituden i   & sidfrekvenser- & LF-signalens    & övermodulering \\
         & båda sidbanden & nas avstånd    & högsta frekvens & och för stor bandbredd \\
         &                & från bärvågen  & & \\
J3E (SSB)& amplituden på  & sidfrekvenser- & skillnaden mellan & för stor bandbredd,\\
         & utsänt sidband & nas avstånd    & LF-signalens      & överstyrning av\\
         &                & från bärvågen  & högsta och lägsta & förstärkarsteg\\
         &                &                & frekvens          & \\
F3E (FM) & deviationen    & hastigheten på & dubbla summan     & för stor deviation,\\
         &                & bärvågens      & av största devia- & för stor bandbredd\\
         &                & frekvens-      & tion och högsta   & \\
         &                & ändring        & LF-frekvens       & \\ \hline
\end{tabular}
\end{center}
\caption{Jämförelse mellan några. vanliga. sändningsslag inom amatörradio}
\end{table*}
