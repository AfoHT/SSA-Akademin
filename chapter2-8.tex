\section{Digitala kretsar}

Särskilt under det senaste decenniet har
digital elektronik blivit vanlig i utrustningar
för radio- och telekommunikation. Även inom
amatörradio används nu denna teknik. Ämnet är mycket omfattande. Utvecklingen är
att enkla digitala funktioner snabbt ersätts av
komplexa datasystem. Här redogörs endast
för några grundläggande digitala funktioner.
Amatörradions kärna finns ännu i analogtekniken, där det under ett förlopp kan förekomma många olika storheter, t.ex. spänningar mellan noll och ett högsta värde.
l digitaltekniken förekommer bara ett bestämt antal tillstånd. l det enklaste digitala
systemet finns två tillstånd, t. ex. Ooch 1 eller
Till och Från eller Hög och Låg eller Fel och
Rätt o.s.v. Ett system med två tillstånd kallas
binärt. En lampa som tänds eller släcks med
en enkel strömställare är ett binärt system.
Strömställaren kan ha olika utföranden. Det
kan vara en mekanisk kontakt som är styrd
för hand eller av en reläspol e. Det kan också
vara en transistor eller annan anordning.

Transistorn som strömställare
Bild II 2-3S

Bilden visar två transistorkopplingar. Den till
vänster är en analog förstärkare för växelspänning. Om det på grund av en viss basspänning flyter en kollektorström av 1 mA
och kollektorresistorn är S n, så blir spänningsfallet över den resistorn SV. Eftersom
matningsspänningen är 12 V, så blir då
spänningen 7V mellan kollektorn och minus.

Kopplingen till höger fungerar som en
binär strömställare. Antag att insignalen intar ett avtvå spänningstillstånd, antingen OV
(låg) eller SV (hög). När inspänningen är
t. ex. SV, så flyter så mycket basström genom
basresistorns 1O k.Q, att transistorn blir fullt
utstyrd.
Därmed är spänningen mellan kollektor
och emitter, d.v.s. utspänningen, nära O V
(0.1 till 0.2V beroende på transistortyp). Man
säger då att utgången är låg (L) eller O(noll).
Om däremot inspänningen ärO V, såspärras kollektorströmmen och utspänningen blir
nära SV. Man säger då att utgången är hög
(H) eller 1.
För NPN-transistorn i bilden, gäller att
• hög inspänning ger låg utspänning,
• låg inspänning ger hög utspänning.
Denna logiska funktion kallas inverterande.

NOT-gate eller inverterande grind
Bild 112-36

Logiska funktioner beskrivs med internationella symboler. En ring vid utgången betyder att utspänningens nivå är motsatt inspänningens. Sambandet mellan in- och utnivåerna beskrivs med en sanningstabe/1.

m
1

o

Bild 1/2-36 NOT-gate

+12V

r
l

Bild II 2-35 Transistorn som analog förstärkare respektive digital strömställare

112- 3S

p

K

Villkorskretsar- s. k. grindar

Det finns olika sätt att bygga grindar. idag är
de flesta grindarna elektroniska lösningar.
Därutöver finns elektromekaniska grindar i
form av strömbrytare och reläkontakter.
Föregångarna till de elektroniska televäxlarna (AXE m.fl.) var stora system av
mestadels elektromekaniska reläer.
För att överskådligt förklara arbetssättet
i de vanligaste grindarna, görs det enklast
med reläsymboler. En reläkontakt kan då
motsvara en transistor eller diod. Reläspolar
kan motsvara logiska nivåer i insignaler.
Elektriska kontakter kan vara normalt
öppna och sluter vid påverkan (s.k. slutande
kontakt). Alternativt kan de vara normalt
slutna och öppnar vid påverkan (s.k. brytande kontakt). l kretsscheman visas kontaktlägena vid systemet i vila.
Bild 112-37
Av bilden framgår att samma villkor kan
skapas med slutande alternativt brytande
kontakter. Observera då placeringen av resistorn på kretsens utgångssida i respektive
fall. När resistorn ligger närmast pluspolen
kallas den pull-up. När den ligger närmast
minuspolen kallas den pull-down. l båda
fallen definierar resistorn den logiska nivån

Bild
Sanningstabellen i bilden säger, att när alla
insignaler är 1 så är utsignalen också 1.

ELLER-grind eller OR-gate
Bild 112-38

Sanningstabellen säger, att när en ellerflera
av insignalerna är 1 så är utsignalen också
1. När alla insignaler är O, så är utsignalen O.

OCH INTE-grind
Bild 112-39

NANO-gate

Sanningstabellen säger, att när ingen eller
någon insignal är 1, men inte alla, så är
utsignalen 1 . När alla insignaler är 1, så är
utsignalen O.

INTE ELLER-grind eller NOR-gate
Bild 112-40

Sanningstabellen säger, att när någon eller
alla insignaler är 1, så är utsignalen O. När
alla insignaler är O, så är utsignalen 1.

c
c

A

l
l
H
H

B

c

A

l

l
l
l
H

o o o
o 1 o
1
o o

H

l
H

1

B

1

c

1

Bild II 2-37 OCH-grind (AND-gate)

112-36

A
B

c

MP NENTER
+

ATBT-

R

c
A

B

A

B

c

L
L
H

L

H

L
H

H

H

H
H

L

c
o o o
o 1 1
1 o 1
A

B

1

1

1

c

~fic

Bild 112-38 ELLER-grind (OR-gate)

c
A

B

c

A

B

c

L
L
H

L

H

H

H

1
1
1

H

H

L

o o
o 1
1
o

L

H

1

1

o

~fic
A~
B~c

Bild 112-39 OCH INTE-grind (NAND-gate)

112-37

PT

K MP

A

c
B

A

c

R

B

A

B

c

A

L
L

H

L

H

L

L
L
L

o o 1
o 1 o
1 o o
1
1 o

H
H

H

B

c

Bild II 2-40 INTE ELLER-grind (NOR-gate)

Inverterad ingång
En ingång kan behöva ha en inverterad
funktion i förhållande till de övriga (s.k. low
active). Man kan då göra på följande sätt
med en OCH-grind som exempel.

A

c

A
B

c
c

A

B

L
L

H

H

L

L
L

H
H

L

H

L

c
o o o
o 1 1
1 o o
1 1 o

A

B

Bild II 2-41 Inverterad ingång

112-38

A
B
A
B

c

R

A

B

c

c

A

L

B

L

C

A

L

o o o

L

H

H

H
H

L
H

H
L

o
1
1

B

1

o
1

C

1
1

o

A

B

A

B

C

L
H

H
L

L
L

H

o o

1

H

H

H

1
1

1

L

L

A
B

Bild II 2-42 Exklusiv ELLER-grind

C

o

o
o o
1

1

c

(EXOR-gate)

Bild II 2-43 Exklusiv INTE ELLER-grind
(EXNOR-gate)

Exklusiv ELLER-grind (EXOR-gate)
Bild II 2-42

Exklusiv INTE ELLER-grind (EXNOR·gate)
Bild 112-43

Sanningstabellen säger, att när alla insignaler antingen är 1 eller O, så är utsignalen O.
När en, men inte alla insignaler är i, så är
utsignalen 1.

Sanningstabellen säger, attnär alla insignaler
antingen är 1 eller O, så är utsignalen 1. När
en, men inte alla insignaler är 1, så är utsignalen O.

112-39

K MP N
Grindar med dioder och transistorer
l stället för reläer i grindar använder man nu

ytterst sällan något annat än kombinationer
av dioder, transistorer och resistorer.
Bild II 2-44
Bilden visar en NANO-grind. Den egentliga
grinden består av tre dioder och en resistor.
Två av dioderna är ingångar och den tredje
är utgång. Grinden styr en digitalt arbetande
transistor liksom den i bild Il 2-35. Resultatet
är en s.k. DTL-Iogik.

Bild II 2-45
Även denna bild visar en NAND-grind. Här
består den egentliga grinden av en ingångstransistor med två emittrar, vilka motsvarar
dioderna vid A och B i föregående bild.
Kollektorn i denna transistor motsvarar ingångsdioden till transistorn i bild Il 2-44.
De övriga tre transisitorerna i bild Il 2-45
bildar en s.k. switch, som ger snabb övergång mellan väl definierade logiska nivåer.
Resultatet är en s.k. TTL-Iogik

A

B

A
B

Bild 112-44 DTL-Iogik

Bild 112-45 TTL-Iogik
