\section{CEPT}
Begreppet CEPT
Vid sidan av folkrättsligt bindande avtal såsom den internationella telekonventionen
(ITC) - har det internationella samarbetet
lett till överenskommelser som inte är tvingande. Sådana avtal görs bl.a. inom CEPT.
CEPTbetyder Conference Europeanne des
Administrations des Poste s et des Telecommunications, d.v.s. Europeiska konferensen
förpost- och teleadministrationerna. "Konferens" är att förstå som ett ständigt arbetande
samarbetsorgan.
Arbetet inom CEPT har huvudsakligen
karaktär av ömsesidiga programförklaringar
mellan länder. Trots att dessa viljeförklaringar
eller rekommendationer inte är bindande har
de visat sig värdefulla för utvecklingen av det
internationella samarbetet.
Länder anslutna till CEPT förenklar handläggningen av ärenden bl.a. rörande amatörradio genom att ömsesidigt bekräfta
rekommendationer inom området.

CEPT-rekommendationerna
Länder anslutna till CEPT förenklar numera
handläggningen av tillståndsärenden om
amatörradio genom att ömsesidigt bekräfta
och inom sitt land tillämpa rekommendationer som länderna utformat i samråd. Det
innebär att svenska amatörradiobestämmelser kan "harmoniseras" till andra länders.
För kompetenskrav vid examinering av radioamatörer finns CEPT-rekommendationerna TIR 61-01 och TIR 61-02.

CEPT-rekommendation TIR 61-02
Rekommendationen T/R 61-02 innebär att
administrationerna i CEPT-Iänder utger ömsesidigt erkända Harmoniserade Amatörradio Examinerings Certifikat (HAREC) till de
personer som vid nationella prov uppfyller
rekommendationens kunskapskrav motsvarande nivå A respektive B. Dessa HARECnivåer motsvarar kraven för de svenska certifikatsklasserna CEPT i respektive CEPT
2. Radioamatörer med ett sådant certifikat
får utöva amatörradio i annat CEPT-Iand,
som godkänt T/R 61-02 och får tilldelas ett
CEPT-certifikat av det landet utan att behöva genomgå ytterligare kunskapsprov.
Det medger också att en person som
uppvisar ett CEPT-certifkat (HAREC), utfärdat av ett annat CEPT-Iand, tilldelas ett
motsvarande tillstånd vid återkomsten till
hemlandet utan att behöva genomgå ytterligare kunskapsprov.
Rekommendationen godkändes år 1990
och reviderades år 1994 med målsättning
att möjliggöra för icke CEPT-Iänder att delta
i systemet.
Sverige tillämpar T/R 61-02.

CEPT-rekommendation TIR 61-01
Rekommendationen T/R 61-01 möjliggörför
radioamatörer från CEPT-länderna att utöva
amatörradio under korta besök i andra CEPTländer, utan att behöva ett tillfälligt tillstånd
från det besökta CEPT-Iandet. Den godkändes år 1985. Erfarenheterna med detta system är goda. År 1992 reviderades rekommendationen med målsättning att möjliggöra för icke CEPT-Iänder att delta i systemet.

