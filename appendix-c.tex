\chapter{OMRÄKNING MELLAN dB OCH KVOTEN AV TAL}

Benämningen Bel kommer från namnet på
amerikanen Alexander Graham Bell, som år
1876 uppfann den första praktiskt användbara telefonen efter ideer från tysken Philipp
ReiB.
Inom teletekniken används begreppet
decibel för att beskriva förlopp av effekt,
ström och spänning. Begreppet förekommer
även i andra sammanhang, t. ex. akustik där
det istället är fråga om ljudtryck.
Måtten i det metriska systemet är alldagliga och ingen finner det märkligt att det
t.ex. gårtio decimeter på en meter. Däremot
är begreppet decibel ovant för många.
Räkning med decibel grundas på användning av logaritmer, som är ett bekvämt
sätt att uttrycka och behandla talvärden.
Detta har har i korthet förklarats i Kapitel1 .9.
Här beskrivs ett omräkningsförfarande med
hjälp av tabeller.

Decibel är ett dimensionslöst uttryck för
graden av dämpning alternativt förstärkning.
Dämpning är följden av att vissa komponenter bromsar elektrisk ström.
Förstärkning innebär att en aktiv komponent kan styra en större elektrisk ström och
därmed större effekt än den själv styrs med.

Bestämning av dB ur effektförhållandet

i) Dela upp effektförhållandet i faktor och
i O-potens
2) Bestäm dB-talet för 1O-potensen
3) Bestäm dB-talet för faktorn med nomogram, tabell eller räknare
4) Addera dB-talen till ett slutresultat
Exempel: dB-talet för 300-faldig effektförstärkning
i) 300 = 3·i00
2) i 00 motsvarar 20 dB
3) 3 motsvarar 4.8 dB
4) 20 + 4.8 = +24.8 dB

l

2
i ji l l

o

iJl
3

i

i

l

l

4
l

l

6

Exempel: dB-talet för 70 000-faldig effektminskning
i) 70 000 = 7 . i 000
2) i O 000 motsvarar 40 dB
3) 7 motsvarar 8.5 dB
4) 40 + 8.5 =- 48.5 dB

o

Exempel: En Vagi-antenn omfördelar den
utstrålade effekten så att den i bästa riktningen blir 56-faldigt bättre än från en
referensantenn. Hur många dB motsvarar
det?
56= 5.6. iO
vilket motsvarar 7.5 + 1O= 17.5 dB
Exempel: l en koaxialkabel förloras hälften
av den inmatade effekten. Hur stor är
dämpningen i dB?
Effektförhållandet är faktor 2 (inverterat),
d.v.s 3 dB dämpning=- 3 dB

Bestämning av effektförhållandet ur dB

i) Dela upp dB-talet i i-tal och i O-tal
2) i-talet dB ger en faktor
3) i O-talet dB ger antal nollor bakom faktorn
(ställvärdet)
4) Multiplicera faktorn med ställvärdet
(antalet nollor).
Exempel: Vilket effektförhållande motsvarar
i3 dB?
1) i 3 dB = i O dB + 3 dB
2) 3 dB motsvarar faktor 2
3) 1O dB ger 1 nolla
4) 2·i O= 20, d v s 20-faldig förstärkning
Exempel: Vilket effektförhållande motsvarar
-i15d8?
1) ii5 dB = i 1O + 5 dB
2) 5 dB motsvarar faktor 3.2
3) 11 O dB ger ii nollor
4) Efter decimaltecknet i faktor 3.2 följer ett
ställvärde av 11 ,
d.v. s. 320 000 000 000-faldig dämpning.

8
l

i

l

l

l

l

9

i

l

10 ggr

l

Effekt

10 dB

C-1

APPENDIX
Exempel: Om ineffekten till ett slutsteg är
100 W och det har en förstärkning av 1odB.
Hur stor är uteffekten?
1O dB motsvarar en 1O-faldig förstärkning.
Uteffekten från slutsteget blir således 10·1 00
= 1000 w.

Exempel: 300-faldig spänningsförstärkning
1) 300 = 3 . 100
2) 100 motsvarar 40 dB, d.v.s två (nollor)
gånger 20 = 40
3) 3 motsvarar 9.5 dB
4) 40 dB+ 9.5 dB = 49.5 dB

Sambandet mellan effektförhållande och dB

Bestämning av ström.. eller spännings..
förhållandet ur dB
1) Dela dB-värdet med 20 dB varvid erhålls
en del och en rest
2) Delen ger 1O-potensen, d.v.s antalet nollor bakom faktorn (ställvärdet)
3) Gå in i nomogrammet och omvandla "resten" till en faktor
4) Multiplicera faktorn med ställvärdet

dB

o

o

1
1
1
1
2

1
2
3

4

5
6
7

8
9

10

o

2
5
9
5
1
9

3
3
5 . o
6
3
7
9

20

o

5
8
9
1
6
8
1

o

4

30

o

8
4
5
1
2
1
1
9
3

40

o

9
8
2
8
2

o

8

5
2

50

o

2
9
6
8
7
7
7
7
8

60

o

5
3
2
6
8
2

2
3

2

Kolumnen längst till vänster upptar 1-tal dB
från Otill 9 och den översta raden upptar 1Otal dB-tal från Otill 60.
Med tabellen kan effektförhållanden bestämmas ur dB-tal från Otill 69 eller omvänt.
Det motsvarar effektförhållanden från 1:1 till
1:8 millioner.
Var uppmärksam på decimaltecknets placering. Avkorta till önskat antal decimaler.
Exempel: Vilket effektförhållande motsvarar
+7dB?
?ligger mellan Ooch 9. Sök därför OdB i den
översta raden. Gå sedan rakt nedåt i kolumnen till raden för 7 dB. Vi kommer då till första
siffran i kvoten för 7 dB. Decimaltecknet står
till höger om denna ruta (mellan kolumnerna
för O och 1O dB).
l sifferfältet kan nu utläsas en effektförstärkning (kvot) av 5.011872.
Bestämning av dB ur ström- eller
spänningsförhållandet
1) Dela upp ström- eller spänningsförhållandet i faktor och 1O-potens
2) Bestäm dB-talet för 1O-potensen
3) Bestäm dB-talet förfaktorn ur nomogrammet
4) Addera dB-talen till ett slutresultat

C-2

Exempel: Vilket spänningsförhållande motsvarar -115 dB?
1) 115 dB/20 dB= 5 rest 15
2) 5 (nollor) motsvarar 100 000
3) 5 dB motsvarar 5.5
4) 5.5 ·1 00 000 =550 000-faldig spänningsdämpning
Tabeller för sambandet mellan ström- eller spänningsförhållande och dB
Kolumnerna längst till vänster upptar dBtalen O- 9. l tabellernas översta rad är dBtalen listade i jämna 1O-tal dB från O - 120
respektive i udda 1O-tal dB från 10-130.

Med tabellerna kan ström- och spänningsförhållanden bestämmas ur dB-tal från Otill
139 dB och omvänt.
Detta motsvarar förhållanden från 1:1 till
1:8.9 millioner.
dB

Jämna 1O-tal dB från O till 120 dB
O 20 40 60 80 100 120

o

o

1
2
3
4
5
6
7
8
9

1
1
1
1
1
1
2
2
2

o o o o o o

1
2
4
5
7
9
2
5
8

2
5
1
8
7
9
3
1
1

2
8
2
4
8
5
8
1
8

o

9
5
8
2
2
7
8
3

1
2
3
9
7
6
2
8
8

8
5
8
3
9
2

1

6
3

PPENDIX
Udda 10-tal dB från 10 till 130 dB
10 30 50 70 90 110 130

o

1
5

3
3
3

1
2
3

9

4
5
5
6
7
7
8

4
5

6

7

8
9

6
4
8
6
1
2

4

o
6

o

7
4
1

9
9

7

8
4
2
6
2

3

o

7

8
8
4
5
4
2
5

3
9
9
3
2

o

3

2
1

2
8
1
6
1

Decibel över 1 m W vid 50 Q [dB(m)]
Som nu beskrivits är uttrycket decibel ett
logaritmiskt mått för hur två effekter förhåller
sig till varandra. När de jämförda effekterna
uppträder över lika stora impedanser, kan
även förhållandet mellan två spänningar eller två strömmar uttryckas i decibel. l samtliga fall rör det sig om förhållandet mellan två
storheter - aldrig absoluta storheter.

3
7
1
7
5
8

3
3
8
2

o

Exempel
Ett drivsteg i en sändare drivs med 1 watt och
avger 1Owatt. Effektförhållandet är 10:1 och
effektförstärkningen är 1O gånger eller 1O
dB. slutförstärkaren i samma sändare drivs
med 1O watt från drivsteget och avger 100
watt till antennen. Även i detta fall är effektförhållandet 10:1 och effektförstärkningen
1O gånger eller 1O dB.
slutförstärkaren hanterar en 1 gånger
så hög effektnivå som drivsteget och ändå är
förstärkningen 1OdB i båda fallen. Decibel är
m.a.o. dimensionslöst
Men om en av två jämförda effekterna
alltid är densamma och väl definierad så
medges nya möjligheter. Den effekt som
skall kvantifieras kan nu ställas i förhållande
till den kända referenseffekten. Med denna
förutsättning kan även de absoluta effektnivåerna, t.ex. genom en sändare uttryckas
i decibel. Detta tillgår på följande sätt.
Det är mycket vanligt att in- och utgångarna i HF-utrustningar utförs med en impedans av 50 n. För god anpassning väljs
då koaxialkablarna mellan apparaterna med
en karaktäristisk impedans av 50 n.

9

Exempel:
En förstärkare med lika in- och utgångsimpedans förstärker spänningen 350-falt.
Hur många dB är det?
Vi söker närmaste 3-ställiga tal i de två
ovanstående tabellerna. l den nedersta tabellen finner vi talet 354 på andra raden.
Över entalet 4 finner vi 50 dB i den översta
raden. Till vänster om 354 finner vi 1 dB.
Som ett närmevärde är alltså förstärkningen
50+ 1 =51 dB.

o

Om kvoten i nedanstående nomogram är en
eller flera 1O-potenser högre än 1O (ggr), så
kan nomogrammet utökas enligt följande
tabell.
Kvot*

Analys
1 har O nollor
1O har 1 nolla
100 har 2 nollor
1 000 har 3 nollor
1O 000 har 4 nollor

1

10
100
1 000
10 000

1

1.1

l

1.2

l

1.3

Skriv

dB

1 . 20
2. 20
3. 20
4. 20

= 20
= 40
= 60
= 80

o. 20 = o

1.4

1.5

1.6

1.7

1.8

1.9

2.0

l

ggr

l
l ....a...........~.........r.....,.l...~..-....\&.........\&................,1~"""'--"-........~......,..j.--l
l
!
l
l
l
.......!..........l......,..j.r-1

l

!

....,...~,.,...........,.1....a...........~...........~.rl

l

i

Spänning
dB

O
2

1

l1PI 1 l 1 1
l

l

0246

8

3
l
l

l
10

l

i

4
l

5

i

l
l

12

14

7

6
l

i

l

l
16

l

li

8

l
l

18

9
i

l

10 g gr

l

20

Spänning
dB
C-3

APPENDIX

C

Det har utbildats en praxis, att referensvärdet vid jämförelse av signalnivåer i radiosystem skall vara en milliwatt (1m W) utvecklad i en belastning med impedansen 50 Q.

Signalnivåer över belastningen 50 Q kan
uttryckas i dB(m), där (m) står för milliwatt,
varvid referenseffekten 1 mW är OdB( m) vid
50 Q.
Det spänningsfall som bildas över belastningen 50 Q vid effektnivån O dB( m) är

U=-.JP·R=1·10-s·50~0.224 V
Den ström som flyter genom belastningen
50 Q vid effektnivån O dB(m) är

l=

(P =~ 1 . 10 -s =0.0045 mA

~Fi

5o

Strömmen 4.5 mA genom belastningen 50 Q
motsvarar således O dB( m).
Varje annan effekt, spänningsfall och ström
som uppstår vid en belastning av 50 Q kan
jämföras med respektive referensvärden
1 mW, 0.22 V och 4.5 mA.

Sambandet mellan spänning över 50 Q
och dB(m)
dB(m) V

-40
-30
-20
- 10

o

1
2
3
4
5
6
7
8
9
10

0.00224
0.00707
0.0224
0.0707
0.224
0.251
0.282
0.316
0.354
0.398
0.446
0.501
0.562
0.630
0.707

dB(m)

V

11
12
13
14
15
16
17
18
19
20

0.793
0.890
0.999
1.121
1.257
1.411
i .583
1.776
1.993
2.236

dB(m) är ett absolut och logaritmiskt mått.

dB(W) är ett annat absolut mått.

Effekt

Effektnivåer över en belastning kan också
uttryckas i dB(W), där (W) står för watt.
Referenseffekten är då 1 W, d.v.s. OdB(W).
Liksom med dB( m) anges impedansen i den
belastning, som effekten utvecklas över.

a [dB(m)]=10 log·

r~nJ

1 m

a

f'so =1

[mW]·1010

Ström
O dB(m) = 4.47 mAsa

l
4.47

a [dB( m)]= 20 log §Q

Isa= 4.47 ·1 0 20
Spänning
O dB(m) = 0.223 Vso

u.

a[dB(m)] = 20 log o.d~
Uso = 0.223 ·1 0 20

C-4

3

sow

J

Exempelvis motsvarar 26 dB(W) 398 W
(se tabellen för sambandet mellan effektförhållande och dB).
