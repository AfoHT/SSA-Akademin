
% Prepare for svenska tecken
\usepackage[T1]{fontenc}
\usepackage[utf8]{inputenc}
\usepackage[swedish]{babel}
\addto\captionsswedish{\renewcommand{\figurename}{Bild}}
\usepackage{amsmath}
\usepackage{fancyhdr}
\usepackage{wrapfig}
\usepackage{caption}
\usepackage{marginnote}
\usepackage{hyperref}
\hypersetup{
    colorlinks,
    citecolor=black,
    filecolor=black,
    linkcolor=black,
    urlcolor=black
}

% % % % % % % % % % % 
% Detta är nya environments för review. De bör vara relativt självförklarande hur de används.
% I princip sätter man bara den del av texten som har en viss status mellan\begin{rev-granskat} och \end{rev-granskat} tex.
% Undvik att nästla dem för det är ingen idé det fungerar inte.
% De är testade med ett antal andra environemnt som tabular mm men kolla att det fungerar med de environments du använder.
% % % % % % % % % % % % % % % % % % % % % % % % % % % % % % % % % % % % % % % % % % % % % % % % % % % % % % % % % % % % % % % 
\usepackage[svgnames,rgb]{xcolor}
\usepackage{pdfcomment}
\newenvironment{rev-ogranskat}{\begin{pdfsidelinecomment}[color=black,linewidth=3px,caption=inline]{Ogranskat}}{\end{pdfsidelinecomment}}
\newenvironment{rev-omarbetas}{\begin{pdfsidelinecomment}[color=red,linewidth=3px,caption=inline]{Omarbetas}}{\end{pdfsidelinecomment}}
\newenvironment{rev-raderas}{\begin{pdfsidelinecomment}[color=red,linewidth=3px,caption=inline]{Raderas}}{\end{pdfsidelinecomment}}
\newenvironment{rev-redo}{\begin{pdfsidelinecomment}[color=yellow,linewidth=3px,caption=inline]{Redo att granska}}{\end{pdfsidelinecomment}}
\newenvironment{rev-granskat}{\begin{pdfsidelinecomment}[color=green,linewidth=3px,caption=inline]{Granskat}}{\end{pdfsidelinecomment}}
\newenvironment{rev-nytt}{\begin{pdfsidelinecomment}[color=brown,linewidth=3px,caption=inline]{Nytt}}{\end{pdfsidelinecomment}}
\newenvironment{rev-releasat}{\begin{pdfsidelinecomment}[color=blue,linewidth=3px,caption=inline]{Klart}}{\end{pdfsidelinecomment}}

\clubpenalty=9990
\widowpenalty=9999
\brokenpenalty=4999


% Make some unicode characters usable
\DeclareUnicodeCharacter{00B0}{\ensuremath{^\circ}} % unicode 00B0 ° degree sign
\DeclareUnicodeCharacter{00B5}{\ensuremath{\mu}} % unicode 00B5 µ micro sign
\DeclareUnicodeCharacter{03C0}{\ensuremath{\pi}} % unicode 3C0 π greek small letter pi
\DeclareUnicodeCharacter{03A9}{\ensuremath{\Omega}} % unicode 3A9 Ω greek capital letter omega
\DeclareUnicodeCharacter{2206}{\ensuremath{\Delta}} % unicode 2206 ∆ increment


% Prepare for tables
\usepackage{multirow}

% Prepare for lists
\usepackage{enumitem}

% Prepare for graphics
\usepackage{xspace,graphicx}

\raggedbottom
