\chapter{SÄNDARE}

För att översiktligt beskriva en komplicerad
sändare eller transeeiver behövs ibland ett
enklare framställningssätt än detaljrika
principscheman. Då kan ett blockschema
vara till stor hjälp.
Hela apparaten kan ses som ett antal
funktionsblock. Hur de samverkar framgår i
stort av blockschemat. Där återfinns oscillatorer, blandare, förstärkare etc. f schemat
kan även finnas uppgifter om frekvenser
och spänningar m.m.
Det finns olika slags funktionsblock kretsar. Kombinationen av block ger apparater med olika egenskaper. Exempel är
s.k. raka sändare med samma frekvens
genom hela sändaren, superheterodynsändare där frekvensblandning används,
frekvensmultiplicerande sändare etc.

Rak sändare
Bild Ii 5-1

Den raka sändaren är det enklaste sändarkonceptet Då är oscillatorns frekvens samma som sändningsfrekvensen och ingen
frekvensomvandling sker i signalvägen. Om
en antenn kopplas till oscillatorn så blir den
en enkel enstegs sändare.

Oscillator

Bild II 5-1 Enstegs sändare

l>
Ose

Buffertsteg

l flerstegs raka sändare följs oscillatorn
av ytterligare funktioner på samma frekvens
som oscillatorn. Buffertsteg, drivsteg och
slutsteg kan vara sådana funktioner.
Bild 115-2
Bilden visar en rak sändare, som består
av oscillator+ buffertsteg 1 + buffertsteg 2 +
drivsteg +effektförstärkare.
Oscillatorn följs av ett avlastande buffertsteg 1. På så sätt blir oscillatorns frekvensstabilitet bättre. Buffertsteg 2 avlastar
ytterligare och matar dessutom ett effekthöjande drivsteg, som ger driveffekt till
slutsteget, samt slutsteget där den slutliga
effekthöjningen sker.
Raka sändare kan användas för CW,
FM, PM och AM, men inte DSB och SSB.
Fördelen med raka sändare är enkelheten.
Nackdelen är att alla steg arbetar på samma
frekvens, varvid risken för återverkan på ett
föregående funktionssteg är större. Oönskad återkoppling kan då bli följden. Genom
att i första hand bygga in VFO och buffertstegen i metallkapslingar, s.k. skärmar, så
minskas denna risk.

Sändare med frekvensmultiplicering

Helst väljer man en arbetsfrekvens för oscillatorn där den är mest frekvensstabiL
Om högre frekvens önskas på nyttasignalen, så kan man t. ex. multiplicera
oscillatorfrekvensen. l olinjära kretsar alstras övertoner, som ofta utnyttjas i detta
syfte.
Endast när kravet på frekvensstabilitet är
lågt används den frekvens, som VFO eller
CO arbetar på, även för nyttosignalen.

l>

l>

l>

Drivsteg

PA

Bild II 5-2 Flerstegs rak sändare

115-1

SÄNDA R
f Q =8.055...... MHz
sving = 55.55 .. Hz

ca

fo

Sändarirekvens f s= n· fQ

fs=435 MHz:

n =2 · 3 · 3 · 3 = 54

sving

=3000 Hz

Bild II 5-3 FM-sändare med frekvensmultiplicering
Bild II 5-3
Oscillatorn svänger här på en låg frekvens,
som multipliceras i olinjära förstärkarsteg till
en hög sändningsfrekvens. Oftast multipliceras frekvensen två eller tre gånger i vart
och ett av förstärkarstegen.
Bilden visar ett blockschema för en FMsändare för435 MHz (70 cm-bandet). Oscillatorfrekvensen är 8.056 MHz. l fyra av de
efterföljande förstärkarna multipliceras frekvensen 2, 3, 3 respektive 3 gånger, alltså
totalt 54 gånger. Sändningsfrekvensen blir
då 8.056 · 54 = 435 MHz.
Variationer i oscillatorfrekvensen blir också multiplicerade. l detta exempel blir sändningsfrekvensens deviation 54 gånger större än oscillatorfrekvensens deviation. En
deviation av max3000Hz från den nominella sändningsfrekvensen motsvaras av följande deviation från oscillatorfrekvensen,
!J. f=

3000
54

= 55. 6

FM-sändare för VHF, UHF och SHF utförs ofta med frekvensmultiplikation. Jämfört med en rak sändare är komponentbehovet större, men i stället ger den låga oscillatorfrekvensen god frekvensstabilitet, vilket
är en fördel. Risken för oönskade självsvängningar är mindre i en frekvensmultiplicerande än i en rak sändare, eftersom inoch utgångsfrekvenserna i flera av stegen
är olika.

De frekvensmultiplicerande stegen i bild
Il 5-3 arbetar i klass C, d.v.s. olinjärt, vilket
medför amplituddistorsion. Vid frekvens-och
fasmodulering saknar emellertid detta betydelse, eftersom amplituden i det fallet inte är
informationsbärande. Övertoner i nyttasignalen bör dock filtreras bort.

[H z]
Sändningsfrekvens

eller

=

fs =
Sändningsfrekvens
fs

=
=

kristallfrekvens - VFO-frekvens
fq -

fVFO

kristallfrekvens -

+

fq

fvFO

HD
fVFO

=

5 -

\

5,5 Mhz

14 -

14,5 MHz

4 -

3,5 MHz

och
VFO

Bild II 5-4 2-bands C W-sändare med frekvensblandning

115-2

VFO-frekvens

SÄND AR

Telegrafisändare (CW) för kortvåg
Bild II 5-4
En VFO är mest stabil på låga •ral.r\ /o,nct::>r
medan en CO har god stabilitet även på
högre frekvenser. När signalerna från dessa
blandas, bildas blandningsprodukter som är
skillnaden och summan av signalernas frekvenser. Bilden visar en telegrafisändare där
detta fenomen används för sändning inom
området 14.0-14.5 eller 3.5-4.0
beroende på passbandet i filtret efter blandaren.
Resultatet är en superheterodyn-VFO
med både variabel och stabil signaL På
bilden har valts ett filter med passband för
det övre av dessa frekvensområden.

len amplitudmoduleras i en balanserad blandare.! en sådan undertrycks bärvågen medan de båda sidbanden släpps fram. Det ena
sidbandet undertrycks med ett bandpassfilter. Denna SSB-signal flyttas till avsett frekvensband genom ännu en frekvensblandoch ytterligare filtrering.
exemplet är GO-frekvensen 9 MHz.
har frekvensområdet 5.0-5.5 MHz. Vid
blandningen fås blandningsprodukter inom
frekvensområdena 14.0-14.5 och 4.0-3.5
MHz. Genom att välja bandpassfilter kan
man sända i ett av dessa frekvensområden.
Efterföljande driv- och slutsteg utförs för
att arbeta i detta frekvensband, antingen
utan särskild avstämning- s.k. bredbandigt
utförande - eller genom avstämning på en
viss frekvens, vilket ger renaste signalen.

Telefonisändare (SSB) för kortvåg
Bild II 5-5
Bilden visar en SSB-sändare förtvå kortvågsband och bygger på sändaren i Bild 5-4.
Filtermetoden är den mest använda för
att bereda en SSB-signal. Oscillatorsigna-

Bild II 5-6
Bilden visar en SSB-sändare som liknar den
i bild 115-5. Den stora skillnaden är att signalfrekvensen kan flyttas till flera olika band
med hjälp av ännu en frekvensblandning.
Därför används fler valbara bandpassfilter.

Sändare med frekvensblandning

bandfilter
balanserad

SSB-

fs

=

14 -

14,5 MHz

HD
VFO

LF

fVFO "' 5 -

5,5 MHz

Bild II 5-5 2-bands SSB-sändare med frekvensblandning

HD

l

•

C02
'

l
l
l

l

1

--L---  ..J
14 -

14,5 MHz

l

"'•• bandomkopplare
C] f
02 valbar

T

Bild II 5-6 Flerbands SSB-sändare med frekvensblandning

115-3

SÄND AR
l en 88B-signal ligger all information i
amplituden, till skillnad från en FM-signal
där all information ligger i frekvensen. En
88B-signal får alltså inte förvrängas. Det
innebär att förstärkarstegen i 88B-sändare
måste arbeta linjärt, d.v.s. en utsignal skall
vara proportionell till insignalen i varje moment.

Pll-styrda sändare

PLL-styrning är inte ett sändarkoncept. Det
är ett sätt att styra frekvensen i en oscillator
och hålla den stabil med hjälp av en likspänning från en PLL - Phase Locked Loop vilket är en digitalt styrd krets.
En PLL kan användas t. ex. i raka sändare och heterodynsändare. l det första fallet
(bild Il 5-2) kan frekvensen i den enda ocillatorn styras av en PLL. l det andra fallet
(bild Il 5-6) kan frekvensen i någon av oscillatorerna styras av en PLL.
En närmare beskrivning av PPL-styrning
av dessa två sändarkoncept följer här.

PLL-styrd FM-sändare för 144- 146 MHz
Bild II 5-7
Bilden visar en PLL-styrd rak sändare med
en VCO (spänningsstyrd oscillator) och ett
PA (effektförstärkare).
VCO ingår som det frekvensstyrda elementet i en PLL. Utfrekvensen från VCO (ärvärdet) avläses och delas periodiskt med
talet 1O och matas in i en programmerbar
frekvensdelare. Eftersom frekvensområdet
för VCO är 144-146 MHz, kommer infrekvensen till den programmerbara delaren att
ligga i området 14.4-14.6 MHz. Delningstalet i denna delare kan progammeras in i steg
om 1 mellan talen 5760 och 5840.
Med den första delarens divisor 1O och
den andra delarens divisor inställd t.ex. på
5760, så avges ur delarkedjan en puls varje
gång som VCO har genomfört 57600 svängningar. Vid en VCO-frekvens av 144 MHz
(144000 kHz) motsvaras divisorn 57600 (=
1O· 5760) av en pulsfrekvens av 2.5 kHz ut
från räknarkedjan. På samma sätt kommer
en VCO-frekvens av 144025 kHz och divisorn 5761 O (= 1O · 5761) också att ge en
pulsfrekvens av 2.5 kHz, likaså 146 MHz
och divisorn 58400 o.s.v.

Avstämningsspänning
med överlagrad LF-

växelspänning

-·14A -14,6 MHz
LF

~

pragramarbar delare
·7 5760 till 7 5840

-o

knappsats
+

CPU

Avstämningsspänning

delare + 10
LP-filter

Bild 1/5-7 PLL-styrd FM-sändare för VHF

115-4

delad ned till 2,5 kHz
referensfrekvens

OH
co

fo =25 kHZ

referensosci Ilator

VCO-frekvensen låses alltså i intervall
om 25 kHz till närmaste delningstal, för att
uppnå en pulsfrekvens av 2.5 kHz. Om
V GO-frekvensen (är-värdet) awikerfrån det
inställda delningstalet (bör-värdet), så kommer pulsfrekvensen att bli högre eller lägre
än 2.5 kHz.
Pulsfrekvensen jämförs i en s.k. fasjämförare med en kristallstyrd referensfrekvens
som efter en delning med 1O också är 2.5
kHz. Utspänningen från jämföraren är en
likspänning, som intar ett medelvärde då
infrekvenserna är lika, men ett högre eller
lägre värde när de skiljer. Denna likspänning
används för att kontinuerlig styra V GO-frekvensen tilllikhet med börvärd et. Regleringsförloppets hastighet bestäms av tidskonstanten i ett lågpassfilter, det s.k.loop-filtret.

9 MHz

Sändningsfrekvensen regleras alltså med
styrspänningen. Med samma spänning går
det också att frekvensmodulera oscillatorn.
Det görs så, att LF-signalen från modulatorn
överlagras på styrspänningen genom additiv blandning (se sidan Il 3-63) via en kondensator. De variationer i reglerspänningen
som kommer av talet är snabbare än laopfiltrets tidskonstant Variationerna av talet
hinner därför inte uppfattas som frekvensavvikelser och blir därför inte utreglerade.
Orossel n efter laop-filtret förhindrar att moduleringssignalen kortsluts av filtrets kondensator.
Frekvensinställningen, d.v.s. programmeringen av delaren kan utföras på flera
sätt. Exempel på inställningsorgan är
tumhjuls-omkopplare, logikkretsar i kombination med en knappsats o.s.v.

70 MHz

HD
LF

% vco

co

'-

CJ

T

fG. = 61 MHz

70 MHz
bärvågstre kvens

40 -

69,5 MHz

l

-

avstämningsspänning

-

l

0,5-

30 MHz

referensoscillator

Bild 115-8 PLL-styrd SSB-sändare för kortvåg
115-5

SÄNDA R
PLL-styrd sändare för kortvåg
Bild II 5-8
Bilden visar ett avancerat koncept för en
kortvågssändare. SSB-signalen alstras på
frekvensen 9 MHz och blandas med 61 MHz
i 1:a blandaren.
Summafrekvensen 70 MHz filtreras fram
som mellanfrekvens. Den önskade sändningsfrekvensen fås genom att blanda 70
MHz MF med frekvensen från VCO och
därefter filtrera fram skillnadsfrekvensen.
VCO i detta exempel täcker frekvensområdet 40-69.5 MHz. Således blir sändarens
täckningsområde 1.5-30 MHz. För att filterfunktionen skall bli optimal, kan den delas
upp på flera valbara filtersektioner, t.ex. ett
per amatörband. Valet kan ske automatiskt
och styrt av frekvensläget på VCO.
Den absoluta ändringen mellan de två
extrema sändningsfrekvenserna är så stor
som 28.5 MHz eller 1:20. Frekvensändringen
i VCO är 29.5 MHz, men där är ändringsförhållandet mellan de extrema frekvenserna
endast 1:1.74, vilket kan täckas av en enda
VCO. Vid en lägre 2:a MF-frekvens skulle
det behövas flera omkopplingsbara VCO för
att täcka hela frekvensområdet
Exempel: Vid en MF på 9 MHz behöver
VCO-funktionen täcka 9.5-39 MHz, d.v.s.
1:4.11 , vilket är för mycket för en VCO.
SSB-signalen efter 2:a blandaren är inte
lämplig att använda i regleringsslingan i
PLL. Anledningen är att bärvågen är undertryckt i denna signal och att därför HFfrekvenserna i det resterande sidbandet
varierar i takt med de modulerande LFfrekvenserna.
l konceptet på bilden rekonstrueras bärvågen i en 1 :a kontrollblandare, genom
blandning av de två GO-frekvenserna 9 och
61 MHz. Den framfiltrerade bärvågen med
frekvensen 70 MHz blandas med VCO-frekvensen i 2:a kontrollblandare och ur denna
signal framfiltreras den rekonstruerade
bärvågen. Denna stämmer perfekt med den
undertryckta bärvågens frekvens och innehåller inga LF-signaler. Bärvågsfrekvensen
delas i en programmerbar frekvensdelare
och jämförs med frekvensen från en kristallstyrd referensoscillator CO. Ur fasjämföraren erhålls en likspänning som styr VCO via
ett loop-filter. Frekvensen ställs in genom
115-6

att programmera delaren i PLL.
l en modern sändare finns ofta en mikroprocessor, som erbjuder talrika möjligheter
bl. a. till frekvensinställning, minnen och avsökning av frekvenser.
Det beskrivna konceptet är avancerat.
Frekvensen i alla oscillatorer styrs av samma referensoscillator. Frekvensstabiliteten
beror alltså enbart på referensoscillatorns
stabilitet.
Omkopplingen mellan LSB och USB kan
göras antingen genom att behålla SSBfiltret och ändra frekvensen 9 MHz med ett
värde så att filtret blir verksamt i det motsatta sidbandet eller genom att behålla frekvensen 9 MHz och byta till ett SSB-filter som
är verksamt i det motsatta sidband et.
En PLL-styrd sändare har både kristalloscillatorns stabilitet och variabel frekvens
över ett stort frekvensområde trots ett litet
antal styrkristaller. En sådan sändare kan
relativt enkelt styras digitalt.
En principiell nackdel med alla sändare
med PLL-oscillator är fasbruset En annan
nackdel är den stora komponentmängden
(sidan 114-11).

SÄNDARE

l
En transeeiver - transmitter receiver - är
både en sändare och mottagare med delvis
gemensamma funktioner. Dessa kan t.ex.
vara oscillatorer, signalbehandlingskretsar,
filter, strömförsörjning o.s. v., vilket innebär
besparing av ingående komponenter, men
också vissa funktionella begränsningar.
Transceiverkoncept är numera vad som
används allra mest av radioamatörer. Eftersom man på olika vis önskar sig så många
sändar- och mottagarfunktioner som möjligt
inom samma skal, så kan det vara svårt att
undvika kompromisser. Så kan t. ex. en
specialiserad, separat mottagare ha bättre
eller fler egenskaper än i en transceiver.

Jämförelse mellan stationskoncept
Bild II 5-9 visar i stort en station med skilda
sändar- och mottagarfunktioner, men att
antennen är gemensam.
Bild II 5-1 O visar i stort en transeeiver där
VFO och antenn är gemensamma, men i
övrigt med skilda funktioner.
Bild II 5-11 visar samma transceiver, men
med ett mer detaljerat blockschema.

Bild II 5-9 Separat sändare och mottagare

l

l

l

l

.··~
ans l utntng
for fl

extra VFO
1
l

,.,.....

.
l

L--------------.J

Bild II 5-1 O Transeeiver med samma VFO

Bild II 5-11 Direktblandad transeeiver med gemensam VFO

115-7

CW-transceiver med direktblandare

Bild II 5-11
Bilden visar en enkel transeeiver för telegrafi. Sändaren är en raksändare och mottagaren arbetar med direktblandning. För ikanaltrafik räcker det med en gemensam
VFO för sändning och mottagning. Om motstationen svarar exakt på sändningsfrekvensen, vilken ju är VFO-frekvensen, så
erhålls svävningsnoll i mottagaren. För att
få hörbara morsetecken är mottagaren utrustad med RIT, som ändrar VFO-frekvensen med ca 800 Hz vid mottagning.
l konstruktionen finns en anordning kallad KOX (Key Operated Xmitter). Denna
kopplar om transeeivern till sändning när
telegrafnyckeln trycks ner och till mottagning igen efter en viss tid sedan nyckeln har
släppts upp. Telegrafnyckeln styr också en
tongenerator som ljuder i takt med de sända
morsetecknen, s.k. medhörning.
Denna transeeiver är utförd för endast
ett frekvensband och i övrigt mycket enkel.

Kristallstyrd FM-transceiver för VHF

Bild 115-12
Bilden visar en kristallstyrd FM-sändare med
frekvensomkopplare för kanalval inom 144146 MHz-bandet.
En kristallfrekvens av c:a 12 MHz multipliceras 12 gånger i en kedja av förstärkarsteg
för att ge sändningsfrekvensen. Bilden visar
räkneexempel för två frekvenskanaler. Det
frekvenssving i oscillatorn, som alstras av
modulatorn, multipliceras också med 12.
För ett sving av 3 kHz på bärvågen är
svinget på oscillatorn bara 250 Hz.
Efter mikrofonförstärkaren följer en amplitudbegränsare, som skall hålla deviationen inom ett givet maxvärd e, oavsett signalstyrkan från mikrofonen. Därefter följer ett
lågpassfilter, som dels dämpar de övertoner
som uppstårvid amplitudbegränsningen och
dels begränsar de höga frekvenserna i den
modulerade signalen. Båda åtgärderna begränsar bandbredden.
Mottagaren är en dubbelsuper. Den mottagna signalen passerar genom ett förselektionsfilter och en H F-förstärkare för att
i 1 :a blandaren blandas med en lokal signaL

Kanalomkopplare på i: simplextrafik på 145,500 MHz
Kanalomkopplare på 6: relätrafik på 145,025/145,625 MHz
145,SOOMHz : 12 ::::
12,125 MHz:

ro :
@

12 MHz

36 MHz 72 MHz

144 MHz

~~--------------~

r-iDH!

145,025 MHz : 12 :: 1
12, 0854 MHz
1

sving

l

l

l
l

r-------------------------------------------~

l

l
l

l

156,200 MHz : 12 =
13,0167 MHz
l

r-iD ,,
r-iDH

156,325 MHz : 12
13,0271 MHz

=

D 11,155 MHz

T

Bild II 5-12 Kristallstyrd 6-kana/s FM-transceiver för VHF

115-8

mellanfrekvensen i en UH F-mottagare väljas ytterligare tre gånger högre. Den relativt
låga 2:a mellanfrekvensen medger en god
närselektering redan med enkla bandfilter.
En eventuell MF-förstärkare ger tillräcklig
signalstyrka till FM-demodulatorn.
För denna lösning behövs det två styrkristaller för varje frekvenskanal, vilket av
kostnadsskäl kan vara en nackdel.

En kristallstyrd lokaloscillator med efterföljande frekvensmultipliceringssteg alstrar
denna signal.
Lokaloscillatorkedjans utfrekvens läggs
1O. 7 MHz över eller under mottagningsfrekvensen och mellanfrekvensen efter den i :a
blandningen blir då i 0.7 MHz. Skilda oscillatorer används vid sändning resp. mottagning varför styrkristalle rna för sändning resp.
mottagning på en given kanal får olika frekvens. Vid omkoppling till en annan kanal
väljs ett annat kristallpar, vilket lämpligen
sker med samma omkopplare.
Den relativt höga i :a mellanfrekvensen
i O. 7 MHz ger ett så stort avstånd till spegelfrekvensen, att bandbredden i förselektionsfiltren är tillräckligt smal för att undertrycka
spegelfrekvensen. Av samma skäl bör i :a

Pll=styrd FM=transceiver för VHF
Bild II 5-i 3
Den PLL-styrda sändare som redan beskrivits i bild 115-7 har här kompletterats med en
svingbegränsare och ett lågpassfilter i modulatorn. Liksom i den station med kanalkristalier, som beskrivits i bild Il 5-12, är mottagaren även i detta fall en dubbelsuper.

VCO-frekvens vid sändning
144 - 146 M Hz
VCO-frekvens vid mottagning 154,7 - 156,7 MHz

begränsare

M

D--{BD--[§}i~

delad
referensfrekvens 2,5 kHz

T

11,155MHz

Bild 115-13 PLL-styrd FM-transceiver för VHF

115-9

TRANSC
VCO används även som lokaloscillator i
mottagaren. Eftersom sändaren och mottagaren skall användas på samma frekvens
(simplextrafik), måste i detta koncept VCOfrekvensen vara olika vid sändning och mottagning. Eftersom mottagarens mellanfrekvens MF är 1O. 7 MHz måste nämligen VCO
ligga 1O. 7 MHz högre eller lägre vid mottagning än vid sändning. Vid sändning däremot, är VCO-frekvensen densamma som
sändningsfrekvensen.
Den programmerbara frekvensdelaren i
PLL-kretsen arbetar därför med olika delningstal vid sändning resp. mottagning. Inställningen av divisorn kan ske med kanalomkopplare, tumhjulssats, knappsats eller
"VFO-ratt" + digitalräknare o.s.v .. PLLstyrningen ger dessutom möjligheter, t.ex.
att ordna en automatisk avsökning över ett
önskat frekvensområde- s.k. scanning.
Sändning
QRG
Del n.MHZ
tal

Mottagning
QRG
vco
MHz
MHz

Simplexkanaler,
exempel
144.000 5760
144.025 5761

144.000 154.700 6188
144.025 154.725 6189

Repeaterkanaler,
exempel
145.000 5800
145.025 5801

145.600 156.300 6252
145.625 156.325 6253

Deln.tal

VCO-frekvensen är lika vid sändning och
mottagning medan delningstalet bestämmer
arbetsfrekvensen.

115- 1o

Kortvågstransceiver för SSB och CW
Bild II 5-14
Vi har redan beskrivit en KV -sändare och
KV-mottagare för SSB. l det koncept på en
kortvågstransceiver, som visas här, ingår
en super-VFO i signalberedningen. VFOsignalen (5 - 5.5 MHz) blandas med signalen från en kristallstyrd CO, vars frekvens är
valbar med en bandomkopplare. Samtidigt
kopplas ett bandpassfilter in efter blandaren
i super-VFO, som svarar till det aktuella
frekvensband et.
För t.ex. 21 MHz-bandet är VFO-filtrets
passband 12-12.5 MHz. När en VFO-signal
12-12.5 MHz blandas med en 9 MHz SSBmodulerad signal erhålls en frekvens i området 3-3.5 MHz och en frekvens i området
21-21.5 MHz. Den önskade av dessa frekvenser filtreras fram med omkopplingsbara
bandpassfilter, vilket sker med den bandkopplare som nämnts tidigare.
l den enkla kortvågssändare som beskrivits tidigare är det tillräckligt med en enda
sats av omkopplingsbara bandpassfilter. Det
större antalet filter i den här beskrivna utrustningen behövs för att även kunna använda super-VFO som en del i mottagaren,
vilken arbetar som enkelsuper. Eftersom en
MF på 9 MHz används även i mottagaren
kommer mottagning och sändning att kunna
ske på samma frekvens.
Mottagaren beskrivs inte närmare. Med
lämpliga omkopplingsanordningar kan vissa funktionsblock i transeeivern användas
både vid mottagning och sändning. Bilden
visar en SSB-transceiver där passbandfilter
i förkretsar, M F-filter och kristalloscillatorer
har dubbel användning. Funktionsblocken
visas inplacerade i sina alternativa funktioner, däremot inte omkopplingsanordningarna.
Vid sändning och mottagning av CW
förbikopplas den balanserade modulatorn
och kristallfiltret i signalbehandlingskretsarna för 9 MHz. För mottagning av CW ändras
BFO-frekvensen i mottagaren så att det
hörs en svävningston när en bärvåg tas
emot. Utan denna frekvensändring skulle
endast bärvågsbruset höras.
Även en RIT och en VOX (Voice Operated Xmitter, talstyrd sändnings-/mottagningsomkoppling) är inritade.

EIVER

9 MHz -

SSB-signalbehandling'

LSB USB CW

LF

omkopplingsbart
bandpassfilter

t

!
D

T

sändning

S1 -

S 5 gangade

( =manövreras samtidigt)
mottagning

omkopplingsbara
kristaller

* ,** ,***

Blockschema
Dessa delar är gemensamma för både sändning och mottagning, men visas med
båda placeringarna exkl. omkopplingsanordningar

Bild 115-14 SSB-transceiver för kortvåg

115- 11

EPT

9 MHz
SSB-

70 MHz-

M

*****

----

t
VCO 40-69,5 MHz
avstämningsspänning

för frekvensval

D

T
1-------------- för 3:e MF för FM

.--------l .A. GC

*****
D

*·**·***·****•*****:

T

61 MHz

Dessa delar är gemensamma för både sändning och mottagning, men visas med
båda placeringarna exkl. omkopplingsanordningar

Bild 115-15 PLL-styrd SSB-transceiver för kortvåg

115- 12

D

T

TRANSCEIVER
PLL-styrd kortvågstransceiver

Bild II 5-15
En modern transeeiver i den högre prisklassen, i s.k. "all-mode"-utförande, erbjuder många funktionella möjligheter. Flera av
dem kommer emellertid endast till användning i speciella situationer. Konceptet
för en sådan transeeiver beskrivs här i stort.
Huvudprincipen för signalbehandlingen kan
beskrivas som en PLL-styrd dubbelsuper.
SSB-signalen bereds på 9 MHz-nivån och
flyttas därefter upp till70 MHz-nivån genom
frekvensblandning och filtrering. De möjliga
sändningsfrekvenserna mellan 0.5 och 30
MHz skapas genom att blanda den fasta
SSB-signalen med en variabel frekvens från
VCO. Den steglösa frekvenstäckningen som
innefattar mellanvågs- och kortvågsområdet är emellertid endast avsedd för mottagningsfunktionen i transceivern. För sändningsfunktionen kan tillkomma blockeringskretsar, som förhindrar sändning utanför
tillåtna frekvensband.
Denna förenklade beskrivning omfattar
inte kristalloscillatorerna för 9 och 61 MHz i
fasregleringskretsen och inte heller SSBmodulatorn, FM-modulatorn och anordningarna för CW-sändning.
Mottagaren är en dubbelsuper med hög
1 :a M F-frekvens. Mottagare för höga frekvenser kan till och med utföras som en
trippelsuper. Samma bandpassfilter, blandare och kristallfilter används både vid sändning och mottagning.
Genom lämplig programmering av
frekvensdelaren kan sändning och mottagning ske på samma frekvens eller på
skilda frekvenser (split-trafik).
En extra VFO-funktion kan åstadkommas genom att frekvensdelaren programmeras med delningstal som hämtas från ett
digitalt minne. Den extra VFO-funktionen
kan sedan efterjusteras genom att ändra
delningstalet med frekvensratten. Minnet
blir ännu mer användbart, om det förutom
frekvenser också kan lagra uppgifter t.ex.
om sändningsslag och andra inställningar.

Sammanfattning

Till skillnad från den raka sändaren är den
här beskrivna PLL-styrda transeeivern
mycket komplicerad. Den tekniska utvecklingen går fort. Nya, bättre och mer invecklade apparater utvecklas ständigt. Men det
är inte alls nödvändigt att använda det senaste och mest avancerade inom apparattekniken för att utöva amatörradio. Det går
mycket bra att börja med enkla medel och
med liten ekonomisk insats.
Det finns ett stort utbud av begagnade
apparater som i olika avseenden är konkurrenskraftiga med senare konstruktioner. Det
ligger i amatörradions traditioner att ta tillvara tillgänglig utrustning och förbättradenna
efter bästa förmåga.
Ytterst beror resultatet och framgången
mest på radiooperatörens skicklighet, val av
frekvens, antenn och tillfälle.

115-13

TRANSC

115-14


