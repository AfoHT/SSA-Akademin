\chapter{VÅGUTBREDNING}

l
Elektromagnetisk vågutbredning är energitransport och förutsättningen för all radiokommunikation. Radiovågornas utbredning
på vägen mellan sändare och mottagare
påverkas emellertid på många sätt. Med
vetskap om radiovågornas utbredningssätt
kan man mer metodiskt försöka uppnå önskade radioförbindelser.

\section{Kraftfälten omkring antenner}

För att sända ut och ta emot radiovågor
behövs antenner. Mycket förenklat är en
antenn en elektrisk krets, som består av en
induktor och en kondensator.
Med kondensatorns elektroder helt isärdragna och förminskade har svängningskretsen fått ett mycket annorlunda mekaniskt utseende. Sedan induktorn i LC-kretsen tagits bort, så återstår mekaniskt sett
endast en enkel ledare, men elektriskt sett
finns kretsen ändå kvar. Ledaren med sin
utsträckning är fortfarande en induktor och
ytorna på dess motstående halvor är fortfarande elektroderna i kondensatorn med omgivningen som dielektrikum.
En elektrisk ledare, en stång, tråd etc. är
alltså en elektrisk svängningskrets, vars resonansfrekvens mest bestäms av längden
och tjockleken. Ledaren (antennen) kan kallas dipol- den har två poler. Se Bild II 7-1.
Vissa likheter finns mellan en mekanisk
pendel och en elektriskt svängningskrets.

Mekanisk pendel
Energin i en mekanisk pendel växlar mellan
två ytterlighetstillstånd. Det ena är när pendeln just vänder i ett ytterläge. Då innehåller
den mest lägesenergi och ingen rörelseenergi. När pendeln rör sig mot mittläget, så
omvandlas lägesenergin till rörelseenergi. l
mittläget, som är det andra ytterlighetstillståndet, innehåller pendeln mest rörelseenergi och ingen lägesenergi etc ..
Elektrisk svängningskrets
Den elektriska svängningskretsen kan sägas motsvara den mekaniska pendeln i den
meningen att det i båda fallen pågår en
ständig pendling mellan lägesenergi och rörelseenergi. Se Bild II 7-2
När strömmen i den elektriska svängningskretsen just upphört för att vända, så
innehåller kondensatorn mestladdning, d. v .s.
det starkaste elektriska fältet mellan elektroderna. Detta fält är lägesenergi. Den utjämningsström som följer, från den ena elektroden över till den andra, omges av ett
magnetiskt fält. Detta fält är rörelseenergi.
Förloppet visas i Bild II 7-2, där det framgår att dipolen omges av det starkaste elektriska fältet vid tidpunkten t=O samt vid t=1/
2T med omvänd polaritet, där T är periodtiden. Vidare att dipolen omges av det starkaste magnetiska fältet vid tidpunkten t=1/
4T samt vid t=3/4T med omvänd strömriktning och fältpolaritet

+

~~+

LTSluten
resonanskrets

Di pol

Bild II 7-1 Från sluten LC-krets till antenn

117- 1

VÅ

-

Max E- fält
Min H- fält

Min E- fält
MaxH-fält

Min E- fält
MaxH-fält

Bild II 7-2 Pendlingen mellan E-fält och H-fält
Med förklaringen av E- och H-fälten som
bakgrund följer nu en enkel framställning av
hur radiovågor uppstår ur dessa fält.
Maxwell påvisade i sina ekvationer bl. a.
sambandet mellan elektroner i rörelse i en
ledare och elektromagnetiska vågor i rummet. Vidare, att elektroner som rör sig med
avtagande eller tilltagande hastighet avger
elektromagnetisk energi.
Hur energi strålar från en ledare kan förklaras med en (tänkt) elementär dipol, som
genomflyts av växelström (Bild II 7-3).

-(])Bild II 7-3 Elementär dipol
Di polen består av två lika stora elektriska
laddningar med motsatt polaritet. När
len matas med en växelström, så rör sig
laddningarna ständigt, omväxlande emot
respektive ifrån varandra. Tänk på två kulor
i var sin ände av en spiralfjäder. Avståndet
mellan laddningarna ändras i takt med styrkan och riktningen på strömmen. Systemet
är alltså under ständig hastighetsändring
(ökning resp. minskning), vilket är förutsättningen för att energi skall strålas ut.
Först är laddningarna nära varandra på
grund av liten laddning. Vid ökande ström
ökar avståndet mellan laddningarna och det

117-2

byggs upp ett mer utbrett och energirikt Efält. Samtidigt byggs även ett H-fält upp
omkring dipolen, vinkelrätt mot E-fälteto.s.v ..
Detta gäller både för en elementär di pol och
en elektrisk ledare med många fria elektroner (verklig antenn).
Formeln för det resulterande s-fältet är
E. vilket visar att den lagrade energin
i di polens närmaste omgivning ökar när avståndet (potentialen) mellan dipolens laddningar ökar.
Bild II 7-4 visar hur ett E-fält byggs upp
omkring en dipol och avskiljs från den. De
visade kraftlinjerna är E- fältet. H-fältet visas
inte, men ligger vinkelrätt mot E-fältet, i cirklar omkring antennen. Se Bild II 7-5.
När dipolens laddningar ändrar riktning
och åter börjar att röra sig emot varandra,
börjar det E-fält som byggts upp att också
byta riktning. Men det kommer inte att falla
tillbaka till dipolens mitt utan sluts till ett eget
kmt!=:lcmn- Maxwells första ekvation. Jämför
med en såpbubbla som lämnat blåsröret.
Omkring dipolen har det nu bildats ett självständigt E-fält, som sin tur alstrar ett eget Hfält.
En period av en elektromagnetisk våg
(ett S-fält) har alstrats och fortsätter att utFörvarje följande period alstras ett
som separeras från antennen och
bildarett
H-fälto.s.v .. Varjegångbildas
alltså en ny "fältbubbla" inne i den föregående, vilken håller på att utvidgas.
Resultatet är ett elektromagnetiskt fält,
d.v.s. en radiovåg.

s=

VÅ UTBREDNING

Bild II 7-4 Ett självständigt E-fält skapas

Bild II 7-5 E-, H- och B-fälten omkring en antenn (förenklad framställning)
Som nämts består en radiovåg av ett högfrekvent elektromagnetiskt fält (S). Det är i
sin tur sammansatt av två andra fält, det
elektriska E- och det magnetiska H-fältet.
Energin i S-fältet fördelas lika mellan E-fältet
och H-fälten, vars krafter korsar varandra
vinkelrätt. S-fältet ligger i plan med både Eoch H-fälten och breder ut sig vinkelrätt mot
dem. s-fältets riktning beror av den inbördes
riktningen på E- och H-fälten.
När E-fältet är vertikalt, sägs vågen vara
vertikalt polariserad. När samma fält är horisontellt sägs vågen vara horisontellt polariserad. När E-fältet roterar i vågfrontens
plan, och därmed även H-fältet, sägs vågen
vara cirkulärt polariserad.

Fälten framställs i text och bild som s.k.
kraftlinjer med pilar som föreställer kraftriktningen. Linjernas längd föreställer fältets
styrka. Bild II 7-6 visar ett avsnitt av en
vågfront S med vertikal polarisation.
E

.E.

Bild II 7-6 E-, H- och S-fält

117-3


VÅGUTBREDNING
Radiovågornas egenskaper

Ett elektromagnetiskt fält, som alstras i ett
givet tidsmoment, breder ut sig åt alla håll i
rymden likt en ständigt växande sfär.
Fältstyrkan inom ett givet avsnitt av sfärens yta sjunker därför alltefter som avståndet från sändaren ökar. Det är därför som en
sändare hörs svagare ju mera avlägsen den
är ifrån mottagaren. Jämför med ljuset från
en rundstrålande lampa.
l rymden breder radiovågor ut sig mycket
långt. Det uppstår dockäven där utbredningsförluster i materia som finns i vägen.
När radiovågorna passerar genom jordatmosfärens olika skikt uppstår mycket större
utbredningsförluster än i rymden och därmed blir räckvidden kortare.
Elektromagnetiska fält från alla slags sändare (emittörer) genomkorsar alla slags material och alstrar strömmar i dem som är
elektriskt ledande.
Radiovågorna
• breder ut sig rätlinjigt i alla riktningari rymden med ljusets hastighet som är ca
300 000 km/s (se även 6.1 ),
e
tränger igenom fasta kroppar, som inte är
elektriskt ledande,
• dämpas eller reflekteras, bl.a. av metaller, joniserade vätskor och joniserade
atmosfärskikt,
e
är polariserade,
e
förstärker eller motverkar varandra.
Radiovågorna breder ut sig
• utmed jordytan,
• upp från jordytan,
e
upp från jordytan efter en första reflexion
mot denna.
Det första sättet kallas för markvåg och
de två senare kallas med ett samlingsbegrepp för rymdvåg.
Radiovågornas riktning kan böjas av genom
• reflexion eller splittring mot naturliga
reflektorer i atmosfären och i jordytan,
e
konstgjorda såväl passiva som aktiva
reflektorer (relästationer) på jordytan och
i rymden.
Radiovågorna kan dämpas
• i jordytan,
• i topografin,
e
i atmosfärsskikten.

117-4

~@~

EPT

Vågutbredningens natur är mycket sammansatt och kan inte enkelt beskrivas. Några starkt påverkande faktorer på vågutbredningen kan ändå urskiljas, t.ex.
• utbredningsvägens höjd över jordytan,
• radiovågens frekvens,
• solstrålningens jonisering av jordatmosfären,
• väderförhållandena..

Olika slags vågavböjning

Olika faktorer påverkar vågutbredningen
inom olika avsnitt i frekvensspektrum. Här
följer de viktigaste:

Reflexion
Reflexion innebär att vågorna böjs tillbaka
från den yta som de träffar. Ljus- och radiovågor reflekteras på samma villkor eftersom
att båda är elektromagnetiska till sin natur.
Den stora skillnaden är vågfrekvensen.
Reflektorns storlek uttrycks i termer av
antal våglängder vid den aktuella frekvensen. En 80-metersvåg reflekteras inte bra
mot en yta med bara någon meters sida.
Däremot reflekteras en 2-metersvåg mycket bättre mot en lika stor yta och en ljusvåg
7.7 J.lm) ojämförligt
(med våglängden 4
mycket bättre.
Olika materials förmåga att reflektera en
infallande radiovåg beror av vågens frekvens samt av materialets tjocklek och elektriska ledningsförmåga. Vågen tränger djupare in i materialet vid låg frekvens respektive vid låg ledningsförmåga.

a

Refraktion
Refraktion (brytning) innebär att vågen ändrar riktning, när den passerar gränsen mellan två media eller material med olika ledningsförmåga. När ledningsförmågan ändras successivt t.ex. i ett atmosfärskikt, blir
vågens avböjning mjuk.
Diffraktion
Diffraktion innebär att vågens infallsriktning
splittras upp i flera nya riktningar, närvågen
passerar nära över ett hinder. Det är p.g.a.
detta fenomen som radiosignaler i viss mån
kan höras även bortom en berg rygg. Diffraktionen tilltar med minskande frekvens.

VÅ UTBREDNING
Jonosfärskikten

Jonosfären har fått namnet från begreppet
jon, som är en fri elektron eller annan laddad
partikel. Jonisering- elektrisk uppladdningav jordatmosfären sker mellan c:a 40 till400
km över jordytan. Där är lufttrycket tillräckligt
lågt för att joner skall kunna röra sig fritt
under avsevärd tid utan att kollidera och
återförena sig till neutrala atomer.
När en radiovåg passerar genom ett
joniseratskikt i atmosfären, kanvågen ändra
riktning, vilket kallas för refraktion. För att
refraktion skall uppstå måste i första hand
två villkor uppfyllas, det är tillräckligt tät jonisering och tillräckligt lång våglängd. Under
"gynnsamma" omständigheter kan vågorna
till och med böjas av ner mot jorden, vilket är
den viktigaste förutsättningen för långväga
radioförbindelser på kortvåg.
Joniseringen av atmosfären är emellertid
oregelbunden och varierar bl. a. med höjden
över jordytan, solinstrålning, tidpunkt m.m.
Ett antal joniserade skikt kan definieras.
Se Bild II 7-7.

D-skiktet
D-skiktet förekommer under den ljusa delen
av dygnet på en höjd av c:a 50-90 km. På 7090 km höjd orsakas joniseringen huvudsakligen av röntgenstrålar från solen, medan
den kosmiska strålningen har störst påverkan på 50-70 km höjd. D-skiktet dämpar de
infallande radiovågorna, med största verkan
i kortvågsområdets lågfrekventa del och under de ljusaste timmarna under sommaren.
D-skiktet har dålig reflexionsförmåga och
verkar hindrande på långdistansförbindelser.

E-skiktet
E-skiktet (Kenelly-Heaviside-skiktet) är det
lägsta reflekterande jonosfärskiktet Det förekommer på en höjd av c:a 90-140 km och
är mest koncentrerat på c:a 11 Okm höjd. Eskiktet alstras av att ultraviolett strålning
joniserar syreatomer. Skiktet reflekterar vågor bäst i kortvågsområdets lågfrekventa del
och är kraftigast under den ljusa delen av
dygnet. På grund av D-skiktets dämpande
verkan under de ljusaste timmarna är Eskiktet mest användbart under grynings- och
skymningstimmarna.
Ett säsongmaximum i reflexionsförmågan inträffar under sommaren. Förbindelseavstånd på upp till 2000 km är möjliga.
Mögel-Dellinger-effekten
Strålning från gasutbrott på solytan kan jonisera D-skiktet så kraftigt, att alla radiovågor
med frekvenser över c:a 1 MHz dämpas helt.
Radiotrafik som baseras på vågutbredning
via jonosfären är då omöjlig att genomföra
under en tidsrymd av ett antal minuter upp till
flera timmar- det blir "black out".
Sporadiska E-skiktet
Den starkare solinstrålningen under sommaren orsakar en kraftigare jonisering i den
lägre jonosfären än under vintern. Inom Eskiktet bildas då sporadiska tunna molnlika
partier med mycket hög joniseringsgrad och
stor reflexionsförmåga, det s.k. sporadiska
E-skiktet (E 8 ). Vågutbredningen via Es är
mycket olika på olika latituder och är bäst
omkring 40:e breddgraden. Mycket goda
långväga förbindelser kan uppnås.

F2 -skikt

km

Avståndet jordyta - skikt icke måttriktigt avbildat

Bild II 7-7 Jonosfärskikten

117-5

VA

UTB EDNIN

F-skiktet
F-skiktet är det högst liggande jonosfärskiktet
Det förekommer såväl dag- som nattetid på
en höjd av 140-500 km. Den nedre del av
skiktet, 140-200 km, uppvisar andra variationer än den övre delen. F-skiktet beskrivs
därför som två skikt, F1 upp till ca 200 km
höjd och F2 över denna höjd.
Liksom E-skiktet, påverkas F1-skiktetkraftigt av instrålningen från solen. Det når sin
högsta joniseringsgrad ungefär en timme
efter högsta lokala solstånd och förekommer
endast under sommaren. Under natten förenar sig F1- och F2-skikten till ett enda Fskikt.
F2 -skiktet är det skikt som varierar mest i
tiden och rummet. Den högsta joniseringsgraden inträffar vanligen sent efter högsta
lokala solstånd, ibland under aftontimmarna.
Skiktets maximala jonisering är på 250-350
km höjd på mellanlatituder och på 350-500
km höjd vid ekvatorn. På mellanlatituder
ligger den största elektrontätheten i skiktet
högre under natten än under dagen. Vid
ekvatorn är förhållandet omvänt.
Reflexioner i F2 -skiktet möjliggör att stora
avstånd kan överbryggas (1 hopp = 30004000 km). Bild 117-8 omjonosfärsutbredning.

PT
Höjd till reflekterande skikt
När en radiovåg, som riktas rakt uppåt, träffar jonosfären kan den antingen
• absorberas, sugas upp,
• reflekteras,
e
tränga igenom.
Vilket som inträffar beror på den använda
frekvensen. Ju högre frekvensen är på den
uppåtriktade radiovågen, desto högre upp i
ett atmosfärskikt kommer avböjningen tillbaka att inträffa. Höjden till skiktet beräknas
ur radiovågens utbredningshastighet och
utbredningstid fram och åter mellan skiktet
och jordytan.
Kritisk frekvens
Vid en viss övre frekvens upphör reflexionen
i atmosfärskiktet och vågen går ut i rymden
i stället för ner till jordytan. Denna frekvens
kallas den kritiska frekvensen, som varierar
med joniseringsgraden i atmosfären. Den
kritiska frekvensen är högst vid högt solfläckstal, såväl i E- som i F-skikten, eftersom
joniseringsgraden då är störst. Den kritiska
frekvensen för E-skiktet varierar mellan c:a
1-4 MHz beroende på tidpunkt i solfläckscykeln och tid på dagen. Den kritiska frekvensen för F-skiktet varierar med tid på

Frankfurt/Main - Osaka, 9 000 km, beamriktning: NO, 3 hopp

Frankfurt/Mai n

Uralbergen

LANGA OCH KORTA VÄGEN

Avståndet jordyta - F -skiktet är icke
2
mättriktigt avbildat

Bild II 7-8 Jonosfärsutbredning

117-6

Bajkalsjön

Osa ka

EDNIN
dagen, årstid och skede i solfläckscykeln.
Den kan variera från 2-3 MHz
natten
3 MHz
under ett solfläcksminimum till 1
på dagen under ett solfläcksmaxi mu m.

Kritisk vinkel
Rymdvågen måste träffa ett joniserat atmosfärskikt med en tillräckligt flack vinkel för att
reflekteras, den s.k. kritiska vinkeln. Denna
vinkel är frekvensberoende. Allt eftersom
den utsända frekvensen ökas ytterligare över
den kritiska frekvensen, måste vågen träffa
atmosfärskiktet i en allt flackare vinkel för att
vågen skall reflekteras mot jordytan.
att sända ut vågen i mycket flack vinkel mot
F2 -skiktet kan långa distanser överbryggas
den
vid frekvenser som är upp till 3.5
kritiska frekvensen.
Så snart den kritiska frekvensen är högre
än frekvensområdet för ett amatörband är
det alltså möjligt att kommunicera över
våg på detta band. Det kan ske över alla
avstånd, allt ifrån skipavståndet till det som
avgörs av utbredningsförlusterna.
Högsta användbara frekvens (MUF)
Radiovågorna vandrar från sändaren till en
avlägsen mottagare genom att reflekteras
en eller flera gånger i jonosfären och jordytan. För detta kan frekvensen inte vara
högre än den högsta användbara frekvensen, Maximum Usab/e Frequency- MUFför en viss överföringssträcka.
MUF är högst mitt på dagen eller
eftermiddag. Allra högst är MUF under perioder av högt solfläckstal och kan då komma
upp till över 30 MHz. Under tidiga morgontimmar sjunker MUF ofta under 5 MHz.
De janesfäriska förlusterna är lägst nära
MUF och ökar snabbt under dagtid för lägre
frekvenser.
Aktuella MU F-data publiceras periodiskt
i olika media, men kan också överslagsberäknas med hjälp av speciella
gram.
Optimal trafikfrekvens (FOT)
l praktiken är det av intresse att veta det
frekvensområde där kommunikation bäst kan
genomföras.
Rekommenderad övre frekvensgräns för
en tillförlitlig radioförbindelse kallas optimal

traffic frequency -FOT- och väljs något under MUF som marginal för oregelbundenheter och turbulens i jonosfären, liksom för
korttidsavvikelser från det förutsagda månatliga medianvärdet för MUF. FOT är vaniigen ungefär 15 °/o lägre än MUF.
Lägsta användbara frekvens (LUF)
lägre sändningsfrekvens som väljs, desto
mer dämpas vågorna i jonosfären, intill den
frekvens då de inte kan uppfattas. Den lägsta användbara rekvensen Lowest Usab/e
Frequency- LUF- är den frekvens som ger
tillfredsställande kommunikation för en viss
utbredningsväg och vid en viss tidpunkt.
Vid frekvenser under LUF är mottagning
inte möjlig eftersom brusnivån då är för hög.
Ju mer frekvensen höjs över LUF, desto
bättre blir signal-brus-förhållandet.
Till skillnad från MUF, som endast påverkas av de janesfäriska förhållandena, kan
till en del påverkas genom utsänd effekt
och bandbredd. Generellt kan LUFsänkas
c:a 2 MHz för varje 1O dB ökning av E. R. P.
Vågutbredningsförutsägelser
Det görs regelmässiga förutsägelser av de
janesfäriska förhållandena. Fortlöpande fysiska observationer, statistisk och matematisk bearbetning ligger till grund för förutsägelserna, vilka bl.a. utnyttjas för att planera
radiotrafiken. Vågutbredningsförutsägelser
(propagation forecasts) görs av både civila
och militära institutioner och upplyser om de
lämpligaste frekvenserna och tiderna för olika förbindelsesträckor. Sådana förutsägelser meddelas i offentliga publikationer, men
även i andra, t.ex. tidskrifter och bulletiner
inom amatörradion.
Regional Warning Center (RWC) samlar
sol- och geofysiska data och sänder dagligen Ursigram per telex eller brev ( URSI =
Union Radio-Scientifique lnternationale).
Ursigram kan erhållas genom årsabonnemang (tyvärr till högt pris). De innehåller
aktuella mätvärden såsom solfläckstal R, i O
cm solflux F, magnetiskt index K, gränsdämpningsvärden, även anvisningar om särskilda händelser (flares,
magnetstormar, polarkalottabsorbtion, Mögei-Dellinger-effekter och liknande) liksom
korttidsprognoser och förvarningar.

117-7

:::i
l

co

<

~

et

):>o

:::::::::

)J

c:o

ffiJ

~

.g
a

cg

~
Ct
"'""\
tu

g
'"""i-

a:
~
ru

~

g.
~

~

"'O

!l>o

~

n3o

(Q

Radioprognos Juni 1997 SSN
Tid/

/GMr

1.8 MHz
000011111222
246802468024

3.5 MHz
000011111222
246802468024

SH
9H

l .. : ... llo2.

lo.: .... 1.11

41.: ••. o2222

621o ... o2244

~4

E: L
!!'
!!'G
JA
KH6
KH6-L

LU
OA
OD
py
T2

trAl
trA9

... : .... :o1o

... : .... : .o1

42o:12112426
•.. : ..•. :oo.

531oo2123546
. .. : .... :o ..

IfK
IfK-L
llU

l lo: .. 1232o.
... :o.o.l.ol
11.: .•. 1oool
••• : ••.. : •. o
o .. : .. 11: .. o
. o.: ... oo1o.

.lllo .. 11. ..
... :.o .. : ...

10 MHz
000011111222
246802468024
.o.: .... oll.
542211113455
2o.: ... o1123
3o.: .... :ool
334544554553
111: .... : .o1
•.• : .... llo.

1..: ... o1122
lolo .... :ool
.0000 ••• : •••

... : .... ooo.

ooo: .... : .. o
•.. : .... ll.o
•.. : .•. ooo .•

.o.: .... :

••••••••••••

•.• : ...• : .••

455455555554
453333445554
443212444554
442lo2334454

.00

. .o: .... : .. .

1\.ntarkt-W ••• : •••• : .••
!Ultarkt-E ... : •••. : •.•

SM 250
SM 500
SM 750
SM 1000

18 MHz
000011111222
246802468024
. .. : ... 11 ...
.0233212452.
.olll. .12o ..
•.. 11111:1..
.o222111221o

oo.: ...• : .. o

:rn

zs

14 MHz
000011111222
246802468024
.lo: .. olll ..
134o .. 355642
o21o.112331.
l .. : .... : ...
. .. o ... o121o
764221224456
222222222322
o .. : ..•. : .. o
••. :o.o.o11o
o .• :oolloo .•
oloooo .. l. .•
.o.: .... o .. .
o .. : .... ooll . . . : ... ooloo .lo: .. o.: .. .
o .. : .••. : ... llo: .... : .o1 .. lo .... : .11
21.: ••• oo123 522oo.112325 2o4llo.l2242
oo.: ••.. : .. o 11.: ..•. : .ol 1.. :oo .. ool2
. . . . . . . . . . . . . .. : .... oo.. .00000 . . oo ..
664333345667 12356665354321 122221122211
o .• : •••. 1221 loo:.ol22222 .111oo1llo .•
••. : ••.. oo .• ••• : .•. 111o.
••• : •••• :.00

W'2
W'6
lm

ZL
ZL-L

=6

7 MHz
000011111222
246802468024
... : .... :.o.
42l: •.. oo234
o .. : ..•. :ool

oo.: .... :.ol

... : .... :oo.
445455554554
454343445554
554332345664
543221234554

o11222112221
223333333432
234565554543
345566665654

ol.: .••• :ool
00.: . . . . :ooo
0000.00. :ooo
.. olo ... o1o.
122222222222
232332222332

. .. : ... o: .. .
0000000.0001
••. : ...• : •. o
•• oo .... olo.
112221112221

21 MHz
000011111222
246802468024

24 MHz
000011111222
246802468024

28 MHz
000011111222
246802468024

. . . : .0 . .

--1

m
m

:0

. .. 110 .. 22 ..

c

• .. oo ... 12 ..
.. l111o111o.

-z

• .• :1 .•• : ...

ooloo ... lo. o
• .. : .... :oo .
.o11ooo.: ...

c:

... : .... o ...
•• 0000 . . : • . .

:11.

. 2344224532.
... :l.o.ollo

.o2221o222 ..
... : .... :1 ..

.o121o.111 ..
• .. 33 ... : .. .

.. oo .... o ..•

... oo ... oo ..

.o.o .... : .. .

... : .... :1 ..

.oooo ... : ...

... : .... :.o .
.. ooooo1: .. .
. .. : ... oo .. .
llooooooooll
oo.: .•.. : .oo

... : .... :.o.

11ooooooool1

oo.: .... : .00

11oooooooo11
00.: .... : .00

1llooo1oo111
00.: .... : .oo

... : .... :o ..

Tabellen visar sannolikheten att få förbindelse för alla amatörband på kortvåg (1.8-28 Mhz) och varannan timme (02-24} GMT. Sannolikheten
anges i procent. "9" betyder 90-100 %, "8" 80-89 %, ... , "2" 20-29 %, "1" 10-19% och "o" 5-9%. Mindre an 5% markeras med"."(":" för timmarna
08 och 18). Vidare förklarina finns i QTC nr 1 1995 samt notis i QTC nr 4 1995. /SM510. Stio

~

©

VÅ UTBR DNIN
Bild II 7-9 visar en radioprognos ur SSAs
medlemstidning QTC. Ny prognos presenteras periodiskt och behandlar kortvågsspektrum. ObseNera det låga solfläckstalet SSN
(Sun Spot Number) på denna bild.
Sträckan SM - UA 1 på 1O MHz Juni -97
Klockan
Kod
Sannolikhet (0/o)
02
2
20-29
04
3
30-39
06
5
50-59
08
6
60-69
10
6
60-69
12
6
60-69
14
5
50-59
16
3
30-39
18
5
50-59
20
4
40-49
22
3
30-39
24
2
20-29

Bild II 7-1 O Detalj av radioprognos i Il 7-9

Solens inverkan på jonosfären
solaktivitet
Solen är ett gasklot, i vars inre pågår en

ständig kärnreaktion där väteatomer omvandlas till helium. Vid denna process frigörs en del av solmaterian som partikelstrålning och elektromagnetisk strålning inom ett
brett frekvensregister, bl.a. kortvågig radiostrålning, gammastrålning. solatmosfärens
yttre består av två skikt, kromasfären och
koronan. Vissa områden på solens yta har
en lägre temperatur och uppfattas som mörka fläckar - solfläckar. Från kromasfären
kastas det ut gasmassor, s.k. protuberanser, ofta från områden nära solfläckarna.
Det förekommer även kortvariga eruptioner, s.k. flares, som syns som lysande fläckar i närheten av solfläckarna. Fiares sänder
~t stark elektromagnetisk strålning och partiklar. Koronan är solatmosfärens yttersta
skikt. Från denna utstrålas partiklar i form av
atomer, elektroner och protoner, som fångas upp av jordens magnetfält och skapar
polarsken, s.k. aurora. Den ökade partikelstrålningen från fiares kan orsaka magnetiska oväder med åtföljande radiostörningar
och ökning av polarskenet. Antalet synliga
solfläckar står i samband med solaktiviteten.

Solfläckstal
Ett mått på solaktiviteten är antalet solfläckar, vilket det görs fortlöpande obseNationer på. Ur detta statistikmaterial beräknas
ett vägt solfläckstal R (Wolf-talet). Med stöd
av solobservationer under mer än 200 år har
det kunnat fastställas att solfläckstalet varierar någorlunda periodiskt mellan ungefär
200 och 5.
En solfläcksperiod varar mellan c:a 7.5
och 17 år, med ett medelvärde av c:a 11 år
-den s.k. 11-årscykeln. Vid utgången av år
1996 noterades ett så lågt solfläckstal som
5, vilket innebar slutet på cykel 22.
När cykel 23 nu börjar betyder det bättre
möjlighetertill DX på kortvåg under några år.
På senare tid har ännu en metod börjat
användas för mätning av solaktiviteten. Då
mäts styrkan av radiobruset från solen (solflux F) i våglängdsområdet 1O cm.
De båda mätmetoderna ger i huvudsak
samma tendenser och det finns ett statistiskt
samband mellan dem.
Vågutbredningen i jonosfären påverkas
av solaktiviteten. Under solfläcksmaximum
blir jonosfären starkt joniserad, speciellt Fskiktet under dagtid. Då reflekteras även
vågor med kortare våglängder mot jonosfären i stället för att passera igenom denna ut
i rymden. 20-metersbandet är då "öppet"
nästan dygnet runt, 15-metersbandet från
före gryningen till efter solnedgången och
10-metersbandet nästan varje dag till efter
solnedgången. Långa förbindelser med
mycket låga effekter är möjliga.
Under solfläcksminimum är det emellertid nödvändigt att använda avsevärt lägre
arbetsfrekvens än vid solfläcksmaximum.
20-metersbandet Jörblir t. ex. inte öppet under hela natten. Oppningar på 15-metersbandet uppstår endast tillfälligtvis och öppningar på 1O- metersbandet är sällsynta.
Goda antenner och högre effekter används
då för att i någon mån kompensera den
sämre vågutbredningen. Vid låg solaktivitet
kan de högre banden vara så tysta, att operatören kan undra om utrustningen verkligen
fungerar.

117-9

VÅGUTB
Vågutbredning på kortvåg
Markvåg
Markvågen breder ut sig längs jordytan utan
kontakt med atmosfären genom reflexion
eller refraktion.
Markvågen har vertikal polarisering och
en vertikal vågfront när jordplanets ledningsförmåga är god. Vid sämre ledningsförmåga
lutar vågfronten framåt.
Markvågens räckvidd står i forhållande
till den använda frekvensen, sändareffekten
och jordplanets ledningsförmåga.
Vid frekvenser under c:a 1O MHz är jordytan är en tämligen god ledare. Markvågsutbredning utnyttjas därför mest vid låga
frekvenser, t.ex. för rundradio i lång- och
mellanvågsbanden då räckvidden kan vara
i storleksordningen 1000 km. På kortvåg är
markvågsräckvidden i 80 m-bandet c:a 100
km och i i O m-bandet c:a 15 km.

Rymdvåg
Under vissa förutsättningar reflekteras radiovågorna mot joniserade atmosfärsskikt
och når åter jordytan på stort avstånd från
utsändningspunkten. Rymdvågsutbredning
utnyttjas mellan platser på jordytan med stort
avstånd.
För att bäst uppnå den önskade reflexionen måste man dels välja lämplig tidpunkt
och frekvens och dels utforma antennen så
att den har sin huvudriktning i en bestämd
vinkel mot det reflekterande skiktet .
Jonosfären är den del av atmosfären på
c:a 50 till 350 km höjd, där instrålningen från
solen skapar fria elektroner och joner i en
sådan mängd att det bildas skikt med god
elektrisk ledningsförmåga. Under vissa villkor reflekterar dessa skikt radiovågorna, men
kan under andra villkor även absorbera dem.

Sändningsfrekvens <övre gränsfrekvens

markvåg

Sändningsfrekvens

Bild II 7-11 Vågutbredning på kortvåg

117-10

>

övre gränsfrekvens

VÅ UTBREDNIN
När vågorna från jordytan reflekterats
mot de joniserade skikten, kan de återträffa
jordytan på ett avstånd av upp till 4000 km
från utsändningspunkten, beroende på frekvens och polarisering. Därefter kan de åter
reflekteras mot jordytan och upp i jonosfären o.s.v. (flerstegshopp). Under gynnsamma förhållanden når rymdvågen mycket
långt genom växelvisa reflexioner mellan
jordytan och jonosfären.
Död zon (skip zone) och skip-avstånd
Rymdvågorna böjs tillbaka mot jorden när
de träffar jonosfären i en vinkel som är
flackare än den s.k. kritiska vinkeln. När
vågorna träffar jonosfären med en brantare
vinkel än den kritiska vinkeln sker det ingen
avböjning utan vågorna passerar genom
jonosfären och rakt ut i rymden. Beroende
på den kritiska vinkeln för tillfället, kommer
därför reflekterade rymdvågor inte att höras
förrän på ett visst avstånd bort från sändaren. Detta avstånd kallas för skip-avstånd.
Men sändarens markvåg har också ett
visst täckningsområde och mellan detta och
zonen där rymdvågen kan höras bildar en
skymningszon-en skip zon e eller död zon.
Grålinjeutbredning -gra y Iine
Med gray Iine menas det smala bälte på
jordytan där det för tillfället råder gryning
eller skymning.
Tidintervallet för gray Iine varierar med
stationens latitud. Vid ekvatorn är det\(\pm\) 5
minuter och i Skandinavien \(\pm\) c:a 1 1/2
timme omkring tidpunkten för solens uppgång respektive nedgång.
När åtminstone en av två stationer befinner sig inom gray Iine kan kortvågsförbindelse erhållas över ett mycket större
avstånd än annars.
Kommunikation längs med gray Iine går
bäst på låga frekvenser, t.ex. på 3.5 MHz
amatörband, under det tidsintervall då Dskiktet just har börjat byggas upp (gryning)
respektive nästan har brutits ned (skymning). Då är joniseringen av D-skiktet liten
och en rymdvåg som träffar skiktet kommer
då snarare att böjas av i D-skiktet än att helt
dämpas. Vågutbredningen sker då både
genom refraktion i D-skiktet och reflexion i
E-skiktet.

Fädning eller signalbortfall
Fältstyrkan på de mottagna vågorna kan
variera kraftigt från ett ögonblick till ett annat
Fenomenet kallas fädning (e ng. fading, uttalas fejding).
Sådana interferensfenomen uppstår när
vågorna samtidigt vandrat flera vägar fram
till mottagarantennen, s.k. flervägsutbredning. När de träffar mottagarantennen kan
de vara tidsförskjutna sinsemellan, med utsläckningseffekter som följd (interferensförluster).
Andra typer av fädning är när
• polari-seringriktningen ändras p.g.a. oregelbundenheter i jonosfären (polariseringsförluster),
• överföringsvägen dämpar vågorna tidsmässigt oregelbundet (absorbtionsförluster),
• vågutbredningsriktningen ändras genom
reflexioner mot hus, bergväggar etc.
(reflexionsförluster, vid t.ex. mobil radiotrafik).

Om amatörradiobanden på kortvåg
1.8 MHz (160m):
Bandet kallas även "top-band": Räckvidden
är normalt relativt liten, nattetid undervintern
c:a 1200 km och i bästa fall några tusen km.
Men under solfläcksminimum kan räckvidden vara mycket större nattetid.
3.5 MHz (80 m):
Under dagtid är räckvidden ca 500 km och
under kvällstid 1000-1500 km. Tidigt på
morgonen under vintermånaderna, särskilt
under solfläcksminimum, är räckvidden tillräcklig för interkontinentala förbindelser (DX
= Iong distance). Under sommarmånaderna
har bandet hög atmosfärisk brus nivå. Döda
zoner förekommer normalt inte.

7 MHz (40 m):
Detta band har större räckvidd än 80 mbandet. Under dagtid har det en räckvidd av
1000-2000 km. Under natten, särskilt under
vintern, kan hela världen nås. Döda zoner är
100 km under dagen och 1000 km under
natten.

117- 11

vA
Vågutbredning på VHF, UHF, SHF

14 MHz (20m):
20m-bandet är ett säkert DX-band för stora
avstånd. Under kvällarna ökar räckvidden
på ett rymdvågshopp upp till ca 4000 km.
Särskilt gynnsam vågutbredning erhålls vid
kontakt genom en skymningszon, dvs där
den ena parten har dag och den andra har
natt. Döda zoner uppträder nästan alltid.

ocn EHF

21 MHz (15 m):
Vågutbredningen i 15 m-bandet är bäst vid
högt solfläckstaL Under solfläcksmaximum
är bandet nästan ständigt öppet för DXförbindelser.
Under solfläcksminimum är bandet i bästa
fall öppet kortare perioder på dagtid under
sommarmånaderna.
Bandet är dött nattetid. Vid reflexioner via
sporadiskt E-skikt kan avstånd av mer än
2000 km överbryggas.

På VHF och högre frekvenser (tidigare UKV)
förekommer sällan någon vågutbredning via
jonosfären annat än under tiden för maximal
sol aktivitet. l stället utnyttjas den lägre delen
av atmosfären och knappast högre än 4 5
km över jordytan. Denna del av atmosfären
kallas för troposfär och vågutbredningen
därför för troposfärisk vågutbredning.
All vågutbredning i troposfären förutsätter i princip optisk sikt. Emellertid förekommer en viss vågavböjning utmed jordytan,
varför den praktiska räckvidden utmed siktlinjen är något längre än till den optiska
horisonten. Man talar om radiohorisont.
Brytningsindex i atmosfären är en viktig
faktor för vågutbredning bortom frisiktsavståndet, speciellt vid frekvenser över 100
MHz. Även den splittring av vågorna som
uppstår när de träffar oregelbundenheter i
atmosfären kan utnyttjas för kommunikation
på avstånd som är flera gången frisiktsavståndet
Vid högre frekvenser begränsas emellertid räckvidden av atmosfärens dämpande
inverkan. likaså förloras vågenergi i den
topografi, vegetation och bebyggelse som
ligger i siktlinjen mellan sändare och mottagare. l gynnsamma fall är det dock möjligt att
överbrygga avstånd på upp till 1000 km
genom troposfären. Sådana avstånd kallas
för överräckvidd.

28 MHz (1 O m):
Bandet är lämpat för närkontakter upp till 50
km nattetid och för DX-kontakter dagtid,
dock ej dagar då E-skiktet är kraftigt joniserat
och skärmar av F-skiktet. Vågutbredningsvägen för DX är på den sida av jorden som
har dagsljus. Döda zoner på upp till4000 km
kan uppstå. Förbindelser över stora avstånd
är möjliga med låg effekt.
Under solfläcksminimum är bandet inte
användbart för DX-kontakter. Då är endast
kortvariga förbindelser på avstånd upp till
2000 km möjliga genom reflexioner via sporadiska E-skikt (short skip).
Bandet har i många fall VHF-karaktäroch
man kan ha kontakter via Aurora och andra
liknande utbredningsformer såsom AuroraE och dubbelt hopp på Auroraringen.
10, 18 och 24 MHz:
Vågutbredningsegenskaperna i dessa senast tillkomna amatörrradioband är ett mellanting av respektive närmast angränsande
amatörradioband.

117-12

Allmänt
Frekvensområdet 30-300000 MHz delas upp
i följande mindre avsnitt som kallas
VHF (Very High Frequency, 30-300 MHz),
UHF (Ultra High Frequency, 300-3000 MHz),
SHF (Super High Frequency, 3-30 GHz) och
EHF (Extra High Frequency, 30-300 GHz).

a

Troposfären - Troposcatter
När en kalluftfront nära jordytan stöter samman med en varmluftfront uppstårturbulenser
i luften med elektriska uppladdningar i gränsskiktet som följd.
Under sådana väderförhållanden kan
radiovågor i VHF-området och däröver att
brytas eller splittras upp när de träffar det
laddade gränsskiktet- troposcatter. Då kan
oväntade radiokontakter uppnås.

N
Temperaturinversion

När ett varmt luftskikt lägger sig över ett
kallare luftskikt uppstår en s.k temperaturinversion.
Vågor på VHF och UHF bryts då mot
gränsskiktet och böjs av mot jordytan. Om
det finns två inversionsskikt samtidigt, så
kan de bilda en slags vågledare, s.k. dukt
(eng. duct = ledning). En räckvidd på 6001300 km kan uppnås. Denna typ av vågutbredning förekommer ofta vid högt atmosfärstryck under sommaren.

Reflexion mot Es (sporadiskt E)

Joniseringen sker när partiklarna passerar genom E-skiktet och brinner upp. Eftersom joniseringen har en varaktighet av endast 0.1- i O sekunder måste MS-förbindelser planeras och förberedas väl. Förbindelserna begränsas vanligen till utbyte av
anropssignaler och signalrapporter med höghastighetstelegrafi med en hastighet av 3003000 tecken per minut. Under de större meteorskurarna kan kontakter uppnås utan överenskommelser på förhand (skeds), både på
telegrafi (CW) och telefoni (SSB).

EME-förbindelser

Vid stark solinstrålning bildas, på de lägre
breddgraderna, joniserade gasmoln på en
höjd av c:a 120 km och med en oregelbunden fördelning.
Den kritiska frekvensen är hög för Esskiktet och det kan även reflektera vågor på
VHF och UHF så effektivt att avstånd av
1000-4000 km kan överbryggas.

Radioförbindelse från en punkt på jorden till
en annan kan åstadkommas genom reflexion av VHF/UHF-signaler mot månen. EMEförbindelser (Earth-Moon-Earth) kallas även
Moon Bounce. EME-förbindelser kräver antenner med mycket hög riktverkan, mycket
hög sändareffekt och känsliga mottagare.

Aurora-reflexion

På VHF och högre frekvenser kan man, som
tidigare beskrivits, endast uppnå radiokontakter hitom den s.k. radiohorisonten.
För att överbrygga detta hinder används
relästationer. Den slags relästation, som allmänt kallas repeater, tar emot det den hör på
en viss fast frekvens och återutsänder detta
på en viss annan fast frekvens. Se frekvensplan i Appendix H.

Markbaserade relästationer

soleruptioner-fiares-utstrålar stora mängder ultraviolett ljus och kastar ut elektriskt
laddade partiklar, som efter 1-2 dagarfångas
upp av jordens m agnetasfär och tränger ner
i polarzonerna. När partiklarna kolliderar med
atmosfären bildas det polarsken i form av
lysande "draperier"- Aurora (kallat norrsken
på norra halvklotet) - samtidigt som atmosfären joniseras. Auroran är joniserade skikt
i samma plan som jordens magnetfält och
speciellt vågor med frekvenser över 30 MHz
reflekteras emot dessa.
VHF- och UHF-kommunikation kan ske
med hjälp av aurorareflexion. De signaler
som reflekteras av Aurora är kraftigt distorderade och har förlorat all ton. Den reflekterade signalen blir bred i frekvens, vilket emellertid gynnar kommunikation med telegrafi
när signalerna är svaga. Oftast är endast
telegrafiförbindelser i långsam takt möjliga.
Vid starkare Aurora går också SSB att använda.

Reflexion mot meteorer- Meteorscatter

Radiovågor på VHF och UHF reflekteras
mot joniserade spår efter det meteorgrus
som faller in i jordatmosfären. Detta fenomen kan utnyttjas för radioförbindelser.

Rymdsateflit-baserade relästationer

Radiovågor med tillräckligt hög frekvens kan
passera genom jonosfärskikten. Detta möjliggör radioförbindelser
VHF/UHF/SHF
mellan stationer på jorden med hjälp av relästationer i rymdsatelliter.
För amatörradiotrafik över rymdsatelliter
används vanligen den slags relästation, som
kallas transponder. En sådan tar emot allt
det den hör inom ett helt frekvensband och
återutsänder detta i ett helt annat frekvensband. På så sätt kan trafik över satellit ske på
ett jämförbart sätt som vid direktkontakt mellan jordbaserade stationer.
satellitbaserade lineartranspondrar med
amatörradioutrustning finns i OsCARsatelliterna (OSCAR = Orbiting Satellite
Carrying Amateur Radio). Dessa har konstruerats och byggts av amatörradiogrupper.

117-13

VÅ UTBREDNIN

E

OSCAR-satelliterna har många olika transpondrar i funktion, vilka var och en arbetar
med olika kombinationer av sändningsslag
(moder) och frekvensband. Detta kallas numera att de har olika konfiguration.
En vanlig konfiguration av transponder
ärCONFIG-V/U (f.d. MOD-J) därupplänken
är på VHF-bandet, t ex. 145.900-146.000
MHz och nerlänken på UHF-bandet t. ex.
435.800-435.900 MHz. Varje upplänk-frekvens motsvarar en bestämd nerlänk-frekvens, t. ex. upp 145.950 och ner 435.850
MHz. Trafiken övertranspondern kan därför
ske i full duplex.

FM- TRAFIK

PA

Man kan då prata och lyssa samtidigt i
båda riktningarna, vilket starkt förbättrar trafiken och gör samtalen roligare och intressantare.
En s.k. linjär transponder kan inte bara
överföra FM, utan även SSB, tontelegrafi
och SSTV. Dessutom även RTTY och andra
digitala trafiksätt
Nästan alla amatörradioband med tillräckligt hög frekvens används i olika kombinationer som upp- och nerlänkar i de olika
OSCAR-satelliterna.
AMSAT är den organisation, som fortlöpande informerar om amatörradiosatelliter.

"2 METER" (144-146 MHz) Exempel: R 1

Mobilstation A

Mobilstation B

Bild II 7-12 Markbaserad repeater

OSCAR-satellit

Markstation 1

Bild II 7-13 Transponder i rymdsatellit

117-14

Markstat i on 2

DNING
Amatörradion utvecklas mycket snabbt
genom den satellitbaserade verksamheten
och det kommer upp allt mer sofistikerade
OSCAR-satelliter. Tendensen är att man
efter hand går över till allt högre frekvensband och allt mera av digitala sändningsslag.
Med hjälp av satellit kan förbindelseavståndet bli mycket stort även med enkel
utrustning och små antenner. En fördel med
kommunikation över rymdsatellit är också
att den till största delen är oberoende av
vågutbredningsvillkoren.
Se Bild il 7-13

117-15

VA

117- 16

UTBR DNIN

E

