\section{Stationsdagbok (loggbok)}

\subsection{Ändamål}

Dina radioförbindelser och övriga händelser med radiostationen bör
antecknas i en \emph{stationsdagbok} (loggbok).

Amatörradioverksamheten bygger på förtroende och då är det viktigt att
själv kunna dokumentera sin verksamhet t.ex. i störningssituationer
m.m. Loggen används också för att kunna visa när man har varit aktiv.

Helt i eget intresse är det ju också trevligt med en loggbok. Tänk
bara på hur bra det är att ha alla underlag för tävlingar och diplom
m.m. dokumenterade.

\subsection{Kunna visa hur man för en loggbok}

Bilden på nästa sida visar ett exempel på hur en loggsida (förminskad)
kan se ut.

Fundera på följande:
\begin{enumerate}
\item Halv tre på eftermiddagen den tionde oktober gör Ulrik (SM7LQQ)
  ett allmänt anrop på den lokala repeatern på 2-metersbandet.  Karin
  (SM7UBM) som är på väg hem från skolan svarar. Ulrik berättar att
  han precis har byggt sitt nya slutsteg på 25 W färdigt och frågar
  Karin om det hörs någon skillnad när han kopplar ur det. Efter lite
  småprat om allt möjligt säger de 73 till varandra och då har det
  gått sju minuter sen de började.  Fyll i loggboken åt Ulrik!
\item Gör ett låtsas-QSO med en kurskamrat. Bokstavera era
  ``signaler''. För in i loggen.
\end{enumerate}

\subsection{Föra in data}
Det man skriver upp i loggen är
\begin{itemize}
\item Tiden i början och i slutet av förbindelsen. Glöm inte datum!
\item Motstationens anropssignal.
\item Din effekt (ineffekt, PEP eller utstrålad effekt)
\item Frekvensband, ev frekvens.
\item Sändningsslag (FM, SSB, CW, paketradio etc).
\item Uppgift om varifrån man sände (eget QTH).
\item Signalrapporter (rapportkoder).
\end{itemize}

Allmänna uppgifter om motstationen, t.ex.  signalrapport, namn, QTH,
motpartens utrustning, QSL-adress o.s.v. brukar också vara bra att ha
med.

Man bör också skriva upp när man har gjort allmänt anrop, sänt ut
bärvåg för prov, experiment och annat som kan vara av intresse.

Om någon annan radioamatör använder din station ska du också skriva
upp hans/hennes namn och anropssignal.

\subsection{Rapportkoder}

Man blir ofta ombedd av motstationen att lämna en s.k. signalrapport
på dennes sändning. Omvänt är det bra att få en signalrapport på den
egna sändningen.

För rapportering mellan radioamatörer används RST-koden.

För lyssnarrapporter t.ex. till rundradiostationer, förekommer ett
kodsystem, som kallas för SINPO eller SINPFEMO.  Se Appendix J.

