\section{Q-koden}

\subsection{Bakgrund}

Vid sändning med morsetelegrafi används sedan år 1912 internationella
``trafikförkortningar'' enligt Q-koden, både för att minska risken för
mottagningsfel på grund av språksvårigheter, störningar m.m. och för
att minska sändningstiden. En trafikförkortning i form av Q-kod har en
entydig innebörd, men kan anpassas något till aktuell situation. Varje
Qkod består av tre bokstäver i bokstavsserien QAA - QZZ.

I CEPT-rekommendation T/R 61-02 nämns följande allmänna
Q-förkortningar som berör amatörradio.  Radioamatörerna använder
emellertid i praktiken fler Q-förkortningar än dessa. En lista kan
beställas från SSA:s kansiL

I reglamentsprovet för radioamatörcertifikat ingår frågor om Q-förkortningar.

Användning
\begin{enumerate}
\item Vissa Q-koder kan ges jakande betydelse genom att bokstaven C
  (vid telefoni uttalad som CHARLE) sänds omedelbart efter
  förkortningen eller ges nekande betydelse med det engelska ordet NO
  omedelbart efter förkortningen.
\item Q-koder kan kompletteras med andra lämpliga förkortningar,
  anropssignaler, frekvenser, tidsuppgifter, person- och ortsnamn,
  siffror, nummer o.s.v. I den beskrivande texten för vissa Q-koder
  lämnas inom en parentes plats för ytterligare uppgifter. Dessa
  uppgifter skall då sändas i den ordning som anges i texten.
\item Q-koderna antar formen av fråga, då de vid radiotelegrafering
  åtföljs av frågetecken liksom då de vid radiotelefonering åtföljs av
  bokstäverna RQ (ROMEO QUEBEC). När kompletterande uppgifter följer
  efter en uttalad fråga, skall ett frågetecken respektive RQ följa
  efter uppgifterna.
\item Q-koder med numrerade alternativa betydelser skall åtföljas av
  motsvarande siffra. Siffran skall sändas omedelbart efter
  förkortningen.
\item I internationell radiotrafik skall, då ej annat anges,
  tidpunkter anges i Universal Time Coordinate (UTC) i stf. det
  tidigare Greenwich Mean Time (GMT). Tidsformatet är fyra siffror,
  vilket även är militär standard.
\end{enumerate}

\begin{table}
  \label{tab:q-kod}
  \caption{Q-koderna}
  \begin{tabular}{lp{6cm}p{6cm}}
    Q-kod & Fråga & Svar eller meddelande \\
    \hline
    QRK &
    Vilken uppfattbarhet har mina (eller \dots:s) signaler?
    &
    Uppfattbarheten hos Dina (eller \dots:s) signaler är
    \begin{enumerate}
    \item dålig
    \item bristfällig
    \item ganska god
    \item god
    \item utmärkt.
    \end{enumerate}
    \\
    QRM &
    Är min sändning störd?
    &
    Störningarna på Din sändning är
    \begin{enumerate}
    \item obefintliga
    \item svaga
    \item måttliga
    \item starka
    \item mycket starka.
    \end{enumerate}
    \\
    QRN
    &
    Besväras Du av atmosfäriska störningar?
    &
    Atmosfäriska störningar är
    \begin{enumerate}
    \item obefintliga
    \item svaga
    \item måttliga
    \item starka
    \item mycket starka.
    \end{enumerate}
    \\
    QRO
    &
    Skall jag öka sändningseffekten?
    &
    Öka sändningseffekten.
    \\
    QRP
    &
    Skall jag minska sändningseffekten?
    &
    Minska sändningseffekten.
    \\
    QRS
    &
    Skall jag minska sändningshastig heten?
    &
    Minska sändningshastigheten
    (sänd \dots ord i minuten).
    \\
    QRT
    &
    Skall jag avbryta sändningen?
    &
    Avbryt sändningen.
    \\
    (QRU)
    &
    Har Ni något till mig?
    &
    Jag har inget till Dig.
    \\
    QRV
    &
    Är Du redo?
    &
    Jag är redo.
    \\
    QRX
    &
    När anropar Du mig igen?
    &
    Jag anropar Dig igen kl \dots på \dots kHz/MHz.
    \\
    QRZ
    &
    Vem anropar mig?
    &
    Du anropas av \dots * (på \dots kHz/MHz).
    \\
    (QSA)
    &
    Vilken styrka har mina
    (eller: \dots *:s) signaler?
    &
    Dina (eller: \dots *:s) signaler är
    \begin{enumerate}
    \item knappast uppfattbara
    \item svaga
    \item ganska starka
    \item starka
    \item mycket starka.
    \end{enumerate}
    \\
    QSB
    &
    Varierar min signalstyrka?
    &
    Din signalstyrka varierar.
    \\
    QSL
    &
    Kan Du ge mig kvittens?
    &
    Jag kvitterar.
    \\
    QSO
    &
    Kan få förbindelse med \dots * direkt?
    &
    Jag kan få förbindelse med \dots * direkt.
    \\
    QSY
    &
    Skall jag gå över till annan frekvens?
    &
    Gå över till annan frekvens.
    \\
    (QTC)
    &
    Hur många telegram har du att sända?
    &
    Jag har telegram till Dig.
    \\
    QTH
    &
    Vilket är Ditt geografiska läge?
    &
    Mitt geografiska läge är \dots
    \\
    QTR
    &
    Kan Du ge mig rätt tid?
    &
    Rätt tid är \dots
    \\
  \end{tabular}
* namn och /eller anropssignal
% TODO Fixa en riktig table note
\end{table}
