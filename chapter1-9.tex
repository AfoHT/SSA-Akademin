\section{Effekt och energi}
\textbf{HAREC a.\ref{HAREC.a.1.9}\label{myHAREC.a.1.9}}

\subsection{Effekt i en sinusformad signal}
För beräkning av effekten av en sinusformad signal använder man effektiwärdet
av spänning och ström.

\(U_{eff} = \frac{U_{max}}{\sqrt{2}}\) och \(I_{eff} = \frac{I_{max}}{\sqrt{2}}\)

\(P = U_{eff} \cdot I_{eff}\)

Bild II 1-32 Effektförhållande

\subsection{Effektändring uttryckt i dB}
Måtten i det metriska systemet är alldagliga och ingen finner det märkligt att
det t. ex. går tio decimeter på en meter. Däremot är begreppet decibel ovant för
många.

I detta avsnitt förklaras det mycket användbara begreppet decibel. Decibel (dB)
är en tiondedel av grundenheten Bel (B).

Räkning med decibel grundas på logaritmer, som är ett bekvämt sätt att uttrycka
och behandla talvärden.

Decibel är ett dimensionslöst uttryck för graden av dämpning alternativt
förstärkning.

Effektdämpning är följden av att vissa komponenter bromsar elektrisk ström. Den
bromsande faktorn kan vara en resistans R, induktans L, kapacitans C eller
sammansatta nätverk av R, L och C.

Effektförstärkning innebär att en transistor, ett elektronrör eller annan s.k.
aktiv komponent kan styra en större elektriskström och därmed större effekt än
den själv styrs med. Vad som förorsakar effektförändringarna går vi inte in på i
detta sammanhang, utan byggdelarna betraktas som "svarta lådor" med
anslutningsklämmor.

En byggdel med två ingångs- och två utgångsklämmor kallas för "fyrpol".

Antag att den inmatade effekten P är 1 W. Om effekten inte ändras vid passagen
genom fyrpolen, så är även den uttagna effekten 1 W.

Effektförhållandet mellan in- och utgångarna är då

\(\frac{P_{in}}{P_{ut}} = \frac{1 watt}{1 watt} = 1 (kvoten = 1)\)

Oförändrad effekt varken dämpas eller förstärks, varför både dämpningen och
förstärkningen har talvärdet 0. Enheten på talvärdet är Bel, dämpningen eller
förstärkningen är således 0 Bel. En tiondel därav är 0 decibel (0 dB).

Omräkning av kvoten av en effektändring till dB görs så, att 10-logaritmen för
kvoten söks och resultatet blir effektändringen uttryckt i Bel (B). Om
resultatet uttrycks i dB, skall Bel-värdet multipliceras med 10.

Logaritmer förklaras i Appendix B.

För att förenkla beräkningen av dB-talet divideras det högre effekttalet med det
lägre. Bokstaven a i följande formler betyder antingen förstärkning (+a) eller
dämpning (-a) beroendet på vilket förtecken som sätts.

\(a[B] = \log \frac{P_{hög}}{P_{låg}}\)

\(a[dB] = 10\log \frac{P_{hög}}{P_{låg}}\)

Att addera eller subtrahera värden på en logaritmisk skala, motsvarar att
multiplicera resp. dividera värden på en linjär skala. Huvudskalorna på en
räknesticka är logaritmiska. (Räknestickan är ett enkelt, förut mycket använt
hjälpmedel).

Med hjälp av följande nomogram kan en effektändring, uttryckt som kvot
(effekterna dividerade med varandra), omvandlas till decibel och omvänt.


Följande avrundade värden kan utläsas:
O dB = 1 1 dB = 1.25 2 dB = 1.6
3 dB = 2 4 dB = 2.5 5 dB = 3.2
6 dB = 4 7 dB = 5 8 dB= 6.3
9 dB = 8 1O dB = 1O 11 dB = 12.5

d. v. s. vid ökning fördubblas effekten för var 3:e dB och vid minskning
halveras effekten för var 3:e dB.

Om kvoten är en eller flera 1O-potenser högre än 1O, så kan nomogrammet utökas
enligt följande tabell.

Kvot av Analys
PhögiPiåg
1
1 har O nollor
1O
1O har 1 nolla
100
100 har 2 nollor
1 000
1 000 har 3 nollor
1O 000 1O 000 har 4 nollor

Skriv

dB

o. 10 = o
1·10=10

2. 10

= 20

3. 10 = 30
4. 10 = 40

\subsection{Strömändring uttryckt i dB}
Förhållandet mellan strömmar liksom mellan spänningar kan även uttryckas i dB, men
annorlunda än mellan effekter. En fyrpol
med inbördes lika ingångs- och utgångsimpedans är förutsättningen för jämförelse.
Enligt Joules lag är P= l

2

•R

(P= U·~

En jämförelse uttryckt i dB kan endast göras
under samma förutsättningar; här att impedanserna (resistanserna) är lika,

111 -54

a[ dB]= 1Olog 'h'og22
flåg

Eftersom log

x2

=

2 ·log x, fås slutligen

a[ dB] = 20 log /hög
flåg

\subsection{Spänningsändring uttryckt i dB}
Förhållandet mellan spänningar kan uttryckas i dB på ett liknande sätt som med
strömmar.

~

2

Enligt Joules lag är P=

(P=

U·~

Två effekter kan ställas i förhållande till varandra på följande sätt:

~ög

-=
~åg

uhög2:R
2
ulåg

:R

R avkortas och efter omskrivning fås en

formel som liknar den för strömmar

~ög

2
uhög

~åg

ulåg

-=--2

log Uhög
qåg

R kan avkortas om in- och utgångsimpedanserna (resistanserna) är lika.

således

Effektförhållandet eller kvadratvärdet på
strömförhållandet kan uttryckas logaritmiskt
i B eller dB

a[ dB] = 20

således

Effekt

~ög

lhög

~åg

flåg

2

-=-2

Med följande nomogram kan kvoten av en
ström- eller spänningsändring omvandlas till
decibel och tvärt om.
Följande avrundade värden kan utläsas:
O dB = 1
1 dB ~ 1.12 2 dB ~ 1.25
3 dB ~ 1.4
4 dB ~ 1.6
5 dB ~ 1.8
6 dB ~ 2
7 dB ~ 2.24 8 dB ~ 2.5
9 dB ~ 2.8
1O dB ~ 3.2 11 dB ~ 3.6
d. v. s. vid ökning fördubblas strömmen resp.
spänningen för var 6:e dB och att vid minskning halveras strömmen resp. spänningen
för var 6:e dB.

L-LÄRA
1

1.2

1.1

l

l

o

i
1

2

lijl l i l i

l
o2 4 6 8
l

1.3
!

l

3
l
i

l
10

l

l
2

l

l

4
l
l
12

l

J

l
3

5
l
l
14

1.6

1.5

1.4
l

l

i

6
l

l

l
4

7
li

l

l
16

Om kvoten är en eller flera 1O-potenser
högre än 1O, så kan nomogrammet utökas
enligt följande tabell.
Skriv
Kvot*
Analys
20
1 har O nollor
1
1 . 20
10
1O har 1 nolla
2. 20
100
100 har 2 nollor
1 000
1 000 har 3 nollor 3. 20
10 000 1 O 000 har 4 nollor 4. 20
*kvot av Uhö/U 1å9 resp. lhö/l,å9

o.

=

dB

o

= 20

= 40

= 60
= 80

Se Appendix C för beräkning med tabeller.

\subsection{Ändring uttryckt i dB vid förstärkande eller
dämpande anordningar kopplade i serie}

Ett räkneexempel på effektändringar:
Fråga:
Vi har en enkel sändaranläggning med
ett drivsteg med en in effekt av 1O W. Drivsteget förstärker med 6 dB. Vidare har vi ett
effektslutsteg som förstärker med 1O dB.
Antennkabeln dämpar med 1 dB.
Med vilken effekt matas själva antennen?
Svar: (två sätt att lösa uppgiften)
1) Drivsteget förstärker fyra gånger, slutsteget förstärker tio gånger och kabeln
dämpar 1/1.25 = 0.8 gånger. Antennen
matas då med 10 · 4 · 10 · 0.8 = 320 W.
2) Drivstegets 6 dB plus slutstegets 1O dB
minus antennkabelns 1 dB = 15 dB.
15 dB är 1O+ 5 dB d.v.s. 1O· 3.2 = 32 ggr.
Antennen matas med 1OW· 32 = 320 W.

1.8

1.7

l

l

l
5

8
l
l
18

2.0 g gr

1.9
l

l

l

l

6

Spänning
dB

10 ggr

9
l
i

Spänning
20

dB

\subsection{Impedansanpassning}
Impedansanpassning är av stor betydelse
inom kommunikationsstekniken. Normaltvill
man nämligen överföra mesta möjliga effekt
från energikällan (t. ex. sändaren) till förbrukaren (t.ex. antennen).
Varje spänningskälla har en inre resistans Ri. Det innebär som först att källan inte
kan avge oändligt stor ström. För att förenkla
det hela antar vi nu att en sändare med den
inre resistansen Ri ansluts direkt till en antenn med resistansen Ra.
Målet med anpassningen är att finna det
optimala förhållandet mellan sändarresistansen och antennresistansen för att kunna
överföra maximal effekt. Vi har de två ytterlighetsfallen obelastad sändare respektive
kortsluten sändare. Sändarens elektromotoriska kraft (EMK) betecknas som E [V]
och sändarens utspänning som U [V].
Fall1.
En obelastad sändare avger ingen ström när
ingen antenn eller en med oändligt stor resistans har anslutits.
Alltså vid obelastad sändare:

Ra =oo

l= O U= E

Fall2.
När sändarutgången är kortsluten, d.v.s.
belastningen (antennresistansen) är noll
ohm, avger sändaren en ström som beror av
EMK och inre resistans. Eftersom såndarutgången är kortsluten är utspänningen Unoll.
Alltså vid kortsluten sändare:

Ra=O !=E
Ri

U=O

111 -55

EL-LÄRA
l båda ytterlighetsfallen är den effekt som
omsätts i Ra lika med noll. För att få ut någon
effekt måste man alltså söka ett värde på Ra
som ligger mellan ytterlighetsvärdena.
Enligt formeln för spänningsdelare är
utspänningen

U=E·

Ra
Ra+Ri

Formeln för uteffektens effektivvärde är

u2
p=ut

R

a

Efter insättning får man
p = p ·Ra

ut

(Ra +Ri)2

För att finna det optimala förhållandet
mellan Ri och Ra, d.v.s. när Ra tar upp maximal effekt, måste man differentiera formeln
med d Pa/dRa, men vi hoppar över denna
utflykt i matematiken.
l stället konstaterar vi helt enkelt att
maximal effektöverföring sker när Ri= Ra.

\subsection{Förhållandet mellan in- och uteffekt uttryckt som \% verkningsgrad}

Antag att en antennkabel har en effektförlust
av 1 dB. Det innebär en effektdämpning av
1.25 gånger, d.v.s. 0.8. Nu matar vi in 1O W
i kabeln och får alltså ut 8 W. Hur stor
verkningsgrad har kabeln uttryckt i o/o ?
Lösning:

8

1J = 1Q ·1 00 = 80o/o

\subsection{Toppvärdeseffekt P.E.P.}

Uteffekten från en sändare kan mätas över
en konstlast (dummy load). En konstlast är
en res istor som kan omsätta sändarens hela
effekt till värme. Med HF-mätprob och en
detektordiod eller en HF-voltmeter kan man
mäta effektivvärdet på spänningen över
konstlasten och beräkna uteffekten med
formeln

u2

p
ut =RU = HF-spänningens effektiwärde
R = resistansen i konstlasten

111-56

På grund av utsignalens karaktär kan
man inte mäta effektiwärdet av uteffekten
från SSB-sändare.
Med oscilloskop kan man emellertid mäta
utspänningen på den största modulationstoppen.
Med detta toppvärde kan man beräkna
spänningen över konstlasten.
Uteffekten definierad som P.E.P. (Peak
Envelope Power) är "den medeleffekt som
matas in i en antennmatarledning under det
högsta effektvärdet inom en frekvenscykel
och mätt under normal drift".

f12

P.E.P.=R
där Oär momentanspänningen på den största
modulationstoppen.


