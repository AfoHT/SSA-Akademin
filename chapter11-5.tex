\section{Anropssignaler}

\subsection{Anropssignalernas sammansättning}

Varje land har unika anropssignaler för all sin
radiotrafik. Signalerna utformas enligt radioreglementet (RR) på sätt,
som beror på syftet med varje särskild radiostation. I RR finns
definitioner för olika slags stationer, t. ex. stationer för fast
radio, landmobila stationer, stationer i fartyg, i
sjöräddningsfarkoster, i flygplan, amatörradiostationer o.s.v.

\subsection{Identifiering av amatörradiostationer}
\textbf{
HAREC b.\ref{HAREC.b.5.1}\label{myHAREC.b.5.1},
 b.\ref{HAREC.b.5.3}\label{myHAREC.b.5.3}
}

En radiostation skall identifieras med den anropssignal, som
tilldelats av det egna landets teleadministration (myndighet). I Sverige
är det Post- och telestyrelsen (PTS).  Anropssignalen meddelas i det
tillstånd för innehav och användning som utfärdats för
tillståndshavaren ifråga. Signalen gäller så länge som tillståndet är
giltigt.

Amatörradiosignaler är uppbyggda på följande sätt:

\begin{tabular}{l|l|l}
                & Antal    &  \\
  Tecken-       & kombi-   &  \\
  kombinationer & nationer & Anmärkningar \\
  \hline
  YOA - Y9Z &
  260 &
  Första tecknet ``y'' räcker ensamt
  \\
  YOAA - Y9ZZ &
  6760 &
  som nationell identitet om det är
  \\
  YOAAA - Y9ZZZ &
  175760 &
  B, F, G, I, K, M, N, R eller W.
  \\
  \hline
  XXOA - XX9Z &
  260 &
  Signaler som börjar med en siffra,
  \\
  XXOAA - XX9ZZ &
  6760 &
  när andra bokstaven är O eller I,
  \\
  XXOAAA - XX9ZZZ &
  175760 &
  är dock ej tillåtna för amatörradio.
  \\
\end{tabular}

(XX är de två första tecknen i en tilldelad signalserie)

Sverige är tilldelat teckenkombinationer i serierna SAA - SMZ, 7SA - 7SZ och
8SA - 8SZ.

Anropssignalerna för svenska amatörradiostationer är uppbyggda på följande
sätt, varvid med distrikt avses amatörradiodistrikt.

Amatörradiotillstånd (CEPT-tillstånd) för
\begin{tabular}{lll}
radioamatörer & SM & + distriktssiffra + bokstäver,  \\
amatörklubbar & SK & + distriktssiffra + bokstäver,  \\
militära förband & SL & + distriktssiffra + bokstäver, \\
amatörklubbar & SI & + distriktssiffra + bokstäver (specialtillstånd), \\
amatörklubbar & SJ & + distriktssiffra + bokstäver (specialtillstånd), \\
amatörklubbar & 7S & + distriktssiffra + bokstäver (specialtillstånd), \\
amatörklubbar & 8S & + distriktssiffra + bokstäver (specialtillstånd), \\
%\multicolumn{3}{l}{SSA-tillstånd inom SSAs utbildningsverksamhet} \\
%& SH & + distriktssiffra + bokstäver (AAA- CZZ). \\
\end{tabular}
% TODO tabellen blir för bred, fixa så att texten flödar runt istället

Sverige är indelat i amatörradiodistrikt med följande numrering och
utsträckning:

\begin{tabular}{rp{5cm}}
Distrikt & Utsträckning \\
0 & Stockholms (AB) län \\
1 & Gotlands (I) län \\
2 & Västerbottens (AC) och Norrbottens (BD) län \\
3 & Gävleborgs (X), Jämtlands (Z) och Västernorrlands (Y) län \\
4 & Örebro (T), Värmlands (S) och Dalarnas (W) län \\
5 & Östergötlands (E), Södermanlands (D), Västmanlands (U) och Uppsala (C) län\\
6 & Hallands (N), Älvsborgs (P), Göteborgs och Bohus (O) län samt Skaraborgs (R) län \\
7 & Malmöhus (M), Kristianstads (L),Blekinge (K), Kronobergs (G),Jönköpings (F) och Kalmar (H) län.\\
\end{tabular}

Distriktssiffran i signalen bestäms av det län som hemadressen är
belägen inom. Vid sändning utanför hemadressen bör det framgå av
tillägg till signalen.


I Post- och telestyrelsens föreskrifter sägs dock inte vilken
distriktssiffra som skall användas, när sändning sker från annan plats
än hemortsadressen.

Med stöd av praxis rekommenderar dock SSA att följande regler
tillämpas:

\begin{itemize}

\item Vid trafik från en regelbundet använd fritidsbostad kan i
  anropssignalen användas den distriktssiffra som utvisar var
  fritidsbostaden är belägen.

\item Vid trafik från annan tillfällig plats bör anropssignalen
  åtföljas av snedstreck och siffran för det distrikt varifrån
  sändningen görs.  Exempel: SMOXYZ/0, SMOXYZ/6 etc.

\item Vid trafik från mobil station bör den ordinarie anropssignalen
  även åtföljas av /M.  Exempel: SMOXYZ/6M.

\item Vid trafik från mobil station inom hemorten kan dock den extra
  distriktssiffran utelämnas.  Exempel: SMOXYZ/M.  e Vid trafik från
  sjöfarkost bör den ordinarie anropssignalen åtföjas av /MM.

\item Vid trafik från luftfarkost bör den ordinarie anropssignalen
  åtföljas av /AM.  e Vid trafik från svensk farkost på
  internationellt territorium kan distriktssiffran 8 användas.

\item Vid sändning från ett annat lands territorium gäller det landets
  bestämmelser.  Vid osäkerhet- Skaffa upplysningar från SSA:s
  reciprokfuntionär!

\item Utländsk radioamatör på besök i Sverige skall använda sin
  anropssignal från det egna landet, föregånget av SM*l där *
  motsvaras av siffran för det svenska distrikt varifrån sändningen
  görs.
\end{itemize}

\section{Användning av anropssignaler}
\textbf{
HAREC b.\ref{HAREC.b.5.2}\label{myHAREC.b.5.2}
}

Både motstationens och den egna anropssignalen skall användas i början
och slutet av varje sändning.  Under sändningen skall anropssignalen
upprepas ``med korta mellanrum'', utan närmare precisering av
mellanrummet.  Även om man inte har kontakt med en motstation, skall
den egna anropssignalen anges vid varje sändning.  Se vidare i PTS
föreskrifter.

$>>>>>$ CH TRAFIKMET DER

%\chapter{Trafikmetoder}

\section{Hur man genomför en radiokontakt}

Det finns många sätt att genomföra en radiokontakt, men det finns
några grundregler för hur man uppträder och utväxlar samtal. Ett
trevligt och kamratligt uppträdande är en hederssak inom
amatörradion. Det behöver inte bli stelt för den skull!

Allmänt anrop är ett sätt att kalla på någon
--- vem som helst --- att kommunicera med.

På telegrafi låter det så här: CQ CQ CQ de SMOXYZ K, d.v.s anropet
först och därefter den egna signalen.  På telefoni låter det så här:
Allmänt anrop, allmänt anrop, allmänt anrop från SMOXYZ Kom. Glöm inte
Kom i slutet!

Riktat anrop gör man, när man vill tala med någon särskild station. Då
sänder man först signalen på den station, som man vill tala med och
därefter sin egen signal.

På telegrafi låter det så här:

--- SMØÅÄÖ SMØÅÄÖ SMØÅÄÖ de SMØXYZ SMØXYZ SMØXYZ K

På telefoni låter det så här:

--- SMØÅÄÖ SMØÅÄÖ SMØÅÄÖ från SMØXYZ SMØXYZ SMØXYZ Kom

Motstationen svarar förhoppningsvis på anropet, alltså

--- SMØXYZ från SMØAAO Kom.

\subsection{Upprättad förbindelse}

När en station svarat på anrop, lämnar man först sin signalrapport
enligt RST-koden och presenterar sig med sitt förnamn och var man
finns. Motstationen kvitterartroligen med sina motsvarande
uppgifter.

Varje gång, som man överlämnar ordet till motstationen
säger man först motstationens signal och därefter sin egen. Därefter
säger man Kom och lyssnar. Om man har en telegrafiförbindelse och bara
vill tala med den stationen kan man sända KN (kom du och ingen annan
(nobody else).

\subsection{Avsluta förbindelse}

När man så småningom avslutar kontakten tackar man för sig på och
utbytter avskedhälsningar.  Då kan det låta så här:

--- Tack för en trevlig förbindelse och på återhörande. SMOÅÄÖ från
SMOXYZ. Klart Slut.

Träna med din instruktör på att klara olika slags trafiksituationer!
