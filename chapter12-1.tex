Tekniskt sett kan radioamatörerna världen
över, med hjälp av sina radiostationer, tämligen lätt skapa kontakt med varandra. Därvid krävs att reglerna i de länder som berörs
vid kontakten respekteras.
En hel serie både internationella och nationella regler styr radiokommunikationerna
i en nation. Varje radioamatör skall känna till
och följa dessa regler så långt de har anslutning till amatörradio. Vissa länder - t.ex.
CEPT-länderna - har i någon utsträckning
harmoniserat sina bestämmelser inbördes.
Nationella avvikelser förekommer likväl och
reglerna i det land, som man gör radiosändningar ifrån, skall alltid följas.

\section{ITU Radioreglemente (RR)}

Amatör- och Amatörsatellittjänsterna är radiokommunikationstjänster med syfte att tillhandahålla nödvändig kommunikation i händelse av naturkatastrofer, träna operatörer
och tekniker i radio- och telekommunikationsteknik till ingen kostnad för stat och
samhälle, bidra till att tidsenlig radiokommunikation främjas och att förbättra internationell förståelse och välvilja.

Artikel1 (RR) Termer och definitioner

Si .56 (RR) Amatörtjänst

En radiokommunikationstjänst avsedd för
självutbildning, inbördes kommunikation och
tekniska undersökningar bedrivet av amatörer, det vill säga av behörigen godkända
personer intresserade av radioteknik, endast av personligt intresse och utan ekonomiskt syfte.
S 1.57 (RR) Amatörsatellittjänst
En radiokommunikationstjänst som använder rymdstationer på jordsatelliter för samma
ändamål som för Amatörradiotjänsten.
81.96 (RR) Amatörradiostation
Radiostation inom amatörradiotjänst

Artikel S25 (RR) (f.d. Artikel 32)
Sektion l. Amatörtjänst
825. i \  1. Radiokommunikation mellan
amatörstationer i olika länder skall vara förbjuden, om administrationen i en av de berörda nationerna har meddelat att den är
emot sådan radiokommunikation.
825.2 \  2. (1) Närsändningarmellan amatörstationer i olika !änder är tillåtna, skall det
ske på klart språk och begränsas till meddelanden av teknisk natur i samband med prov
och till personliga kommentarer, som på
grund av sin oviktighet inte är skäl nog för att
ta den allmänna telekommunikationstjänsten i anspråk.
(2) Det är absolut förbjudet att
825.3
använda amatörradiostationer för internationell radiokommunikation för tredje parts
räkning.
825.4
(3) De föregående bestämmelserna får ändras genom särskilda överenskommelser mellan administrationerna i berörda länder.
825.5 \ 3
(1) Varje person som söker en
licens för att använda apparaterna i en
amatörradiostation skall bevisa sin förmåga
att för hand sända rätt och med hörseln rätt
ta emot texter i form av morsesignaler. Berörda administrationer får emellertid bortse
från detta krav för stationer som endast
används på frekvenser över 30 MHz.
825.6
(2) Administrationerna skall vidta sådana åtgärder som de finner nödvändiga för att kontrollera de handhavandemässiga och tekniska kvalifikationerna hos varje
person som önskar använda apparaterna i
en amatörradiostation.
825.7 \ 4
Den högsta effekten från en
amatörstation skall fastställas av berörda
administrationer, med hänsyn till operatörernas tekniska kvalifikationer och under vilka förhållanden dessa stationer skall användas.

1112- 1

LER

C TRAFIKMET DER

S25.8 \ 5
(1) Alla allmänna regler i överenskommelsen och de i denna artikel skall
tillämpas på amatörradiostationer. Särskilt
den utsända frekvensen skall vara så stabil
och så fri från sidafrekvenser som den tekniska utvecklingen för sådana stationer medger.
S25.9
(2) Under loppet av sändningarna skall amatörstationer sända sina anropssignaler med korta mellanrum.

Sektion Il. Amatörsatellittjänst
S25.1 O \ 6 Bestämmelserna i Sektion 1 i
denna artikel skall gälla i all tillämplig omfattning även för amatörsatellittjänst
S25.1 O \ 7. Rymdstationer i amatörsatellittjänst, som arbetar i band som delas med
andra tjänster, skall förses med lämplig utrustning för att kontrollera utstrålningen om
skadlig störning rapporteras, allt i överensstämmelse med den procedur som föreskrivs i Artikel S15 *.Administrationer som
godkänner sådana rymdstationer skall informera RRB (Radio Registrations Board)
och skall tillse att tillfredställande jordkontrollstationer upprättas före uppskjutningen för
att säkerställa att varje rapporterad skadlig
störning skall kunna avbrytas omedelbart av
den bemyndigande administrationen. Se
S22.1 **.
* S15 behandlar "lnterference"
** 822 behandlar "Space Services"

Bild III 2-1 ITU Regionkarta (ur RRB-2)

1112-2

~©

Artikels (RR 8..1) Frekvenstilldelning
Inledning
391 § 1. l Unionens alla dokument där
termerna allocation, alfatment och assignment används skall de ha den betydelse
som ges i nummer 17 till19, varvid termerna
på de tre arbetsspråken skall vara som följer
(franska, engelska och spanska):
Frekvensfördelning till:
Tjänster
Allocation (tilldelning)
Allotment
(fördelning)
Områden
stationer
Assignment (anvisning) .... etc.
(För enkelhetens skull återges här endast
betydelserna på engelska språket).
sektion l. Regioner och områden
392 § 2. Förtilldelning av frekvenserhar
världen delats in i tre Regioner så som visas
på följande karta och som beskrivs i 393 till
399 .... etc.
Det innebär att tilldelning, fördelning och
anvisning av frekvenser mycket väl kan skilja
mellan ITU-regionerna. Skillnaderna förklaras t.ex. av regionalt olika behovsstruktur,
befolkning etc.
Det förekommer också likheter. På nedanstående karta har markerats en tropisk
zon, vilket förklaras av den annorlunda vågutbredningen där. T.ex. behöver särskild
hänsyn tas vid frekvenstilldelning (allokering) till rundradiotjänsten i zonen.

