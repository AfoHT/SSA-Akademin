\section{Digital Signal Processing (DSP)}
\index{Digital Signal Processing (DSP)}
\index{DSP}
\index{Software Defined Radio (SDR)}
\index{SDR}
\index{FPGA}

\begin{rev-nytt}[MAD]

\emph{Digital signal-processing} (eng. \emph{Digital Signal Processing (DSP)})
har blivit allt viktigare i vardagen och så även inom amatörradion i och med
att \emph{Software Defined Radio (SDR)} blivit en viktig del i allt fler
radios och även användning av vanliga datorer.

I grunden så bygger det på att man digitaliserar signalerna, processar det
digitalt i t.ex. en processor eller programerbar logik (FPGA), och sedan
omvandlar det till analoga signaler igen. När man gör detta i mjukvara i en
processor kallar man det för SDR.

Har man en dedikerad processor för att göra det kallar man det för en
\emph{Digital Signal Processor (DSP)}. Processingen kan även göras av dedikerad
logik som inte kan programmeras i normal form som en processor, det är
fortfarande \emph{Digital Signal Processing}, men används
nu mer mest för de delarna av processing där man behöver utföra samma
standardiserade jobb fort och effektivt så att en processor kan utföra det
enklare jobbet.

En GPS-mottagare är ett exempel på en sådan mottagare, där dedikerad hårdvara
hanterar många miljoner samples per sekund, men processar dem till några värden
per milisekund som sedan processas vidare i en processor.

För att kunna förstå detta behöver vi gå igenom grunderna i konvertering av
signalerna mellan analogt och digitalt, och tillbaka.

\end{rev-nytt}

\subsection{Sampling och kvantisering}
\textbf{HAREC a.\ref{HAREC.a.1.10.1}\label{myHAREC.a.1.10.1}}
\index{sampling}
\index{sample-takt}
\index{sample rate}
\index{sample period}
\index{Sample (S)}
\index{enheter: Sample (S)}
\index{tidsdiskret}
\index{kvantisering}
\index{quantize}
\index{Pulse Code Modulation (PCM)}
\index{PCM}

\begin{rev-nytt}[MAD]

Analoga signaler är vad vi kallar för kontinuerliga i tid, de varierar spänning
och ström som ett kontinuerligt variation av värdet, så snabbt att vi kan
hantera det fulla radio-spektrat och mer därtill. Detta fungerar dock inte så
väl i den digitala världen. Dels vill man ha värden i digital form, så vi
behöver omvandla våra spänningar och strömmar till tal, och dels behöver vi
göra det i en jämn takt.

\emph{Sampling} (från engelskan) är vad det låter som, vi tar ett prov-värde
(sample) då och då, och i detta sammanhang gör vi det i en jämn takt,
\emph{sample-takten} (eng \emph{sample rate}. Denna benämner vi ofta med
\(f_S\) och dess \emph{sample period-tid} \(T_S=\frac{1}{f_S}\) förekommer
också. Sample-takten är alltså den jämna takt varmed vi får värden. Det
förekommer lite slarvigt att man benämner den för att vara 1 MHz, men det mer
korrekta är att man har 1 MS/s dvs 1 miljon samples per sekund, där S
representerar Samples.

\hilight{TODO: illustrera sampling}

Medans sampling är den process som ger oss \emph{tids-diskreta} värden istället
för tids-kontinuerliga värden så är värdena fortfarande inte representerade som
tal, dvs. värdes-diskreta istället för värdes-kontinuerliga. För att åstakomma
detta behöver man omvandla värdena till fasta värden, en process som kallas för
\emph{kvantisering} (eng. \emph{quantize}).

För att kvantisera värden har man ofta ett fixt avstånd mellan stegen på en
trappstege av värden, varje steg kallas ibland för kvantiserings-steg och
storleken på varje kvantiserings-steg avgör därmed hur hög upplösning man får.
Har man t.ex. ett kvantiserings-steg på 0,1 V så blir 0 till 0,1 V tolkat som
0, 0,1-0,2V tolkat som 1 osv.

\hilight{TODO: illustrera kvantisering och PCM}

Denna sista del att omvandla de kvantiserade talen till värden kallas
\emph{Pulse Code Modulation (PCM)}, men det ingår idagligt tal i kvantiserings-
processen idag som en naturlig representation. Denna omvandling kan göras
olinjär, vilket nyttjats i telefoni-system för kompression, men man har börjat
frångå det annat än av kompatibilitetsskäl.

\end{rev-nytt}

\subsection{Minsta samplingsfrekvensen}
\textbf{HAREC a.\ref{HAREC.a.1.10.2}\label{myHAREC.a.1.10.2}}
\index{Nyquist-frekvens}
\index{Nyquist-Shannon samplings-teorem}

\begin{rev-nytt}[MAD]

\infobox{
Denna frekvens kallas för Nykvist-frekvensen efter Harry Nyquist (1889-1976),
från Stora Kil i Värmland, efter hans banbrytande arbete på Bell laboratories
där han publicerade 1924 och 1928. Det ingår i \emph{Nyquist-Shannon samplings-
teoremet} (eng. \emph{Nyquist-Shannon sampling theorem}).
}

Vår nya begreppsvärld har några inneboende begränsningar, en av dem är minsta
samplingsfrekvensen. Den lägsta frekvensen vi kan hantera i vårt samplade
material är fasta värden (eller DC som man oftast säger) medans den högsta är
den när man alternerar mellan två värden, säg -1 +1 -1 +1 vilket ju ger
hälften av samplingstakten \(f_S\), för perioden för sekvensen blir
\(T = 2T_S\) och därmed \(f=\frac{1}{T}=\frac{1}{2T_S}=\frac{f_S}{2}\).

\end{rev-nytt}

\subsection{Faltning}
\textbf{HAREC a.\ref{HAREC.a.1.10.3}\label{myHAREC.a.1.10.3}}
\index{faltning}
\index{convolution}
\index{konvolution}
\index{linjär tids-invariant filter}
\index{linear time-invariant filter}
\index{LTI}

\begin{rev-nytt}[MAD]

Filtrering i den digitala domänen, eller egentligen den tids-diskreta domänen,
kan beskrivas som att filtrets impuls-respons appliceras på signalen, denna
process kallas för \emph{faltning} eller ibland \emph{konvolution} (eng. \emph{convolution}). Man kan se det
som att varje enskilt sample kommer att spela upp hela filtrets svängning med
sin amplitud, och responsen från alla samples blir därför summan av alla dessa.
För varje utgående sample så kommer man därför ha summan av filter-responsen
i en viss fördröjning från ett sample som ligger på samma fördröjning bort.

Den matematiskt sinnade kan då använda formeln

\(y(n) = \sum_{m=0}^{N-1} x(n-m)h(m)\)

där \(x(n)\) är den inkommande sample-strömmen och \(n\) är indexet för det
n:de samplet, \(h(m)\) är filtrets respons och slutligen \(y(n)\) är de utgående
samplen. Denna summering är densamma som beskrivet ovan och beskriver processen
i tids-planet, dvs. när vi jobbar med tid.

Motsvarande process kan utföras i frekvens-planet, dvs. när vi har konverterat
signalen som amplitud av frekvens istället för amplitud av tid. Har man då
även konveterat filtrets egenskaper så gör man helt enkelt en multiplikation av
signal och filter för varje frekvens:

\(Y(f) = X(f)H(f)\)

Bägge representerar faltning, och är viktig för förståelsen av \emph{linjära
tids-invarianta} filter (eng. \emph{linear time-invariant (LTI)}) filter,
som är det vi i allmänhet fokuserar på.

\end{rev-nytt}

\subsection{Anti-vikningsfilter}
\textbf{HAREC a.\ref{HAREC.a.1.10.4}\label{myHAREC.a.1.10.4}}
\index{vikning}
\index{aliasing}
\index{anti-viknings-filter}
\index{anti-aliasing filter}

\begin{rev-nytt}[MAD]

Medans bandbredden vi kan representera är begränsad av Nykvist-frekvensen så
är däremot inte frekvensen det. Själva samplingen ger upphov till
\emph{vikning} (eng. \emph{aliasing}),
sådan att spektrumet efter halva samplings-frekvensen blir vänt så att högre
frekvenser blir lägre. Denna vikning vänder sedan igen när frekvensen blir
den hos samplings-frekvensen, och spektrumet upprepar sig. Detta fenomen
uppstår alltid när man går mellan tids-kontinuerlig och tids-diskret tid.

\hilight{TODO: illustrera viknings-spektrum}

Vid sampling så kan alltså högre frekvenser vika ned sig i spektrat. Detta är
oftast oönskat, vardvid man har ett filter före ingången som undertrycker
oönskade signaler. För t.ex. tal-signaler använder man ett lågpass filter för
att undertrycka de oönskade signalerna högre upp. Detta filter kan istället
användas för ett visst frekvensband för att konvetera ned detta band i
processen, något som är väldigt populärt i SDR sammanhang. I bägge dessa fall
är filtret ett \emph{anti-vikningsfilter} (eng. \emph{anti-aliasing filter}).

Omvänt, när man skall konvertera från tids-diskret till tids-kontinuerlig
signal så viker sig signalen uppåt i frekvens, och för att undertrycka dessa
oönskade frekvenser används på samma sätt ett anti-vikningsfilter. På samma
sätt som förut kan man antingen få de låga frekvenserna som för tal med ett
lågpass-filter eller högre upp i ett band med ett lämpligt bandpass-filter.

Anti-vikningsfilter kan många gånger vara relativt branta, för de måste
undertrycka andra delar av spektrat så att de inte blir en störning.

Vid varje fall när man använder en annan frekvens än den lägsta upp till
Nykvist-frekvensen får man vara omsorgsfull för att se till att man inte viker
det tänkta bandet. Ofta kombinerar man därför med en separat mixer för att
flytta bandet på ett behändigt sätt, men det förekommer också att man väljer
samplingstakten för att inte vika bandet.

\end{rev-nytt}

\subsection{ADC/DAC}
\textbf{HAREC a.\ref{HAREC.a.1.10.5}\label{myHAREC.a.1.10.5}}
\index{ADC}
\index{DAC}

\begin{rev-nytt}[MAD]

För att hantera dessa delar använder man analog-till-digital konverterare
(eng. \emph{Analog-Digital Conversion (ADC)}) samt digital-till-analog
konverterare (eng. \emph{Digital-Analog Conversion (DAC)}). En ADC tar hand om
sampling, kvantisering och PCM-kodning medans en DAC omvandlar PCM-koden till
analog spänning. Ofta behöver man kompletera med analoga filter, men moderna
sigma-delta omvandlare har kraftigt reducerat kraven.

ADC och DAC köper man idag som färdiga integrerade kretsar, inte sällan med
flera kanaler och det finns även dem som har bägge integrerade i samma krets.
Utvecklingen har gjort att man idag kan köpa 24-bitars 48 kS/s ADC och DAC med
dynamiskt område bättre än 100 dB för väldigt låg kostnad.

\end{rev-nytt}
