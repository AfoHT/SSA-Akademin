
\section{CEPT}

\subsection{Begreppet CEPT}

Vid sidan av folkrättsligt bindande avtal såsom den internationella
telekonventionen (ITC) -- har det internationella samarbetet lett till
överenskommelser som inte är tvingande.
Sådana avtal görs bland annat inom \emph{CEPT}.

% FIXME
%\emph{CEPT} betyder \emph{Conf\´{e}rence Europ\´{e}enne des administrations des
%	Postes et T\´{e}l\´{e}communications}, d.v.s. Europeiska konferensen
\emph{CEPT} betyder \emph{Conference Europeenne des administrations des
	Postes et Telecommunications}, det vill säga Europeiska konferensen
för post- och teleadministrationerna. ''Konferens'' är att förstå som
ett ständigt arbetande samarbetsorgan.

Arbetet inom CEPT har huvudsakligen karaktär av ömsesidiga programförklaringar
mellan länder.
Trots att dessa viljeförklaringar eller rekommendationer inte är bindande har de
visat sig värdefulla för utvecklingen av det internationella samarbetet.

\subsection{CEPT-rekommendationerna}

Länder anslutna till CEPT förenklar numera handläggningen av
tillståndsärenden om amatörradio genom att ömsesidigt bekräfta och
inom sitt land tillämpa rekommendationer som länderna utformat i
samråd.
Det innebär att svenska amatörradiobestämmelser kan harmoniseras till andra
länders.
För kompetenskrav vid examinering av radioamatörer finns CEPT-rekommendationen
T/R~61-02 \cite{TR6102}.

\subsubsection{CEPT-rekommendation T/R 61-01}
\textbf{
HAREC b.\ref{HAREC.c.2.1}\label{myHAREC.c.2.1},
 b.\ref{HAREC.c.2.2}\label{myHAREC.c.2.2},
 b.\ref{HAREC.c.2.3}\label{myHAREC.c.2.3}
}

Rekommendationen T/R 61-01 \cite{TR6101} möjliggör för radioamatörer från
CEPT-länderna att utöva amatörradio under korta besök i andra CEPT-länder, utan
att behöva ett tillfälligt tillstånd från det besökta CEPT-landet.
Erfarenheterna av detta system är goda.

\subsubsection{CEPT-rekommendation T/R 61-02}

Rekommendationen T/R 61-02 \cite{TR6102} innebär att administrationerna i
CEPT-länder utger ömsesidigt erkända amatörradiocertifikat (Harmonised Amateur
Radio Examination Certificate -- HAREC) till de personer som vid nationella
prov uppfyller rekommendationens kunskapskrav.

Radioamatörer med ett CEPT-certifikat (HAREC) får utöva amatörradio i annat
land som accepterat T/R 61-01 och får tilldelas ett tillstånd av det landet
utan att behöva genomgå ytterligare kunskapsprov.

Det svenska amatörradiocertifikatet motsvarar kraven för HAREC och Sverige
tillämpar T/R 61-01 och T/R 61-02.
