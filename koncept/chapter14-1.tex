
Isolation och jordning är samlingsbegrepp för ett antal viktiga koncept för
att reducera störningar, som även berör EMC och elsäkerhet.
Detta är viktiga koncept både när man bygger en installation och när man
designar utrustning.
Det skapar också förståelse för hur utrustning är designad, vilket gör det
enklare att använda den på ett korrekt sätt.

\section{Isolation}
\index{isolation}
\index{isolator}
\index{galvanisk isolation}
\index{isolation!galvanisk}

\emph{Isolation} (eng. \emph{isolation}) är ett samlingsbegrepp för att separera
olika signaler.
Den första enkla separationen är den hos en \emph{isolator}, dvs. ett material
som inte leder ström så bra. Det är den mest grundläggande formen av isolation
som förhindrar elektrisk ledning mellan ledningar.

Man brukar prata om \emph{galvanisk isolation} (eng. \emph{galvanic isolation})
för en isolation som inte kan leda likström.
Transformatorer används ofta för att åstadkomma galvanisk isolation.

Nu är isolation inte begränsat till enbart likström, utan även växelspänning
kan behöva isoleras. Hur god isolationen är beror kraftigt på frekvensen, och
de åtgärder man gör bör anpassas för hur god isolation man behöver eller vill
ha för olika frekvenser.
Man kan t.ex. vilja ha god isolation vid sändar- och mottagarfrekvensen 14~MHz,
men vill inte ha galvanisk isolation för det gemensamma 12~V kraftaggregatet.

\section{Jordning}
\index{jordning}
\index{bonding}
\index{jordnät}
\index{bonding network}
\index{blomjord}
\index{skyddsjord}
\index{nolla}

\emph{Jordning} (eng. \emph{bonding}) eller dagligt tal \emph{jord} (eng.
\emph{ground}, \emph{earth}) är en kopplingsstrategi för att få samma
referenspotential i olika delar av en elektrisk koppling.
Man bygger ett \emph{jordnät} (eng. \emph{bonding network (BN)}
\cite[3.2.1]{K27-1991} och \emph{earthing network}) \cite[3.1.3]{K27-1991}
för att koppla samman de olika jordpunkterna.

Den engelska termen \emph{bonding} och även \emph{bondning network} ger en
indikation på vad det handlar om, nämligen en metod att knyta samman flera
olika delar av en design eller installation för att få en gemensam
referensspänning.
Det är helt enkelt en galvanisk sammankoppling.

Många gånger kallas den referenspotentialen för \emph{jordpotential} för att
det är väldigt behändigt att använda jorden som referens, helt enkelt gräva ned
en ledare i marken, t.ex. jordspett (eng. \emph{earth electrode})
\cite[3.1.2]{K27-1991}, för att den vägen få tillgång till jordpotentialen.

Jorden och jordning är dock ofta missförstått, det finns en övertro på att man
kan ta ned störningar med enbart jordning.
Det förekommer också att man upplever att man har problem där jorden upplevs
skapa störningar, varvid en del felaktigt bryter jorden, och därmed
\emph{skyddsjorden}, något man inte får göra av elsäkerhetsskäl.

På samma sätt tror många att man kan göra sig av med en stor växelström i
jorden.
Detta kallas ibland lite skämtsamt för \emph{blomjordning}, för att man inte
tagit hänsyn till jordledarens resistans och induktans, vilket gör att
växelström inte tar sig så långt då ledaren motarbetar den, och då kan man lika
gärna lägga sin jordanslutning ned i en blomkruka för där gör den lika
god nytta.

Inom elkraft förekommer även termen \emph{nolla} (eng. \emph{neutral}), den
kan lätt förväxlas med jorden, men ska hanteras separat från skyddsjord utom
där elsäkerhetsföreskrifter föreskriver att de ska vara sammankopplade.
Nollan är den ledare som är returledare för strömmen.
Nollan är i TN-C system sammankopplad med skyddsjorden i elcentralen, men ut
från elcentralen hanteras den som en separat ledare.
Man får inte koppla ihop dem för att spara ledare!

\subsection{Seriekoppling av jord}

Den enklaste uppkopplingen av jordförbindelse är att seriekoppla jorden
\cite[3]{ott1988} mellan ett antal strömförbrukare.

\begin{figure}
  \begin{center}
    \begin{circuitikz}[american voltages]
      \draw (0,0) node[ground]{};
      \draw (0,0) -- (1,0);
      \draw (1,0)
      to [R, l^=$Z_1$] (3,0)
      to [R, l^=$Z_2$] (5,0)
      to [R, l^=$Z_3$] (7,0);
      \draw (3,1) to [short, i^=$I_1$] (3,0);
      \draw (5,1) to [short, i^=$I_2$] (5,0);
      \draw (7,1) to [short, i^=$I_3$] (7,0);
      \draw (2.75,-1) to [short, <-, i=$I_1+I_2+I_3$] (1.25,-1);
      \draw (4.75,-1) to [short, <-, i=$I_2+I_3$] (3.25,-1);
      \draw (6.75,-1) to [short, <-, i=$I_3$] (5.25,-1);
    \end{circuitikz}
  \end{center}
  \caption{Seriekopplat jordsystem}
  \label{fig:kap4-1}
\end{figure}

I bild \ref{fig:kap4-1} att vi har tre strömförbrukare som var och en
bidrar med en ström \(I_1\), \(I_2\) och \(I_3\), och att dessa är
seriekopplade till en jordanslutning.
Från jordanslutningen till strömbidraget \(I_1\) har vi impedansen \(Z_1\),
och från den punkten har vi impedansen \(Z_2\) fram till strömbidragen \(I_2\)
och slutligen impedansen \(Z_3\) fram till \(I_3\).

En naiv tolkning är att spänningen \(U_1\) för strömbidraget \(I_1\) blir
\(U_1 = Z_1 I_1\), vidare \(U_2 = (Z_1 + Z_2) I_2\) och
\(U_3 = (Z_1 + Z_2 + Z_3) I_3\) för det blir det ju om varje ström ansluts var
och en för sig, dvs. normal seriekoppling av impedanserna.
Denna analys är dock för enkel för att ta hänsyn till fallet när strömmarna
ansluts samtidigt, eftersom strömmar och spänningar kommer samverka.

Den totala strömmen genom första impedansen \(Z_1\) blir ju summan av de tre
strömmarna, därför måste också spänningen höjas med det bidraget.
Den första spänningen blir därför \(U_1=Z_1 (I_1 + I_2 + I_3)\).
På liknande sätt beräknas den andra spänningen med de bägge strömmarna \(I_2\)
och \(I_3\) plus spänningen \(U_1\) och därför blir
\(U_2 = U_1 + Z_2 (I_2 + I_3)\).
Slutligen blir den sista spänningen \(U_3 = U_2 + Z_3 I_3\).
Med förenkling får vi

\(
\begin{array}{ll}
U_1 & = Z_1 I_1 + Z_1 I_2 + Z_1 I_3 \\
U_2 & = Z_1 I_1 + (Z_1 + Z_2) I_2 + (Z_1 + Z_2) I_3 \\
U_3 & = Z_1 I_1 + (Z_1 + Z_2) I_2 + (Z_1 + Z_2 + Z_3) I_3
\end{array}
\)

Vi ser då att störningen blir

\(
\begin{array}{ll}
\Delta U_1 & =  Z_1 I_2 + Z_1 I_3 \\
\Delta U_2 & = Z_1 I_1 + (Z_1 + Z_2) I_3 \\
\Delta U_3 & = Z_1 I_1 + (Z_1 + Z_2) I_2
\end{array}
\)

Vilket är ett tydligt exempel på hur strömmarna stör varandras spänningar och
därmed har avsaknad av isolation.

Fördelen med seriekopplad jord är förstås att man får flera korta anslutningar
men däremot kommer summeringen av de olika strömmarna göra att man får dålig
isolation mellan de olika jordströmmarna och hur nollpotentialen upplevs.

\subsection{Parallellkoppling av jord}
\index{stjärnjordning}
\index{jordning!stjärn-}
\index{skyddsjord}
\index{nolla}

Om vi istället ansluter våra tre laster med individuella ledare till jord
kommer de olika strömmarna inte att samverka, detta är en parallellkoppling
av jord \cite[3]{ott1988}.
Vi har därmed åstadkommit en isolation mellan strömmarna med avseende på
jordanslutningen.

\begin{figure}
  \begin{center}
\begin{circuitikz}[american voltages]
  \draw
  (0,0) to node[ground]{}
  (0,0) to (1,0)
  to [R, l^=$Z_1$] (1,2)
  (1,0) to (3,0)
  to [R, l^=$Z_2$] (3,2)
  (3,0) to (5,0)
  to [R, l^=$Z_3$] (5,2)
  (1,3) to [short, i^=$I_1$] (1,2)
  (3,3) to [short, i^=$I_2$] (3,2)
  (5,3) to [short, i^=$I_3$] (5,2);
\end{circuitikz}
  \end{center}
  \caption{Parallellkopplat jordsystem}
  \label{fig:kap4-2}
\end{figure}

Dock kommer varje strömkälla uppleva en förskjutning i spänningen av
sin jord som beror på dess egen ström och impedansen den har till jord.
För att minska denna effekt kan en minskad strömförbrukning förstås användas,
eller oftare med en förbättrad jord.

Givetvis kan även varje strömförbrukare ha två jordar, parallellt.
Elkraftsystemens användning av både \emph{skyddsjord} och \emph{nolla} är
just ett sådant system, där nollan är den som har strömmen och tillåts få
åka runt i spänning, medans skyddsjorden i allmänhet enbart har små strömmar.
Skyddsjordens funktion är också att kunna hantera stora strömmar vid fel,
för att kunna bryta tillförseln.
Skyddsjorden har egentligen det som sitt huvudsyfte, men ger ofta en bra
jordreferens.

I apparater och även inne på kretskort kan man ha parallellkoppling.
Det är även känt som \emph{stjärnjordning} (eng. \emph{star grounding}) eftersom
kopplingsschemat ser ut att ha en stjärna från en gemensam punkt.
Det kan vara nyttigt att isolera jord för analoga signaler från digitala eller
rent av reläer, PA mm.
Man försöker sätta stjärnan direkt vid anslutningen till kraftaggregatet för
att hålla dem så gemensamt som möjligt men med så lite påverkan av
seriejordning som möjligt.
Samma teknik används ofta för själva kraftdistributionen av samma skäl.

\subsection{Sammankoppling av apparater}
\label{sammankopplingavapparater}
\index{jordbrum}

I ett system där man har gjort parallella jordar i matningen,
vill man nu koppla samman två apparater för att överföra en signal.
En första naiv lösning är ju att helt enkelt bara dra en tråd från den ena
apparaten över till den andra.
Eftersom de har jordanslutning så har de ju en gemensam jordreferens.

\begin{figure}
  \begin{center}
\begin{circuitikz}[american voltages]
  \draw
  (0,0) to node[ground]{}
  (0,0) to (1,0)
  to [R, l^=$Z_1$] (1,2)
  to [american voltage source, l^=$U_{ut}$] (1,5)
  to (3,5)
  to [R, l_=$R_3$, v^=$U_3$] (3,2)
  (1,0) to (3,0)
  to [R, l^=$Z_2$] (3,2)
  (0,2) to [short, i^=$I_1$] (1,2)
  (4,2) to [short, i_=$I_2$] (3,2);
\end{circuitikz}
  \end{center}
  \caption{Sammankopplat system}
  \label{fig:kap4-3}
\end{figure}

Problemet är ju att för den första apparaten som har strömmen \(I_1\) att gå i
anslutningsimpedansen \(Z_1\) till jorden så ger det en spänning
\(U_1 = Z_1 I_1\) på den jordanslutningen.
På samma sätt kommer den andra apparaten att uppleva jorden med en förskjutning
av jordspänningen på \(U_2 = Z_2 I_2\). Om den tänka utspänningen är \(U_{ut}\)
så kommer den egentliga utspänningen vara \(U_3 = U_{ut} + U_1\).
Om vi för stunden antar att det inte går någon anmärkningsvärd ström i ledaren
över till den andra apparaten så kommer den uppleva det som en inspänning
\(U_{in}\) i förhållande till sin jordpotential \(U_2\) dvs.
\(U_{in} = U_3 - U_2 = U_{ut} + U_1 - U_2\).

Vi ser här att skillnaden i jordpotential kommer förskjuta den upplevda
inspänningen \(U_{in}\) från den avsedda spänningen \(U_{ut}\) med skillnaden i
jordpotential, dvs. \(U_1 - U_2\) som i sin tur beror på anslutningarnas 
impedans och strömmarna.
Överföringen kan därför ha problem med sin isolation.

Det här illustrerar grunden i hur \emph{jordbrum} (eng. \emph{hum}) brukar
uppstå när man kopplar ihop två apparater.

\subsection{Isolerad jordning}
\index{isolerad jordning}
\index{jordning!isolerad}
\index{IBN}
\index{signaljord}
\index{flytande}
\index{jordning!flytande}
\index{jordbrum}
\index{chassijordning}
\index{jordning!chassi}
\index{ledningsbunden störning}

En strategi för att skapa isolation från jordvägen är att helt enkelt
isolera signalerna och dess jord från den av kraftförsörjningen, detta kallas
för \emph{isolerad jordning} (eng. \emph{isolated bonding} även \emph{isolated
	bonding network (IBN)}) \cite[3.2.4]{K27-1991}.
Man börjar plötsligt prata om \emph{skyddsjord} skilt från \emph{signaljord}
(eng. \emph{signal ground}).

För apparater med växelströmsmatning har man redan en transformator, som ju
tillhandahåller en galvanisk isolation mellan primärsidan (elkraft) och
sekundärsidan (elektroniken).
Genom att helt enkelt hålla signaljorden \emph{flytande} (eng.
\emph{floating}), dvs. utan någon galvanisk koppling till skyddsjord, så kan
man istället koppla samman signaljord på två apparater med separata ledare.

Om vi återgår till de bägge två apparaterna, så kan vi nu istället för att
använda oss av elnätets skyddsjord låta apparaternas signaljord vara kopplad
med en kabel parallell med signalledaren.
Har vi en förhållandevis låg ström genom den impedans som kabeln har så
kommer det fungera fint.

Det här scenariot liknar t.ex. det hos en normal hemmastereo och ändå kan det
uppstå \emph{jordbrum} i denna koppling.
Det finns flera skäl.
Ett skäl är att transformatorer visserligen erbjuder en galvanisk isolation,
men de är även kapacitiva spänningsdelare för den spänning som finns över
primärlindningen, med 230~VAC spänning så behövs bara lite läckage över för att
man ska uppleva att isolationen brister.
Det brukar vara rekommendabelt att helt enkelt lasta denna spänningsdelare med
ett motstånd, så att signaljord och skyddsjord sitter ihop med ett någorlunda
högt motstånd, ofta med en kondensator parallellt, för att se till att reducera
det bidraget utan att få för mycket störningar från den ström som kommer flyta
mellan jordarna.

Ett annat scenario som skapar jordbrum är när man i någon ände råkar hårt
koppla samman signaljord och skyddsjord, typiskt att det blir oavsiktlig
kontakt mot chassi, som ska vara skyddsjordad.
Själva chassit brukar man prata om som \emph{chassijordat}, men det är
egentligen bara skyddsjord på de flesta system.

\begin{figure}
  \begin{center}
\begin{circuitikz}[american voltages]
  \draw
  (0,0) to node[ground]{}
  (0,0) to (1,0)
  to [R, l^=$Z_1$] (1,2)
  to [american voltage source, l^=$U_{ut}$] (1,5)
  to (3,5)
  to [R, l^=$R_3$, v_=$U_3$] (3,2)
  (1,0) to (3,0)
  to [R, l^=$Z_2$] (3,2)
  (1,3) to [short, i^=$I_1$] (1,2)
  (3,3) to [short, i^=$I_2$] (3,2);
  \draw (1,2) to [R, l^=$R_4$, v_=$U_4$] (3,2);
\end{circuitikz}
  \end{center}
  \caption{Sammankopplat system med utjämningsledare}
  \label{fig:kap4-4}
\end{figure}

För att isolationsjordning ska fungera måste alla kontakter vara isolerade
från chassit.
Även signaljord som inte får ha kontakt med chassi inuti apparaten.
Man behöver alltså försäkra sig om isolationsavstånd, vilket väldigt lätt kan
missas av att man har en skruv som råkar skrapa sig igenom skyddslack t.ex.

En annan nackdel med isolationsjordning är att den gör det svårare att designa
för god EMC-täthet.
För \emph{ledningsbunden störning} (eng. \emph{conductive emission}) så vill
man helst att kontaktens och kabelns skärm sitter i chassijorden med så låg
impedans (induktans) som möjligt.
Isolationsjordning kräver då att man monterar kondensatorer som kopplar ihop
ledarens jord med chassijord och helst runt om för att få lägsta induktans.

Isolationsjordning rekommenderas inte för större system, då den blir svår
att upprätthålla.

Det förekommer att man för att minska störningarna i ett isolationsjordat
system väljer att koppla bort skyddsjorden, för att på det sättet ha mindre
störningar.
Detta är oftast inte tillåtet göra då man normalt inte bryter mot
elsäkerhetsregler och anläggningen riskerar bli farlig, då personskyddet
sätts ur spel.
\textbf{Varje gång som skyddsjorden kopplas bort för att lösa ett problem så
  har man skaffat sig ett större problem, vilket indikerar att man valt en
  felaktig lösning.}

\subsection{Sammankopplad jordning}
\index{sammankopplad jordning}
\index{jordning!sammankopplad}
\index{jordloop}
\index{vagabonderande jordström}
\index{jordbrum}

En annan strategi är \emph{sammankopplad jordning} (eng. \emph{mesh bonding}
och \emph{mesh bonding network (mesh-BN)}) \cite[3.2.3]{K27-1991}
där man istället för att isolera satsar på att koppla samman jordarna, hårt.
Varje signalkabel sitter ansluten mot chassijord och därmed skyddsjord och
man låter därmed jordarna sammankopplas. Varje apparat har en ordentlig
jordanslutning som man ansluter till stativjord eller jordskenor. Kablar läggs
på kabelstegar som jordas. I detta system kommer varje extra kabel att koppla
samman jordarna hårdare, eftersom man parallellkopplar många impedanser.

\begin{figure}
  \begin{center}
\begin{circuitikz}[american voltages]
  \draw
  (0,0) to node[ground]{}
  (0,0) to (1,0)
  to [R, l^=$Z_1$] (1,2)
  to [american voltage source, l^=$U_{ut}$] (1,5)
  to (3,5)
  to [R, l^=$R_3$, v_=$U_3$] (3,2)
  (1,0) to (3,0)
  to [R, l^=$Z_2$] (3,2)
  (1,3) to [short, i^=$I_1$] (1,2)
  (3,3) to [short, i^=$I_2$] (3,2);
  \draw (1,2) to [R, l^=$R_4$, v_=$U_4$] (3,2);
\end{circuitikz}
  \end{center}
  \caption{Sammankopplat system med utjämningsledare}
  \label{fig:kap4-5}
\end{figure}

I ett system som har sammankopplad jord kommer man ofrånkomligen att behöva
hantera vad man kallar för \emph{jordloop} (eng. \emph{ground loop}) eller även
vagabonderande jordströmmar. Många gånger förklaras det som att man får en
loop som agerar antenn för ett magnetfält. Det är dock sällan som ett
magnetfält är så starkt att det inducerar flera ampere av vanlig 50~Hz ström.

Om vi går tillbaka till sammankoppling av apparater (kapitel
\ref{sammankopplingavapparater}) där vi fick en skillnad av spänning mellan
jordpunkterna så kommer vi ha den även här, men nu ansluter vi ju en ledning
mellan dessa punkter, och då kommer det gå en ström som försöker utjämna
potentialen mellan de bägge jordanslutningarna, som då kommer närmare varandra.
Det är impedansen på kabeln som kommer att avgöra hur stor strömmen blir och hur
nära de kommer varandra. Denna ström kan bli ansenlig och har man då en kabel
som har t.ex. tunn skärm så kommer kabeln helt enkelt bli varm. Det är därför
lämpligt att lägga en jordkabel parallellt med signalkabeln, för att låta den
med sin större tvärsnittsarea ta merparten av strömmen och därmed undviker man
värme och ström i signalkabeln.

Med en större kabel mellan kommer spänningen sjunka och den vägen kommer
\emph{jordbrummet} minska.

Fördelen med sammankoppling av jordar är att det blir enklare (och billigare)
att designa ur EMC-perspektiv, då man direkt kopplar jordströmmarna i chassit.
Man har inte heller problem med att man skulle råka jorda eller att man skulle
tappa den enda jordvägen. Istället försöker man koppla ihop jordarna väl.

Ett vanligt problem är om man låter jordströmmarna gå genom kretskort, vilket
gör att man skapar lokala problem med seriejordning. Man ska se till att
jordströmmarna knyter hårt till chassit, men svagt genom kortet för att på det
sättet få bästa möjliga isolation. Denna princip är också lämplig för att
kunna hantera t.ex. ESD-skador.

En annan fördel är att man bygger en vana att jorda allt, och för varje
kompletterande jordning gör man systemet starkare.

\subsection{Balanserad signal}
\index{balanserad signal}
\index{jordbrum}
\index{galvanisk isolation}

För att ytterligare få isolation från jordbrum kan man använda en
\emph{balanserad signal} (eng. \emph{balanced signal}). Grundprincipen är att
man skickar samma signal två gånger, men med omvänt tecken, och sedan ta emot
den och bara titta på skillnaden mellan dem. Skulle nu en störning introducera
sig på dessa ledare gemensamt så påverkar detta inte skillnaden i spänning
mellan dem.

Redan tidigare har vi gjort liknande och försökt efterlikna
egenskaperna, för redan när vi skickade en signal på en enkel ledare så skickar
vi en spänning i förhållande till en referensspänning och vi tittar på den
inkommande spänningen i förhållande till referensspänningen. Dock har vi haft
problem att ha en bra gemensam sådan, och det är uppenbart att vi egentligen
observerar skillnaden i spänning.

Med balanserad signal tar vi steget fullt ut och separerar jord från signal
och skickar en signal som vars summa är en fix spänning medans skillnaden är
hela signalen. Det är som om signalen är neutral. Ofta är dock signalen av
praktiska skäl förskjuten.

Den balanserade signalen har jord, \emph{pluspol} och \emph{minuspol}.
\emph{Pluspolen} kallas även +, \emph{positiv polaritet}, \emph{het} (eng.
\emph{plus pole}, \emph{positive polarity} och \emph{hot}) medans
\emph{minuspolen} kallas även -, \emph{negativ polaritet}, \emph{kall} (eng.
\emph{minus pole}, \emph{negative polarity} och \emph{cold}).

Transformatorer passar väl för att både generera och ta emot balanserade
signaler, då de har en \emph{galvanisk isolation} för \emph{gemensam spänning}
men transformerar den \emph{differentiella spänningen}. Detta kan även göras med
aktiv elektronik så som op-ampar men även färdiga kretsar finns.

Transformatorer har fördelen att man kan få den galvaniska skillnaden genom
att helt enkelt bryta jordförbindelsen på ledaren. Dock, transformatorer har
inte fulländad isolation men kan däremot ofta hantera ganska stora spänningar,
vilket kan krävas i besvärliga sammanhang. För RF är dock transformatorer inte
balanserade och ger dålig isolation. Förbättrad isolation hos transformatorer
kan uppnås med ett eller två skärmlager mellan lindningarna. Skärmlagren kan
anslutas till respektive sidas jord.

Aktiv elektronik för balansering har sällan galvanisk isolation, men däremot
kan man upprätthålla hög impedans för den gemensamma spänningen, vilket kan
vara nog så tillräckligt.

Differentiell signal i RF kan uppnås genom att använda en RF-choke som
undertrycker den gemensamma spänningen i RF men inte i likspänning.

\section[Gemensam och diff]{Gemensam och differentiell spänning och ström}

När man har ett treledarsystem som vi har med differentiell matning eller
även om man bara har två ledare men mellan system som har gemensam jord
(gäller också om de bara har RF-koppling en annan väg) så kan man betrakta
de två signalledarna antingen som att de har sin individuella spänning och
ström, eller som att de har gemensam och differentiell spänning och ström.

\subsection{Gemensam och differentiell spänning}
\label{comdiffv}

Gemensam spänning och differentiell spänning är ett alternativt sätt att
betrakta spänning på de bägge ledarna, där man delar upp spänningen i det som
är gemensamt för de bägge spänningarna och det som skiljer dem åt. Man kan
alltså betrakta dem på detta alternativa och oberoende (ortogonala) sättet.

Tänk om vi har två ledare med 10~V och 12~V på sig. De har en tydlig gemensam
spänning och de har en liten skillnad. Den gemensamma spänningen är medeltalet
och det som skiljer dem åt är då den differentiella spänningen. Om vi låter
12~V gå till vår plus pol, dvs. \(V_+ = 12\ V\), och 10~V gå till vår minus pol,
dvs. \(V_- = 10\ V\) så kan vi nu formulera medelvärdet som

\(V_g = \frac{V_+ + V_-}{2} = \frac{12+10}{2} = 11\ V\)

På samma sätt kan man nu formulera det differentiella värdet, avvikelsen från
det gemensamma som

\(V_d = \frac{V_+ - V_-}{2} = \frac{12-10}{2} = 1\ V\)

Försöker vi vända på det och uttrycka de bägge spänningarna uttryckt i de nya
begreppen så finner vi att det låter sig enkelt göras. Genom siffrorna och
resonemang är det ju rimligt att \(V_+\) och \(V_-\) bägge har den gemensamma
spänningen \(V_g\) men att man sedan adderar den med \(V_d\) för \(V_+\)
respektive subtraherar för \(V_-\). Det ger följande formler:

\(V_+ = V_g + V_d\)

\(V_- = V_g - V_d\)

Ett sätt att illustrera skillnaden är t.ex. med en transformator.
En transformator med 1:1 lindning kopplas in mellan två balanserade signaler.
Transformatorns primärlindning kommer att omvandla den differentiella spänningen
\(V_d\) till en motsvarande spänning på utgången. Däremot kommer den gemensamma
spänningen att inte överföras. Transformatorn blir då en isolator för den
gemensamma spänningen precis som vi förväntar oss av en galvanisk isolation.

Isolationen för den gemensamma spänningen i en transformator är dock främst ett
likströmsbeteende, så ju högre frekvens desto bättre koppling, dvs. sämre
isolation. Detta beror på den kapacitiva kopplingen mellan lindningarna som
skapar en ström.

Eftersom nyttosignalen är differentiell kan man ibland medvetet använda den
gemensamma spänningen för att överföra matningsspänning till t.ex. en mikrofon.
Denna form av matningsspänning kallas för \emph{fantommatning}
(eng. \emph{phantom power}). En vanligt förekommande spänning är 48~V, som då
symboliseras med P48. Det förekommer även på modern Ethernet-utrustning och
kallas då för \emph{Power over Ethernet (PoE)}.

\subsection{Gemensam och differentiell ström}
\label{comdiffi}
\index{RF-choke}
\index{strömbalun}
\index{current balun}

Precis som för spänning kan man beskriva strömmarna i samma ledare som
gemensam och differentiell ström. Vi kan därför återanvända formlerna och bara
byta ut V mot I genomgående och får då:

\begin{eqnarray}
I_+ = & I_g + I_d\\
I_- = & I_g - I_d\\
I_g = & \frac{I_+ + I_-}{2}\\
I_d = & \frac{I_+ - I_-}{2}
\end{eqnarray}

Om vi återgår till transformator exemplet så kommer det vara den differentiella
strömmen på primärlindningen som ger upphov till magnetfältet i transformatorn
och som sedan inducerar en differentiell ström i sekundärlindningen.

Isolationen mellan lindningarna förhindrar att det går en ström mellan dem,
och därför förhindras den gemensamma strömmen vid låga frekvenser. Vid högre
frekvenser kommer dock den kapacitiva kopplingen mellan de två sidorna att
ske varvid en gemensam ström kommer uppstå för högre frekvenser, dvs. för
högre frekvenser kommer isolationen att bli sämre.

Ett intressant specialfall är om vi sätter en ringkärna på vår kabel, lindar
kabeln flera varv om den, eller bara lindar den runt luft. Då kommer strömmen
i den ena ledaren inducera en ström i den andra ledaren och vice versa.
Denna koppling kan liknas vid att vrida en 1:1 transformator 90 grader fel.
Eftersom den inducerade strömmen har motsatt riktning så kommer den motverka
den gemensamma strömmen, men inte den differentiella strömmen. Dessutom kommer
denna koppling bli starkare för högre frekvenser (i den fina teorin) och
därmed skapa en högre isolation för gemensam ström. Detta kallas för
bland annat \emph{RF-choke} (eng. \emph{RF-choke}) och \emph{strömbalun} (eng.
\emph{current balun}). Den kompletterar isolationen hos en transformator
eller löser den nödvändiga isolationen helt på egen hand.

RF-choke är ett oerhört användbart verktyg för att undertrycka RF-strålning
och det man ofta i EMC sammanhang kallar ledningsbunden strålning, som är en
gemensam ström ut på ledarna. Att det är den gemensamma strömmen förstås lätt
eftersom den differentiella strömmen från de bägge ledarna kommer att motverka
varandra i utstrålat magnetfält medans den gemensamma strömmen samverkar och
därför är det enbart den som ger ett utstrålat magnetfält.

Det är därför man ofta hittar klumpar som sitter
på kablar till t.ex. skärmar. Dessa klumpar är helt enkelt en ringkärna som
förstärker kopplingen mellan ledarna för att undertrycka den gemensamma strömmen
för RF och därmed minska störningen.

\subsection{Generell gemensam och differentiell analys}
\label{comdiffgeneric}
\index{gemensam}
\index{gemensam strömöverföriing}
\index{differentiell}
\index{differentiell strömöverföring}
\index{common mode (CM)}
\index{CM}
\index{differential mode (DM)}
\index{DM}

Efter att ha studerat gemensam och differentiell spänning (kapitel \ref{comdiffv})
och gemensam och differentiell ström (kapitel \ref{comdiffi}) kan vi
sammanfattningsvis konstatera att den grundläggande metoden att omvandla
de individuella spänningarna och strömmarna till \emph{gemensam överföring}
(eng. \emph{common mode (CM)}) och \emph{differentiell överföring}
(eng. \emph{diffrential mode (DM)}) är en kraftfull metod både för att
förstå och avhjälpa problem och uppnå isolation.

För spänning har vi ekvationerna

\begin{eqnarray}
V_+ = & V_{CM} + V_{DM}\\
V_- = & V_{CM} - V_{DM}\\
V_{CM} = & \frac{V_+ + V_-}{2}\\
V_{DM} = & \frac{V_+ - V_-}{2}
\end{eqnarray}

För ström har vi ekvationerna

\begin{eqnarray}
I_+ = & I_{CM} + I_{DM}\\
I_- = & I_{CM} - I_{DM}\\
I_{CM} = & \frac{I_+ + I_-}{2}\\
I_{DM} = & \frac{I_+ - I_-}{2}
\end{eqnarray}

\subsection{Gemensam och differentiell impedans}

Precis som man har impedans på ingångar så har man det på ingångar i
treledarsystem. Det som är den normala impedansen för en transmissionsledare
t.ex. är egentligen den differentiella impedansen, dvs. förhållande mellan den
differentiella spänningen och differentiella strömmen. Den gemensamma impedansen
är på samma sätt förhållandet mellan gemensam spänning och gemensam ström

\begin{eqnarray}
Z_{DM} = & \frac{U_{DM}}{I_{DM}}\\
Z_{CM} = & \frac{U_{CM}}{I_{CM}}
\end{eqnarray}

Egentligen är det inte så konstigt, om man har en koaxialkabel i ett 50~Ohm
system så har sändare och mottagare idealt 50~Ohm som differentiell impedans.
I ett system som har isolerad jordning så kan den gemensamma impedansen vara
många MegaOhm eller högre, eftersom den ju är isolerad.

\subsection{Obalans}
\index{strömbalun}
\index{obalans}

Så här långt har huvudsakligen antagit att vi har balans, dvs. att
transformatorer, induktorer mm är ideala och ger lika bra koppling till bägge
sidor. Givetvis finns inte detta i verkligheten, och man har en obalans.
Vid obalans får man en signal som är gemensam att läcka över till den som är
differentiell och omvänt att differentiell läcker över till den gemensamma.
Det resulterar dels i minskad isolation och dels i minskad signal.
I allmänhet är den minskade isolationen värre än förlusten av signal, som i
allmänhet är försumbar.

I en transformator ligger lindningarna ofta så att den kapacitiva kopplingen
från ena polen på en spole är starkare än från den andra polen.
Det ger därför en obalans i hur de kopplar kapacitivt. Genom att lägga ett
skärmlager mellan lindningarna kan den kapacitiva kopplingen jämnas ut, då de
kopplar kapacitivt till skärmlagret istället, som kan lågresitivt hindra
koppling. En ännu bättre lösning är att ha dubbla lager med isolation, för då
kan de kopplas mot respektive sidas jord, och kvar blir bara den kapacitiva
kopplingen mellan jordarna, som oftast är ett mindre problem. Med dessa metoder
fås bättre isolation än vad en oskärmad transformator kan erbjuda, på grund av
just obalans.

Den kapacitiva kopplingen har väldigt hög impedans vid 50~Hz, så man kan
använda relativt höga motståndsvärden för att lasta ned den hårt. Fördelen är
att man kan undvika direkt koppling, vilket kan skapa andra problem som när man
vill ha relativ isolation galvaniskt.

I en strömbalun kan den ena ledaren ha något lite längre varv runt kärnan än
den andra. Det ger inte en perfekt 1:1 relation i kopplingen och därmed en
obalans.

I en transformator med mitt-tapp kan mitt-tappen sitta lite förskjuten från
riktiga mitten, så att anslutningen av mitt-tappen till jord skapar en
obalans.

Dessa exempel på imperfektion ska man vara medveten om, så att man inte
tillskriver en transformator eller strömbalun en perfekt isolation av
egenskaperna. Snarare ska man förvänta sig att den inte är perfekt och
anpassa sin design efter det.

Ett enkelt fall i ljudsammanhang är 50~Hz 230~V men man vill hålla störningen
mindre än säg 1~mV. Det kräver mer än 106~dB isolation mellan 230~V
differentiellt på primärlindningen och 1~mV gemensamt på sekundärlindningen.
Så god balans kan vara svår att finna i enskilda komponenter. Principen
återkommer oavsett spänning och frekvens, det är en imperfektion man behöver
lära sig att förstå och hantera.

\subsection{Obalans i antennsystem}
\index{obalans!antennsystem}

Obalans kan även förekomma i antennsystem, där en obalanserad antenn omvandlar
den utsända signalen, som är differentiell, till att delvis bli gemensam.
Detta gör att som reflex från den obalanserade antennen går en ström i
matningsledningen som gör att den strålar. Detta har traditionellt uttryckts
som att strömmen vänder och går på utsidan av skärmen, men det som hänt är att
den differentiella strömmen, som ju motverkar utstrålning plötsligt får en
pålagd gemensam komponent som då kommer stråla. Man kan uppleva det om man
berör ledningen så kan man känna denna som en ström, vilket man upplever går
på utsidan. Kabeln har då blivit en strålande del av antennen. Det är också
denna ström som behöver motverkas för att operatören inte ska skada sig.
Detta görs med en strömbalun, lämpligtvis en kvartsvåg ned från anslutningen
till antennen. Strömbalunen motverkar ju den gemensamma strömmen utan att
nämnvärt påverka den differentiella, så det är ett fint exempel på en bra
åtgärd.

De allra flesta antenner har en annan impedans i matningspunkten än vad dess
matarledning har. Detta kräver en impedansanpassning för optimal
energiöverföring. En annan aspekt är att för en koaxial matning så överförs
energin enkelsidigt (single-ended) dvs. det är mittledaren i förhållande till
skärm/jord som överför energi. När vi ansluter denna ledare till en dipolantenn
vill vi se till att strömmen går balanserat ut i de bägge ledarna,
så att mittpunkten är nära noll, så att det inte går en ström med gemensam
mod ut i matarledningen. Vi har alltså dels behovet att omvandla obalanserad
signal till balanserad samt undertrycka gemensam signal i ledaren, och därtill
impedanskonvertera den. Detta brukar man låta en balun (balanced-unbalanced)
göra, vilket som namnet anger bara ger indikation på konverteringen, men den
gör alltså flera saker. Eftersom ingen balun är perfekt designad så kommer
den i sig själv ha en obalans, varvid den ändå kommer ge viss gemensam ström.
För högre effekter kan därför en separat spärr komma att behövas.

Utöver balun finns även unun (unbalanced-unbalanced) som gör
impedanskonvertering enbart.

Även om man har en bra balun riskerar man att få mantelströmmar, ty antennen
kan vara av en obalanserad typ, t.ex. Off-Center-Feed (OCF)/Windom, eller
för att den kopplar olika med miljön som träd och torn m.m.

Att undvika att det går gemensam ström, även kallad mantelström kan krävas av
många olika anledningar, och det är viktigt dels för att få ut energin där den
ska, dvs. radierat ut i luften på ett korrekt sätt, men även av säkerhetsskäl
så att inte utrustning eller personskada uppstår.
