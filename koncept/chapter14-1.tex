\chapter{Bestämmelser}

\emph{Tekniskt sett kan radioamatörerna världen över, med hjälp av
  sina radiostationer, tämligen lätt skapa kontakt med varandra.
  Därvid krävs att reglerna i de länder som berörs vid kontakten respekteras.}

\emph{En hel serie både internationella och nationella regler styr
  radiokommunikationerna i en nation.
  Varje radioamatör ska känna till och följa dessa regler så långt de har
  anslutning till amatörradio.
  Vissa länder -- till exempel CEPT-länderna -- har i någon utsträckning harmoniserat
  sina bestämmelser inbördes.
  Nationella avvikelser förekommer likväl och reglerna i det land, som man gör
  radiosändningar ifrån, ska alltid följas.}

\section{ITU Radioreglemente (RR)}
\index{Internationella Teleunionen (ITU)}
\index{ITU}
\index{ITU RR}
\index{ITU Radioreglemente (ITU RR)}
\index{RR}
\index{radioreglemente}
\index{PTS}

Internationella Teleunionen (ITU) är det internationella samarbetsorgan där
olika länders myndigheter (administrationer) för telekommunikation samarbetar
och koordinerar sig, bland annat genom gemensamt regelverk och standarder.
Det är viktigt för att koordinera användning av spektrum och signalerna i det.

ITU Radioreglemente (RR) \cite{ITU-RR} är det övergripande regelverket för att
koordinera spektrumanvändning, det vill säga för alla former av
radiorelaterad verksamhet.
Det är det gemensamma ramverk som används, och varje land utgår från det för att
sedan skriva de nationella föreskrifterna och tilldelningarna.
Dock, den ansvariga myndigheten behöver inte strikt följa ITU RR och det
förekommer flera fall där saker i ITU RR ser tillåtna ut men de nationella
föreskrifterna inte tillåter samma sak.
Man ska därför inte tolka ITU RR som gällande istället för de nationella
föreskrifterna utan snarare en utgångspunkt.
Det kan ha skett förändringar i ITU RR men där existerande frekvensallokeringar
nationellt förhindrar möjligheten att följa ITU RR.
Omvänt så är det ofta svårt för de nationella föreskrifterna att gå utanför
ITU RR eftersom det kan kräva svåra förhandlingar, och därför försöker man ofta
få till i förändringen av ITU RR istället.

I Sverige är det Post- och telestyrelsen (PTS) som är ansvarig för
administration gällande telekommunikation och spektrum Det är deras
föreskrifter som reglerar all radio inklusive amatörtjänsten.
Där ITU RR nämner begreppet ''administration'' så avses för Sveriges del PTS.

Som del av ITU RR definieras ''Amateur services'' \cite[Article 25]{ITU-RR}.
Amatör- och Amatörsatellittjänsterna är radiokommunikationstjänster
med syfte att tillhandahålla nödvändig kommunikation i händelse av
naturkatastrofer, träna operatörer och tekniker i radio- och
telekommunikationsteknik till ingen kostnad för stat och samhälle,
bidra till att tidsenlig radiokommunikation främjas och att förbättra
internationell förståelse och välvilja.

\subsection{Artikel 1 (RR) Termer och definitioner}
\harec{b}{1.1}{1.1}
\harec{b}{1.2}{1.2}
\index{amatörtjänst}
\index{amatörsatellittjänst}
\index{amatörradiostation}
\index{radiostation}

1.56 (RR) \emph{Amatörtjänst} \cite[1.56]{ITU-RR}\\
En radiokommunikationstjänst avsedd för självutbildning, inbördes
kommunikation och tekniska undersökningar bedriven av amatörer, det
vill säga av behöriga personer intresserade av radioteknik,
endast av personligt intresse och utan ekonomiskt syfte.

1.57 (RR) \emph{Amatörsatellittjänst} \cite[1.57]{ITU-RR}\\
En radiokommunikationstjänst som använder rymdstationer på
jordsatelliter för samma ändamål som för \emph{Amatörradiotjänsten}.

1.96 (RR) \emph{Amatörradiostation} \cite[1.96]{ITU-RR}\\
Radiostation inom \emph{amatörradiotjänst}.

\subsection{Artikel 25 (RR) Amateur services}
\harec{b}{1.3}{1.3}
\harec{b}{1.4}{1.4}

\subsubsection{Sektion I. Amatörtjänst}
25.1 \S1 Radiokommunikation mellan amatörstationer i olika länder
skall vara tillåten, om inte administrationen i en av de berörda
nationerna har meddelat att den är emot sådan radiokommunikation.
\cite[25.1]{ITU-RR}

25.2 \S2 1) Sändning mellan amatörstationer i olika länder skall vara
begränsad till spontan kommunikation med syftet att nyttja amatörtjänsten,
som definierad i 1.56, och av personlig karaktär.
\cite[25.2]{ITU-RR}

25.2A \S2 1A) Sändning mellan amatörstationer i olika länder skall
inte vara kodad med syfte att dölja dess mening, annat än för kontrollsignaler
utbytta mellan jordstation och satellitstation i amatörradiotjänst.
\cite[25.2A]{ITU-RR}

25.3 \S2 2) Amatörradiostationer får användas för internationell
radiokommunikation för tredje parts räkning enbart vid nöd eller
krishantering.
\cite[25.3]{ITU-RR}

25.5 \S3 1) Administrationerna avgör huruvida en person som söker licens
att använda en amatörstation skall bevisa sin förmåga att sända och ta
emot text i morsesignaler.
\cite[25.5]{ITU-RR}

25.6 \S3 2) Administrationerna skall kontrollera de handhavandemässiga och
tekniska kvalifikationerna hos varje person som önskar använda en
amatörradiostation. En guide för den kompetens som krävs kan man finna i
senaste upplagan av ITU-R rekommendation M.1544.
\cite[25.6]{ITU-RR}

25.7 \S4 Den högsta effekten från en amatörstation skall fastställas
av berörda administrationer.
\cite[25.7]{ITU-RR}

25.8 \S5 1) Alla allmänna regler i överenskommelsen och de i denna
artikel skall tillämpas på amatörradiostationer.
\cite[25.8]{ITU-RR}

25.9 \S5 2) Under loppet av sändningarna skall amatörstationer sända
sina anropssignaler med korta mellanrum.
\cite[25.9]{ITU-RR}

25.9A \S5A Administrationer uppmuntras att vidta nödvändiga steg för att
tillåta amatörstationer att förbereda sig för och möta kommunikationsbehov
vid katastroftillstånd.
\cite[25.9A]{ITU-RR}

25.9B \S5B En administration kan avgöra huruvida en person som har tillstånd
att använda en amatörstation hos en annan administration kan tillåtas använda
en amatörstation medan denna person befinner sig på tillfälligt besök landet,
samt vilka villkor och begränsningar de väljer att ange.
\cite[25.9B]{ITU-RR}

\subsection{Sektion II. Amatörsatellittjänst}

25.10 \S6 Bestämmelserna i Sektion 1 i denna artikel skall gälla i all
tillämplig omfattning även för amatörsatellittjänst.
\cite[25.10]{ITU-RR}

25.10 \S7 Administrationer som godkänner rymdstationer i amatörsatellittjänst
ska tillse att tillfredsställande jordkontrollstationer upprättas före
uppskjutningen för att säkerställa att varje rapporterad skadlig störning
skall kunna avbrytas omedelbart av den bemyndigande administrationen.
Se 22.1 **.
\cite[25.11]{ITU-RR}

** 22 behandlar ''Space Services''

\subsection{Artikel 5 Frekvenstilldelning}

\subsubsection{Inledning}

5.1 I Unionens alla dokument där termerna \emph{allocation},
\emph{allotment} och \emph{assignment} används skall de ha den
betydelse som ges i 1.16 till 1.18, varvid termerna på de tre
arbetsspråken skall vara som följer (franska, engelska och spanska):
\cite[5.1]{ITU-RR}

Frekvensfördelning till:
\begin{tabular}{lll}
  Tjänster & Allocation & (tilldelning) \\
  Områden & Allotment & (fördelning) \\
  Stationer & Assignment & (anvisning) .... etc. \\
\end{tabular}

(För enkelhetens skull återges här endast betydelserna på engelska språket).

\subsubsection{Sektion I. Regioner och områden}
\harec{b}{1.5}{1.5}

5.2 För tilldelning av frekvenser har världen delats in i tre
Regioner så som visas på följande karta och som beskrivs i 5.3 till
5.9 ... etc.
\cite[5.2]{ITU-RR}

\emph{ Det innebär att tilldelning, fördelning och anvisning av frekvenser
  mycket väl kan skilja mellan ITU-regionerna.
  Skillnaderna förklaras till exempel av regionalt olika behovsstruktur, befolkning
  etc.}

\emph{Det förekommer också likheter.
  På nedanstående karta har markerats en tropisk zon, vilket förklaras av den
  annorlunda vågutbredningen där.
  Till exempel behöver särskild hänsyn tas vid frekvenstilldelning (allokering) till
  rundradiotjänsten i zonen.}

%%\mediumfig[0.9]{images/cropped_pdfs/bild_3_2-01.pdf}{ITU Regionkarta (ur RRB-2)}{fig:bildIII2-1}
\mediumfig[0.9]{images/ITU_Regions-Map.png}{ITU Regionkarta (ur RRB-2)}{fig:bildIII2-1}
