\chapter*{INLEDNING}

\Huge{VAD, HUR, VAR?}\normalsize

\section*{VAD behöver en radioamatör kunna?}

CEPT är ett samarbetsorgan mellan europeiska länders teleadministrationer
(myndigheter). En av dem är svenska Post- och telestyrelsen -- PTS.

Dessa administrationer har antagit rekommendationer om sinsemellan
harmoniserade krav på radioamatörers kompetens.

Sverige har antagit CEPT-rekommendationen T/R 61-02 \cite{TR6102}.
Vid genomförandet av kompetensprov ska de i den rekommendationen
angivna kraven särskilt beaktas.

För den som godkänts i ett sådant prov utfärdas ett harmoniserat
amatörradiocertifikat (HAREC).
Rekommendationen anger kompetensnivån HAREC.
Den svenska certifikatet bygger på CEPT HAREC krav \cite{TR6102},
med anpassning till svensk bandplan i Bilaga \ref{bandplaner}.
De detaljerade CEPT HAREC kraven finns i Bilaga \ref{CEPT HAREC}, där även
referenser till den eller de del-kapitel som avses uppfylla utbildningenkraven.

\subsection*{HUR blir man radioamatör?}

För att få sända med amatörradiosändare måste man ha amatörradiocertifikat.
Man kan antingen söka sig till någon av de klubbar som har kurs, eller skaffa
SSAs utbildningspaket och studera på egen hand. Post- och telestyrelsen har
dessutom övningsprov online som man kan testa sina kunskaper på.
När man är mogen för att avlägga certifikatprov så skriver man för någon av de
provförrättare som finns. De klubbar som har utbildning brukar planera prov
med den grupp elever de har.

Efter avlagt och godkänt prov kan man sedan ansöka om signal och certifikat,
något som SSA sköter enligt delegation från Post- och telestyrelsen.

Till tillståndet knyts en internationellt unik anropssignal. Man har möjlighet
att föreslå signal, men i brist på förslag så tas en ledig signal ur serien.

\subsection*{VAR hålls det certifikatskurser?}

Vissa amatörradioklubbar, militära förband, FRO-förbund och andra
sammanslutningar håller certifikatskurser.
Det går också att studera på egen hand.

\subsection*{VILKA läromedel behöver man?}

Denna bok omfattar hela teorin för CEPT HAREC och PTS krav.
Den ingår i det utbildningspaket som kan köpas från SSA.
