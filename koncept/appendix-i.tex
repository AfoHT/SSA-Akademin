\chapter{Frekvenser för svenska amatörradiorepeatrar}
Vid direktförbindelser på höga frekvenser är
räckvidden begränsad, särskiltvid låg effekt
och små antenner. Med repeatrar
tio ner) med högt belägna antenner
räckvidden förbättras, vilket underlättar kommunikation med rörliga (mobila) radiostationer.
Eftersom sändaren och
i en
repeater arbetar samtidigt, måste avståndet
mellan deras arbetsfrekvenser vara så stort
att det inte uppstår ömsesidiga störningar.
Dessa arbetsfrekvenser kallas frekvenspar eller kanal och avståndet mellan dem
kallas repeaterskift, vilket är enhetligt inom
repeaterbandet

Frekvensparet i en repeater måste arbeta med omvänt frekvensläge i förhållande
till det i de stationer som den betjänar.
Kanalavståndet mellan repeatrarna i ett
band är också enhetligt och sändningar över
repeatrarna måste naturligtvis ha mindre
bandbredd än kanalavståndet
Inom IARU har man enats bl.a. om frekvensparen för smalbandiga FM-repeatrar.
Se IARU:s bandplaner i Appendix F.
Frekvensplaner finns för repeatrar inom
banden 51--52 MHz (6 m), 145--146 MHz (2
432--438 MHz (70 cm), 1240--1300 MHz
cm) samt 28000--29700~kHz (1 O m).

Nya kanalnumreringsmetoden
l och med införandet av i
kHz kanalavstånd på 2 meters- och 70 cm-banden har
ett nytt enkelt system införts. Man börjar med
en bokstav som talar om vilket band det är:
F för 51 MHz, kanalavstånd 1O
V för 145 MHz, kanalavstånd 12,5 kHz,
kHz.
U för 430 KHz, kanalavstånd 1
Kanalnumret börjar med 00 på varje sådant band och ökar med ett (1) för varje kanal
i bandet. På 51 och 145 MHz används
siffrig numrering och på 430 MHz tresiffrig.
För repeaterfrekvenser sätts ett R före bandbokstaven.

70-centimetersbandet (skift 1600 kHz)
Kanal
Din sändarDin mottagarnr
frekvens MHz frekvens MHz
433,000
RU368
434,600
433,0125
RU369
434,6125
RU370
433,025
434,625
RU371
433,0375
434,6375
433,050
RU372
434,650
434,6625
433,0625
RU373
434,675
433,075
RU374
434,6875
433,0875
RU375
434,700
433,100
RU376
433,1125
434,7125
RU377
433,125
434,725
RU378
433,1375
434,7375
RU379
RU380
433,150
434,750
RU381
433,1625
434,7625
RU382
433,175
434,775
RU383
433,1875
434,7875
433,200
434,800
RU384
433,2125
434,8125
434,825
RU386
433,2375
434,8375
RU387
433,250
434,850
RU388
433,2675
434,8675
RU389
RU390
434,875
433,2875
RU391
434,8875
RU392
433,300
434,900
433,3125
434,9125
RU393
433,325
434,925
RU394
433,3375
434,9375
RU395
433,350
434,950
RU396
434,9675
RU397
433,375
434,975
RU398
433,3875
434,9875
RU399

2-metersbandet (repeaterskift 600
Kanal
Din sändarDin mottagarnr
frekvens MHz
frekvens MHz
RV48
145,000
145,600
RV49
145,0125
25
1
RV 50
145,025
1
RV51
145,0375
i
RV 52
145,050
145,650
RV 53
145,0625
1
RV 54
i45,075
1
RV55
145,0875
1
RV 56
145,100
i
RV 57
145,1125
1
RV 58
145,125
1
RV59
145,1375
i
RV60
145,150
1
RV61
145,1625
1
RV62
145,175
1
RV63
145,1875
i

H-1

APPENDIX

E

H

23-centimetersbandet (skift 6000 kHz)
Kanal
nr

RMO

RM1
RM2
RM3
RM4
RM5
RM6
RM7
RM8
RM9
RM10
RM11
RM12
RM13
RM14
RM15
RM16
RM17
RM18
RM19

Din sändarfrekvens MHz
1291,000
1291,025
1291,050
1291,075
1291 '1 00
1291 '125
1291 '150
1291,175
1291,200
1291,225
1291,250
1291,275
1291,300
1291,325
1291,350
1291,375
1291,400
1291,425
1291,450
1291,475

Din mottagarfrekvens MHz
1297,000
1297,025
1297,050
1297,075
1297,100
1297,125
1297,150
1297,175
1297,200
1297,225
1297,250
1297,275
1297,300
1297,325
1297,350
1297,375
1297,450
1297,475
1297,450
1297,475

Repeaterband med speciella egenskaper

6-metersbandet (skift 600 kHz)
Din sändarDin mottagarKanal
nr
frekvens MHz frekvens MHz
RF81
51.21 O
51.81 O
RF83
51.230
51.830
RF85
51.250
51.850
RF87
51 .270
51 .870
RF89
51 .290
51 .890
RF91
51.31 O
52.91 O
RF93
51.330
52.930
RF95
51 .350
52.950
RF97
51.370
52.970
RF99
51.390
52.990
(dvs endast udda kanalnummer används).
Observera att det i Sverige, utöver amatörradiotillståndet, t.v. krävs särskilda tillstånd
för amatörradioanvändning i detta band.
På grund av den relativt låga frekvensen
uppnås ofta överräckvidder p.g.a. sporadisk
vågutbredning via E-skiktet. Man kan då
uppnå förbindelser utan hjälp av repeater.
Observera, att i Sverige f.n. inga repeatrar
finns i 6-metersbandet.
1O- metersbandet (skift 100 kHz)
Kanal
Din sändarDin mottagarnr
frekvens kHz
frekvens kHz
29560
29660
29570
29670
29580
29680
29590
29690
På grund av den relativt låga frekvensen
uppnås stora räckvidder genom jonosfärisk
vågutbrednin.g, särskilt under år med högt
solfläckstal. Aven sporadisk vågutbredning
via E-skiktet förekommer. I båda fallen bör
repeatertrafik undvikas.

H-2
