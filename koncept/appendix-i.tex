\chapter{Frekvenser för svenska amatörradiorepeatrar}
\label{svenska repeatrar}

Vid direktförbindelser på höga frekvenser är räckvidden begränsad,
särskilt vid låg effekt och små antenner.
Med repeatrar med högt belägna antenner kan räckvidden förbättras,
vilket underlättar kommunikation med rörliga (mobila) radiostationer.
Eftersom sändaren och mottagaren i en repeater arbetar samtidigt, måste
avståndet mellan deras arbetsfrekvenser vara så stort att det inte uppstår
ömsesidiga störningar.
Dessa arbetsfrekvenser kallas frekvenspar eller kanal och avståndet mellan dem
kallas repeaterskift, vilket är enhetligt inom repeaterbandet.

Frekvensparet i en repeater måste arbeta med omvänt frekvensläge i förhållande
till det i de stationer som den betjänar.
Kanalavståndet mellan repeatrarna i ett band är också enhetligt och sändningar
över repeatrarna måste naturligtvis ha mindre bandbredd än kanalavståndet.
Inom IARU har man enats bl.a. om frekvensparen för smalbandiga FM-repeatrar.
Se IARU:s bandplaner i Appendix F.
Frekvensplaner finns för repeatrar inom banden 51--52~MHz (6~m), 145--146~MHz
(2~m), 432--438~MHz (70~cm), 1240--1300~MHz (10~cm) samt 28000--29700~kHz (10~m).

\section{Kanalnumreringsmetod}
Vid införandet av 12,5 kHz kanalavstånd på 2~meters- och 70~cm-banden infördes
ett nytt system.
Man börjar med en bokstav som talar om vilket band det är
\begin{itemize}
  \item F för 51~MHz, kanalavstånd 10~kHz
  \item V för 145~MHz, kanalavstånd 12,5~kHz
  \item U för 430~kHz, kanalavstånd 12,5~kHz.
\end{itemize}
Kanalnumret börjar med 00 på varje sådant band och ökar med ett (1) för varje
kanal i bandet.
På 51 och 145~MHz används tvåsiffrig numrering och på 430~MHz tresiffrig.
För repeaterkanaler sätts ett R före bandbokstaven.

\section{70-centimetersbandet}
Repeaterskift 1600~kHz.

\begin{tabular}{ l | l | l }
  Kanal & Din sändar- & Din mottagar- \\
        & frekvens [MHz] & frekvens [MHz] \\
  \hline
  RU368 & 433,0000 & 434,6000 \\
  RU370 & 433,0250 & 434,6250 \\
  RU372 & 433,0500 & 434,6500 \\
  RU374 & 433,0750 & 434,6750 \\
  RU376 & 433,1000 & 434,7000 \\
  RU378 & 433,1250 & 434,7250 \\
  RU380 & 433,1500 & 434,7500 \\
  RU382 & 433,1750 & 434,7750 \\
  RU384 & 433,2000 & 434,8000 \\
  RU386 & 433,2250 & 434,8250 \\
  RU388 & 433,2500 & 434,8500 \\
  RU390 & 433,2750 & 434,8750 \\
  RU392 & 433,3000 & 434,9000 \\
  RU394 & 433,3250 & 434,9250 \\
  RU396 & 433,3500 & 434,9500 \\
  RU398 & 433,3750 & 434,9750 \\
\end{tabular}

\section{2-metersbandet}
Repeaterskift 600~kHz.

\begin{tabular}{ l | l | l }
  Kanal & Din sändar- & Din mottagar- \\
        & frekvens [MHz] & frekvens [MHz] \\
  \hline
  RV48 & 145,000 & 145,600 \\
  RV49 & 145,0125 & 145,6125 \\
  RV50 & 145,025 & 145,625 \\
  RV51 & 145,0375 & 145,6375 \\
  RV52 & 145,050 & 145,650 \\
  RV53 & 145,0625 & 145,6625 \\
  RV54 & 145,075 & 145,675 \\
  RV55 & 145,0875 & 145,6875 \\
  RV56 & 145,100 & 145,700 \\
  RV57 & 145,1125 & 145,7125 \\
  RV58 & 145,125 & 145,725 \\
  RV59 & 145,1375 & 145,7375 \\
  RV60 & 145,150 & 145,750 \\
  RV61 & 145,1625 & 145,7625 \\
  RV62 & 145,175 & 145,775 \\
  RV63 & 145,1875 & 145,7875 \\
\end{tabular}

\section{23-centimetersbandet}
Repeaterskift 6000~kHz.

\begin{tabular}{ l | l | l }
  Kanal & Din sändar- & Din mottagar- \\
        & frekvens [MHz] & frekvens [MHz] \\
  \hline
  RM0 & 1291,000 & 1297,000 \\
  RM1 & 1291,025 & 1297,025 \\
  RM2 & 1291,050 & 1297,050 \\
  RM3 & 1291,075 & 1297,075 \\
  RM4 & 1291,100 & 1297,100 \\
  RM5 & 1291,125 & 1297,125 \\
  RM6 & 1291,150 & 1297,150 \\
  RM7 & 1291,175 & 1297,175 \\
  RM8 & 1291,200 & 1297,200 \\
  RM9 & 1291,225 & 1297,225 \\
  RM10 & 1291,250 & 1297,250 \\
  RM11 & 1291,275 & 1297,275 \\
  RM12 & 1291,300 & 1297,300 \\
  RM13 & 1291,325 & 1297,325 \\
  RM14 & 1291,350 & 1297,350 \\
  RM15 & 1291,375 & 1297,375 \\
  RM16 & 1291,400 & 1297,450 \\
  RM17 & 1291,425 & 1297,475 \\
  RM18 & 1291,450 & 1297,450 \\
  RM19 & 1291,475 & 1297,475 \\
\end{tabular}

\section{Repeaterband med speciella egenskaper}
\subsection{6-metersbandet}
Repeaterskift 600~kHz.

\begin{tabular}{ l | l | l }
  Kanal & Din sändar- & Din mottagar- \\
        & frekvens [MHz] & frekvens [MHz] \\
  \hline
  RF81 & 51,210 & 51,810 \\
  RF83 & 51,230 & 51,830 \\
  RF85 & 51,250 & 51,850 \\
  RF87 & 51,270 & 51,870 \\
  RF89 & 51,290 & 51,890 \\
  RF91 & 51,310 & 52,910 \\
  RF93 & 51,330 & 52,930 \\
  RF95 & 51,350 & 52,950 \\
  RF97 & 51,370 & 52,970 \\
  RF99 & 51,390 & 52,990 \\
\end{tabular}

(dvs endast udda kanalnummer används).

Observera att det i Sverige, utöver amatörradiotillståndet, t.v. krävs
särskilda tillstånd för amatörradioanvändning i detta band.
På grund av den relativt låga frekvensen uppnås ofta överräckvidder p.g.a.
sporadisk vågutbredning via E-skiktet.
Man kan då uppnå förbindelser utan hjälp av repeater.

\subsection{10-metersbandet}
Repeaterskift 100~kHz.

\begin{tabular}{ l | l | l }
  Kanal & Din sändar- & Din mottagar- \\
        & frekvens [kHz] & frekvens [kHz] \\
  \hline
  - & 29560 & 29660 \\
  - & 29570 & 29670 \\
  - & 29580 & 29680 \\
  - & 29590 & 29690 \\
\end{tabular}

På grund av den relativt låga frekvensen uppnås stora räckvidder genom
jonosfärisk vågutbredning, särskilt under år med högt solfläckstal.
Även sporadisk vågutbredning via E-skiktet förekommer.
I båda fallen bör repeatertrafik undvikas.
