\chapter{Frekvenser för svenska amatörradiorepeatrar}
Vid direktförbindelser på höga frekvenser är räckvidden begränsad,
särskiltvid låg effekt och små antenner.
Med repeatrar med högt belägna antenner kan räckvidden förbättras,
vilket underlättar kommunikation med rörliga (mobila) radiostationer.
Eftersom sändaren och mottagaren i en repeater arbetar samtidigt, måste avståndet
mellan deras arbetsfrekvenser vara så stort att det inte uppstår ömsesidiga störningar.
Dessa arbetsfrekvenser kallas frekvenspar eller kanal och avståndet mellan dem
kallas repeaterskift, vilket är enhetligt inom repeaterbandet.

Frekvensparet i en repeater måste arbeta med omvänt frekvensläge i förhållande
till det i de stationer som den betjänar.
Kanalavståndet mellan repeatrarna i ett band är också enhetligt och sändningar över
repeatrarna måste naturligtvis ha mindre bandbredd än kanalavståndet.
Inom IARU har man enats bl.a. om frekvensparen för smalbandiga FM-repeatrar.
Se IARU:s bandplaner i Appendix F.
Frekvensplaner finns för repeatrar inom banden 51--52~MHz (6~m), 145--146~MHz
(2~m), 432--438~MHz (70~cm), 1240--1300~MHz (10~cm) samt 28000--29700~kHz (10~m).

Nya kanalnumreringsmetoden
I och med införandet av i kHz kanalavstånd på 2 meters- och 70 cm-banden har
ett nytt enkelt system införts.
Man börjar med en bokstav som talar om vilket band det är:
F för 51~MHz, kanalavstånd 10 kHz,
V för 145~MHz, kanalavstånd 12,5 kHz, 
U för 430~kHz, kanalavstånd 1
Kanalnumret börjar med 00 på varje sådant band och ökar med ett (1) för varje
kanal i bandet.
På 51 och 145 MHz används siffrig numrering och på 430 MHz tresiffrig.
För repeaterkanaler sätts ett R före bandbokstaven.

70-centimetersbandet (skift 1600~kHz)

\begin{tabular}{ l | l | l }
  Kanal & Din sändar- & Din mottagar- \\
        & frekvens [MHz] & frekvens [MHz] \\
  \hline
  RU368 & 433,0000 & 434,6000 \\
  RU370 & 433,0250 & 434,6250 \\
  RU372 & 433,0500 & 434,6500 \\
  RU374 & 433,0750 & 434,6750 \\
  RU376 & 433,1000 & 434,7000 \\
  RU378 & 433,1250 & 434,7250 \\
  RU380 & 433,1500 & 434,7500 \\
  RU382 & 433,1750 & 434,7750 \\
  RU384 & 433,2000 & 434,8000 \\
  RU386 & 433,2250 & 434,8250 \\
  RU388 & 433,2500 & 434,8500 \\
  RU390 & 433,2750 & 434,8750 \\
  RU392 & 433,3000 & 434,9000 \\
  RU394 & 433,3250 & 434,9250 \\
  RU396 & 433,3500 & 434,9500 \\
  RU398 & 433,3750 & 434,9750 \\
\end{tabular}

2-metersbandet (repeaterskift 600~kHz)

Kanal
Din sändarDin mottagarnr
frekvens MHz
frekvens MHz
RV48
145,000
145,600
RV49
145,0125
25
1
RV 50
145,025
1
RV51
145,0375
i
RV 52
145,050
145,650
RV 53
145,0625
1
RV 54
i45,075
1
RV55
145,0875
1
RV 56
145,100
i
RV 57
145,1125
1
RV 58
145,125
1
RV59
145,1375
i
RV60
145,150
1
RV61
145,1625
1
RV62
145,175
1
RV63
145,1875
i


E

H

23-centimetersbandet (skift 6000 kHz)
Kanal
nr

RMO

RM1
RM2
RM3
RM4
RM5
RM6
RM7
RM8
RM9
RM10
RM11
RM12
RM13
RM14
RM15
RM16
RM17
RM18
RM19

Din sändarfrekvens MHz
1291,000
1291,025
1291,050
1291,075
1291 '1 00
1291 '125
1291 '150
1291,175
1291,200
1291,225
1291,250
1291,275
1291,300
1291,325
1291,350
1291,375
1291,400
1291,425
1291,450
1291,475

Din mottagarfrekvens MHz
1297,000
1297,025
1297,050
1297,075
1297,100
1297,125
1297,150
1297,175
1297,200
1297,225
1297,250
1297,275
1297,300
1297,325
1297,350
1297,375
1297,450
1297,475
1297,450
1297,475

Repeaterband med speciella egenskaper

6-metersbandet (skift 600~kHz)
Din sändarDin mottagarKanal
nr
frekvens MHz frekvens MHz
RF81
51.21 O
51.81 O
RF83
51.230
51.830
RF85
51.250
51.850
RF87
51 .270
51 .870
RF89
51 .290
51 .890
RF91
51.31 O
52.91 O
RF93
51.330
52.930
RF95
51 .350
52.950
RF97
51.370
52.970
RF99
51.390
52.990
(dvs endast udda kanalnummer används).

Observera att det i Sverige, utöver amatörradiotillståndet, t.v. krävs
särskilda tillstånd för amatörradioanvändning i detta band.
På grund av den relativt låga frekvensen uppnås ofta överräckvidder p.g.a.
sporadisk vågutbredning via E-skiktet.
Man kan då uppnå förbindelser utan hjälp av repeater.
Observera, att i Sverige f.n. inga repeatrar finns i 6-metersbandet.

10-metersbandet (skift 100~kHz)

Kanal
Din sändarDin mottagarnr
frekvens kHz
frekvens kHz
29560
29660
29570
29670
29580
29680
29590
29690
På grund av den relativt låga frekvensen uppnås stora räckvidder genom
jonosfärisk vågutbredning, särskilt under år med högt solfläckstal.
Även sporadisk vågutbredning via E-skiktet förekommer.
I båda fallen bör repeatertrafik undvikas.
