% !TeX spellcheck = <none>
\section{Anropssignaler}

\subsection{Anropssignalernas syfte}

Alla radiosändare ska vara identifierbara, så att man kan veta vem
som sänder \cite[\S19.1]{ITU-RR}.
Detta görs genom att man sänder en anropsssignal.
Anropssignaler är internationellt koordinerade och unika, vilket är nödvändigt
när signalerna kan komma att höras över hela världen.
Systemet är gemensamt för kommersiell trafik och amatörradio, men vi kommer
enbart beröra de anropssignaler som är aktuella för amatörradio.

Alla sändningar med falsk eller missledande identifiering är förbjuden
\cite[\S19.2]{ITU-RR}!

Alla amatörradiosändningar ska vara identifierade \cite[\S19.4, \S19.5]{ITU-RR}.

Identifiering sker normalt i tal eller på morsetelegrafi, men även andra former
kan förekomma som är anpassade till modulationsmetoden som används.

Som sekundärt syfte kan man använda anropssignalen för att ropa upp en annan
station, vilken då kan unikt identifieras som mottagare.

Det finns flera sätt på vilka personen bakom en anropssignal kan identifieras.
För svenska anropssignaler tillhandahåller SSA en Callbook
(\url{http://www.ssa.se/}).
En annan populär variant är QRZ (\url{https://www.qrz.com/}) där man kan
registrera sig.
Anropssignalen används även för online-loggning av QSOn, så som
Logbook of the World (LoTW) (\url{https://lotw.arrl.org/}).

\subsection{Anropssignalernas sammansättning}

Varje land har unika anropssignaler för all sin radiotrafik.
Dessa utformas enligt ITU RadioReglemente (RR) \cite[\S19]{ITU-RR} på sätt,
som beror på syftet med varje särskild radiostation.
I RR finns definitioner för olika slags stationer, t.ex. stationer för fast
radio, landmobila stationer, stationer i fartyg, i sjöräddningsfarkoster,
i flygplan, amatörradiostationer o.s.v.

\subsection{Identifiering av amatörradiostationer}
\textbf{
HAREC b.\ref{HAREC.b.5.1}\label{myHAREC.b.5.1},
 b.\ref{HAREC.b.5.3}\label{myHAREC.b.5.3}
}

En radiostation ska identifieras med den anropssignal, som tilldelats av det
egna landets teleadministration (myndighet).
I Sverige är det Post- och telestyrelsen (PTS) som har ansvaret och genom
beslut delegerat handläggningen amatörradiosignaler till föreningen
Sveriges Sändareamatörer (SSA).

Anropssignalen meddelas i det amatörradiocertifikat som erhålls efter godkänt
kompetensprov.
Certifikat och anropssignaler som utfärdats före 1~oktober 2004, då undantaget
från tillståndsplikt infördes, gäller utan tidsgräns.

Amatörradiosignaler är uppbyggda på följande sätt
\cite[\S19.68, \S19.69]{ITU-RR}:

Ett tecken (ett av B, F, G, I, K, M, N, R eller W) och en siffra, följt av en
grupp av inte mer än 4 tecken, varav sista ska vara en bokstav.

Två tecken och en siffra, följt av en grupp av inte mer än 4 tecken, varav
sista ska vara en bokstav.

Till bokstav räknas A--Z och till siffra räknas 0--9 \cite[\S19.45]{ITU-RR}.
Ett tecken är antingen en bokstav eller siffra.

Sverige är tilldelat teckenkombinationer i serierna SAA--SMZ, 7SA--7SZ och
8SA--8SZ \cite[Appendix 42]{ITU-RR}.

Tillfälliga eventsignaler, för tillfällig användning, kan använda fler tecken
(0--9 och A--Z) än vad som anges ovan \cite[\S19.68A]{ITU-RR}.

\textbf{Exempel:} DL65DARC är en eventsignal för tyska (DL)
amatörradioföreningen DARC:s 65:års jubileum.

PTS regler för tilldelning av svenska signaler kan skilja sig från grundreglerna
i RR som anges ovan, men följer i allmänhet grundreglerna som angivits i RR.

Anropssignalerna för svenska amatörradiostationer är uppbyggda på följande
sätt, varvid med distrikt avses amatörradiodistrikt.

Amatörradiotillstånd (CEPT-tillstånd) för

\begin{tabular}{lll}
radioamatörer & SA & + distriktssiffra + bokstäver (SSA tilldelat) \\
radioamatörer & SM & + distriktssiffra + bokstäver (PTS tilldelat) \\
amatörklubbar & SK & + distriktssiffra + bokstäver \\
militära förband & SL & + distriktssiffra + bokstäver \\
amatörklubbar & SI & + distriktssiffra + bokstäver (specialtillstånd) \\
amatörklubbar & SJ & + distriktssiffra + bokstäver (specialtillstånd) \\
amatörklubbar & 7S & + distriktssiffra + bokstäver (specialtillstånd) \\
amatörklubbar & 8S & + distriktssiffra + bokstäver (specialtillstånd) \\
%\multicolumn{3}{l}{SSA-tillstånd inom SSA:s utbildningsverksamhet} \\
%& SH & + distriktssiffra + bokstäver (AAA- CZZ). \\
\end{tabular}

Signalserien SM är tilldelad av Televerket och sedemera PTS fram till 2004.
Signalserien SA är tilldelad av SSA från 2004.
Äldre signaler i SM-serien är tilldelad med två bokstäver, medan nyare SM och
SA signaler är treställiga bokstäver.

Utöver huvudsignalen finns även signaler tilldelade med en tilldelad
bokstav i de olika tillgängliga serierna.

\textbf{Exempel:} SM0XXX är en radioamatör som fått sin tilldelning av PTS.

\textbf{Exempel:} SA0XXX är en radioamatör som fått sin tilldelning av SSA.

\textbf{Exempel:} SK2ÄÖ är en amatörklubb.

\textbf{Exempel:} SM7Ö är en radioamatör med kort signal.

Sverige är indelat i amatörradiodistrikt med följande numrering och
utsträckning:

\begin{tabular}{rp{10cm}}
Distrikt & Utsträckning \\
0 & Stockholms (AB) län \\
1 & Gotlands (I) län \\
2 & Västerbottens (AC) och Norrbottens (BD) län \\
3 & Gävleborgs (X), Jämtlands (Z) och Västernorrlands (Y) län \\
4 & Örebro (T), Värmlands (S) och Dalarnas (W) län \\
5 & Östergötlands (E), Södermanlands (D), Västmanlands (U) och Uppsala (C) län\\
6 & Hallands (N), Älvsborgs (P), Göteborgs och Bohus (O) län samt Skaraborgs (R) län \\
7 & Malmöhus (M), Kristianstads (L), Blekinge (K), Kronobergs (G), Jönköpings (F) och Kalmar (H) län.\\
\end{tabular}

Distriktssiffran i signalen bestäms av det län som hemadressen är belägen inom.
Vid sändning utanför hemadressen bör det framgå av tillägg till signalen.

\textbf{Exempel:} SA0XXX är en radioamatör hemmahörande i Stockholms län.

\textbf{Exempel:} SM7YYY är en radioamatör hemmahörande i Jönköpings län.

\textbf{Exempel:} SK7AX är en amatörklubb hemmahörande i Jönköping län.

I Post- och telestyrelsens föreskrifter sägs dock inte vilken distriktssiffra
som ska användas, när sändning sker från annan plats än hemortsadressen.

Med stöd av praxis rekommenderar dock SSA att följande regler tillämpas:

\begin{itemize}
\item Vid trafik från en regelbundet använd fritidsbostad kan i
  anropssignalen användas den distriktssiffra som utvisar var
  fritidsbostaden är belägen.

\item Vid trafik från annan tillfällig plats bör anropssignalen
  åtföljas av snedstreck och siffran för det distrikt varifrån
  sändningen görs.
  \textbf{Exempel:} SM0XYZ/0, SM0XYZ/6 etc.

\item Vid trafik från mobil station bör den ordinarie anropssignalen
  även åtföljas av /M.
  \textbf{Exempel:} SM0XYZ/6M.

\item Vid trafik från mobil station inom hemorten kan dock den extra
  distriktssiffran utelämnas.
  \textbf{Exempel:} SM0XYZ/M.

\item Vid trafik från sjöfarkost bör den ordinarie anropssignalen
 åtföjas av /MM.

\item Vid trafik från luftfarkost bör den ordinarie anropssignalen
  åtföljas av /AM.

\item Vid trafik från svensk farkost på internationellt territorium
 kan distriktssiffran 8 användas.

\item Vid sändning från ett annat lands territorium gäller det landets
  bestämmelser.
  Vid osäkerhet -- Skaffa upplysningar från SSA:s reciprokfunktionär!

\item Utländsk radioamatör på besök i Sverige ska använda sin
  anropssignal från det egna landet, föregånget av SM*/ där *
  motsvaras av siffran för det svenska distrikt varifrån sändningen
  görs. \textbf{Exempel:} SM5/LA9XX.
\end{itemize}

\subsection{Nationella prefix}
\textbf{HAREC 
	b.\ref{HAREC.b.5.4}\label{myHAREC.b.5.4}
}

Nationella prefix att kunna

\begin{tabular}{llllll}
    Prefix & Land& & & &  \\
    LA & Norge & OH & Finland & OH0 & Åland\\
    OZ & Danmark & DL & Tyskland & EA & Spanien\\
    ES & Estland & F & Frankrike & G & Storbritannien\\
    HB & Schweiz & I & Italien & LY & Litauen\\
    OK & Tjekien & ON & Belgien & PA & Holland\\
    S5 & Slovenien & SP & Polen & SV & Grekland\\
    UA & Ryssland & YL & Lettland & EA8 & Kanarieöarna\\
    ZS & Sydafrika & HS & Thailand & JA & Japan\\
    K & USA & VE & Kanada & LU & Argentina\\
    PY & Brazilien & VK & Australien & ZL & Nya Zeeland\\
\end{tabular}


\section{Användning av anropssignal}
\textbf{HAREC
  b.\ref{HAREC.b.5.2}\label{myHAREC.b.5.2},
  b.\ref{HAREC.b.7.2.2}\label{myHAREC.b.7.2.2}
}

Både motstationens och den egna anropssignalen ska användas i början
och slutet av varje sändning.
Under sändningen ska anropssignalen upprepas ''med korta mellanrum'', utan
närmare precisering av mellanrummet.
Även om man inte har kontakt med en motstation, ska den egna anropssignalen
anges vid varje sändning.
Se vidare i PTS föreskrifter.

\section{Exempel på kontakt}
\textbf{HAREC
  b.\ref{HAREC.b.7.2.1}\label{myHAREC.b.7.2.1}
}

Det finns många sätt att genomföra en radiokontakt, men det finns
några grundregler för hur man uppträder och utväxlar samtal.
Ett trevligt och kamratligt uppträdande är en hederssak inom amatörradion.
Det behöver inte bli stelt för den skull!

Allmänt anrop är ett sätt att kalla på någon
-- vem som helst -- att kommunicera med.

På telegrafi låter det så här: CQ CQ CQ de SM0XYZ K, d.v.s anropet
först och därefter den egna signalen.
På telefoni låter det så här:
Allmänt anrop, allmänt anrop, allmänt anrop från SM0XYZ Kom.
Glöm inte Kom i slutet!

Riktat anrop gör man, när man vill tala med någon särskild station.
Då sänder man först signalen på den station, som man vill tala med och
därefter sin egen signal.

På telegrafi låter det så här:

-- SM0ÅÄÖ SM0ÅÄÖ SM0ÅÄÖ de SM0XYZ SM0XYZ SM0XYZ K

På telefoni låter det så här:

-- SM0ÅÄÖ SM0ÅÄÖ SM0ÅÄÖ från SM0XYZ SM0XYZ SM0XYZ Kom

Motstationen svarar förhoppningsvis på anropet, alltså

-- SM0XYZ från SM0ÅÄÖ Kom.

\subsection{Upprättad förbindelse}

När en station svarat på anrop, lämnar man först sin signalrapport
enligt RST-koden och presenterar sig med sitt förnamn och var man finns.
Motstationen kvitterar troligen med sina motsvarande uppgifter.

Varje gång, som man överlämnar ordet till motstationen
säger man först motstationens signal och därefter sin egen.
Därefter säger man Kom och lyssnar.
Om man har en telegrafiförbindelse och bara vill tala med den stationen kan man
sända KN (kom du och ingen annan (nobody else).

\subsection{Avsluta förbindelse}

När man så småningom avslutar kontakten tackar man för sig på och
utbyter avskedshälsningar. Då kan det låta så här:

-- Tack för en trevlig förbindelse och på återhörande. SM0ÅÄÖ från
SM0XYZ. Klart Slut.

Träna med din instruktör på att klara olika slags trafiksituationer!

\section{Innehåll i förbindelse}
\textbf{HAREC
  b.\ref{HAREC.b.7.2.3}\label{myHAREC.b.7.2.3}
}

Tidigare har det i Sverige varit reglerat vad innehållet får vara i
förbindelser, eller snarare vad de inte får innehålla.
Den regleringen är numera borttagen.
Man ska vara medveten om att samma regler och förutsättningar inte gäller i alla
länder och för deras radioamatörer.
Därför uppmanas att använda sunt förnuft, hålla god ton och respektera alla
amatörer. Se även IARU etik och trafikmetoder.

\section[Hederskod]{Radioamatörens hederskod}
\textbf{HAREC
  b.\ref{HAREC.b.7.1.1}\label{myHAREC.b.7.1.1}
}

Radioamatören är

\begin{tabular}{lp{9cm}}
  \textbf{HÄNSYNSFULL} &
     Han agerar aldrig medvetet på ett sätt som minskar nöjet för andra. \\
  & \\

  \textbf{LOJAL} &
  Han erbjuder lojalitet, uppmuntran och stöd åt andra amatörer, lokala klubbar,
  IARU organisationen i hans land genom vilken amatörradio i hans land
  representeras nationellt och internationellt.\\
  & \\

  \textbf{PROGRESSIV} &
  Han håller sin station på en hög teknisk nivå.
  Den är välbyggd och effektiv.
  Hans operationsteknik är oantastlig.\\
  & \\

  \textbf{VÄNLIG} &
  Han kommunicerar sakta och tålmodigt när så begärs;
  erbjuder kamratligt stöd och ger nybörjaren goda råd;
  vänlig assistans, samarbete och omtanke i andras intresse.
  Detta är kännetecknen för amatörandan.\\
  & \\

  \textbf{BALANSERAD} &
  Radio är en hobby och får aldrig orsaka konflikt i förpliktelser gentemot
  familj, arbete, skola eller samhälle.\\
  & \\

  \textbf{PATRIOTISK} &
  Hans station och hans kunnande står alltid till förfogande för att
  assistera land och samhälle.\\
\end{tabular}

-- anpassad från den ursprungliga Amateur's Code, skriven av Paul M. Segal, W9EEA, 1928.

\section[Ordningsregler]{Radioamatörens ordningsregler}
\textbf{HAREC
  b.\ref{HAREC.b.7.1.2}\label{myHAREC.b.7.1.2}
}

\subsection{Grundläggande principer}
\textbf{Grundläggande principer} som ska styra vårt \textbf{uppträdande} på
amatörbanden är:

\begin{itemize}
\item \textbf{Samhörighet, broderskap och kompiskänsla}: många, många av oss
  är aktiva i etern (vår spelplan).
  Vi är aldrig ensamma.
  Alla andra amatörer är våra kollegor, våra bröder och systrar, våra vänner.
  Agera därefter.
  Var alltid hänsynsfull.

\item \textbf{Tolerans}: inte alla amatörer delar nödvändigtvis samma
  uppfattning som du, och din uppfattning är kanske inte den bästa.
  Förstå att det finns andra med en annan uppfattning om ett visst tema.
  Var tolerant.
  Du har inte denna värld för dig själv.

\item \textbf{Anständighet}: aldrig får svordomar och oanständigheter yttras
  på banden.
  Ett sådant beteende säger ingenting om den person som de är avsedda för men
  mycket om den person som uttalar dem.
  Behåll ditt lugn i alla situationer.

\item \textbf{Förståelse}: Var snäll och förstå att alla inte är så smarta,
  så professionella eller så mycket expert som du.
  Om du vill göra något åt detta agera positivt (hur kan jag hjälpa till,
  hur kan jag förbättra, hur kan jag lära ut) i stället för negativt
  (med svordomar, förolämpningar etc.).
\end{itemize}

\subsection{Risken för konflikter}
\textbf{Endast en spelplan, etern}: alla radioamatörer vill spela sitt spel
eller utöva sin sport men det måste göras på en enda spelplan: våra amatörband.
Hundratusentals spelare på en enda spelplan leder ibland till konflikter.

Ett exempel: Plötsligt hör du någon ropa CQ på din frekvens (den frekvens du
har kört på en stund).
Hur är detta möjligt?
Du har varit igång här mer än en halvtimme på en helt ren frekvens!
Jo, visst är det möjligt; den där andra stationen tror kanske också att du stör
honom på HANS frekvens.
Kanske har skippet eller konditionerna ändrats?

\subsection{Hur undvika konflikter?}
\begin{itemize}
\item Genom att förklara för alla spelare vilka regler som gäller och genom
  att motivera dem att tillämpa dessa regler.
  De flesta konflikter orsakas av \textbf{okunskap}:
  många spelare känner inte till reglerna tillräckligt väl.

\item Dessutom hanteras många konflikter dåligt återigen på grund av
  \textbf{okunskap}.

\item Den IARU-etikhandbok som finns översatt på SSA:s webbplats avser att
  åtgärda denna brist på kunnande i huvudsak genom att lära ut hur man kan
  undvika konflikter av alla slag.
\end{itemize}

\subsection{Moraliska aspekter}
\begin{itemize}
\item I de flesta länder bryr sig myndigheterna inte om i detalj hur
  amatörerna uppför sig på banden, förutsatt att de håller sig till reglerna
  som myndigheten fastslagit.
\item Radioamatörerna anses vara \textbf{självstyrande}, detta betyder att
  självdisciplin måste utgöra basen i vårt agerande. Det betyder emellertid
  inte att radioamatörerna har en egen polisiär funktion!
\end{itemize}

\subsection{Förhållningsregler}

Vad menar vi med \textbf{förhållningsregler} (code of conduct)?
De är en  uppsättning regler baserade på såväl \textbf{etiska} principer som
\textbf{trafikmässiga hänsyn}.

\begin{itemize}
\item \textbf{Etik}: Etik bestämmer vår attityd och vårt allmänna uppförande
  som radioamatörer.
  Etik har med moral att göra.
  Etik utgör principerna för moral.

  Exempel: etiken säger oss att aldrig medvetet störa andra stationers
  radiotrafik.
  Detta är en moralisk regel.
  Det är omoraliskt att inte följa denna regel, likvärdigt med att fuska i en
  tävling.
\item \textbf{Praktiska regler}: för att hantera alla olika aspekter av
  vårt uppförande behövs utöver etik också en uppsättning regler baserade på
  \textbf{trafikmässiga hänsyn} och på \textbf{praxis och sedvänja}.
  För att undvika konflikter behöver vi också praktiska regler som styr
  vårt beteende på amatörbanden eftersom vårt huvudintresse är att köra
  radio på de olika banden.
  Vi avser här mycket \textbf{praktiska regler} och \textbf{riktlinjer} för
  situationer som ej är etikrelaterade.
  De flesta trafikmetoder (hur genomföra ett QSO, var får man köra,
  vad betyder QRZ, hur använda Q-koderna) hör hit.
  Respekt för dessa trafikmetoder säkerställer optimalt resultat och
  effektivitet i våra QSO och kommer att vara nyckeln till att undvika
  konflikter.
  Dessa trafikmetoder har tillkommit som ett resultat av daglig radiotrafik
  under många år och som ett resultat av den pågående tekniska utvecklingen.
\end{itemize}
