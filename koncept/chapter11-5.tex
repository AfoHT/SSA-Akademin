\section{Anropssignaler}

\subsection{Anropssignalernas syfte}

Alla radiosändare ska vara identifierbara, så att man kan identifiera vem
som sänder \cite[§19.1]{ITU-RR}. Detta görs genom att skicka en
identifikationssignal. Denna identifikationssignal är internationellt
koordinerad och unik, vilket är nödvändigt när signalerna kan komma att
höras över hela världen. Detta system är gemensamt för kommersiell trafik och
amatörradio, men vi kommer enbart beröra de signaler som är aktuella för
amatörradio.

Alla sändningar med falsk eller missledande identifiering är förbjuden \cite[§19.2]{ITU-RR}!

Alla amatör-radiosändningar ska vara identifierade \cite[§19.4, §19.5]{ITU-RR}.

Identifiering sker normalt i tal eller Morse, men även andra former kan
förekomma som är anpassad till modulationsmetoden som används.

Som sekundärt syfte kan man använda signalen för att ropa upp en annan station,
vilken då kan unikt identifieras som mottagare.

Det finns flera som personen bakom en signal kan identifieras. För svenska
signaler tillhandahåller SSA en Callbook (\url{http://www.ssa.se/}). En annan
populär variant är QRZ (\url{https://www.qrz.com/}) där man kan registrera sig.
Signalen används även för online-loggning av QSOn, så som Logbook of the World (LoTW) (\url{https://lotw.arrl.org/}).

\subsection{Anropssignalernas sammansättning}

Varje land har unika anropssignaler för all sin radiotrafik.
Signalerna utformas enligt ITU RadioReglemente (RR) \cite[§19]{ITU-RR} på sätt,
som beror på syftet med varje särskild radiostation. I RR finns
definitioner för olika slags stationer, t. ex. stationer för fast
radio, landmobila stationer, stationer i fartyg, i
sjöräddningsfarkoster, i flygplan, amatörradiostationer o.s.v.

\subsection{Identifiering av amatörradiostationer}
\textbf{
HAREC b.\ref{HAREC.b.5.1}\label{myHAREC.b.5.1},
 b.\ref{HAREC.b.5.3}\label{myHAREC.b.5.3}
}

En radiostation ska identifieras med den anropssignal, som
tilldelats av det egna landets teleadministration (myndighet). I Sverige
är det Post- och telestyrelsen (PTS) som har ansvaret och genom
delegationsbeslut delegerat handläggningen amatörradiosignaler till föreningen Sveriges SändareAmatörer (SSA).

Anropssignalen meddelas i det radioamatörcertifikat som erhålls efter godkänt kompetensprov. Certifikat och anropssignaler som utfärdats före den 1 oktober 2004 då undantaget för tillståndsplikt infördes gäller utan tidsgräns.

Amatörradiosignaler är uppbyggda på följande sätt \cite[§19.68, §19.69]{ITU-RR}:

Ett tecken (ett av B, F, G, I, K, M, N, R eller W) och en siffra, följt av en
grupp av inte mer än 4 tecken, varav sista ska vara en bokstav.

Två tecken och en siffra, följt av en grupp av inte mer än 4 tecken, varav
sista ska vara en bokstav.

Till bokstav räknas A-Z och till siffra räknas 0-9 \cite[§19.45]{ITU-RR}.
Ett tecken är antingen en bokstav eller siffra.

En kombinatorisk analys av de bägge reglerna ger:

\begin{tabular}{l|l|l}
                & Antal    &  \\
  Tecken-       & kombi-   &  \\
  kombinationer & nationer & Anmärkningar \\
  \hline
  Y0A - Y9Z & 260 & \\
  Y00A - Y99Z & 2600 & \\
  Y0AA - Y9ZZ & 6760 & \\
  Y000A - Y999Z & 26000 & \\
  Y00AA - Y99ZZ & 67600 & \\
  Y0A0A - Y9Z9Z & 67600 & \\
  Y0AAA - Y9ZZZ & 175760 & Första tecknet ``Y'' räcker ensamt \\
  Y0000A - Y9999Z & 260000 & som nationell identitet om det är \\
  Y000AA - Y999ZZ & 676000 & B, F, G, I, K, M, N, R eller W. \\
  Y00A0A - Y99Z9Z & 676000 & \\
  Y00AAA - Y99ZZZ & 1757600 & \\
  Y0A00A - Y9Z99Z & 676000 & \\
  Y0A0AA - Y9Z9ZZ & 1757600 & \\
  Y0AA0A - Y9ZZ9Z & 1757600 & \\
  Y0AAAA - Y9ZZZZ & 4569760 & \\
  \hline
  XX0A - XX9Z & 260 & \\
  XX00A - XX99Z & 2600 & \\
  XX0AA - XX9ZZ & 6760 & \\
  XX000A - XX999Z & 26000 & \\
  XX00AA - XX99ZZ & 67600 & \\
  XX0A0A - XX9Z9Z & 67600 & \\
  XX0AAA - XX9ZZZ & 175760 & ``XX'' är de två första tecknen \\
  XX0000A - XX9999Z & 260000 & i en tilldelad signalserie \\
  XX000AA - XX099ZZ & 676000 & \\
  XX00A0A - XX99Z9Z & 676000 & \\
  XX00AAA - XX99ZZZ & 1757600 & \\
  XX0A00A - XX9Z99Z & 676000 & \\
  XX0A0AA - XX9Z9ZZ & 1757600 & \\
  XX0AA0A - XX9ZZ9Z & 1757600 & \\
  XX0AAAA - XX9ZZZZ & 4569760 & \\
\end{tabular}

Sverige är tilldelat teckenkombinationer i serierna SAA - SMZ, 7SA - 7SZ och
8SA - 8SZ \cite[Appendix 42]{ITU-RR}.

Tillfälliga eventsignaler, för tillfällig användning, kan använda fler tecken
(0-9 och A-Z) än vad som anges ovan \cite[§19.68A]{ITU-RR}.

\textbf{Exempel:} DL65DARC är en eventsignal för Tyska (DL) amatörradio
föreningen DARCs 65:års jubileum.

PTS regler för tilldelning av svenska signaler kan skilja från grundreglerna
i RR som anges ovan.

Anropssignalerna för svenska amatörradiostationer är uppbyggda på följande
sätt, varvid med distrikt avses amatörradiodistrikt.

Amatörradiotillstånd (CEPT-tillstånd) för

\begin{tabular}{lll}
radioamatörer & SA & + distriktssiffra + bokstäver (SSA tilldelat) \\
radioamatörer & SM & + distriktssiffra + bokstäver (PTS tilldelat) \\
amatörklubbar & SK & + distriktssiffra + bokstäver \\
militära förband & SL & + distriktssiffra + bokstäver \\
amatörklubbar & SI & + distriktssiffra + bokstäver (specialtillstånd) \\
amatörklubbar & SJ & + distriktssiffra + bokstäver (specialtillstånd) \\
amatörklubbar & 7S & + distriktssiffra + bokstäver (specialtillstånd) \\
amatörklubbar & 8S & + distriktssiffra + bokstäver (specialtillstånd) \\
%\multicolumn{3}{l}{SSA-tillstånd inom SSAs utbildningsverksamhet} \\
%& SH & + distriktssiffra + bokstäver (AAA- CZZ). \\
\end{tabular}

Signalserien SM är tilldelad av Televerket och sedemera PTS fram till 2004.
Signalserien SA är tilldelad av SSA från 2004. Äldre signaler i SM-serien är
tilldelad med två bokstäver, medan nyare SM och SA signaler är treställiga
bokstäver.

Utöver huvudsignalen finns även signaler tilldelade med en tilldelad
bokstav i de olika tillgängliga serierna.

\textbf{Exempel:} SM0UTY är en radioamatör som fått sin tilldelning av PTS.

\textbf{Exempel:} SA0MAD är en radioamatör som fått sin tilldelning av SSA.

\textbf{Exempel:} SK0UX är en amatörklubb.

\textbf{Exempel:} SM0W är en radioamatör med kort signal.

Sverige är indelat i amatörradiodistrikt med följande numrering och
utsträckning:

\begin{tabular}{rp{10cm}}
Distrikt & Utsträckning \\
0 & Stockholms (AB) län \\
1 & Gotlands (I) län \\
2 & Västerbottens (AC) och Norrbottens (BD) län \\
3 & Gävleborgs (X), Jämtlands (Z) och Västernorrlands (Y) län \\
4 & Örebro (T), Värmlands (S) och Dalarnas (W) län \\
5 & Östergötlands (E), Södermanlands (D), Västmanlands (U) och Uppsala (C) län\\
6 & Hallands (N), Älvsborgs (P), Göteborgs och Bohus (O) län samt Skaraborgs (R) län \\
7 & Malmöhus (M), Kristianstads (L), Blekinge (K), Kronobergs (G), Jönköpings (F) och Kalmar (H) län.\\
\end{tabular}

Distriktssiffran i signalen bestäms av det län som hemadressen är
belägen inom. Vid sändning utanför hemadressen bör det framgå av
tillägg till signalen.

\textbf{Exempel:} SA0MAD är en radioamatör hemmahörande i Stockholms län.

\textbf{Exempel:} SM7NTJ är en radioamatör hemmahörande i Jönköpings län.

\textbf{Exempel:} SK7AX är en amatörklubb hemmahörande i Jönköping län.

I Post- och telestyrelsens föreskrifter sägs dock inte vilken
distriktssiffra som ska användas, när sändning sker från annan plats
än hemortsadressen.

Med stöd av praxis rekommenderar dock SSA att följande regler
tillämpas:

\begin{itemize}

\item Vid trafik från en regelbundet använd fritidsbostad kan i
  anropssignalen användas den distriktssiffra som utvisar var
  fritidsbostaden är belägen.

\item Vid trafik från annan tillfällig plats bör anropssignalen
  åtföljas av snedstreck och siffran för det distrikt varifrån
  sändningen görs. \textbf{Exempel:} SM0XYZ/0, SM0XYZ/6 etc.

\item Vid trafik från mobil station bör den ordinarie anropssignalen
  även åtföljas av /M.  \textbf{Exempel:} SM0XYZ/6M.

\item Vid trafik från mobil station inom hemorten kan dock den extra
  distriktssiffran utelämnas.  \textbf{Exempel:} SM0XYZ/M.

\item Vid trafik från sjöfarkost bör den ordinarie anropssignalen
 åtföjas av /MM.

\item Vid trafik från luftfarkost bör den ordinarie anropssignalen
  åtföljas av /AM.

\item Vid trafik från svensk farkost på internationellt territorium
 kan distriktssiffran 8 användas.

\item Vid sändning från ett annat lands territorium gäller det landets
  bestämmelser.  Vid osäkerhet- Skaffa upplysningar från SSA:s
  reciprokfunktionär!

\item Utländsk radioamatör på besök i Sverige ska använda sin
  anropssignal från det egna landet, föregånget av SM*/ där *
  motsvaras av siffran för det svenska distrikt varifrån sändningen
  görs. \textbf{Exempel:} SM5/LA8PV.
\end{itemize}

\section{Användning av anropssignaler}
\textbf{
HAREC b.\ref{HAREC.b.5.2}\label{myHAREC.b.5.2}
}

Både motstationens och den egna anropssignalen ska användas i början
och slutet av varje sändning.  Under sändningen ska anropssignalen
upprepas ``med korta mellanrum'', utan närmare precisering av
mellanrummet.  Även om man inte har kontakt med en motstation, ska
den egna anropssignalen anges vid varje sändning.  Se vidare i PTS
föreskrifter.

%\chapter{Trafikmetoder}

\section{Hur man genomför en radiokontakt}

Det finns många sätt att genomföra en radiokontakt, men det finns
några grundregler för hur man uppträder och utväxlar samtal. Ett
trevligt och kamratligt uppträdande är en hederssak inom
amatörradion. Det behöver inte bli stelt för den skull!

Allmänt anrop är ett sätt att kalla på någon
--- vem som helst --- att kommunicera med.

På telegrafi låter det så här: CQ CQ CQ de SM0XYZ K, d.v.s anropet
först och därefter den egna signalen.  På telefoni låter det så här:
Allmänt anrop, allmänt anrop, allmänt anrop från SM0XYZ Kom. Glöm inte
Kom i slutet!

Riktat anrop gör man, när man vill tala med någon särskild station. Då
sänder man först signalen på den station, som man vill tala med och
därefter sin egen signal.

På telegrafi låter det så här:

--- SM0ÅÄÖ SM0ÅÄÖ SM0ÅÄÖ de SM0XYZ SM0XYZ SM0XYZ K

På telefoni låter det så här:

--- SM0ÅÄÖ SM0ÅÄÖ SM0ÅÄÖ från SM0XYZ SM0XYZ SM0XYZ Kom

Motstationen svarar förhoppningsvis på anropet, alltså

--- SM0XYZ från SM0ÅÄÖ Kom.

\subsection{Upprättad förbindelse}

När en station svarat på anrop, lämnar man först sin signalrapport
enligt RST-koden och presenterar sig med sitt förnamn och var man
finns. Motstationen kvitterar troligen med sina motsvarande
uppgifter.

Varje gång, som man överlämnar ordet till motstationen
säger man först motstationens signal och därefter sin egen. Därefter
säger man Kom och lyssnar. Om man har en telegrafiförbindelse och bara
vill tala med den stationen kan man sända KN (kom du och ingen annan
(nobody else).

\subsection{Avsluta förbindelse}

När man så småningom avslutar kontakten tackar man för sig på och
utbyter avskedshälsningar.  Då kan det låta så här:

--- Tack för en trevlig förbindelse och på återhörande. SM0ÅÄÖ från
SM0XYZ. Klart Slut.

Träna med din instruktör på att klara olika slags trafiksituationer!
