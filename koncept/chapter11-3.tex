\section{Trafikförkortningar, vanliga i amatörradio}
\textbf{
HAREC b.\ref{HAREC.b.3.1}\label{myHAREC.b.3.1},
 b.\ref{HAREC.b.3.2}\label{myHAREC.b.3.2},
 b.\ref{HAREC.b.3.3}\label{myHAREC.b.3.3},
 b.\ref{HAREC.b.3.4}\label{myHAREC.b.3.4},
 b.\ref{HAREC.b.3.5}\label{myHAREC.b.3.5},
 b.\ref{HAREC.b.3.6}\label{myHAREC.b.3.6},
 b.\ref{HAREC.b.3.7}\label{myHAREC.b.3.7},
 b.\ref{HAREC.b.3.9}\label{myHAREC.b.3.9},
 b.\ref{HAREC.b.3.10}\label{myHAREC.b.3.10},
 b.\ref{HAREC.b.3.11}\label{myHAREC.b.3.11},
 b.\ref{HAREC.b.3.12}\label{myHAREC.b.3.12}
}

Utöver Q-koden och klartext används vid morsetelegrafering även andra
trafikförkortningar. Eftersom det internationella radiospråket är
engelska, är förkortningar av engelska ord vanligast.  Förkortningar
bör emellertid inte användas i onödan. En ovan operatör vid
motstationen kan då få svårt att förstå meddelandet.

\subsection{Urval för radioamatörer}

I CEPT-rekommendation T/R 61-02 nämns utöver Q-koden följande övriga
trafikförkortningar, som berör amatörradio.  Radioamatörerna använder
i praktiken många fler trafikförkortningar än dessa.

I reglementsprovet för radioamatörcertifikat ingår frågor om
trafikförkortningar, se tabell \ref{tab:trafikförkortningar}.

\begin{table}
  \label{tab:trafikförkortningar}
  \caption{Trafikförkortningar -- urval för radioamatörer}
  \begin{tabular}{lll}
    Förkort- & & \\
    ning & Engelskt uttryck & Svensk betydelse \\
    \hline
    BK & break & avbryt(-a) (sändningen) \\
    CQ & ''seek you'' & allmänt anrop, till alla \\
    CW & continuous waves & telegrafi (A1A) \\
    DE & franska ''de'' & från ..... (anropssignal) \\
    K  & come & ''kom'' \\
    MSG & message & meddelande, telegram \\
    PSE & please & var god (att \dots) \\
    R & received & allt uppfattat, mottaget \\
   RST & readability, & läsbarhet \\
   & signal-strength, & signalstryka \\
   & tone-report & ton \\
    RX & receiver & mottagare \\
    TX & transmitter & sändare \\
    UR & your & din, ditt, dina, er \\
  \end{tabular}
\end{table}

Utöver ovanstående trafikförkortningar upptas i CEPT-rekommendationen
även följande bokstavskombinationer, vilka används i teleprintertrafik
i stället för motsvarande morsetecken, slagna utan tecken mellanrum.
(Strecket ovanför bokstäverna betecknar att det inte finns något
mellanrum).

\begin{tabular}{lll}
  \(\overline{\mathrm{AR}}\) & sluttecken & \(+\) \\
  \(\overline{\mathrm{VA}}\) eller \(\overline{\mathrm{SK}}\) & avslutningstecken & @ \\
\end{tabular}

Ett exempel på en avsnitt ur en amatörradiosändning, där
trafikförkortningar används särskilt flitigt:

''gm es tnx vy much om fer ur rprt. u are cmg in hr ufb. my tx is
.... and rx .... anta 3 el beam . condx hr gud mni dx stns hrd . wl nw
nil so tks es 73''

I klartext ser exemplet ut så här: ''good morning and thank you very
much Old Man for your report. You are coming in here ultra fine
business. My transmitter is .....  and receiver .. ... antenna is a 3
element beam. Conditions here are good many stations heard. Weil now
nothing for you so thanks and kindest regards''
