\chapter{REGLER OCH TRAFIKMETODER}

Naturlagarna begränsar frekvensområdet som kan användas för radiosändningar.
En del sändningar riktar sig till många lyssnare, andra sändningar sker
mellan två personer. Många radiosändningar delar på samma frekvensområde och
ingen part vill bli störd av någon annan.

För att minimera risken för störningar och utnyttja frekvensområdet effektivt
sker ett samarbete inom ITU mellan många länders administrationer och
radiotjänster.

Samarbetet tar fram hur frekvenserna och utrymmet för radiokommunikation ska
fördelas och användas samt prioriteringar om nyttjanderätt, sändningsslag,
effekter, räckvidder med mera.

Liksom det finns lagar och trafikbestämmelser för flyg, sjöfart och landtrafik
så regleras sedan mycket länge även radiotrafik av alla de slag. Utöver
nationella regler finns det mellanstatliga (bilaterala), regionala och
internationella överenskommelser om radiotrafik. Detta gäller även
amatörradiotrafik, som är en internationell radiotjänst.

Frekvensbanden för amatörradio har tilldelats vid internationella konferenser
där IARU har representerat radioamatörerna. De svenska föreskrifterna för
amatörradio påverkas därför av internationella intressen och överenskommelser.

Den \emph{Internationella TeleKonventionen (ITC)} är den
överenskommelse på vilken verksamheten inom den \emph{Internationella
TeleUnionen (ITU)} -- bygger. Konventionen kompletteras av det internationella
Radioreglementet (ITU-RR), vilket omfattar huvudregler som överenskommits mellan
alla medlemmar inom ITU.

Konventionen och radioreglementet är bindande för alla stater som ratificerat
konventionen. De avsteg, som ett land vill göra och övriga länder godtar,
skrivs in som s.k. fotnoter. Radioreglementet omfattar alla radiotjänsters
verksamhet, däribland Amatör- och amatörsatellittjänsterna.

\textbf{OBSERVERA!  Som radioamatör är du skyldig att följa gällande
  bestämmelser för amatörradioanvändning i det land som du vistas i.
  Förvissa dig om att du har senaste utgåvan!  Vid osäkerhet --
  rådfråga PTS!}
