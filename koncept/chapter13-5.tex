% !TeX spellcheck = <none>
\section{Anropssignaler}

\subsection{Anropssignalernas syfte}

Alla radiosändare ska vara identifierbara, så att man kan veta vem
som sänder \cite[\S19.1]{ITU-RR}.
Identifiering görs med hjälp av en anropssignal, som är en kombination av
bokstäver, (A--Z) och siffror (0--9). \cite[\S19.45]{ITU-RR}.
Ett tecken är antingen en bokstav eller siffra.
Nationella bokstäver som Å, Ä och Ö samt andra specialtecken används inte.
Anropssignaler är internationellt koordinerade och unika, vilket är nödvändigt
när signalerna kan komma att höras över hela världen.
Systemet är gemensamt för kommersiell trafik och amatörradio, men vi kommer
enbart beröra de anropssignaler som är aktuella för amatörradio.

Alla sändningar med falsk eller missledande identifiering är förbjuden
\cite[\S19.2]{ITU-RR}!

Alla amatörradiosändningar ska vara identifierade \cite[\S19.4, \S19.5]{ITU-RR}.

Identifiering sker normalt i tal eller på morsetelegrafi, men även andra former
kan förekomma som är anpassade till modulationsmetoden som används.

Det finns flera sätt på vilka personen bakom en anropssignal kan identifieras.
För svenska anropssignaler tillhandahåller SSA en Callbook
(\href{http://www.ssa.se/}{www.ssa.se}).
En annan populär variant är QRZ (\href{https://www.qrz.com/}{www.qrz.com}) där
man kan registrera sig.
Anropssignalen används även för online-loggning av kontakter, så som
Logbook of the World (LoTW) (\href{https://lotw.arrl.org/}{lotw.arrl.org}).

\subsection{Anropssignalernas sammansättning}

Varje land disponerar en eller flera serier med unika anropssignaler för all
sin radiotrafik.
Dessa utformas enligt ITU Radioreglemente (RR) \cite[\S19]{ITU-RR} på sätt,
som beror på syftet med varje särskild radiostation.
I RR finns definitioner för olika slags stationer, till exempel stationer för fast
radio, landmobila stationer, stationer i fartyg, i sjöräddningsfarkoster,
i flygplan, amatörradiostationer osv.

\subsection{Identifiering av amatörradiostationer}
\textbf{
HAREC b.\ref{HAREC.b.5.1}\label{myHAREC.b.5.1},
 b.\ref{HAREC.b.5.3}\label{myHAREC.b.5.3}
}

En radiostation ska identifieras med den anropssignal, som tilldelats av det
egna landets teleadministration (myndighet).
I Sverige är det Post- och telestyrelsen (PTS) som har ansvaret och som genom
beslut har delegerat handläggningen av amatörradiosignaler till Föreningen
Sveriges Sändareamatörer (SSA).

Anropssignalen meddelas i det amatörradiocertifikat som erhålls efter godkänt
kompetensprov.

Anropssignaler för amatörradio är uppbyggda av ett prefix, en siffra och ett
suffix på följande sätt \cite[\S19.68, \S19.69]{ITU-RR}:

Prefixet består vanligtvis av två tecken, exempelvis SM~(Sverige), 9A~(Kroatien)
eller S5~(Slovenien).
Prefixet kan ibland bestå av en ensam bokstav, som i så fall måste vara någon
av B, F, G, I, K, M, N, R eller W.

Sverige är tilldelat prefix i serierna SA--SM, 7S och 8S
\cite[Appendix 42]{ITU-RR}.

Prefixet följs av en siffra och ett suffix. Suffixet består av minst ett och
högst fyra tecken, där det sista tecknet inte får vara en siffra.

Anropssignaler för speciella ändamål, exempelvis för att fira något jubileum,
kan ha suffix som består av fler än fyra tecken \cite[\S19.68A]{ITU-RR}.
Sådana anropssignaler, eller andra som inte följer formatmallen, behöver i så
fall godkännas av PTS innan de kan tilldelas av SSA.

\textbf{Exempel:} DL65DARC är en eventsignal för tyska (DL)
amatörradioföreningen DARC:s 65:års jubileum.

PTS regler för tilldelning av svenska anropssignaler kan skilja sig från
grundreglerna i RR som anges ovan, men följer i allmänhet dessa.

Anropssignaler för svenska amatörradiostationer är uppbyggda på följande
sätt, varvid med distrikt avses amatörradiodistrikt.

\begin{tabular}{lll}
enskilda radioamatörer & SA & + distriktssiffra + treställigt suffix (grundsignal) \\
enskilda radioamatörer & SM & + distriktssiffra + två- eller treställigt suffix (grundsignal) \\
amatörradioklubbar & SA & + distriktssiffra + tvåställigt suffix \\
amatörradioklubbar & SK & + distriktssiffra + tvåställigt suffix \\
militära förband och FRO & SL & + distriktssiffra + två- eller treställigt suffix \\
\end{tabular}

Signalserien SM är tilldelad av Televerket och sedemera PTS fram till 2009.
Signalserien SA är tilldelad av SSA från 2004.
Äldre anropssignaler i SM-serien är tilldelade med tvåställiga suffix, medan
nyare SM- och SA-signaler har treställiga suffix.

Utöver grundsignalen finns även extra anropssignaler tilldelade i de övriga tillgängliga serierna.

\textbf{Exempel:} SM0XXX är en radioamatör som fått sin tilldelning av PTS.

\textbf{Exempel:} SA0XXX är en radioamatör som fått sin tilldelning av SSA.

\textbf{Exempel:} SK2XX är en amatörklubb.

\textbf{Exempel:} SM7X är en radioamatör med kort anropssignal.

Sverige är indelat i amatörradiodistrikt med följande numrering och
utsträckning:

\begin{tabular}{rp{10cm}}
Distrikt & Utsträckning \\
0 & Stockholms (AB) län \\
1 & Gotlands (I) län \\
2 & Västerbottens (AC) och Norrbottens (BD) län \\
3 & Gävleborgs (X), Jämtlands (Z) och Västernorrlands (Y) län \\
4 & Örebro (T), Värmlands (S) och Dalarnas (W) län \\
5 & Östergötlands (E), Södermanlands (D), Västmanlands (U) och Uppsala (C) län\\
6 & Hallands (N) och Västra Götalands (O) län \\
7 & Skåne (M), Blekinge (K), Kronobergs (G), Jönköpings (F) och Kalmar (H) län.\\
\end{tabular}

Distriktssiffran i anropssignalen bestäms av det län som hemadressen är belägen inom.
Vid sändning utanför hemadressen bör det framgå av tillägg till anropssignalen.

\textbf{Exempel:} SA0XXX är en radioamatör hemmahörande i Stockholms län.

\textbf{Exempel:} SM7YYY är en radioamatör hemmahörande i Jönköpings län.

\textbf{Exempel:} SK7AX är en amatörklubb hemmahörande i Jönköping län.

I Post- och telestyrelsens föreskrifter sägs dock inte vilken distriktssiffra
som ska användas, när sändning sker från annan plats än hemortsadressen.

Med stöd av praxis rekommenderar dock SSA att följande regler tillämpas:

\begin{itemize}
\item Vid trafik från en regelbundet använd fritidsbostad kan i
  anropssignalen användas den distriktssiffra som utvisar var
  fritidsbostaden är belägen.

\item Vid trafik från annan tillfällig plats bör anropssignalen
  åtföljas av snedstreck och siffran för det distrikt varifrån
  sändningen görs.
  \textbf{Exempel:} SM0XYZ/0, SM0XYZ/6 etc.

\item Vid trafik från mobil station bör den ordinarie anropssignalen
  även åtföljas av /M.
  \textbf{Exempel:} SM0XYZ/6/M.

\item Vid trafik från mobil station inom hemorten kan dock den extra
  distriktssiffran utelämnas.
  \textbf{Exempel:} SM0XYZ/M.

\item Vid trafik från sjöfarkost bör den ordinarie anropssignalen
 åtföjas av /MM.

\item Vid trafik från luftfarkost bör den ordinarie anropssignalen
  åtföljas av /AM.

\item Vid trafik från svensk farkost på internationellt territorium
 kan distriktssiffran 8 användas.

\item Vid sändning från ett annat lands territorium gäller det landets
  bestämmelser.
  Vid osäkerhet -- Vänd dig till SSA!

\item Utländsk radioamatör på besök i Sverige ska använda sin
  anropssignal från det egna landet, föregånget av SM/. \textbf{Exempel:} SM/LA9XX \cite{TR6101}.
\end{itemize}

\subsection{Nationella prefix}
\textbf{HAREC 
	b.\ref{HAREC.b.5.4}\label{myHAREC.b.5.4}
}

Nationella prefix att kunna

\begin{tabular}{llllll}
    Prefix & Land& & & &  \\
    LA & Norge & OH & Finland & OH0 & Åland\\
    OZ & Danmark & DL & Tyskland & EA & Spanien\\
    ES & Estland & F & Frankrike & G & Storbritannien\\
    HB & Schweiz & I & Italien & LY & Litauen\\
    OK & Tjekien & ON & Belgien & PA & Holland\\
    S5 & Slovenien & SP & Polen & SV & Grekland\\
    UA & Ryssland & YL & Lettland & EA8 & Kanarieöarna\\
    ZS & Sydafrika & HS & Thailand & JA & Japan\\
    K & USA & VE & Kanada & LU & Argentina\\
    PY & Brazilien & VK & Australien & ZL & Nya Zeeland\\
\end{tabular}


\section{Användning av anropssignal}
\textbf{HAREC
  b.\ref{HAREC.b.5.2}\label{myHAREC.b.5.2},
  b.\ref{HAREC.b.7.2.2}\label{myHAREC.b.7.2.2}
}

Både motstationens och den egna anropssignalen ska användas i början
och slutet av varje sändning.
Under sändningen ska anropssignalen upprepas ''med korta mellanrum'', utan
närmare precisering av mellanrummet.
Även om man inte har kontakt med en motstation, ska den egna anropssignalen
anges vid varje sändning.
Se vidare i PTS föreskrifter.

\section{Exempel på kontakt}
\textbf{HAREC
  b.\ref{HAREC.b.7.2.1}\label{myHAREC.b.7.2.1}
}

Det finns många sätt att genomföra en radiokontakt, men det finns
några grundregler för hur man uppträder och utväxlar samtal.
Ett trevligt och kamratligt uppträdande är en hederssak inom amatörradion.
Det behöver inte bli stelt för den skull!

Allmänt anrop är ett sätt att kalla på någon
-- vem som helst -- att kommunicera med.

På telegrafi låter det så här:

-- CQ CQ CQ de SM0XYZ K

det vill säga anropet först och därefter den egna anropssignalen.

På telefoni låter det så här:

-- Allmänt anrop, allmänt anrop, allmänt anrop från SM0XYZ Kom

Glöm inte Kom i slutet!

Riktat anrop gör man, när man vill tala med någon särskild station.
Då sänder man först anropssignalen på den station, som man vill tala med och
därefter sin egen anropssignal.

På telegrafi låter det så här:

-- SM0ZYX SM0ZYX SM0ZYX de SM0XYZ SM0XYZ SM0XYZ K

På telefoni låter det så här:

-- SM0ZYX SM0ZYX SM0ZYX från SM0XYZ SM0XYZ SM0XYZ Kom

Motstationen svarar förhoppningsvis på anropet, alltså

-- SM0XYZ från SM0ZYX Kom

\subsection{Upprättad förbindelse}

När en station svarat på anrop, lämnar man först sin signalrapport
enligt RST-koden och presenterar sig med sitt förnamn och berättar var man finns.
Motstationen kvitterar troligen med sina motsvarande uppgifter.

När man överlämnar ordet till motstationen avslutar man meningen med Kom och lyssnar.
Om man har en telegrafiförbindelse och bara vill tala med den stationen kan man
sända KN (kom du och ingen annan (nobody else).

Om förbindelsen varar länge, är det lämpligt att upprepa anropssignalerna
ungefär var tionde minut vid överlämning.

-- SM0ZYX från SM0XYZ Kom

\subsection{Avsluta förbindelse}

När man så småningom avslutar kontakten tackar man för sig på och utbyter
avskedshälsningar. Då kan det låta så här:

-- Tack för en trevlig förbindelse och på återhörande. SM0ZYX från
SM0XYZ. Klart Slut.

Träna med din instruktör på att klara olika slags trafiksituationer!

\subsection{Second operator}
\index{Second operator}
\label{secondoperator}

Den som självständigt använder en amatörradiosändare ska ha ett
amatörradiocertifikat.
Det finns ett undantag från kravet på amatörradiocertifikat då en person
tillfälligt använder en amatörradiosändare under uppsikt av någon som har ett
amatörradiocertifikat.
Detta kallas \emph{second operator} och innebär att en person som saknar
amatörradiocertifikat kan agera operatör jämte en person som har ett.

I Sverige är det reglerat i undantagsföreskriften PTSFS 2015:4 som tas upp i
avsnitt \ref{PTSFS2015:4}.
Detta medger att man kan förevisas hobbyn och även träna under kontrollerade
förhållanden.
För att detta ska fungera krävs att den med amatörradiocertifikat instruerar
om hur man ska bete sig i etern, hur handhavandet går till och kan övervaka
att detta följs.

Självklart används anropssignalen för innehavaren av amatörradiocertifikatet.
Det är bra att det tydligt framgår att det är en second operator som är aktiv.
Antingen ropar amatören upp och sedan berättar att han lämnar över till second
operator Simon.
Alternativt kan en second operator göra anropen själv och då ropa till exempel
"SM5XYZ second operator Anna".

Möjligheten att använda second operator skall användas med klokhet, och kan rätt
använd skapa en god förståelse för hobbyn och utgöra en morot för att få både
ungdomar och vuxna intresserade av amatörradio.

\subsection{CQ DX och split}
\index{CQ DX}
\index{split}
\index{DX expedition}
\index{rar DX}
\index{pile-up}

Det förekommer att man hör någon ropa \emph{''CQ DX''}, vilket betyder att
stationen söker långväga kontakter, i allmänhet utanför sin egen världsdel.

-- ''CQ DX, CQ DX, CQ DX, SM0XYZ calling CQ DX and standing by''

I detta fallet är det SM0XYZ som söker att nå någon utanför Europa.
Är du själv inte ett DX, det vill säga om du befinner dig i samma världsdel så
ska du undvika att svara.

Ibland genomförs så kallade \emph{DX expeditioner} då man beger sig till en
plats som sällan aktiveras.
Man brukar tala om \emph{rara DX} (eng. \emph{rare DX}), då ett ovanligt
landområde aktiveras, som många vill ha i sin logg.

En station som ropar CQ kan få svar från många stationer samtidigt.
Då uppstår ett sammelsurium av signaler som kallas för \emph{pile-up}.

När stationen betar av en pile-up kan stationen även fråga \emph{''QRZ?''},
alltså ''vem där''?

Ett rart DX kan drabbas av enorma pile-ups, och det kan bli svårt
motstationerna att höra DX-stationen bland alla andra som ropar samtidigt.
Det kan kan också vara svårt för DX-stationen att urskilja vilka motstationer
som svarar, om alla svarar på samma frekvens.
En strategi för att få detta att fungera effektivare är att köra split
\cite{LowBandDX}, det vill säga att DX-stationen sänder och lyssnar på olika
frekvenser (men fortfarande inom samma frekvensband).

Oftast väljer DX-stationen att lyssna på en frekvens som ligger några kilohertz
högre än den egna sändningsfrekvensen, och anger detta genom att sända
exempelvis \emph{''listening up''} eller \emph{''listening five up''}.

Genom att använda sig av split undviker DX-stationen att störas ut av sin egen
pile-up.
DX-stationen kan också välja sprida ut sin pile-up, genom att inte lyssna på
endast en frekvens, utan genom att svepa över ett lite större område.

\emph{''Listening five to ten up''} betyder då att DX-stationen lyssnar i ett
område mellan 5 och 10~kHz över den egna sändningsfrekvensen, och
motstationerna får försöka gissa var i detta frekvensområde som DX-stationen
lyssnar just för tillfället.

För trafik på morsetelegrafi använder man vanligtvis ett mindre avstånd mellan
sändar- och mottagarfrekvens än för trafik på SSB-telefoni, eftersom
telegrafisignalerna upptar mindre bandbredd.

Moderna transceivrar har nästan alltid möjlighet att ställa in split genom att
man använder ''VFO A/B'', ''RIT/XIT'' eller ''clarifier''.
Mer avancerade transceivrar kan ha möjlighet att separera de två frekvenserna i
hörlurarnas vänster- respektive högerkanal.

När en station ropar CQ och gör paus för anropande stationer, ange då din egen
signal kort och tydligt en gång i varje pass.
Istället för att ropa flera gånger varje pass, skrika eller på annat sätt
ta utrymme, ha tålamod och vänta ut bra tillfälle.
Ropa inte under den tid som DX-stationen sänder sitt CQ. Då hör han dig ju ändå inte.

Det kan vara nyttigt att lyssna in sig på operatörens stil.
Var medveten om att DX-stationen kan höra helt andra stationer starkare än vad du hör,
eftersom konditionerna kan vara helt annorlunda för DX-stationen.

Vid stora pile-ups kan operatören välja att bara lyssna efter vissa stationer,
och därför fråga efter ''only number five stations please'' eller efter ''only
European stations please''.
Detta syftar till att dela upp en stor pile-up för en chans att lättare
uppfatta vilka som anropar.

Vid uppdelning efter nummer kommer operatören avverka några stationer med ett
visst nummer i anropssignalen, för att sedan gå vidare till nästa och så
vidare, tills 0 till 9 är genomgångna.

Alternativet att gå efter regioner eller landsprefix kan vara att föredra om
operatören upplever att konditionerna dit snart försvinner och därför vill ge
dem en extra förtur innan de helt tappar chansen.

Lär dig ''DX:arens ordningsregler'' (\href{http://www.dx-code.org/}{www.dx-code.org}) :

\begin{itemize}
	\item Jag ska lyssna, och lyssna, och sedan lyssna lite till.
	\item Jag ska ropa endast om jag kan läsa DX-stationen ordentligt.
	\item Jag ska icke lita på cluster-information, utan vara helt klar över DX-stationens anropssignal innan jag ropar.
	\item Jag ska icke störa DX-stationen, eller någon som anropar denne, och jag ska aldrig stämma av på DX:ets egen frekvens, eller i det segment där denne lyssnar.
	\item Jag ska vänta tills DX:et avslutat föregående kontakt innan jag ropar själv.
	\item Jag ska alltid ange min fullständiga anropssignal.
	\item Jag ska ropa och lyssna med lämpliga intervaller.
	\item Jag ska icke ropa kontinuerligt.
	\item Jag ska icke ropa då DX:et svarar någon annan än mig.
	\item Jag ska icke ropa då DX:et frågar efter en anropssignal som icke liknar min egen.
	\item Jag ska icke ropa då DX:et söker efter ett annat geografiskt område än mitt eget.
	\item När DX:et svarar mig så ska jag icke upprepa min anropssignal, annat än om jag tror att denne ej uppfattat den korrekt.
	\item Jag ska vara tacksam om och när jag får kontakt.
	\item Jag ska respektera mina amatörkamrater och uppträda så jag förtjänar deras respekt.
	\end{itemize}

\textbf{Försök inte agera polis och rätta andra stationer, som du anser bryter mot reglerna!}

\section{Innehåll i förbindelse}
\textbf{HAREC
	b.\ref{HAREC.b.7.2.3}\label{myHAREC.b.7.2.3}
}

Tidigare har det i Sverige varit reglerat vad innehållet får vara i
förbindelser, eller snarare vad de inte får innehålla.
Den regleringen är numera borttagen.
Man ska vara medveten om att samma regler och förutsättningar inte gäller i
alla länder och för deras radioamatörer.
Därför uppmanas du att använda sunt förnuft, hålla god ton och respektera alla
amatörer.
Se även IARU etik och trafikmetoder.

\subsection{Tystnadsplikt}
\index{tystnadsplikt}
\index{LEK}

Innehållet i en radioförbindelse skyddas av
\emph{Lagen för Elektronisk Kommunikation (LEK)} \cite{SFS2003:389}.
I LEK regleras tystnadsplikt för radiobefordrade meddelanden i kapitel 6.

\begin{quote}
	Den som i annat fall än som avses i 20 \S~första stycket och 21 \S~i
	radiomottagare har avlyssnat eller på annat sätt med användande av sådan
	mottagare fått tillgång till ett radiobefordrat meddelande i ett
	elektroniskt kommunikationsnät som inte är avsett för honom eller henne
	själv eller för allmänheten får inte obehörigen föra det vidare.
	Lag (2012:285).\cite[kap 6, \S23]{SFS2003:389}
\end{quote}

Tystnadsplikten gäller alla radiomeddelanden som avlyssnats, oavsett ursprung.

Detta innebär att om du själv varit part i radiomeddelandet eller om 
radiomeddelandet var en nyhetsbulletin avsett för många så du föra det vidare.

En stor del av radioamatörhobbyn bygger dock på radiokommunikation med andra och
att andra kan höra dig när du sänder.
En radioamatör kan därför inte anses vara omedveten om att någon annan lyssnar
på det som sänds ut.
Därför är mycket accepterat inom amatörradio som annars skulle vara förbjudet.

Tips om rara DX, tips om någon som ropar CQ, QSL från lyssnaramatörer, att
berätta att du hörde någon ha förbindelse med någon annan anses därför normalt
inte vara ett brott mot tystnadsplikten.

Att koppla en radiomottagare till webben så att någon kan lyssna på radiotrafik
i realtid är tillåtet.

\emph{Observera även texten i andra punkten i kapitel 6 \S~20 gällande den som i
	samband med tillhandahållande av ett elektronisk kommunikationstjänst har fått
	del av eller tillgång till innehållet i ett elektroniskt meddelande inte
	obehörigen får föra vidare eller utnyttja det han fått del av eller tillgång
	till.}

Detta kan vara aktuellt då någon som tillhandahåller en elektronisk
kommunikationstjänst från punkt A till punkt B får tillgång till innehållet i
ett elektroniskt meddelande när det har lämnat punkt A och innan det når fram
till punkt B.

\subsection{Inspelning av radiomeddelande}
\index{inspelning}
\index{GDPR}
\index{Dataskyddsförordningen}

Radiosamtal som du själv deltar i får spelas in utan att andra deltagare i
samtalet informeras om att du spelar in samtalet.

Grundregeln är att inspelning av radiomeddelanden är tillåten såvida inte
inspelningen är förbjuden för att skydda personers personliga integritet.

Uppspelning av de inspelade meddelandena får inte bryta mot bestämmelserna om
tystnadsplikt.
Det vill säga att meddelandet inte obehörigen får föras vidare.

Alla radiomeddelanden får inte spelas in.
Lagstiftningen skiljer även på analoga- och digitala inspelningar.
Personuppgiftslagen SFS 1998:204 \cite{SFS1998:204} och dess ersättare från och
med 25 maj 2018 Dataskyddsförordningen \cite{GDPR} samt 4 kap \S~9a i
Brottsbalken \cite{SFS1962:700} är exempel på lagar som begränsar inspelning av
avlyssnade radiomeddelanden.

Av ovanstående följer att det inte är tillåtet att lagra inspelad radiotrafik
för senare lyssning via webbaserade medier då det kan anses kränka den
personliga integriteten.

\subsection{Kryptering av radiomeddelande}
\index{kryptering}

Inom Sveriges gränser är kryptering av radiomeddelanden på amatörradiofrekvenser
tillåten under villkor att en anropssignal regelbundet sänds ut, anropssignalen
ska då kunna avläsas med kända tekniker.
Trots detta rekommenderas inte användning av kryptering för amatörradiotrafik.

Tekniken för kryptering av radiomeddelanden har blivit mera lättillgänglig i
samband med införandet av digitala radiosystem typ DMR (Digital Mobile Radio) på
amatörbanden.
Ett flertal av dessa radiosystem är dock ihopkopplade via internationella
nätverk och därigenom hörbara i flera länder där kryptering inte är tillåten.

Användningen av krypteringsteknik på amatörradiofrekvenser riskerar därför att
medföra begränsningar i de rättigheter vi har enligt PTSFS 2015:4.

\section[Hederskod]{Radioamatörens hederskod}
\textbf{HAREC
  b.\ref{HAREC.b.7.1.1}\label{myHAREC.b.7.1.1}
}
\index{Radioamatörens hederskod}

Radioamatören är

\begin{tabular}{lp{9cm}}
  \textbf{HÄNSYNSFULL} &
     Han agerar aldrig medvetet på ett sätt som minskar nöjet för andra. \\
  & \\

  \textbf{LOJAL} &
  Han erbjuder lojalitet, uppmuntran och stöd åt andra amatörer, lokala klubbar,
  IARU organisationen i hans land genom vilken amatörradio i hans land
  representeras nationellt och internationellt.\\
  & \\

  \textbf{PROGRESSIV} &
  Han håller sin station på en hög teknisk nivå.
  Den är välbyggd och effektiv.
  Hans operationsteknik är oantastlig.\\
  & \\

  \textbf{VÄNLIG} &
  Han kommunicerar sakta och tålmodigt när så begärs;
  erbjuder kamratligt stöd och ger nybörjaren goda råd;
  vänlig assistans, samarbete och omtanke i andras intresse.
  Detta är kännetecknen för amatörandan.\\
  & \\

  \textbf{BALANSERAD} &
  Radio är en hobby och får aldrig orsaka konflikt i förpliktelser gentemot
  familj, arbete, skola eller samhälle.\\
  & \\

  \textbf{PATRIOTISK} &
  Hans station och hans kunnande står alltid till förfogande för att
  assistera land och samhälle.\\
\end{tabular}

-- anpassad från den ursprungliga Amateur's Code, skriven av Paul M. Segal, W9EEA, 1928.

\section[Ordningsregler]{Radioamatörens ordningsregler}
\textbf{HAREC
  b.\ref{HAREC.b.7.1.2}\label{myHAREC.b.7.1.2}
}

\subsection{Grundläggande principer}
\textbf{Grundläggande principer} som ska styra vårt \textbf{uppträdande} på
amatörbanden är:

\begin{itemize}
\item \textbf{Samhörighet, broderskap och kompiskänsla}: många, många av oss
  är aktiva i etern (vår spelplan).
  Vi är aldrig ensamma.
  Alla andra amatörer är våra kollegor, våra bröder och systrar, våra vänner.
  Agera därefter.
  Var alltid hänsynsfull.

\item \textbf{Tolerans}: inte alla amatörer delar nödvändigtvis samma
  uppfattning som du, och din uppfattning är kanske inte den bästa.
  Förstå att det finns andra med en annan uppfattning om ett visst tema.
  Var tolerant.
  Du har inte denna värld för dig själv.

\item \textbf{Anständighet}: aldrig får svordomar och oanständigheter yttras
  på banden.
  Ett sådant beteende säger ingenting om den person som de är avsedda för men
  mycket om den person som uttalar dem.
  Behåll ditt lugn i alla situationer.

\item \textbf{Förståelse}: Var snäll och förstå att alla inte är så smarta,
  så professionella eller så mycket expert som du.
  Om du vill göra något åt detta agera positivt (hur kan jag hjälpa till,
  hur kan jag förbättra, hur kan jag lära ut) i stället för negativt
  (med svordomar, förolämpningar etc.).
\end{itemize}

\subsection{Risken för konflikter}
\textbf{Endast en spelplan, etern}: alla radioamatörer vill spela sitt spel
eller utöva sin sport men det måste göras på en enda spelplan: våra amatörband.
Hundratusentals spelare på en enda spelplan leder ibland till konflikter.

Ett exempel: Plötsligt hör du någon ropa CQ på din frekvens (den frekvens du
har kört på en stund).
Hur är detta möjligt?
Du har varit igång här mer än en halvtimme på en helt ren frekvens!
Jo, visst är det möjligt; den där andra stationen tror kanske också att du stör
honom på HANS frekvens.
Kanske har skippet eller konditionerna ändrats?

\subsection{Hur undvika konflikter?}
\begin{itemize}
\item Genom att förklara för alla spelare vilka regler som gäller och genom
  att motivera dem att tillämpa dessa regler.
  De flesta konflikter orsakas av \textbf{okunskap}:
  många spelare känner inte till reglerna tillräckligt väl.

\item Dessutom hanteras många konflikter dåligt återigen på grund av
  \textbf{okunskap}.

\item Den IARU-etikhandbok som finns översatt på SSA:s webbplats avser att
  åtgärda denna brist på kunnande i huvudsak genom att lära ut hur man kan
  undvika konflikter av alla slag.
\end{itemize}

\subsection{Moraliska aspekter}
\begin{itemize}
\item I de flesta länder bryr sig myndigheterna inte om i detalj hur
  amatörerna uppför sig på banden, förutsatt att de håller sig till reglerna
  som myndigheten fastslagit.
\item Radioamatörerna anses vara \textbf{självstyrande}, detta betyder att
  självdisciplin måste utgöra basen i vårt agerande. Det betyder emellertid
  inte att radioamatörerna har en egen polisiär funktion!
\end{itemize}

\subsection{Förhållningsregler}

Vad menar vi med \textbf{förhållningsregler} (code of conduct)?
De är en uppsättning regler baserade på såväl \textbf{etiska} principer som
\textbf{trafikmässiga hänsyn}.

\begin{itemize}
\item \textbf{Etik}: Etik bestämmer vår attityd och vårt allmänna uppförande
  som radioamatörer.
  Etik har med moral att göra.
  Etik utgör principerna för moral.

  Exempel: etiken säger oss att aldrig medvetet störa andra stationers
  radiotrafik.
  Detta är en moralisk regel.
  Det är omoraliskt att inte följa denna regel, likvärdigt med att fuska i en
  tävling.
\item \textbf{Praktiska regler}: för att hantera alla olika aspekter av
  vårt uppförande behövs utöver etik också en uppsättning regler baserade på
  \textbf{trafikmässiga hänsyn} och på \textbf{praxis och sedvänja}.
  För att undvika konflikter behöver vi också praktiska regler som styr
  vårt beteende på amatörbanden eftersom vårt huvudintresse är att köra
  radio på de olika banden.
  Vi avser här mycket \textbf{praktiska regler} och \textbf{riktlinjer} för
  situationer som ej är etikrelaterade.
  De flesta trafikmetoder (hur genomföra ett QSO, var får man köra,
  vad betyder QRZ, hur använda Q-koderna) hör hit.
  Respekt för dessa trafikmetoder säkerställer optimalt resultat och
  effektivitet i våra QSO och kommer att vara nyckeln till att undvika
  konflikter.
  Dessa trafikmetoder har tillkommit som ett resultat av daglig radiotrafik
  under många år och som ett resultat av den pågående tekniska utvecklingen.
\end{itemize}
