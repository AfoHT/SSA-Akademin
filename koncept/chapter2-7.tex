\section{Elektronrör}
\label{elektronrör}
\index{elektronrör}

\subsection{Allmänt}

Ett elektronrör består av två eller flera elektroder i en lufttom behållare,
vanligen av glas eller ett keramiskt material.

\subsection{Vakuumdioden (tvåelektrodröret)}
\harec{a}{2.8.1}{2.8.1}
\label{vakuumdioden}
\index{vakuumdioden}
\index{elektronrör!diod}
\index{anod}
\index{katod}
\index{diod!anod}
\index{diod!katod}
\index{diod!vakuumdiod}
\index{diod!elektronrör}

\smallfig{images/cropped_pdfs/bild_2_2-24.pdf}{Schemasymboler för dioder}{fig:BildII2-24}

Bild \ref{fig:BildII2-24}

Dioden innehåller två elektroder, anod (a) och katod (k), samt i förekommande
fall en glödtråd (f) (eng. \emph{filament}).

\emph{Anoden} ska dra elektronerna från katoden.
\emph{Katoden} ska avge elektronerna och måste därför hettas upp.

Upphettningen av katoden kan göras direkt, det vill säga att katoden i sig
själv utgör glödtråd, vanligen med en 4- till 6-volts strömkälla.
Alternativt kan katoden uphettas indirekt med en separat glödtråd som omsluter
och hettar upp ett speciellt katodmaterial.
I det senare fallet är en 1,5- till 12,6-volts glödströmkälla vanlig.

\subsubsection{Edisoneffekten}
\index{Edisoneffekten}
\index{elektronrör!Edisoneffekten}

\smallfig{images/cropped_pdfs/bild_2_2-25.pdf}{Edisoneffekten}{fig:BildII2-25}

Bild \ref{fig:BildII2-25} illustrerar \emph{Edisoneffekten}.
När katoden upphettas lossnar fria elektroner från den och bildar ett moln.
Med en spänning mellan anod och katod, där anoden är positiv, kommer
elektronerna att dras mot anoden.
En anodström börjar att flyta.

\subsubsection{\(I_a/U_a\)-karaktäristikan för en vakuumdiod}

\mediumfig{images/cropped_pdfs/bild_2_2-26.pdf}{Diodens karaktäristik}{fig:BildII2-26}

Bild \ref{fig:BildII2-26} illustrerar vakuumdiodens karaktäristik.
När anoden ges positiv potential (anodspänning) flyter en elektronström från
katod till anod (anodström).
Om anodspänningen \(U_a\) ökar så ökar anodströmmen \(I_a\).
Varje par av talvärden representerar en punkt i ett diagram, som det på bilden.
När anodspänningen ökat till ett visst värde, ökar inte anodströmmen ytterligare.
I ett mellanområde, det linjära området, är kurvan i det närmaste rak.

\subsubsection{Likriktarverkan}
\index{elektronrör!likriktarverkan}

När anoden i en vakuumdiod ges positiv potential i förhållande till katoden
flyter en så kallad anodström, förutsatt att katoden upphettas så att den avger
fria elektroner.

När anoden ges en negativ potential i förhållande till katoden flyter däremot
ingen anodström.

Vakuumdioden kan därför användas för likriktning av växelströmmar.
Den har en likriktande funktion.

\subsubsection{Halvvågslikriktning}

\mediumfig{images/cropped_pdfs/bild_2_2-27.pdf}{Halvvågslikriktning}{fig:BildII2-27}

Bild \ref{fig:BildII2-27} illustrerar halvvågslikriktning.
När anoden ges en omväxlande positiv och negativ potential, en växelspänning,
flyter anodström under varje positiv halvperiod av växelspänningen.
En likströmspuls uppstår under varannan halvperiod.

\subsubsection{Helvågslikriktning}

\mediumfig{images/cropped_pdfs/bild_2_2-28.pdf}{Helvågslikriktning}{fig:BildII2-28}

Bild \ref{fig:BildII2-28} illustrerar helvågslikriktning.
Med ett el\-ek\-tron\-rör med dubbla anoder och en transformator med mittuttag på
sekundärlindningen, kan växelspänningens båda halvperioder utnyttjas, så att
anodström flyter i samma riktning under alla halvperioder.

\smallfig[0.45]{images/cropped_pdfs/bild_2_2-29.pdf}{Likriktande funktion}{fig:BildII2-29}

Bild \ref{fig:BildII2-29} illustrerar hur växelspännng via två
två halvvågslikriktningar formar en helvågslikriktning.

\subsection{Vakuumtrioden (treelektrodröret)}
\index{vakuumtrioden}
\index{trioden}
\index{elektronrör!triod}

\smallfig{images/cropped_pdfs/bild_2_2-30.pdf}{Symboler för triod och pentod}{fig:BildII2-30}

Bild \ref{fig:BildII2-30} illustrerar symboler för triod och pentod.

Trioden innehåller de tre elektroder anod (a), styrgaller (\(g_1\)) och katod
(k) samt en glödtråd (f = filament).

\subsubsection{Triodens funktion}

Bild \ref{fig:BildII2-31} illustrerar en triod och dess elektronström.
Styrgallret kan ges positiv, neutral eller negativ potential (förspänning) i
förhållande till katoden.
Valet av förspänning avgör triodens arbetssätt.
När styrgallret ges samma potential som katoden fungerar trioden som en diod.
Med styrgallret positivt ökar anodströmmen.
Med gallret negativt minskar den.

Trioden har en \emph{förstärkande} funktion eftersom anodströmmen kan styras med
styrgallret. En liten ändring av gallerspänningen medför stor ändring av anodströmmen.
Vid positiv förspänning flyter en gallerström, som inte får bli för hög.
Vanligen väljs en negativ förspänning.

\mediumfig{images/cropped_pdfs/bild_2_2-31.pdf}{Elektronstömmen i en triod}{fig:BildII2-31}

\subsubsection{Triodens strömkretsar och strömkällor}

\begin{tabular}{lll}
Glödströmskrets      & Anodkrets        &  Gallerkrets \\
Glödbatteri          & Anodbatteri      &  Gallerbatteri \\
Glödspänning \(U_f\) & Anodsp. \(U_a\)  &  Gallersp. \(U_{g1}\) \\
Glödström \(I_f\)    & Anodstr. \(I_a\) &  Gallerstr. \(I_{g1}\) \\
\end{tabular}

Vanligen används nätdrivna strömkällor i stället för batterier.

Valet av gallerförspänning är avgörande för triodens arbetssätt.

\newpage

\subsection{Pentoden (femelektrodröret)}
\index{pentod}
\index{elektronrör!pentod}

Pentoden innehåller fem elektroder.

\begin{tabular}{ll}
  a       & anod \\
  \(g_3\) & bromsgaller \\
  \(g_2\) & skärmgaller \\
  \(g_1\) & styrgaller \\
  k      & katod med glödtråd (f = filament) \\
\end{tabular}

Bromsgallret förbinds med katoden. Skärmgallret ges en potential som är något
lägre än anodspänningen.
Broms- och skärmgallren förhindrar elektronerna att studsa tillbaka till
styrgallret efter anslaget mot anoden.


\subsection{Tetroden (fyraelektrodröret)}
\index{tetrod}
\index{elektronrör!tetrod}

Denna rörtyp innehåller fyra elektroder. Uppbyggnaden är densamma som pentodens,
men bromsgallret saknas.

\subsection{Karaktäristika för elektronrör}

\mediumfig{images/cropped_pdfs/bild_2_2-32.pdf}{Karaktäristika för elektronrör}{fig:BildII2-32}

Bild \ref{fig:BildII2-32} illustrerar ett \(I_a/U_{gt}\)-diagram för en triod
eller pentod, vid konstant \(U_a\).

\(I_a/U_a\)-diagram för en triod, vid konstant \(U_{g1}\)

\(I_a/U_a\)-diagram för en pentod, vid konstant \(U_{g1}\)

Tre kurvor visas i \(I_a/U_a\)-diagrammen, med olika värden på
\(U_{g1}\). (\(U_{g1}\) är en så kallad parameter).

\smallfig{images/cropped_pdfs/bild_2_2-33.pdf}{Branthet}{fig:BildII2-33}

\subsection{Branthet $S$ och inre resistans $R_i$}

Bild \ref{fig:BildII2-33} visar brantheten.
Om man vid konstant anodspänning ändrar gallerförspänningen med värdet
\(\Delta U_{g1}\), ändrar sig anodströmmen med värdet \(\Delta I_a\).

Branthet \(S = \dfrac{\Delta I_a}{\Delta U_{g1}}\)

\(S\ [mA/V]\) \(\Delta I_a\ [mA]\) \(U_{g1}\ [V]\)

Bild \ref{fig:BildII2-34} visar den inre resistansen.
Om man vid konstant gallerförspänning ändrar anodspänningen med
\(\Delta U_a\), ändras anodströmmen med värdet \(\Delta I_a\).

\newpage

Inre resistans \(R_i = \dfrac{\Delta U_a}{\Delta I_a}\)

\(R_i\ [k \omega]\)  \(\Delta U_a\ [V]\)  \(\Delta I_a\ [mA]\)

\mediumfig{images/cropped_pdfs/bild_2_2-34.pdf}{Inre resistans}{fig:BildII2-34}

Om man vill ändra anodströmmen med \(\Delta I_a\) ges två möjligheter.
Antingen ändrar man gallerförspänningen med värdet \(\Delta U_{g1}\), eller så
ändrar man anodspänningen med värdet \(\Delta U_a\).
Genom att ändra gallerförspänningen med värdet \(U_{g1}\) kan man åstadkomma
samma anodströmsändring \(\Delta I_a\) som med en ändring av anodspänningen
med värdet \(\Delta U_a\).

\subsection{Barkhausens elektronrörsformler}
\index{Barkhausen elektronrörsformler}
\index{elektronrör!Barkhausen formler}

Förstärkningsfaktorn \(\mu \) illustreras av följande samband som gäller mellan
de så kallade rörkonstanterna

\(\mu = S \cdot R_i\)

\emph{Exempel:}
Beräkna \(\mu\)  om \(S = 2\ mA/V\) \(R = 10\ k\Omega\) \(\mu = ?\)

\emph{Svar:} \(\mu = 20\) (\(\mu\)  är dimensionslös)

\subsection{Transistor jämförd med elektronrör}

Transistorer har fördelar som lågt pris, små dimensioner, lång livslängd, enkel
strömförsörjning (glödström behövs inte) och låg driftspänning (6~V, 12~V
\ldots ). Vanliga nackdelar är känslighet för överbelastning och höga
temperaturer.

Elektronrör har fördelen av tålighet mot överbelastning, men bland nackdelarna
kan nämnas att de kräver hög anodspänning, att de behöver glödström och att de
är utrymmeskrävande.

Transistorer ersätter numera nästan helt elektronrören, men man bör ändå känna
till elektronrörens egenskaper och arbetssätt.

Ett användningsområde där elektronrör ännu är vanliga är i större
sändarslutsteg.
