\chapter{Trafikreglemente}

\section{Fonetiska alfabet}
\textbf{
HAREC b.\ref{HAREC.b.1.1}\label{myHAREC.b.1.1},
 b.\ref{HAREC.b.1.2}\label{myHAREC.b.1.2},
 b.\ref{HAREC.b.1.3}\label{myHAREC.b.1.3},
 b.\ref{HAREC.b.1.4}\label{myHAREC.b.1.4},
 b.\ref{HAREC.b.1.5}\label{myHAREC.b.1.5},
 b.\ref{HAREC.b.1.6}\label{myHAREC.b.1.6},
 b.\ref{HAREC.b.1.7}\label{myHAREC.b.1.7},
 b.\ref{HAREC.b.1.8}\label{myHAREC.b.1.8},
 b.\ref{HAREC.b.1.9}\label{myHAREC.b.1.9},
 b.\ref{HAREC.b.1.10}\label{myHAREC.b.1.10},
 b.\ref{HAREC.b.1.11}\label{myHAREC.b.1.11},
 b.\ref{HAREC.b.1.12}\label{myHAREC.b.1.12},
 b.\ref{HAREC.b.1.13}\label{myHAREC.b.1.13},
 b.\ref{HAREC.b.1.14}\label{myHAREC.b.1.14},
 b.\ref{HAREC.b.1.15}\label{myHAREC.b.1.15},
 b.\ref{HAREC.b.1.16}\label{myHAREC.b.1.16},
 b.\ref{HAREC.b.1.17}\label{myHAREC.b.1.17},
 b.\ref{HAREC.b.1.18}\label{myHAREC.b.1.18},
 b.\ref{HAREC.b.1.19}\label{myHAREC.b.1.19},
 b.\ref{HAREC.b.1.20}\label{myHAREC.b.1.20},
 b.\ref{HAREC.b.1.21}\label{myHAREC.b.1.21},
 b.\ref{HAREC.b.1.22}\label{myHAREC.b.1.22},
 b.\ref{HAREC.b.1.23}\label{myHAREC.b.1.23},
 b.\ref{HAREC.b.1.24}\label{myHAREC.b.1.24},
 b.\ref{HAREC.b.1.25}\label{myHAREC.b.1.25},
 b.\ref{HAREC.b.1.26}\label{myHAREC.b.1.26}
}

Ibland behöver man göra förtydliganden genom att bokstavera.
Det internationella finns i ITU radioreglemente (RR) \cite[Appendix 14]{ITU-RR-appendices}
och kravställs i CEPT för HAREC \cite[Annex 6]{TR6102}.

Svenska radioamatörer ska kunna två fonetiska alfabet, dels det
internationella och dels det svenska.

Det kan vara värt att veta att det förekommer slang med andra ord vid
bokstavering.
Det finns därtill bokstavering på flera språk.
Medan dessa inte ska användas vid internationell trafik, så kan det vara bra
att känna till.
%\url{https://en.wikipedia.org/wiki/NATO_phonetic_alphabet}

\begin{table}[htbp]
  \label{tab:bokstavering-internationell}
  \label{tab:bokstavering-svenska}
  \caption{Det internationella och svenska fonetiska alfabetet}
  \begin{tabular}{llll}
    Bokstav & Kodord & Uttal & Svenskt kodord \\ \hline
    A & Alfa  & \underline{ALL} FA & Adam \\
    B & Bravo & \underline{BRA} VO & Beril \\
    C & Charlie & \underline{TJAR} LI & Ceasar \\
    D & Delta & \underline{DELL} TA & David \\
    E & Echo & \underline{ECK} Å & Erik \\
    F & Foxtrot & \underline{FÅCKS} TRÅTT & Filip\\
    G & Golf & \underline{GÅLF} & Gustav \\
    H & Hotel & HÅ \underline{TELL} & Helge \\
    I & India & \underline{IN} DIA & Ivar \\
    J & Juliett & \underline{DJO} LI \underline{ETT} & Johan \\
    K & Kilo & \underline{KI} LÅ & Kalle \\
    L & Lima & \underline{LI} MA & Ludvig \\
    M & Mike & MAJK & Martin \\
    N & November & NO \underline{VEM} BÖ(RR) & Niklas \\
    O & Oscar & \underline{ÅSSK} A(RR) & Olof \\
    P & Papa & PA \underline{PA} & Petter \\
    Q & Quebec & KE \underline{BECK} & Qvintus \\
    R & Romeo & \underline{RÅ} MIO & Rudolf \\
    S & Sierra & SI \underline{ERR} RA & Sigurd \\
    T & Tango & \underline{TÄNG} GÅ & Tore \\
    U & Uniform & \underline{JO} NI FORM & Urban \\
    V & Victor & \underline{VICK} TÖ(RR) & Viktor \\
    W & Whiskey & \underline{OISS} KI & Wilhelm \\
    X & X-ray & \underline{ECKS} REJ & Xerxes \\
    Y & Yankee & \underline{JÄNG} KI & Yngve \\
    Z & Zulu & \underline{ZO} LO & Zäta \\
    Å & Alfa Alfa & \underline{ALL} FA \underline{ALL} FA & Åke \\
    Ä & Alfa Echo & \underline{ALL} FA \underline{ECK} Å & Ärlig \\
    Ö & Oscar Echo & \underline{ÅSSK} A \underline{ECK} Å & Östen \\
    & & \\
    0 & Zero & \underline{ZE} RO & Nolla \\
    1 & One & O \underline{ANN} & Ett (inte ETTA)\\
    2 & Two & TO & Tvåa \\
    3 & Three & TRI & Trea \\
    4 & Four & FÅR & Fyra \\
    5 & Five & FAJV & Femma \\
    6 & Six & SICKS & Sexa \\
    7 & Seven & \underline{SE} VEN & Sju (inte SJUA)\\
    8 & Eight & EJT & Åtta \\
    9 & Nine & \underline{NAJ} NÖ(RR) & Nia \\
    & & \\
    , & Decimal & \underline{DE} SI MAL & Komma \\
    . & Stop & STOPP & Punkt \\
    \multicolumn{3}{l}{Ungefärligt uttal. Betona det understrukna.}
  \end{tabular}
\end{table}
