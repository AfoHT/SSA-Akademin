\section{Mottagare}
% sid 198
\label{mottagare}
\index{mottagare}

\emph{Energin i de elektromagnetiska magnetfält, som omger oss,
  alstrar högfrekventa strömmar i alla metallföremål. För att
  effektivt fånga upp dessa fält används antenner.  Fastän energin i
  fälten kan få en lampa att lysa om sändarantennen är tillräckligt
  nära, så går det ändå inte att uppfatta den information som fälten
  också kan innehålla.  För det behövs en radiomottagare för att dels
  förstärka de oftast mycket svaga signalerna och dels uttyda
  informationen i dem.}

\emph{Lyssna på amplitudmodulerade rundradiosändningar på mellanvåg
  kan man enklast göra med hjälp av en detektormottagare. Speciellt
  under dygnets mörka timmar vintertid kan man höra utländska sändare
  med denna enkla mottagare, låt vara att det hörs mycket svagt. I
  detektormottagaren omvandlas fältens energi till elektricitet och
  sedan till ljud. Så länge som ingen förstärkare används, förbrukas
  ingen annan energi än den som fångas ur fälten -- radiovågorna.}
