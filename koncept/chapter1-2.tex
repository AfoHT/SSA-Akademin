\section{Elektriska kraftkällor}
\textbf{HAREC a.\ref{HAREC.a.1.2}\label{myHAREC.a.1.2}}

\subsection{Elektromotorisk kraft -- EMK}
\textbf{HAREC a.\ref{HAREC.a.1.2.1}\label{myHAREC.a.1.2.1}}
\index{elektromotorisk kraft (EMK)}
\index{EMK}
\index{volt (V)}

Det som driver ström genom en elektrisk strömkrets är kretsens elektromotoriska
kraft (EMK).

Måttenheten för EMK är \(volt\ [V]\).

EMK är summan av de potentialökningar som uppstår i kretsen.

De vanligaste slagen av EMK är
\begin{itemize}
\item elektromagnetisk EMK som uppkommer i strömledare i magnetfält som
varierar (t.ex. lindningarna i en roterande generator)
\item elektrokemisk EMK som uppkommer i beröringsytan mellan en metallisk
ledare och en elektrolyt (t.ex. battericell)
\item elektrostatisk EMK, t.ex. i kondensatorer
\item kontakt-EMK i beröringsytan mellan metaller med olika termoelektrisk
potential eller mellan metall och luftens syre (t.ex. korrosion mellan metaller)
\item termo-EMK som uppkommer i en strömkrets där två sammanlödda metaller med
olika temperatur ingår (t.ex. termokors för strömmätning).
\end{itemize}

\subsection{Polspänning}

Den spänning, som kan mätas mellan kretsens anslutningspoler då kretsen är öppen.

\subsection{Inre resistans}

I likhet med att komponenterna i en strömkrets har en viss resistans, har också en
strömkälla en inre resistans.
Den inre resistansen i en strömkälla ingår i kretsens totala resistans.

\subsection{Kortslutningsström}

Om man på kortaste väg förbinder strömkällans anslutningspoler blir kretsen
totala resistans lika med källans inre resistans.

Den kortslutningsström som då uppstår begränsas enbart av strömkällans
polspänning och inre resistans.

Eftersom den inre resistansen oftast är mycket liten blir kortslutningsströmmen
motsvarande hög.

\subsection{Serie- och parallellkopplade kraftkällor}
\textbf{HAREC a.\ref{HAREC.a.1.2.2}\label{myHAREC.a.1.2.2}}

\subsubsection{Seriekopplade kraftkällor}

För att uppnå en högre total spänning (EMK) kan flera kraftkällor
(delspänningar) kopplas i en slinga efter varandra. Detta kallas seriekoppling.

Seriekopplade delspänningar verkar med eller mot varandra, beroende på
deras inbördes polariteter.

Den totala spänningen över kopplingen är summan av de ingående
delspänningarna, med hänsyn taget till deras polariteter.

\subsubsection{Parallellkopplade kraftkällor}

För att erhålla högre ström, kan flera svagare kraftkällor parallellkopplas.
Vid parallellkoppling erhålls däremot inte högre spänning.

Vid parallellkoppling av kraftkällor \textbf{måste} deras polaritet vara lika.

För minsta utjämningsström mellan parallellkopplade kraftkällor bör även deras
polspänning och inre resistans vara så lika som möjligt.

\textbf{NOT: Parallellkoppling av kraftkällor är ofta olämpligt eftersom det
i praktiken är svårt att få en balans, varvid enbart den ena källan levererar.
Det finns kraftaggregat utformade för att parallellkopplas.}
