\section{Internationell nödtrafik}
\textbf{
	HAREC b.\ref{HAREC.b.4.1}\label{myHAREC.b.4.1},
	b.\ref{HAREC.b.4.2}\label{myHAREC.b.4.2},
	b.\ref{HAREC.b.4.3}\label{myHAREC.b.4.3},
	b.\ref{HAREC.b.4.4}\label{myHAREC.b.4.4}
}

\subsection{Nödsignaler}
\index{nödsignaler}
\index{SOS}
\index{MAYDAY}

I CEPT-rekommendation T/R 61-02 \cite{TR6102} ställs krav på att radioamatörer
ska ha känna till de internationella nödsignalerna SOS och MAYDAY.

Nödsignalen på morsetelegrafi består av teckendelarna . . . --- --- --- . . .
sända i en följd, där längden på de långa teckendelarna betonas så att de klart
skiljer sig från de korta.

Signalen skrivs som bokstäverna $\overline{SOS}$ med ett streck ovanför.

Nödsignalen på radiotelefoni består av ordet MAYDAY uttalat som det franska
uttrycket ''m'aider''. I Sverige kan man även ropa ''NÖDANROP''.

\subsection{Internationella nödfrekvenser}
\index{nödfrekvenser}

Nödsignaler på telefoni sänds i första hand på frekvenserna:
\begin{itemize}
	\item 121,5~MHz AM (Flygradio)
	\item 156,8~MHz FM (Marin VHF kanal 16).
\end{itemize}

En äldre nödfrekvens är 2182 kHz, men den är ej längre en primär frekvens för
nöd- och säkerhetstrafik.
Det finns inte längre krav på att fartyg ska ha radiopassning på frekvensen
vilket framgår av \emph{Transportstyrelsens föreskrifter och allmänna råd
	om radioutrustning på fartyg} TSFS 2009:95 \cite[\S22]{TSFS2009:95}.
(Läs mer om nödfrekvens i avsnitt \ref{nödtrafik})

Kustradion i Sverige upphörde med sin radiopassning, vakthållning, av frekvensen
i början av 2005 och US-coast guard slutade radiopassningen i augusti 2013.

\textbf{I händelse av nöd, med omedelbart behov av assistans, är det därför
	olämpligt att i första hand söka hjälp på frekvensen 2182 kHz.}

Frekvensen 2182 kHz är fortfarande reserverad i ITU Radioreglemente (RR)
\cite{ITU-RR} för ''Distress and safety communications'' och radiopliktiga
fartyg som trafikerar vatten i och utanför kustnära områden ska ha
radioutrustning för frekvensen.

\subsection{Nödtrafik}
\index{nödtrafik}
\index{nödfrekvens}
\index{GMDSS}
\label{nödtrafik}

I CEPT-rekommendation T/R 61-02 \cite{TR6102} ställs krav på att radioamatörer
ska ha känna till bestämmelser om nödtrafik och användningen av
amatörradiostationer vid naturkatastrofer.

ITU Radioreglemente (RR) \cite{ITU-RR} har sedan WRC-07 inte längre information
om ''Distress and safety communications'' för annat än
GMDSS (\emph{Global Maritime Distress and Safety System})

Med ''nödfrekvens'' avses en frekvens som radiopassas av exempelvis flyg- eller
sjöräddningscentral 24/7 (dygnet runt, året runt).
Även om termen ''nödfrekvens'' ibland förekommer i svenska bandplaner för
amatörradio, så finns inga egentliga sådana frekvenser inom amatörradiobanden.

År 1998 hölls en internationell konferens i Tammerfors i Finland
(\emph{ICET-98}).
Konferensen ledde fram till Tampere-konventionen ''The Tampere Convention on
the Provision of Telecommunication Resources for Disaster Mitigation and Relief
Operations'' \cite{TampereConvention}.
Konventionen trädde i kraft 8~januari 2005.

I enlighet med konventionen har IARU infört rekommendationer om regionala och
globala frekvenser för \emph{Emergency Centre of Activity}.
Det vill säga centerfrekvenser för radiokommunikation som kan användas i
händelse av naturkatastrofer.

\textbf{IARUs rekommendationer och förändringen av ITU RR innebär alltså att
  det inte finns någon speciell nödsignal för amatörradiobanden och inga
  nödfrekvenser inom amatörradiobanden.}

För vidare läsning rekommenderas
\emph{IARU Emergency Telecommunications Guide} \cite{IARU-ETG}.

\subsection{Om du hör en nödsignal på radio}

Avbryt omedelbart din egen sändning när du hör en nödsignal. Lyssna på
nödmeddelandet och \textbf{skriv ner} vad som sägs. Notera position, frekvens, tidpunkt
etc. Anmäl vad du hört på följande sätt.
%tagit bort telefonnummer till UD och räddningscentraler. NTJ

\subsubsection{Nödsignal från radioamatör i utlandet}

Om du uppfattar ett nödanrop från Sveriges närområde, så som Nordsjön och
Östersjön eller närliggande länder, så använd sunt förnuft och ring 112 och
berätta att du uppfattat en nödsignal från utlandet via radio.

För de fall att det är längre bort i Europa, Ryssland och Baltikum kan det
kanske vara läge att vara lite mer restriktiv.

Oavsett bör man först avvakta en stund för att monitorera om anropet verkar
besvaras innan man själv besvarar det, då man i och med det påtar sig ansvaret
för att räddningsinsatsen påbörjas. Eftersom 112 kan välja att prioritera ned
det sitter man då med ansvaret utan att kunna göra mer.

\subsection{Nödsignal från svenskt landområde}

Ring 112 för att kalla på Ambulans, Polis, Räddningstjänst, Sjöräddning,
Flygräddning etc.
Ditt telefonnummer visas automatiskt i larmoperatörens display.
För att undvika missförstånd och feldirigering av räddningsinsatserna
\textbf{måste} du meddela operatören att nödanropet kommit via radio.
Själva olycksplatsen kan ligga i ett helt annat riktnummerområde än det som ditt
telefonsamtal kommer ifrån.

\subsection{Nödsignal från fartyg eller luftfarkost}

Om nödsignalen inte besvaras av någon kust- eller markstation, ring 112
och begär Sjöräddning respektive Flygräddning och meddela dina iakttagelser.

\emph{Vidarebefordra nödmeddelandet utan att ändra på det!}

\subsection{Du själv sänder nödsignal över radio}

I första hand bör du välja andra signalvägar än amatörradio, så som fast och
mobil telefoni, båt eller flygradio eller därför avsedda nödsändare om möjligt.

Välj gärna en frekvens med mycket trafik utifall du inte använder en
nödfrekvens, så det finns en chans att den är bevakad så att någon kan höra
ditt nödanrop. Att använda en repeater för att höras bättre är en bra strategi.

Uppträd lugnt och sansat, när du kallar på hjälp över radion.
Tänk först och sänd sedan.
Som ovan sagts måste den som svarar dig och sedan ringer 112 meddela
larmoperatören att ditt nödanrop kommit via radio.

\subsection{Åtgärder}

Nyckelordet för dina åtgärder är LARMA:

\begin{tabular}{lp{9cm}}
	\textbf{L}äge &
        Ange olycksplatsens läge.
        Du kan ange gatu- eller vägnamn eller riktmärken som
        till exempel vägkorset, gränsen, bron, järnvägen etc.
	\\
	\textbf{A}nalysera
	&
        Gör en överblick över olycksplatsen och tala om vad som hänt.
        Några skadade? Några innestängda?
        Brinner det? Släpps farliga ämnen ut?
	\\
	\textbf{R}opa &
        Ropa på hjälp.
        Använd gärna en repeater på 2-metersbandet så att du når många,
        men även andra frekvenser kan användas.
        Anropa med NÖDANROP FRÅN SMXxxx.
        Fråga efter någon med telefon.
        Ge inte upp om du inte får svar genast.
	\\
	\textbf{M}eddela &
        Meddela när du fått kontakt med någon med telefon, sänd NÖDTRAFIK PÅGÅR
        för att freda frekvensen och NÖDMEDDELANDET med de viktigaste
        uppgifterna.
        Begär att uppgifterna repeteras och ta löfte på att de sänds vidare.
        Begär att få veta när så har skett.
        Påminn annars!
	\\
	\textbf{A}vvakta &
        Vänta på platsen tills hjälp har anlänt.
        Passa radion så att du kan svara på frågor.
        Behövs inte längre din hjälp, avsluta då med\\
	& NÖDTRAFIK UPPHÖR FRAN SMXxxx \dots KLART SLUT.
\end{tabular}
