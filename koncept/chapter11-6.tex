\section{Bandplaner}

\subsection{Introduktion}

Det allra vanligaste är att en radiostation eller ett nät av stationer tilldelas
en eller ett fåtal frekvenser samt väl preciserade villkor i övrigt. Amatörradio
är däremot en radiotjänst, som tilldelas inte bara enstaka frekvenser utan hela
frekvensband samt inom dessa band förhållandevis stor frihet till personligt val
av frekvens, sändningsslag etc.  Därvid kan den enskilde radioamatören inte
ställa anspråk på ostörda frekvenser. l stället är det upp till radioamatörerna,
att själva samråda och rekommendera varandra om hur de tilldelade frekvensbanden
bör fördelas på olika slags användning. Denna fördelning av trafiken kallas
bandplan.

Frekvensbanden och effekt-gränser sätts av varje lands ansvarig myndighet,
kallad administration, och i Sverige är det Post- och Telestyrelsen (PTS) som
meddelar i \emph{Post- och telestyrelsens föreskrifter om undantag från
tillståndsplikt för användning av vissa radiosändare} PTSFS 2015:4
\cite{PTSFS2015:4}. För att koordinera frekvensanvändningen samarbetar olika
länders ansvariga myndigheter i Internationella Teleunionen (ITU) för att
ha en gemensam utgångspunkt för sina beslut, som publiceras som ITU
Radioreglemente \cite{ITU-RR}.

ITUs Radioreglemente är dock inte ett tvingande dokument för
administrationerna, och därmed gäller det inte per automatik över
nationell lag, utan man behöver alltid hålla sig bekant med vad gällande lag
och föreskrifter säger.

För att radioamatörer sedan ska använda banden på ett liknande sätt, så har
sedan IARU skrivit ihop en bandplan, som enbart kan tolkas som en
rekommendation inom den ram som nationella lagar och föreskrifter sätter på
radiosändning.

Det är viktigt att förstå dessa samband rätt, för det förekommer tyvärr att
betydelsen av dokument övertolkas, eftersom det kan resultera i att man sänder
på frekvenser som inte är tilldelat amatörtjänsten, eller sända med högre
effekt än tillåtet.

\subsection{IARU:s bandplaner, syfte och ändamål}
\textbf{
HAREC b.\ref{HAREC.b.6.1}\label{myHAREC.b.6.1},
 b.\ref{HAREC.b.6.2}\label{myHAREC.b.6.2}
}
\index{Internationella Amatörradiounionen (IARU)}
\index{IARU}

Internationella Amatörradiounionen (IARU) är det enda organ på internationell
nivå, där samråd om amatörradions intressen sker regelbundet, dels i arbetsmed
olika inriktning och dels i generella konferenser.

IARU har som syfte att

\begin{itemize}
\item verka för att av ITU tilldelade frekvensband för amatörradio bevaras,
\item förbättra amatör- och amatörsatellittjänsternas status inom tilldelade
  frekvensband,
\item verka för tilldelning av ytterligare frekvensband för amatörradio,
\item frekvensplanera amatörradiotrafiken inom tilldelade amatörradioband genom
  samråd och rekommendationer.
\end{itemize}

Syftet med en bandplan är att ge utrymme för alla aspekter inom amatörradio
självträning, kommunikation och tekniska undersökningar.

Radioamatörernas bandplaner siktar på att ge möjlighet till så många olika
amatöraktiviteter som möjligt, såväl sändningsslag som tekniker, både nu och i
framtiden. För att utnyttja banden på bästa sätt är det normalt att minsta
möjliga bandbredd samt optimal sändarutrustning och teknik används.

För att alla ska kunna utöva amatörradio med ett minimum av störningar,
förutsätts att man använder utrustningar som är ``state of the art''.  God
insikt i frekvensplanering, tillräckliga resurser, gott anseende samt
internationellt samarbete behövs för att främja amatörradion. De flesta
nationella amatörradioorganisationer har sedan många år ett världsomfattande
samarbete genom sitt organ The International Amateur Radio Union (IARU) som är
organiserat som tre regioner. Dessa regioner sammanfaller geografiskt med ITU :s
regioner. Region 1 omfattar Afrika, Europa och västra Asien.

\section{Svenska bandplaner, sändningsslag}

Tilldelningen av frekvensband för amatörradioanvändning sker enligt
överenskommelser mellan telemyndigheterna i de länder som är anslutna till
ITU. Tilldelningen är därvid i stort sett lika i de flesta länder. Av olika skäl
förekommer dock skillnader såväl mellan ITU-regioner som länder.

I Sverige regleras amatörradioanvändningen främst genom Radiolagen och Post- och
telestyrelsens (PTS) föreskrifter \cite{PTSFS2015:4}. I anslutning till
frekvenstilldelningen anges tillåtna sändningsslag och amatörradiostatus
i respektive band. Inom denna ram
är det upp till radioamatörerna själva att utnyttja sina möjligheter bästa sätt.
Bandplaner fungerar som radioamatörernas rekommendationer till varandra. Endast
i minsta utsträckning medverkar PTS till reglering inom dessa planer.

\hilight{TODO: Se Appendix F.}

\hilight{TODO: Se Appendix G.}

Föreningen Sveriges Sändareamatörer SSA --- företräder de svenska
radioamatörerna i IARU Region 1.
