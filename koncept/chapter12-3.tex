\section{Faror}

\subsection{Överhettning}

Elektricitet kan lätt vålla både personskador och materiella skador.
Det är viktigt att veta hur skador kan undvikas.
Elektrisk utrustning ska vara beröringsskyddad med fullgod kapsling.
Samtidigt får värmen inne i kapslingen inte bli så hög att det innebär
brandrisk.
Spontana fel kan trots allt uppstå.

Isolationsfel medför risk vid beröring och brand kan utvecklas snabbt.
När utrustning under spänning lämnas obevakad, ska det ske med särskild
aktsamhet.

Hur elektriska apparater och anläggningar får utföras, regleras av
lagar och föreskrifter.
Elektriska apparater ska uppfylla vissa krav för att få marknadsföras och
användas.
Utförande och ursprung ska vara dokumenterat på föreskrivet sätt.

Även självbyggda apparater ska uppfylla kraven på elsäkerhet -- dvs. säkerhet
mot elchock och brand -- och byggaren bär ensam ansvaret för att utförandet och
hanterandet av apparaterna är betryggande.

Den som bygger och använder en elektrisk apparat bör därför ha
nödvändiga kunskaper om elsäkerhet.

\subsection{Höga spänningar}
\textbf{
HAREC a.\ref{HAREC.a.10.3}\label{myHAREC.a.10.3}
}
\index{batteri}
\index{ackumulator}

Ingrepp i elektriska apparater under spänning innebär personfara.
Öppna aldrig en apparat om spänningen är tillslagen.
Vid ingrepp t.ex. i sändare, mottagare och kraftförsörjningsaggregat är
det lätt att utsätta sig för höga likspänningar.
I sändare med elektronrör förekommer spänningar i storleksordningen hundratals
till tusentals volt.
Så är det också i bildskärmar.

Observera att även apparater som drivs med batteri eller ackumulatorer kan
innehålla kretsar som omvandlar den låga spänningen till direkt livsfarlig hög
spänning.

\subsection{Höga strömmar}

Höga strömmar ger häftiga muskelkramper och brännskador.
Man vet att det skiljer mellan skador av lik- respektive växelström.

Lågfrekvent växelström ger upphov till muskelkramper, som kan göra det
omöjligt att släppa det strömförande föremålet.

Högfrekvent växelström i MHz-området värmer upp kroppsvävnaderna,
snarare än att förorsaka muskel reaktioner.

Likström påverkar kroppen annorlunda än växelström.
Genom det elektriska motståndet i kroppens vävnader och vätskor utvecklas det
värme.
Detta kan leda till brännskador både på huden och inne i kroppen.
Om likströmmen pulserar uppstår dessutom muskelreaktioner på liknande sätt som
vid växelström.

Höga spänningar är alltid farliga.
Det är däremot inte så känt att även låga spänningar kan vara det.
Ackumulatorer och anslutna apparater kan ge ifrån sig höga strömmar även om
spänningen är låg.
Oavsiktliga strömvägar t.ex. kortslutning genom en klocka eller fingerring kan
medföra allvarliga brännskador.

\subsection{Antenner}

Placera helst antennerna utom räckhåll för obehöriga.
På sändarantenner kan det nämligen uppstå höga HF-spänningar redan vid
låg sändareffekt.
HF bränns vid beröring och en reflexrörelse gör det lätt att tappa balansen och
falla.
Sätt gärna upp skyltar på eller invid antennerna, med varning för högfrekvent
spänning samt uppgift om ägarens namn, adress och telefonnummer.

En obalanserad antenn kan resultera i stor spänning även på ansluten
antenn-kabel, och kan därför innebära samma risker som att ta i själva antennen.
T-antennen är en antenn som är designad för att utnyttja denna balans då själva
antenn-kabeln är del av antennen utstrålande delar, medan de flesta andra
antenner så ska antennkabeln inte stråla och ska därför inte innebära fara,
men iakttag försiktighet.
Obalans bör åtgärdas, inte enbart av personsäkerhet utan även för att få en
effektiv antenn.

Antenner får inte korsa eller placeras nära högspännings-, lågspännings- eller
telefonlinjer.
Det är en olycksrisk om antenner och kraft- eller teleledningar av någon
anledning slår ihop.
Det är också en olycksrisk om antenner faller ner över dessa ledningar.

Endast efter tillstånd från berörd myndighet och/eller linjeägare får
man dra ledningar av något slag över väg eller offentlig plats även antenner.
Höga likspänningar från sändaren får inte komma ut i antennen.
Se till att antennernas matarledningar är kopplade till god likströmsjord
via HF-drosslar eller försedda med överspänningsavledare.
Som extra säkerhetsåtgärd bör sändaren anslutas till antennledningen över en
stor kondensator.

Undvik att beröra antenner utan att de jordats, särskilt vid vistelse
på tak eller i träd.

Under åskväder, snöfall, regn eller dimma då laddade partiklar är i
rörelse, kan antennerna laddas upp till höga statiska spänningar.
Arbetar man då med antennen kan man överraskas av en elektrisk stöt.
Det är då lätt hänt att tappa taget och falla ner.

\subsection{Restladdning i kondensatorer}

Kondensatorer kan behålla en betydande restladdning under många timmar
sedan kraften brutits.
\begin{itemize}
\item Koppla urladdningsresistorer (bleeder) över filterkondensatorer,
  så att de laddas ur när matningen stängs av.
  Av säkerhetsskäl ska urladdningsresistorerna tåla fyra gånger så stor effekt
  som de själva förbrukar under drift.
\item Varning: När du laddar ur en kondensator, kortslut den inte!
  Använd en urladdningsresistor som tål den effekt som utvecklas vid urladdning!
\end{itemize}

\subsection{Säkerhetsåtgärder}
\index{batteri}
\index{ackumulator}
\index{litium batteri}
\index{LiPo batteri}

Transformator med förstärkt säkerhet
\begin{itemize}
\item Om du är osäker på det elsäkerhetsmässiga utförandet på en
  apparat, t.ex. en gammal sändare, använd då en skiljetransformator
  (fulltransformator) -- helst av klass II (extraisolerad).
\end{itemize}

Vid reparation ska utrustningen vara spänningslös.
Före arbetet ska du
\begin{itemize}
  \item stänga av utrustningens nätströmbrytare
  \item dra ur stickproppen ur vägguttaget (dubbel säkerhet).
\end{itemize}

Om trimning eller felsökning måste ske under spänning ska följande iakttas:
\begin{itemize}
\item Arbeta inte med anläggningen när du är trött eller omotiverad.
  Då är du minst vaksam mot olyckor.
\item Se till att du inte får ström genom kroppen, arbeta helst bara med höger
  hand och håll den andra borta från den utrustning som du arbetar med.
  Stoppa gärna den fria handen i fickan!
\item Ha inga hörtelefoner på huvudet.
  Använd högtalare om du trimmar med hörseln.
\item Helst bör någon finnas i närheten när du arbetar i apparater under
  spänning.
  Visa var nätströmbrytaren sitter.
  Se gärna till att han/hon kan elolycksfallshjälp.
\end{itemize}

Vid arbete med ackumulatorbatterier:
\begin{itemize}
\item Trots att spänningen är låg kan ackumulatorbatterier lämna
  mycket höga strömmar vid kortslutning.
  Tag därför av fingerringar, armbandsur m.m.

Använd isolerade verktyg vid arbete med batteripolskor.
\item Akta dig för elektrolyten i ackumulatorbatterierna -- den är
  starkt frätande.
\item Varning för explosionsrisk av knallgas och syrastänk i ögonen.
\item Moderna litium polymer LiPO batterier är oerhört energirika.
  Dessa kan börja brinna med hög temperatur, och bör behandlas varsamt samt
  läggas i därför lämpliga skyddspåsar.
  Olika varianter är olika känsliga, så det är rekommenderat att läsa på om
  de alternativ som finns och hur bäst hantera dem.
  Akta för överladdning!
\end{itemize}
