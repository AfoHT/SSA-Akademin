\section{Svenska lagar, bestämmelser}

\hilight{TODO: Behöver uppdateras med aktuella lagar och förordningar.}

\textbf{Lagar, föreskrifter och anvisningar tillämpas för
  amatörradioanvändning.}

\textbf{Märk, att ändringar kan förekomma.}

\textbf{Använd därför aktuella versioner!}


\subsection{Lag om elektronisk kommunikation m.fl.}
\textbf{
HAREC b.\ref{HAREC.c.3.1}\label{myHAREC.c.3.1}
}

Lag (2003:389) om elektronik kommunikation reglerar all radiokommunikation Sverige. Tillstånd behövs för all radiosändning som inte är undantagen
tillståndsplikt.

\emph{Post- och telestyrelsen - PTS} - är enligt Förordning (2003:396) om
elektronisk kommunikation den svenska myndighet (administration) som handlägger
ärenden gällande telekommunikation. PTS skall bland annat svara för att
möjligheterna till radiokommunikationer utnyttjas effektivt och har därvid att
beakta den internationella regleringen inom området. Regleringen av
amatörradioanvändningen begränsas nu till den minsta omfattning som
följer av internationella avtal och europeiska rekommendationer,
CEPT-rekommendationer.

\subsection{Post- och telestyrelsens föreskrifter om undantag från tillståndsplikt för användning av vissa radiosändare}
\textbf{
HAREC b.\ref{HAREC.c.3.2}\label{myHAREC.c.3.2}
}

Post- och telestyrelsens föreskrifter om undantag från tillståndsplikt för
användning av vissa radiosändare PTSFS 2015:4.

Post- och telestyrelsen föreskriver med stöd av 12 § förordningen (2003:396)
om elektronisk kommunikation att användningen av amatörradiosändare är
undantagen tillståndsplikt.

Den som använder en amatörradiosändare ska ha ett amatörradiocertifikat. För
att få ett amatörradiocertifikat krävs kunskaper i enlighet med Annex 6 i 
CEPT Rekommendation T/R 61-02, Examinering för amatörradio certifikat

Undantag från kravet på amatörradiocertifikat gäller för den som under en
tidsbegränsad period utbildar sig för att få ett sådant certifikat och för
den som under en förevisning tillfälligt använder amatörradiosändare, under
förutsättning att användningen av radiosändaren sker under uppsikt av en
innehavare av amatörradiocertifikat.

Den som innehar amatörradiocertifikat ska ha en egen anrops signal.
Denna framgår av certifikatet, eller tidigare av amatörradiotillståndet.

\subsection{Litteraturhänvisning om lagar och föreskrifter}

\begin{itemize}
\item CEPT rekommendation T/R 61-01,
\item CEPT rekommendation T/R 61-02,
\item Lag (2003:389) om elektronik kommunikation,
\item Förordning (2003:396) om elektronisk kommunikation,
\item Post- och telestyrelsens föreskrifter om undantag från tillståndsplikt för
användning av vissa radiosändare PTSFS 2015:4,
\end{itemize}


