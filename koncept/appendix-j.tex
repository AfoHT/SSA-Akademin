\chapter{Rapportkoder}
\label{Rapportkoder}

Det finns olika sätt och system att rapportera hur en radiostation hörs.
%Plockat bort rapport om Modulation då den inte längre förekommer vid telefoni.

\section{Amatörradiotrafik}

I amatörradiotrafik används RST-koden vid rapportering av
telegrafisignaler och RS-koden för telefonisignaler.
Namnet kommer av begynnelsebokstäverna i de engelska orden

\begin{description}
  \item[Readability] (läsbarhet),
  \item[Signal strenght] (signalstyrka),
  \item[Tone] (ton),
\end{description}

\subsection{R-skala (läsbarhet)}

\begin{tabular}{p{0.02\textwidth}p{0.98\textwidth}}
1 & Oläsbar \\
2 & Knappt läsbar, enstaka ord tydbara \\
3 & Läsbar med stor svårighet \\
4 & Läsbar med obetydlig svårighet \\
5 & Helt läsbar \\
\end{tabular}

\subsection{S-skala (signalstyrka)}

\begin{tabular}{p{0.02\textwidth}p{0.98\textwidth}}
1 & Signalerna knappt uppfattbara \\
2 & Mycket svaga signaler \\
3 & Svaga signaler \\
4 & Något svaga signaler \\
5 & Ganska goda signaler \\
6 & Goda signaler \\
7 & Mycket goda signaler \\
8 & starka signaler \\
9 & Mycket starka signaler \\
\end{tabular}

\subsection{T-skala (ton)}

\begin{tabular}{p{0.02\textwidth}p{0.98\textwidth}}
1 & Mycket rå växelströmston, ostabil och omusikalisk \\
2 & Mycket rå växelströmston, stabil men musikalisk \\
3 & Rå växelströmston, ostabil och omusikalisk \\
4 & Rå växelströmston. stabil och någorlunda musikalisk \\
5 & Tydligt växelströmsmodulerad ton, ostabil men musikalisk \\
6 & Tydligt växelströmsmodulerad ton, stabil och musikalisk \\
7 & Nästan ren likströmston, ostabil och med tydligt brum \\
8 & Nästan ren likströmston, med spår av brum eller ojämnheter \\
9 & Absolut ren likströmston, stabil \\
\end{tabular}

Dessutom kan följande tillägg till T-skalan förekomma:

\begin{tabular}{p{0.02\textwidth}p{0.98\textwidth}}
  X & Absolut ren likströmston, mycket stabil, kristallklar, mjuka tecken utan
      knäppar \\
C & Absolut ren, men ostabil likströmston vid nycklingen \\
K & Knäppar alstras vid nycklingen \\
\end{tabular}

\section{Kommersiell sjö- och luftradiotrafik}

I kommersiell sjö- och luftradiotrafik används t.ex. Q-förkortningarna
QSA (signalstyrka), ORM (störningar från annan station), QRN
(atmosfäriska störningar), QSB (fädning) och ORK (uppfattbarhet)
åtföljda av en siffra för graden i skala 1--5. Jämför med SINPO-koden
i avsnitt \ref{sinpo}.

\textbf{Exempel:}

''QSA 5, ORK 3, QRN 1'', vilket betyder
''ljudstyrka mycket god, uppfattbarhet ganska god, störningar från andra
stationer måttliga, atmosfäriska störningar obefintliga''.

Se mer om Q-förkortningar i avsnitt \ref{q-koden}.

\section{Rundradiosändningar m.m.}
\label{sinpo}

\begin{wraptable}{R}{0.5\textwidth}
\begin{tabular}{lll}
  Kod & Grad & Bedömning \\
  S   & 1    & Knappt uppfattbar \\
      & 2    & Dålig \\
      & 3    & Tillfredsställande \\
      & 4    & God \\
      & 5    & Utmärkt \\
  & & \\

  Kod     & Grad & Bedömning \\
  I, N, P & 1    & Mycket stark \\
          & 2    & Stark \\
          & 3    & Måttlig \\
          & 4    & Svag \\
          & 5    & Ingen \\
  & & \\

  Kod & Grad & Bedömning \\
  F   & 1    & Mycket snabb \\
      & 2    & Snabb \\
      & 3    & Måttlig \\
      & 4    & Långsam \\
      & 5    & Ingen \\

  & & \\
  Kod & Grad & Bedömning \\
  E   & 1    & Mycket dålig \\
      & 2    & Dålig \\
      & 3    & Tillfredsställande \\
      & 4    & God \\
      & 5    & Utmärkt \\

  & & \\
  Kod & Grad & Bedömning \\
  M   & 1    & Ständig övermodulering \\
      & 2    & Dålig eller ingen \\
      & 3    & Tillfredsställande \\
      & 4    & God \\
      & 5    & Maximal \\

  & & \\
  Kod & Grad & Bedömning \\
  O   & 1    & Oanvändbar \\
      & 2    & Dålig \\
      & 3    & Tillfredsställande \\
      & 4    & God \\
      & 5    & Utmärkt \\
\end{tabular}
\end{wraptable}

För rapportering till rundradiostationer m.m. används ett system
som kallas SINPO eller tidigare SINPFEMO.
Numera används enbart SINPO-koden.

Namnet på koden kommer av begynnelsebokstäverna i orden

\begin{description}[style=nextline]
\item[Signal strength]
  (signalstyrka),
\item[Interference]
  (störningar från annan radiosändning),
\item[Noise]
  (atmosfäriska störningar),
\item[Propagation disturbance]
  (vågutbredningsstörningar),
\item[Frequency of fading]
  (fädningsfrekvens),
\item[Emission quality]
  (modulationskvalitet)
\item[Modulation depth]
  (modulationsgrad),
\item[Over all merit]
  (sammanfattande omdöme).
\end{description}

Rapporten inleds med koden SINPO följd av fem siffror, vilka var och en i tur
och ordning graderar egenskaperna i skala 1--5.
För icke bedömda egenskaper ska bokstaven X ersätta siffran för egenskapen.
