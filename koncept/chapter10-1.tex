\section{Människokroppen}
\textbf{
HAREC a.\ref{HAREC.a.10.1}\label{myHAREC.a.10.1}
}

\subsection{Elektrisk chock}

Människokroppen är ett komplicerat elektrokemiskt system, som främst
kontrolleras av hjärnan. Musklerna styrs av svaga elektriska
strömimpulser genom nervsystemet. Främmande strömmar genom kroppen kan
störa kroppsfunktioner och kan i olyckliga fall göra stor
skada. Styrkan och frekvensen på strömmarna avgör skadans art och
omfattning.

Elektrisk chock kan döda av flera orsaker.

En orsak är att hjärtrytmen störs. Hjärtkammarflimmer och
hjärtstillestånd kan lätt uppstå. Flimmer innebär att hjärtat arbetar
okontrollerat och med kraftigt nedsatt eller helt upphävd
pumpfunktion. Hjärtstillestånd inträffar lätt av hög spänning. Av
otillräcklig blodtillförsel blir det syrebrist i hjärncellerna, som då
skadas snabbt. Medvetslöshet inträder redan efter ett fåtal sekunder.

En annan orsak är andningsstillestånd genom att andningscentum
blockeras. Det kan hända när strömmen från en högspänningskondensator
går genom kroppen.

\subsection{Hjärt- och lungräddning, HLR}
\index{hjärt- och lungräddning (HLR)}
\index{HLR}

Vid hjärtstillestånd, hjärtkammarflimmer eller andningsstillestånd ska
hjärt- och lungräddning påbörjas omedelbart då obotliga hjärnskador av
syrebrist kan uppstå inom några få minuter. Finns en hjärtstartare,
AED, i närheten bör den användas så skyndsamt som möjligt.

\textbf{Glöm inte att ringa efter hjälp! Ring 112!}

Broschyren \emph{Vägledning vid elskada} kan laddas ner eller beställas från
Elsäkerhetsverkets hemsida (\url{http://www.elsakerhetsverket.se}).

Vårdguidens hemsida (\url{https://www.1177.se}) har instruktioner för hjärt- och lungräddning, HLR.

Svenska rådet för Hjärt- och Lungräddning (\url{http://www.hlr.nu}) har beskrivningar
och instruktionsfilmer för hjärt- och lungräddning.

\subsection{Resistansen genom människokroppen}

Vid kontakt med ett strömförande föremål kommer kroppen att bli en del
av strömkretsen. Det flyter då en främmande ström genom kroppen.

Strömstyrkan följer Ohms lag och beror av strömkällans spänning och
inre resistans samt av övergångsresistansen i huden och kroppens inre
resistans.

Övergångsresistansen minskar med fuktigare hud samt med större
kontaktyta och större kontakttryck. Beröringsspänningen inverkar
också. Vid spänningar över ca 75 V minskar övergångsresistansen med
ökande spänning. Vid allvarliga förbränningar minskar
övergångsresistansen särskilt mycket. Den totala resistansen genom
kroppen blir då nära lika med dess inre resistans -- ungefär 500~Ω.

\textbf{VARNING. Experimentera inte med detta!}


\subsection{Strömmens inverkan på människan}

Sjukvården skiljer på verkan av strömstöt, strömgenomgång och ljusbåge.

En strömstöt kan tyckas ofarlig men kan leda till okontrollerade rörelser,
fallskada eller beröring av andra spänningsförande föremål.

Vid en strömgenomgång utjämnas en elektrisk potentialskillnad genom kroppen
vilket utöver hjärtstillestånd, hjärtkammarflimmer, och andningsstillestånd
kan leda till blodpropp, muskelskador, njurskador eller inre brännskador.

Vid en ljusbågsolycka ökar risken för kraftiga brännskador på grund av den
höga temperaturen i ljusbågen. En ljusbåge kan även orsaka skador på ögonen
på grund av bländning eller den stora mängden UV-ljus.

\emph{
Personer som drabbats av olycka med:
\begin{itemize}
\item högspänning,
\item lågspänning med strömgenomgång genom bålen
\item som är omtöcknade eller medvetslösa efter strömolycka
\item som har drabbats av brännskada
\item som visar tecken på nervskada till exempel förlamning
\end{itemize}
Ska omedelbart till sjukhus för akut behandling.}

Starka strömmar ger häftiga muskelkramper och/eller brännskador. Muskelkramp
kan förekomma redan vid strömmar under 10~mA. För vuxna, friska människor är
det direkt farligt när strömmen överstiger detta värde. För unga eller sjuka
kan strömmar under 10~mA vara direkt farliga.

Strömstyrkan påverkar kroppen olika från fall till fall och det är osäkert
vilken strömstyrka som är farlig. Det finns både de som överlevt höga strömmar
och de som inte har klarat någon milliampere. Strömmar som går genom hjärta
eller hjärna är särskilt farliga. När man arbetar med elektriska apparater
under spänning, bör man för säkerhets skull hålla den ena handen i fickan!

\subsection{Påverkan från elektromagnetiska fält}

Undersökningar har visat att vistelse i starka elektromagnetiska fält
kan kan påverka människan. Personer som har varit utsatta för kraftig
exponering av fält har bl.a. klagat över svettningar och
huvudvärk. Det forskas omkring dessa fenomen.

Elektromagnetiska fält kan förorsaka fel i
elektronikutrustningar. Halvledare är särskilt känsliga för
kraftfält. Det är möjligt att känsliga instrument, hjärtstimulatorer
(pacemaker) etc. kan påverkas av högfrekventa elektromagnetiska
fält från radiosändare. När du använder en sändare, mobiltelefon etc.
och någon får svårigheter med hjärta eller andning så ska du
omedelbart stänga av din apparat helt! Med tiden utvecklas
störningsokänsligare elektronik, men säker mot störningar kan man
aldrig vara.

\subsection{Normer för fältstyrkor}

Det finns flera olika normer och rekommendationer för elektromagnetiska
fältstyrkor. Några av dessa normer har till exempel syftet att olika slags
apparater ska kunna samexistera och därför fungera utan att påverkas av
elektromagnetiska fält eller stråla ut elektromagnetiska fält överstigande
givna gränsvärden (EMC).

Andra normer och råd har till syftet att skydda arbetstagare eller individer
ur allmänheten från akuta biologiska effekter när de exponeras för
elektromagnetiska fält.

Strålsäkerhetsmyndigheten har genom utgivandet av SSMFS 2008:18 publicerat
allmänna råd om begränsning av allmänhetens exponering för elektromagnetiska
fält. Dessa råd bygger på rekommendationer från Europeiska unionens råd. 
