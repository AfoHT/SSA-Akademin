\chapter{EMC}

Endast en del av frekvensspektrum av elektromagnetiska vågor används för
radiosändningar.
Samtidigt används detta utrymme av allt fler intressenter och för allt fler
ändamål.
Samhället blir alltmer tekniskt avancerat och elektroniktätheten tilltar
kraftigt.
Den ökande mängden och komplexiteten hos apparaterna kräver därför regler, som
styr både utförande och användning med rimligt bibehållen säkerhet och funktion.

\section{Störningar och störkänslighet}

\subsection{Om EMC-lagen}
\label{EMC-lagen}

Det kan inte längre ges enkla svar på vad som är att vara störd och att störa.
Internationella och nationella väl preciserade regler för radio- och
teletekniskt samexistens är numera helt nödvändiga.

Samlingsbegreppet är \emph{Electromagnetic Compatibility} (EMC), dvs. en
apparats förmåga att fungera tillfredsställande i sin elektromagnetiska
omgivning utan att alstra elektromagnetiska störningar som överstiger en nivå,
som tillåter radio- och teleutrustning och andra apparater att fungera som
avsett.
Vidare ska apparater ha en sådan tillräcklig inbyggd tålighet mot
elektromagnetiska störningar, att de kan fungera som avsett.

Till skydd för liv, personlig säkerhet och hälsa samt kommunikationer och
näringsverksamheter har därför Lag om elektromagnetisk kompatibilitet införts.
Denna lag är anpassad efter av EG/EES utfärdade direktiv angående bl.a.
radiostörningar.

Förordning om elektromagnetisk kompatibilitet \emph{SFS 2016:363}
\cite{SFS2016:363} definierar nyckelbegreppen; apparater, EMC, elektromagnetisk
störning och tålighet.
Elsäkerhetsverket är ansvarig myndighet, med rätt att utfärda föreskrifter om
bl.a. skyddskraven, kontroll och märkning samt om vissa undantag.
Föreskrifter utfärdade av Elsäkerhetsverket publiceras som ELSÄK-FS.

Post- och telestyrelsens föreskrifter om undantag från tillståndsplikt för
användning av vissa radiosändare \emph{PTSFS 2015:4} \cite{PTSFS2015:4} hänvisar
till\\
Lag om elektronisk kommunikation (\textbf{LEK}) \emph{SFS 2003:389}
\cite{SFS2003:389}.
Där kan följande läsas om åtgärder vid störningar.

\subsection{Ur LEK}

\begin{quote}
3 Kap. 13\S Om det uppkommer skadlig störning, skall tillståndshavaren
omedelbart se till att störningen upphör eller i möjligaste mån minskar, om
inte störningen är tillåten.
Detsamma gäller den som använder en radiomottagare som stör användningen av en
annan radiomottagare.
\end{quote}

\begin{quote}
3 Kap 14\S Elektriska eller elektroniska anläggningar som, utan att vara
radioanläggningar, är avsedda att alstra radiofrekvent energi för
kommunikationsändamål i ledning eller för industriellt, vetenskapligt,
medicinskt eller något annat liknande ändamål, får användas endast i enlighet
med föreskrifter som meddelas av regeringen eller den myndighet som regeringen
bestämmer.

Regeringen eller den myndighet som regeringen bestämmer får meddela
föreskrifter om förbud mot att inneha elektriska eller elektroniska
anläggningar som inte omfattas av första stycket och som, utan att vara
radioanläggningar, är avsedda att sända radiovågor.
\end{quote}

I \textbf{LEK} definieras bl.a. radioanläggning:
\begin{quote}
anordning som möjliggör radiokommunikation eller bestämning av position,
hastighet eller andra kännetecken hos ett föremål genom sändning av radiovågor
(radiosändare) eller mottagning av radiovågor (radiomottagare)
\end{quote}

\subsection{Utstrålning från amatörradiosändare}

Vad som sägs i 3 Kap 4\S Lag om elektronisk kommunikation och skrivningen i
Post- och telestyrelsens föreskrifter om undantag från tillståndsplikt för
användning av vissa radiosändare 3 Kap 14\S \emph{De tekniska egenskaperna hos
amatörradiosändaren ska anpassas så att de inte stör användningen av andra
radioanläggningar.}
Samt skrivningen i Strålsäkerhetsmyndighetens SSMFS 2008:18.
Medför att sändareffekten alltid ska anpassas så att styrkan av utstrålade
fält inte förorsakar störningar eller för höga nivåer av elektromagnetiska fält.

Den enligt undantagsföreskrifterna högsta tillåtna effekten kan alltså inte
användas hinderslöst.

Om störningarna inte kan avhjälpas kan PTS komma att anvisa om restriktioner
(begränsningar i sändningstillståndet), det kan vara sändningsförbud under
vissa tider, på vissa frekvenser, över viss sändareffekt etc.

\subsection{PM vid störningsproblem}
\begin{itemize}
\item Störningar är alltid förenade med obehag och ställer grannsämjan på prov.
  Håll dig väl med dem som bor i omgivningen!
\item Om det väcks klagomål på dig om störningar, ska du först
  kontrollera din egen sändare och antennanläggning.
\item Be därefter att få undersöka antennanläggning och apparater hos
  den som besväras av störningar.
\item Om du ser en lösning, berätta om vad som kan göras.
  Kom överens om vad som får göras.
  Ändra då inte något inne i apparater, men prova gärna ut yttre,
  kompletterande filter etc.
\item Om det inte går att komma till rätta med störningarna bör de som
  levererat och installerat anläggningen anlitas.
\item Störningsanmälan kan även ske till PTS närmaste tillsynsområde.
%PHU: Finns fortfarande "PTS tillsynsområde?
\end{itemize}

\subsection{Arbeta aktivt med avstörning}
\begin{itemize}
\item Låna hem en av SSA:s avstörningslådor och försök att finna en lösning.
  I lådan finns ett sortiment av frekvensfilter för avstörning.
\item Undvik att störa i onödan.
  Sänk sändareffekten och begränsa sändningstiden under utprovningen av en
  lösning.
\end{itemize}

Lyckas du inte själv med att störa av
\begin{itemize}
\item ta gärna hjälp av en radioamatör med erfarenhet av avstörning
\item anlita annan sakkunnig hjälp.
\end{itemize}
