\section{Åska}
\textbf{
HAREC a.\ref{HAREC.a.10.4}\label{myHAREC.a.10.4}
}

\subsection{Faror}

Vid åska utvecklas det mycket starka, elektromagnetiska fält, som breder ut sig
och alstrar mycket korta spänningsstötar i alla metallföremål, till exempel i antenner.
Stötarna vandrar genom kablarna in i apparaterna.
Är stötströmmen hög, kommer saker i strömvägen att förstöras på något sätt.
Förbränning och nersmältning är vanligt.

Men om blixturladdningen sker på långt håll, kan stötströmmen bli så låg att
man någorlunda kan undgå skada på apparater och hus.

Om blixturladdningen däremot sker mycket nära antennen eller som direkt nedslag,
då uppstår definitivt stora skador.

\subsection{Skydd och jordning}

Antenner och antennkablar kan man aldrig skydda mot blixtnedslag.
De är till sin natur en slags åskledare.
Det man kan försöka att göra är att leda en eventuell blixturladdning i ett
antennsystem bort från hus och människor.
Observera, att man inte får ''haka på'' husets ordinarie åskledare.
Då gäller inte husförsäkringen.

Antennkabeln, som fungerar som en (för klent dimensionerad) åskledare,
ska naturligtvis inte i onödan dras in i huset utan kortaste vägen
utanför huset till en avgrening.

Från avgreningen fortsätter dels kabeln in till apparaterna genom ett
överspänningsskydd och dels en jordlina kortaste vägen ner till
jordtaget över en gniststräcka.
Det bästa sättet att skydda apparaterna mot åska är fortfarande att koppla bort
dem helt från antennkabeln och vägguttag.

Om man bor i ett hyreshus är det tyvärr oftast svårt att få vidta
åtgärder som dem här ovan.
Då får man nöja sig med att koppla bort antennledningarna från apparaterna och
lägga dem väl åt sidan -- gärna utanför husväggen.

Som permanent, men otillräckligt skydd kan man förse de olika
anslutningsställena med lämpliga överspänningsskydd.

Mer information om hur man kan skydda sin amatörradiostation mot blixten
finns på webbplatsen för Uppsala universitet, avdelningen för elektricitetslära.
Dokumentet \emph{Att skydda sin amatörradiostation mot blixten}
(\url{http://www.hvi.uu.se/meny/m5.html}).
