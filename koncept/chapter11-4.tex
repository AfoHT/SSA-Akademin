\section{Internationell nödtrafik och trafik vid naturkatastrofer}
\textbf{
	HAREC b.\ref{HAREC.b.4.1}\label{myHAREC.b.4.1},
	b.\ref{HAREC.b.4.2}\label{myHAREC.b.4.2},
	b.\ref{HAREC.b.4.3}\label{myHAREC.b.4.3},
	b.\ref{HAREC.b.4.4}\label{myHAREC.b.4.4}
}

\section{Internationell nödtrafik och trafik vid naturkatastrofer}
\textbf{
	HAREC b.\ref{HAREC.b.4.1}\label{myHAREC.b.4.1},
	b.\ref{HAREC.b.4.2}\label{myHAREC.b.4.2},
	b.\ref{HAREC.b.4.3}\label{myHAREC.b.4.3},
	b.\ref{HAREC.b.4.4}\label{myHAREC.b.4.4}
}

\subsection{Nödsignaler}
\index{SOS}
\index{MAYDAY}
\index{GMDSS}

I CEPT-rekommendation T/R 61-02 ställs krav på att radioamatörer ska ha känna
till de internationella nödsignalerna SOS och MAYDAY.

ITU Radioreglemente (RR) har sedan WRC-07 inte längre information om
``Distress and safety communications'' för annat än GMDSS (\emph{Global Maritime Distress and Safety System})

Nödsignalen på morsetelegrafi består av teckendelarna . . . --- --- --- . . .
sända i en följd, där längden på de långa teckendelarna betonas så att de klart
skiljer sig från de korta.

Signalen skrivs som bokstäverna SOS med ett streck ovanför.

Nödsignalen på radiotelefoni består av ordet MAYDAY uttalat som det franska
uttrycket ``m'aider''. I Sverige kan man även ropa ``NÖDANROP''.

\subsection{Internationella nödfrekvenser}

Nödsignaler på telefoni sänds i första hand på frekvenserna:
\begin{itemize}
	\item 121,5~MHz (flygtrafik)
	\item 156,8~MHz VHF kanal 16 (sjöfart)
\end{itemize}

\subsection{Nödtrafik}

I CEPT-rekommendation T/R 61-02 ställs krav på att radioamatörer ska ha känna
till bestämmelser om nödtrafik och användningen av amatörradiostationer vid
naturkatastrofer.

1998 hölls en internationell konferens i Tapere, Finland (\emph{ICET-98}).
Konferensen ledde fram till \textbf{``THE TAMPERE CONVENTION''} \emph{the
	Tampere convention on the provision of telecommunication resources for disaster
	mitigation and relief operations}. Konventionen trädde i kraft 8~januari 2005.

Sverige signerade konventionen 10~juni 2003 med deklarationen att den behöver
beredas inom EU för att helt implementeras. Deklarationen bekräftades vid
ratifikationen. (\emph{United Nations Treaty Collection Chapter XXV Telecommunications 4.Tampere Convention})

Arbetet med konventionen har medfört att IARU har infört rekommendationer om
regionala och globala frekvenser för \emph{Emergency Centre of Activity}. Det
vill säga centerfrekvenser för radiokommunikation som kan användas i händelse
av naturkatastrofer. Notera att i den Svenska bandplanen för kortvåg har dessa
frekvenser benämnts ``Nödfrekvenser''.

\textbf{IARUs rekommendationer och förändringen av ITU RR innebär att det inte
	finns någon speciell nödsignal för amatörradiobanden och inga nödfrekvenser
	inom amatörradiobanden.}   

För vidare läsning rekommenderas
\emph{IARU Emergency Telecommunications Guide 1 september 2016}

\subsection{Om Du hör en nödsignal på radio}

Avbryt omedelbart din egen sändning när du hör en nödsignal. Lyssna på
nödmeddelandet och SKRIV NER vad som sägs. Notera position, frekvens, tidpunkt
etc. Anmäl vad du hört på följande sätt.
%tagit bort telefonnummer till UD och räddningscentraler. NTJ

\subsubsection{Nödsignal från radioamatör i utlandet}

Ring 112 och berätta att du uppfattat en nödsignal från utlandet via radio.

\subsection{Nödsignal från svenskt landområde}

Ring 112 för att kalla på Ambulans, Polis, Räddningskår, Sjöräddning,
Flygräddning etc. Ditt telefonnummer visas automatiskt i larmoperatörens
display. För att undvika missförstånd och feldirigering av
räddningsinsatserna MÅSTE du meddela operatören att nödanropet kommit via radio.
Själva olycksplatsen kan ligga i ett helt annat riktnummerområde än det som ditt
telefonsamtal kommer ifrån.

\subsection{Nödsignal från fartyg eller luftfarkost}

Om nödsignalen inte besvaras av någon kust- eller markstation, ring 112
och begär Sjöräddning respektive Flygräddning och meddela dina
iakttagelser.

\emph{Vidarebefordra nödmeddelandet utan att ändra på det!}

\subsection{Du själv sänder nödsignal över radio}

Uppträd lugnt och sansat, när du kallar på hjälp över radion. Tänk först och
sänd sedan. Som ovan sagts måste den som svarar dig och sedan ringer 112 meddela
larmoperatören att Ditt nödanrop kommit via radio.

\subsection{Åtgärder}

Nyckelordet för dina åtgärder är LARMA:

\begin{tabular}{lp{9cm}}
	\textbf{L}äge &
	Ange olycksplatsens läge. Du kan ange gatu- eller vägnamn eller riktmärken som
	t.ex. vägkorset, gränsen, bron, järnvägen etc.
	\\
	\textbf{A}nalysera
	&
	Gör en överblick över
	olycksplatsen och tala om vad som hänt.  Några skadade? Några innestängda?
	Brinner det? Släpps farliga ämnen ut?
	\\
	\textbf{R}opa &
	Ropa på hjälp. Använd gärna en repeater på 2-metersbandet så att du når många,
	men även andra frekvenser kan användas.  Anropa med NÖDANROP FRÅN SMXxxx. Fråga
	efter någon med telefon. Ge inte upp om du inte får svar genast.
	\\
	\textbf{M}eddela &
	Meddela när du fått kontakt med någon med telefon, sänd NÖDTRAFIK PÅGÅR för att
	freda frekvensen och NÖDMEDDELANDET med de viktigaste uppgifterna. Begär att
	uppgifterna repeteras och ta löfte på att de sänds vidare. Begär att få veta när
	så har skett. Påminn annars!
	\\
	\textbf{A}vvakta &
	Vänta på platsen tills hjälp har anlänt. Passa radion så att du kan svara på
	frågor. Behövs inte längre din hjälp, avsluta då med\\
	& NÖDTRAFIK UPPHÖR FRAN SMXxxx \dots KLART SLUT.
\end{tabular}