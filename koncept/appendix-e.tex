\chapter{Beskrivningskod typ av sändning}
\label{sändslag}
\index{bandbredd}
\index{basband}
\index{sidband}
\index{använd bandbredd}
\index{nödvändig bandbredd}
\index{tilldelat frekvensband}
\index{frekvenstolerans}

Radiosändningar beskrivs enligt ITU-RR Appendix 1 \cite{ITU-RR-appendices} med
standardiserade kombinationer av siffror och bokstäver som beskriver
sändningens nödvändiga bandbredd och sändningsklass.

Den fullständiga beskrivningen av en radiosändning inleds med fyra tecken
som beskriver den nödvändiga bandbredden.
Detta följs av tre tecken som beskriver sändningsklassen.
Vid behov kan sändningsklassens tre tecken kompletteras med ytterligare två
tecken som tydligare beskriver signalen.

Detta benämningssystem är dock inte utan problem.
Det tar mer hänsyn till metoden hur en signal alstras, snarare än hur en signal
helt enkelt ser ut när den sänds.

Direkt modulation av huvudbärvågen benämns på ett sätt, medan modulation
av en underbärvåg i en sändare för enkelt sidband med undertryckt bärvåg
benämns på ett annat sätt.
Om man t.ex. nycklar ett RTTY-modem med en direktskrivande fjärrskrivare och
sedan byter till en dator för att göra samma sak, så ändras benämningen av
sändningsslaget.

\subsection{Bandbredd}
\index{nödvändig bandbredd}
\index{basband}
\index{sidband}
\index{Lower Side Band}
\index{USB}

Den nödvändiga bandbredden (eng. \emph{necessary bandwidth}) beskrivs med
tre siffror och en bokstav.
Bokstaven placeras på platsen för decimaltecknet och representerar enheten för
bandbredd.
Bokstäverna H (Hz), K (kHz), M (MHz) och G (GHz) används, medan varken 0 eller
K, M eller G får vara det första tecknet.
Numeriska värden med mer än tre signifikanta siffror avrundas.

Decimaltecknen används på följande sätt:\\
\begin{tabular}{lll}
	bandbredd & 0,001--999 Hz & (decimaltecken H),\\
	bandbredd & 1,00--999 kHz & (decimaltecken K),\\
	bandbredd & 1,00--999 MHz & (decimaltecken M),\\
	bandbredd & 1,00--999 GHz & (decimaltecken G).\\
\end{tabular}

\textbf{Exempel:}

\begin{tabular}{lll}
	0,002 Hz & skrivs & H002\\
	12,5 kHz & skrivs & 12K5\\
	0,1 Hz & skrivs & H100\\
	2,4 kHz & skrivs & 2K40\\
	25,3 Hz & skrivs & 25H3\\
	6 kHz & skrivs & 6K00\\
	180 kHz & skrivs & 181K\\
	6,25 MHz & skrivs & 6M25\\
\end{tabular}

Det är särskilt viktigt att komma ihåg bandbredden vid utsändningar nära
bandgränserna.
T.ex. kommer sidbandet (USB) i en telefonisignal med bärvågsfrekvensen 29,699,
att tydligt överskrida den övre bandgränsen för 10-metersbandet.
Bandgränserna får INTE överskridas och det gäller även för sändningens sidband.

\emph{Basbandet} är det frekvensområde, som upptas av signaler innan de
modulerar bärvågen.
Signaler i basbandet ligger vanligen mycket lägre i frekvens än bärvågen.
I den låga änden av basbandet kan frekvensen närma sig eller vara likström
(0~Hz).
I den höga änden beror frekvensen på det värde där information finns liksom att
det finns underbärvågor eller andra speciella signaler inom basbandet.
Det finns ett basband för alla typer av signaler, vare sig de är analoga eller
digitala.
Det ska också förstås, att termen basband är relaterad till den modulation som
avses från fall till fall.

Det kan finnas mer än ett basband i en komplett modulationsprocess.
Till exempel, en nycklad ton som går till sändaren genom mikrofoningången är
dess analoga basband medan nycklingspulserna till tongeneratorn är dess digitala
basband.

\emph{Sidband} alstras alltid när en bärvåg moduleras.
De är blandningsprodukter på båda sidor om bärvågen, som resultat av att
signaler från basbandet modulerar bärvågen på något sätt.
Det övre sidbandet kallas USB (eng. \emph{upper sideband (USB)}
och det undre sidbandet LSB (eng. \emph{lower sideband (LSB)}.

I system för amplitudmodulation är bredden på sidbanden i stort lika med den
högsta frekvenskomposanten i basbandet.
Sidbanden är spegelbilder av varandra och innehåller exakt samma information.
För att spara bandbredd räcker det alltså med att överföra det ena sidbandet,
varvid det andra sidbandet kan undertryckas, liksom även bärvågen.

I andra modulationssystem än för amplitudmodulation kan däremot bredden på
sidbanden mycket överstiga den högsta frekvenskomposanten i basbandssignalen.

\emph{Använd bandbredd} (eng. \emph{occupied bandwidth}) är avståndet mellan
(-23~dB) av den totala medeleffekten.
För amatörer är det inte alltid lätt att översta och nedersta delen av ett
spektrum, där medeleffekten är lägre än 0,5~\% bestämma den använda bandbredden.
Den kan mätas med en spektrumanalysator, men ett sådant instrument är
svårtillgänglig för de flesta amatörer.
Den använd bandbredden kan även beräknas, men det kräver matematikkunskaper i
informationsteori och behandlas inte här.

\emph{Nödvändig bandbredd} är den del av den använda bandbredden, som räcker
för att säkra informationsöverföringen i den omfattning och kvalitet som krävs.
Förenklade sätt att beräkna nödvändig bandbredd vid specifika modulationssystem
finns i kapitel \ref{modulation}.

\emph{Tilldelat frekvensband} är den nödvändiga bandbredden plus två gånger den
absoluta frekvenstoleransen.

\emph{Frekvenstolerans} \emph{eng. frequency tolerance} uttryckt i PPM (del per
\(10^6\)) , procent eller i Hz är den maximalt tillåtna frekvensavvikelsen från
den korrekta frekvensen.

\subsection{Sändningsklass}
Sändningklass anges med tre tecken där:\\
\begin{tabular}{ll}
	Första tecknet beskriver &  Huvudbärvågens modulation \\
	Andra tecknet beskriver & Den modulerade signalens karaktär \\
	Tredje tecknet beskriver & Typ av information \\
\end{tabular}

\subsubsection{Huvudbärvågens modulation}
\textbf{Första tecknet} -- Huvudbärvågens modulation.\\
\begin{tabular}{ll}
	Ingen modulation & N\\
	Utsändning där huvudbärvågen är amplitudmodulerad &\\
	(även i fall med vinkelmodulerad underbärvåg) &\\
	Dubbla sidband & A\\
	Enkelt sidband, full bärvåg & H\\
	Enkelt sidband, reducerad bärvåg eller bärvåg av varierande nivå & R\\
	Enkelt sidband, undertryckt bärvåg & J\\
	Sinsemellan oavhängiga sidband & B\\
	Stympat sidband & C\\
	Utsändning där huvudbärvågen är vinkelmodulerad &\\
	Frekvensmodulation & F\\
	Fasmodulation & G\\
	Utsändning vars huvudbärvåg är amplitud- och vinkelmodulerad &\\
	antingen samtidigt eller i viss förutbestämd följd. & D\\
	Utsändning av huvudbärvågen som en sekvens av pulser \emph{not}. &\\
	Omodulerade pulser & P \\
	Amplitudmodulerade pulser & K\\
	Bredd- eller tidmodulerade pulser & L\\
	Faslägesmodulerade pulser & M\\
	Vinkelmodulerad bärvåg under pulsens varaktighet & Q\\
	Kombination av ovanstående eller alstrat på annat sätt & V\\
	Övriga fall där utsändningens huvudbärvåg är modulerad, &\\
	antingen samtidigt eller i förutbestämd följd på två eller &\\
	flera av sätten amplitud-, vinkel- eller pulsmodulering & W\\
	Övriga fall & X\\
\end{tabular}

\textbf{Not: Utsändning där huvudbärvågen är direkt modulerad av en signal,
  vilken är kodad i kvantiserad form (t.ex. pulskodmodulation) ska hänföras till
  amplitud eller vinkelmodulation.}

\subsubsection{Den modulerande signalens karaktär}
\textbf{Andra tecknet} -- Den modulerande signalens karaktär\\
\begin{tabular}{ll}
	Ingen modulerande signal & 0\\
	En enda kanal med kvantiserad eller digital information, &\\
	utan användning av modulerande underbärvåg & 1\\
	En enda kanal med kvantiserad eller digital information, &\\
	med användning av modulerande underbärvåg & 2\\
	En enda kanal med analog information & 3\\
	Två eller flera kanaler med kvantiserad eller digital information & 7\\
	Två eller flera kanaler med analog information & 8\\
	Sammansatta system av en eller flera kanaler med kvantiserad eller & \\
	digital information samt en eller flera kanaler med analog information & 9\\
	Övriga fall & X\\
\end{tabular}

I fråga om bassignalens karaktär skiljer man å ena sidan på kanaler för
kvantiserad eller digital information, dvs. där signalen växlar språngvis
mellan vissa givna tillstånd, och på kanaler för analog information, där
signalen kan variera kontinuerligt inom givna gränser.

\emph{Att fastställa arten av huvudbärvågens modulation kan kräva viss
  eftertanke.
  I många fall får den information som ska överföras, modulera en underbärvåg,
  som i sin tur påtrycks modulatorn för huvudbärvågen.}

\subsubsection{Informationens form}
\textbf{Tredje tecknet} -- Informationens form\\
\begin{tabular}{ll}
	Ingen överförd information & N\\
	Telegrafi för hörselmottagning & A\\
	Telegrafi för automatisk mottagning & B\\
	Faksimil & C\\
	Dataöverföring, fjärrmätning, fjärrstyrning & D\\
	Telefoni, även rundradio & E\\
	Television, video & F\\
	Kombination av ovanstående fall & W\\
	Övriga fall & X\\
\end{tabular}

Telegrafisignaler är kvantiserade (till/från, mark/space).
Telefonisignaler har mestadels varit analoga, men är allt oftare
kvantiserade (digitala).
Faksimilsignaler är analoga eller kvantiserade, beroende på om gråtoner
överförs eller ej.


\subsection{Tilläggstecken}
De två avslutande tecknen är tilläggstecken som ger en mer komplett
beskrivning av signalen.
Om ingen komplettering behövs ska tecknen ersättas med två streck ( - - )

\subsubsection{Närmare beskrivning av signalen}
\textbf{Första tilläggstecknet}- Närmare beskrivning av signalen.\\
\begin{tabular}{ll}
	Tvåtillståndskod med element av: &\\
	Olika antal och/eller olika varaktighet- morsetelegrafi & A\\
	Samma antal och varaktighet, utan felkorrigering -- fjärrskrift & B\\
	Samma antal och varaktighet, med felkorrigering- fjärrskrift, &\\
	AMTOR, paketradio m.m. & C\\
	Fyratillståndskod där: &\\
	Varje tillstånd företräder ett tillstånd om ett antal bitar & D\\
	Flertillståndskod där: &\\
	Varje tillstånd företräder ett signalelement om ett antal bitar & E\\
	Varje tillstånd eller kombination av tillstånd företräder ett tecken & F\\
	Ljud av rundradiokvalitet: & \\
	Monatoniskt & G\\
	stereofoniskt eller kvadrafoniskt & H\\
	Ljud av kommersiell kvalitet: &\\
	Alla fall utom K och L enligt nedan & J\\
	Med användning av frekvensinversion eller banduppdelning & K\\
	Med särskilda frekvensmodulerade signaler för styrning av &\\
	den demodulerade signalens nivå & L\\
	Video &\\
	Monokrom & M\\
	Färg & N\\
	Kombination av ovanstående fall & W\\
	Övriga fall & X\\
\end{tabular}

\subsubsection{Arten av multiplex}
\textbf{Andra tilläggstecknet} -- Arten av multiplex.\\
\begin{tabular}{ll}
	Ingen multiplex & N\\
	Koddelning & C\\
	Frekvensdelning & F\\
	Tiddelning & T\\
	Kombination av Frekvens- och Tidsdelning & W\\
	Andra arter av multiplex & X\\
\end{tabular}


\subsection{Exempel på beskrivningskod}
\begin{tabular}{ll}
	N0N & Omodulerad bärvåg, ingen överförd information.\\
	100H A1A AN & Morsetelegrafi genom nyckling av bärvåg,\\
	& 125-takt, bandbredd 100 Hz, s.k. CW.\\
	16K0 F2A AN & Morsetelegrafi, frekvensmodulation med nyckling av ton,\\
	& t.ex. i repeater, s.k. tontelegrafi.\\
	254H F1B BN & Fjärrskrift genom frekvensskiftnyckling av bärvåg (FSK),\\
	& utan felkorrigering, hastighet 50 Bd, frekvensskift i 70Hz, s.k. RTTY.\\
	254H J2B BN & Fjärrskrift genom frekvensskiftnyckling av modulerande\\
	& tonpar (AFSK), vid sändning av enkelt sidband med undertryckt bärvåg,\\
	& bandbredden beroende av hastighet och frekvenser i tonparet\\
	& Jfr 254H F1B BN\\
	304H F1B CN & Fjärrskrift genom frekvensskiftnyckling av bärvåg (FSK),\\
	& med felkorrigering, hastighet 100 Bd, frekvensskift 170Hz, t.ex. AMTOR.\\
	& Jfr 254H F1B BN.\\
	6K00 A3E JN & Telefoni, amplitudmodulation med dubbla sidband och full\\
	& bärvåg, bandbredd 6kHz, s.k. AM.\\
	2K70 J3E JN & Telefoni, enkelt sidband och undertryckt bärvåg,\\
	& bandbredd 2,7 kHz, s.k. SSB.\\
	16K0 F3E JN & Telefoni, frekvensmodulation, bandbredd 16 kHz,\\
	& s.k. NBFM (smalbands-FM).\\
	2K12 F3C MN & Faksimil med halvtoner (telefoto), kooperationsindex 264,\\
	& avsökningshastighet 90 linjer/minut,\\
	& frekvensmodulering med \(\pm\) 400 Hz skift.\\
	6M25 C3F MN & Television i svartvitt enligt det europeiska 625-linjerssystemet.\\
	3K00 F3F MN & Smalbandstelevision enligt amatörradiostandard, s.k. ATV.\\
\end{tabular}

Exempel på sändningsslag utan ITU beskrivningskod enligt ovan
- Telefoni, amplitudmodulation med dubbla sidband och reducerad bärvåg.
En enda kanal med analog information.

Sändningsslaget tillämpas i effektbesparande syfte bl.a. för rundradiosändningar
på AM, varvid traditionella rundradiomottagare fortfarande kan användas.
