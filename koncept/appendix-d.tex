\chapter{S-ENHETER OCH dB}
\label{s-enhet}

I kommunikationsradiomottagare brukar det nästan alltid finnas en
anordning som mäter och visar styrkan av mottagna signaler.

Eftersom spänningen från antennen in i mottagaren kan variera mycket,
är det praktiskt att uttrycka styrkevärdena i en logaritmisk måttenhet,
s.k. S-enhet.

Signalspänningen mäts över en impedans av 50 Ω.

Eftersom S-enheten är logaritmisk, så motsvarar t. ex. signalstyrkan
S8 halva signalspänningen, d.v.s. 25 µV eller -6 dB jämfört med S9. Om
halveringen fortsätts, fås att S0 (noll) motsvarar en kvarvarande
signalstyrka av 0.1 µV.

I en kortvågsmottagare alstras det ett internt brus med en nivå av
åtminstone 0.1 µV.  Detta brus blandas med den inkommande signalen. En
insignal med en styrka under under brusnivån kommer alltså inte att
kunna höras, alltså S0. Vid högre signalstyrkor än S9 anges styrkan
som S9 +ett antal dB. Det är då frågan om mycket starka signaler.

Följande tabell gäller för det ideala sambandet mellan S-enheter och
signalstyrkor över två alternativa brusnivåer.

Signalstyrkan mäts vid mottagarens antenningång, varför skillnaden i
signalstyrkan olika antenner och mottagningsriktningar samt dämpningen
i antenn och nedledning kan behöva bedömas.

I kortvågsområdet (under 30 MHz) uppträder ett atmosfäriskt
bredbandigt brus tillsammans med bruset från den stora mängden
rundradio- m.fl. andra starka sändare.  Detta brus är mer dominerande
än mottagarens interna brus. I praktiken har de flesta KV-mottagare en
högre brusnivå än 0.1 µV.

Över 30 MHz däremot, är det mest mottagarens interna brus som sätter
gränsen för hörbarheten av svaga signaler. Med samma S-skala som för
kortvågsområdet, börjar man uppfatta signaler i bruset utan att
S-metern ger utslag.

Vid IARU Region i-konferensen 1978 i Miskolcz föreslog de nationella
föreningarna VERON (Nederländerna) och RSGB (Storbritannien) en annan
S-skala över 30 MHz.  Vid konferensen 1981 i Brighton antogs förslaget
som rekommendation.

Mätningar ska i båda fallen göras med en kvasi-toppvärdesdetektor
med en stigtid av 10 ms \(\pm\)0.2 ms och en falltid av 500 ms.

\begin{table}[h]
  \begin{tabular}{l|lll|lll}
    S-Meter  & \multicolumn{3}{c}{Under 30 MHz} & \multicolumn{3}{c}{Över 30 MHz} \\
    värde    & dBm & (U vid 50 Ω) & dBµV & dBm & (U vid 50 Ω) & dBµV \\
    \hline
    S9+ 40 dB & -33  & 5.0 mV  & 74  & -53  & 500  & 54  \\
    S9+ 30 dB & -43  & 1.6 mV  & 64  & -63  & 160  & 44  \\
    S9+ 20 dB & -53  & 500 µV  & 54  & -73  & 50   & 34  \\
    S9+ 10 dB & -63  & 160 µV  & 44  & -83  & 16   & 24  \\
    S9        & -73  & 50 µV   & 34  & -93  & 5    & 14  \\
    S8        & -79  & 25 µV   & 28  & -99  & 2.5  & 8   \\
    S7        & -85  & 12.6 µV & 22  & -105 & 1.26 & +2  \\
    S6        & -91  & 6.3 µV  & 16  & -111 & 0.63 & -4  \\
    S5        & -97  & 3.2 µV  & 10  & -117 & 0.32 & -10 \\
    S4        & -103 & 1.6 µV  & +4  & -123 & 0.16 & -16 \\
    S3        & -109 & 0.8 µV  & -2  & -129 & 0.08 & -22 \\
    S2        & -115 & 0.4 µV  & -8  & -135 & 0.04 & -28 \\
    S1        & -121 & 0.21 µV & -14 & -141 & 0.02 & -34 \\
  \end{tabular}
  \caption{S-enheter, rekommenderade normvärden inom IARU Region 1}
  \label{s-enhet tabell}
\end{table}

