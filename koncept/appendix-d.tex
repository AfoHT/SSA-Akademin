\chapter{S-enheter och dB}
\label{s-enhet}

I kommunikationsradiomottagare brukar det nästan alltid finnas en anordning som
mäter och visar styrkan av mottagna signaler.

Eftersom spänningen från antennen in i mottagaren kan variera över ett stort
område, är det praktiskt att uttrycka styrkevärdena med en logaritmisk
måttenhet, så kallad S-enhet.

Signalspänningen mäts över en impedans av 50~\(\Omega\).

Eftersom S-enheten är logaritmisk, så motsvarar till exempel signalstyrkan S8
halva signalspänningen, det vill säga 25~\(\mu V\) eller -6~dB jämfört med S9.
Om halveringen fortsätts, fås att S0 (noll) motsvarar en signalstyrka av 0,1~\(\mu V\).

I en kortvågsmottagare alstras det ett internt brus med en nivå av
åtminstone 0,1~\(\mu V\).
Detta brus blandas med den inkommande signalen.
En insignal med en styrka under brusnivån (S0) kommer alltså inte att
kunna höras.
Vid högre signalstyrkor än S9 anges styrkan som S9 +ett antal dB.
Det är då frågan om mycket starka signaler.

Följande tabell gäller för det ideala sambandet mellan S-enheter och
signalstyrkor över två alternativa brusnivåer.

\newpage

Signalstyrkan mäts vid mottagarens antenningång, varför skillnaden i
signalstyrkan olika antenner och mottagningsriktningar samt dämpningen
i antenn och nedledning kan behöva bedömas.

I kortvågsområdet (under 30~MHz) uppträder ett atmosfäriskt bredbandigt brus
tillsammans med bruset från den stora mängden rundradio- m.fl. andra starka
sändare.
Detta brus är mer dominerande än mottagarens interna brus.
I praktiken har de flesta KV-mottagare en högre brusnivå än 0,1~\(\mu V\).

Över 30~MHz däremot, är det mest mottagarens interna brus som sätter
gränsen för hörbarheten av svaga signaler.
Med samma S-skala som för kortvågsområdet, börjar man uppfatta signaler i
bruset utan att S-metern ger utslag.

Vid IARU Region~1-konferensen 1978 i Miskolcz föreslog de nationella
föreningarna VERON (Nederländerna) och RSGB (Storbritannien) en annan
S-skala över 30~MHz.
Vid konferensen 1981 i Brighton antogs förslaget som rekommendation.

Mätningar ska i båda fallen göras med en kvasi-toppvärdesdetektor
med en stigtid av 10~ms \(\pm\)0,2~ms och en falltid av 500~ms.

\begin{table*}[ht]
  \begin{center}
  \begin{tabular}{l|lll|lll}
    S-Meter  & \multicolumn{3}{c}{Under 30~MHz} & \multicolumn{3}{c}{Över 30~MHz} \\
    värde    & dBm & (U vid 50 \(\Omega\)) & dB\(\mu V\) & dBm & (U vid 50 \(\Omega\)) & dB\(\mu V\) \\
    \hline
    S9+ 40 dB & -33  & 5,0 mV  & 74  & -53  & 500 \(\mu V\) & 54  \\
    S9+ 30 dB & -43  & 1,6 mV  & 64  & -63  & 160 \(\mu V\) & 44  \\
    S9+ 20 dB & -53  & 500 \(\mu V\)  & 54  & -73  & 50 \(\mu V\) & 34  \\
    S9+ 10 dB & -63  & 160 \(\mu V\)  & 44  & -83  & 16 \(\mu V\) & 24  \\
    S9        & -73  & 50 \(\mu V\)   & 34  & -93  & 5 \(\mu V\) & 14  \\
    S8        & -79  & 25 \(\mu V\)   & 28  & -99  & 2,5 \(\mu V\) & 8   \\
    S7        & -85  & 12,6 \(\mu V\) & 22  & -105 & 1,26 \(\mu V\) & +2  \\
    S6        & -91  & 6,3 \(\mu V\)  & 16  & -111 & 0,63 \(\mu V\) & -4  \\
    S5        & -97  & 3,2 \(\mu V\)  & 10  & -117 & 0,32 \(\mu V\) & -10 \\
    S4        & -103 & 1,6 \(\mu V\)  & +4  & -123 & 0,16 \(\mu V\) & -16 \\
    S3        & -109 & 0,8 \(\mu V\)  & -2  & -129 & 0,08 \(\mu V\) & -22 \\
    S2        & -115 & 0,4 \(\mu V\)  & -8  & -135 & 0,04 \(\mu V\) & -28 \\
    S1        & -121 & 0,21 \(\mu V\) & -14 & -141 & 0,02 \(\mu V\) & -34 \\
  \end{tabular}
  \caption{S-enheter, rekommenderade normvärden inom IARU Region~1}
  \label{s-enhet tabell}
  \end{center}
\end{table*}
