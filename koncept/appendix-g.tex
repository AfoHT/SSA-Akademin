\section{VHF/UHF/SHF/EHF bandplan}

\hilight{TODO: F.2 Behöver fixas.}

Sammanfattad av SM7GVF Kjell

Denna bandplan reviderades vid IARU Region 1-konferensen i Tel-Aviv 1996.
\hilight{TODO: Referens till senare bandplan}
Den vänstra delen är själva bandplanen, medan den högra delen rekommenderar användning/mötespunkter.
(PTS bandplan och status för amatörradio i Sverige, framgår av Kapitel !111.6 samt Appendix G och H.)
\hilight{TODO: Fixa referenserna}

\hilight{TODO: Här används en annan tabell-form än i F.1, unifiera}

\setlongtables
\begin{longtable}{lll}
\caption{50 MHz Användning: Experimentband, rundradio primär, landmobil radio tillåten} \\
Segment & Trafiksätt & Rekommenderad användning \\
kHz     &            & \\ \hline
\endhead
50,000 - 50,100 & CW & \\
50,020 - 50,080 & & Fyrar \\
50,090          & & CW aktivitetscenter \\

50,100 - 50,500 & Alla smalbandsmoder & \\
                & (CW, SSB, AM, RTTY, & \\
                & SSTV, ETC)          & \\
                & Smalband = 6 kHz    & \\
50,100 - 50,130 &  & SSB/CW internationellt (interkontinentalt) \\
50,110          &  & DX anropsfrekvens interkontinentalt \\
50,150          &  & SSB aktivitetscenter \\
50,185          &  & Aktivitetscenter för crossband \\
50,200          &  & MS aktivitetscenter \\

50,500 - 50,190 & Alla moder & \\
50,510          & & SSTV (AFSK) \\
50,550          & & FAX arbetsfrekvens \\
50,600          & & RTTY (FSK) \\
50,620 - 50,750 & & Digital kommunikation \\
\end{longtable}

\hilight{TODO: 50,500-50,190 har fel bakre gräns, hitta rätt}

51,210

51,390
51,410

l

51,590
51,810

l

51,990
52,000



RF81
NBFM repeater infrekvenser, 20~kHz kanaldelning, i 0~kHz kanalbredd, varannan kanal används
RF99
F41
NBFM, simplex
51,510
NBFM anropsfrekvens
F59
RF81
NBFM repeater utfrekvenser, 20~kHz kanaldelning, 10~kHz kanalbredd, varannan kanal används
RF99

ANMÄRKNINGAR
Sändningsslag

Telegrafi är tillåtet över hela bandet, men är exklusivt i området 50,000--50,100~MHz

Anropsfrekvenser

50,110~MHz är interkontinental DX anropsfrekvens och bör inte användas för trafik inom Europa.

Kanaltrafik

För kanaltrafik är kanaldelningen 20~kHz, förskjutet 10~kHz.

Tillståndskrav i Sverige

För amatörradiotrafik i bandet 50--52~MHz, krävs i Sverige utöver amatörradiotillståndet ett särskilt tillstånd !

F-3

APPENDIX F2
144 MHz

Användning: Amatörradio primär

144,000

CW(a)
144,150
SSB

l

144,400

144,000- 144,035
144,050
144, i 00
144,140 - i 44,150
i 44,150 - 144,160
144,195 - i 44,205
144,300
144,390 - 144,400

EME (månstuds) exklusiv användning
CW anropsfrekvens
CW MS referensfrekvens, random
CW, FAl (Field Aligned lrregularities)
SSB, FAl (Field Aligned lrregularities)
SSB MS (Meteorscatter), Random
SSB anropsfrekvens
SSB MS (Meteorscatter), Random

144,490
144,500
144,525
144,600
144,700
144,750

SAREX uplink, temporär
SSTV anropsfrekvens
ATV SSB talk back center
RTTY anropsfrekvens
FAX anropsfrekvens
ATV anropsfrekvens

Fyrar (b)

l

144,490
144,500
Alla moder (c)

144,800

l

144,990
145,000

l

145, i 875
145,200

l

i45,5875
145,600

l

i45,7875
145,800

l

i46,000

Digital kommunikation (d)
RV48
NBFM repeater infrekvenser, 12,5 kHz kanalseparation, 600kHz skift (e)
RV 63
145,200
Bemannad rymdtrafik, upplänk
V16
RTTY lokal
i 2,5 kHz NBFM
i45,300
(Mobil) anropsfrekvens
i45,500
simplex
V47
RV 48
NBFM repeater utfrekvenser, 12,5 kHz kanalseparation, 600 kHz skift
RV63
145,800
Bemannad rymdtrafik, nerlänk
satellitservice

ANMÄRKNINGAR
Generella
i) I Europa skall inga in- eller utfrekvenser för NBFM repeatrar förekomma inom segmentet 144 - 145 MHz.
ii) Med undantag för satellitsegmentet tillåts inte in- eller utfrekvenser i 2-metersbandet för repeatrar i andra band.
iii) Inga nya nät för packet radio skall sättas upp i 2-metersbandet.
Access från nät i 2-metersbandet till nät i andra band skall inte förekomma. Emellertid får detta förekomma
under begränsad tid i delar av Region 1 för att där introducera packet radio. De delar av regionen som avses är
där amatörtätheten är låg och/eller i regionens utkanter där sådan access inte påverkar trafiken i de delar av
regionen där stort tryck på tillgång till spektrum motiverar att bandplanen följs metodiskt. Denna andra del av
fotnoten skall aldrig användas för att legitimera att första stycket ignoreras för avsevärd tid.
iv) Fyrar skall oavsett ERP ligga i fyrbandet
Särskilda
(a) Telegrafi är tillåtet över hela bandet, exklusivt i segmentet 144,035- 144,150 MHz.
(b) Fyrar med ERP över 50 W koordineras av IARU Region 1 fyrkoordinator. Förfyrar med 1OW eller mer skall denne
meddelas. Under begränsad tid- inte längre än att noviser i Nederländerna har detta segment tillgängligt- är
även SSB och CW tillåtet i detta segment.
(c) Inga obemannade stationer skall användas i all mode-segmentet.

F-4

F2

APPENDIX

(d) stationer i nätverk för digital trafik skall använda den digitala delen av bandet och tillåtas för en begränsad tid.
Dessa bör ha access till portar på andra VHF-, UHF- eller mikrovågsband och bör inte använda 2-metersbandet
för forward-trafik till andra nätverksstationer. Nya nätverksstationer uppmuntras inte. Obemannade stationer
tillåts endast i segmentet 144,800 - 144,990 MHz. Utanför detta segment skall sidband inte överstiga -60dB i 12
kHz band bredd. Hänsyn skall tas till den använda bandbredde n, så att sidband ej faller utanför segmentet. För
stationer med 12,5 kHz bandbredd betyder det att kanalerna 144,8125 - 144,975 MHz då kan användas.
(e) En övergång till genuint i 2,5 kHz kanalsystem uppmuntras.
144,140 - 144,160 MHz utgör ett alternativt segment för EME.

432 MHz
432,000

l

432,150

l

432,500

l

432,600

Användning: Amatörradio och radiolokalisering delat primär
CW (a)

432,000 - 432,025
432,050

Månstuds
CW aktivitetscenter

SSB/CW

432,200
432,350

SSB aktivitetscenter
Mikrovågor "talk-back" center

Linjära transpondrar, in

432,500
432,600
432,700
432,700- 432,775

SSTV (smalband)
RTTY (FSK/PSK)
FAX (FSK)
Digital kommunikation, ej mer än
25 kHz kanalseparation

Linjära transpondrar, ut

l

432,800

l

432,990
433,000

l

433,3875
433,400

l

433,5875
433,600

Fyrar (b)
RU368
NBFM repeater infrekvenser, 12,5 kHz kanalseparation, 1,6 MHz skift
RU399
U272
Simplex, i 2.5 kHz
SSTV (FM/AFSK)
433,400
433,500
(Mobil) FM anropskanal
U287

Alla moder

434,575
434,600

433,600
433,625 - 433,775
433,700
434,450 - 434,575

RTTY (FM)
Digital kommunikation
FAX (FM/AFSK)
Digital kommunikation, ej mer än
25 kHz kanalseparation

l

RU 368
NBFM repeater utfrekvenser, 12,5 kHz kanalseparation, i ,6 MHz skift
RU 399

l

satellitservice

434,9875
435,000
438,000

Förslag till packet duplex frekvenser: 432,700/434,500 - 432,775/434,575 MHz.
ANMÄRKNINGAR

Generella

i) I Europa skall inga in- eller utfrekvenser för NBFM repeatrar förekomma inom segmentet 432 - 433 MHz.
ii) Fyrar skall oavsett ERP placeras i fyrbandet

Särskilda

(a) Telegrafi är tillåtet över hela bandet, exklusivt i segmentet 144,035- 144,150 MHz.
(b) Fyrar med ERP över 50 W koordineras av IARU Region 1 fyrkoordinator.

F-5

APPENDIX
1296 MH:z
1240.000

Användning: Amatörradio sekundär
Alla moder

1240,000-1241,000
1242,025- 1242,700
i 242,725- 1243,250

Digital kommunikation
Repeater ut, RS1 - RS28
Packet duplex, RS29 - RS50

l

Amatörtelevision

1258, i 50- i 259,350

Repeater ut, R20 - R68

l

satellitservice
1270,025- 1270,700
1270,725- 1271,250

Repeater in, RSI - RS28
Packet duplex, RS29 - RS50

l

1243,250
1260,000
1270,000

l

1272,000

l

12l000
1291,475
1291,500

Alla moder

Amatörtelevision
RMO
(används i Sverige)
NBFM repeater infrekvenser
25 kHz kanalseparation, 6 MHz skift
RM19

l

Alla moder

i 293,150 - i 294,350

Repeater in, R20 - R68

l

CW(a)

i 296,000 - 1296,025

Månstuds

1296,200
1296,400 - 1296,600
1296,500
1296,600
i 296,600 - i 296,800
1296,700

Smalbands aktivitetscenter
Linjär transponder infrekvens
SSTV
RTTY
Linjär transponder utfrekvens
FAX

1296,000
1296,150

SSB

1296,800

l

1296,990
1297,000

l

1297,475
1297,500

l

1297,975
1298,000

1300,000

Fyrar

RMO
(används i Sverige)
NBFM repeater utfrekvenser,
25 kHz kanalseparation, 6 MHz skift
RM19
SM20
NBFM simplex kanaler,
25 kHz kanalseparation,
SM39

1297,500

FM aktivitetscenter

Alla moder

1298,025- 1298,700
i 298,500 - i 300,000
1298,725- 1299,000

Repeater ut, RS1 - RS28
Digital kommunikation
Packet duplex, RS29 - RS40

ANMÄRKNINGAR
Särskilda
(a) Telegrafi är tillåtet över hela smalbandsegmentet, exklusivt i segmentet 1296,000 - i 296, i 50 MHz.
(b) Fyrar med ERP över 50 W koordineras av IARU Region 1 fyrkoordinator.

F-6

APPENDIX F2
2300 Mhz
2300,000

l

Användning: Amatörradio sekundär
Subregional planering

2320,000

l
l

CW

2320,000- 2320,025

Månstuds

CW/SSB

2320,200

SSB aktivitetscenter

2320,150
2320,800

l

Fyrar

2320,990
2321,000

Simplex och repeater, NBFM

2322,000
Alla moder

2400,000

l

2322,000- 2355,000
2355,000- 2365,000
2365,000- 2370,000
2370,000- 2392,000
2392,000 - 2400,000

Amatörtelevision
Digital kommunikation
Repeatrar
Amatörtelevision
Digital kommunikation

satellitservice

2450,000
5650 MH:z
5650,000

l

5670,000
5661'000
567,,000
5700,000

l
l
5760,000
l
5762,000
l
5790,000
l

5720,000

Användning: Amatörradio sekundär
Satellitservice, upplänk
Smalband, CW/SSB/FM

5668,200

Aktivitetscenter

5760,200

Aktivitetscenter

Digital kommunikation
Amatörtelevision
Alla moder
Smalband, CW/SSB/FM
Alla moder
Satellitservice, nerlänk

5850,000

10000 MH:z
10000,000

l

10150,000

l
l
10350,000

10250,000

Användning: Amatörradio sekundär
Digital kommunikation
Alla moder: ATV, data, FM simplex/duplex/repeatrar
Digital kommunikation

l
l

Alla moder

l

Alla moder

l

satellitservice

10368,000
10370,000
10450,000

Smalband CW/SSB/fyrar

10368,200

Aktivitetscenter

10500,000

F-7

PPENDIX

F2

24000 MHz Användning: Amatörradio sekundär
24000,000

l

satellitservice

l

CW/SSB/fyrar

24048,200

Aktivitetscenter, smalbandsmoder

Alla moder

24125,000

Aktivitetscenter,
bredbandiga moder

24048,000
24050,000

l
24250,000

47000 MHz Användning: Amatörradio primär
47000,000

l

47200,000

F-8

47088,000

Aktivitetscenter, smalbandsmoder
