\section{VHF/UHF/SHF/EHF bandplan}

Sammanfattad av SM7GVF Kjell

Denna bandplan baseras på IARU Region 1 2009 \cite{IARU1}.

Den vänstra delen är själva bandplanen, medan den högra delen rekommenderar användning/mötespunkter.
(PTS bandplan och status för amatörradio i Sverige, framgår av Kapitel \ref{bandplaner} samt Bilaga \ref{svenska bandplaner} och \ref{svenska repeatrar}.)

\subsection{50 MHz bandplan}

\setlongtables
\begin{longtable}{llll}
\caption{50 MHz Användning: Experimentband, rundradio primär, landmobil radio tillåten} \\
Segment & Trafiksätt & delband & Rekommenderad användning \\
kHz     &            &         & \\ \hline
\endhead
50,000 & CW &                 & \\
       &    & 50,020 - 50,080 & Fyrar \\
       &    & 50,050          & CW aktivitetscenter \\
       &    & 50,090          & CW aktivitetscenter \\
50,100 &    &                 & \\ \hline

50,100 & Alla smalbandsmoder & & \\
       & (CW, SSB, AM, RTTY, & & \\
       & SSTV, ETC)          & & \\
       & Smalband = 2,7 kHz  & & \\
       &    & 50,100 - 50,130 & SSB/CW internationellt (interkontinentalt) \\
       &    & 50,110          & DX anropsfrekvens interkontinentalt \\
       &    & 50,150          & SSB aktivitetscenter \\
50,500 &    &                 & \\ \hline

50,500 & Alla moder & & \\
       &    & 50,510          & SSTV (AFSK) \\
       &    & 50,550          & FAX arbetsfrekvens \\
       &    & 50,600          & RTTY (FSK) \\
       &    & 50,620 - 50,750 & Digital kommunikation \\
50,190 &    &   & \\ \hline

51,210 & RF81 & & \\
       & \multicolumn{3}{l}{NBFM repeater infrekvenser, 20~kHz kanaldelning, 10~kHz kanalbredd} \\
51,390 & RF99 & & \\ \hline

51,410 & F41 & & \\
       & NBFM, simplex & 51,510 & NBFM anropsfrekvens\\
51,590 & F59 & & \\ \hline

51,810 & RF81 & & \\
       & \multicolumn{3}{l}{NBFM repeater utfrekvenser, 20~kHz kanaldelning, 10~kHz kanalbredd} \\
51,990 & RF99 & & \\
52,000 & & & \\
\end{longtable}

\textbf{ANMÄRKNINGAR}
\textbf{Sändningsslag}

Telegrafi är tillåtet över hela bandet, men är exklusivt i området
50,000--50,100~MHz

\textbf{Anropsfrekvenser}

50,110~MHz är interkontinental DX anropsfrekvens och bör inte användas
för trafik inom Europa.

\textbf{Kanaltrafik}

För kanaltrafik är kanaldelningen 20~kHz, förskjutet 10~kHz.

\subsection{144 MHz bandplan}

\setlongtables
\begin{longtable}{llll}
\caption{144 MHz Användning: Amatörradio primär} \\

144,000 & & & \\
        &       & 144,000 - 144,035 & EME (månstuds) exklusiv användning \\
        &       & 144,050           & CW anropsfrekvens \\
        & CW(a) & 144,100           & CW MS referensfrekvens, random \\
        &       & 144,140 - 144,150 & CW, FAl (Field Aligned lrregularities) \\
144,150 &       & 144,150 - 144,160 & SSB, FAl (Field Aligned lrregularities) \\
        &       & 144,195 - 144,205 & SSB MS (Meteorscatter), Random \\
        & SSB   & 144,300           & SSB anropsfrekvens \\
144,400 &       & 144,390 - 144,400 & SSB MS (Meteorscatter), Random \\
        & Fyrar (b) & & \\
144,490 &       & 144,490 & SAREX uplink, temporär \\
144,500 &       & 144,500 & SSTV anropsfrekvens \\
        &       & 144,525 & ATV SSB talk back center \\
        & Alla moder (c) & 144,600 & RTTY anropsfrekvens \\
        &       & 144,700 & FAX anropsfrekvens \\
        &       & 144,750 & ATV anropsfrekvens \\
144,800 & & & \\
        & \multicolumn{3}{l}{Digital kommunikation (d)} \\
144,990 & & & \\
145,000 & RV48 & & \\
        & \multicolumn{3}{l}{NBFM repeater infrekvenser, 12,5 kHz kanalseparation, 600kHz skift (e)} \\
145,1875 & RV63 & & \\
145,200 & V16   & 145,200 & Bemannad rymdtrafik, upplänk \\
        & 12,5 kHz NBFM & 145,300 & RTTY lokal \\
        & simplex & 145,500 & (Mobil) anropsfrekvens \\
145,5875 & V47 & & \\
145,600 & RV48 & & \\
        & \multicolumn{3}{l}{NBFM repeater utfrekvenser, 12,5 kHz kanalseparation, 600 kHz skift} \\
145,7875 & RV63 & 145,800 & Bemannad rymdtrafik, nerlänk \\
145,800 & & & \\
        & \multicolumn{3}{l}{Satellitservice} \\
146,000 & & & \\
\end{longtable}

\textbf{ANMÄRKNINGAR}

\textbf{Generella}
\begin{enumerate}[label=\alph*.]
\item I Europa skall inga in- eller utfrekvenser för NBFM repeatrar
  förekomma inom segmentet 144 - 145 MHz.
\item Med undantag för satellitsegmentet tillåts inte in- eller
  utfrekvenser i 2-metersbandet för repeatrar i andra band.
\item Inga nya nät för packet radio skall sättas upp i 2-metersbandet.
  Access från nät i 2-metersbandet till nät i andra band skall inte förekomma.
  Emellertid får detta förekomma under begränsad tid i delar av Region 1
  för att där introducera packet radio.
  De delar av regionen som avses är där amatörtätheten är låg och/eller
  i regionens utkanter där sådan access inte påverkar trafiken i de delar
  av regionen där stort tryck på tillgång till spektrum motiverar att
  bandplanen följs metodiskt.
  Denna andra del av fotnoten skall aldrig användas för att legitimera
  att första stycket ignoreras för avsevärd tid.
\item Fyrar skall oavsett ERP ligga i fyrbandet
\end{enumerate}

\textbf{Särskilda}
\begin{itemize}

\item[(a)] Telegrafi är tillåtet över hela bandet, exklusivt i segmentet
  144,035 - 144,150 MHz.

\item[(b)] Fyrar med ERP över 50 W koordineras av IARU Region 1 fyrkoordinator.
  Förfyrar med 1OW eller mer skall denne meddelas. Under begränsad tid- inte
  längre än att noviser i Nederländerna har detta segment tillgängligt- är
  även SSB och CW tillåtet i detta segment.

\item[(c)] Inga obemannade stationer skall användas i all mode-segmentet.

\item[(d)] Stationer i nätverk för digital trafik skall använda den
digitala delen av bandet och tillåtas för en begränsad tid.
Dessa bör ha access till portar på andra VHF-, UHF- eller mikrovågsband och bör
inte använda 2-metersbandet för forward-trafik till andra nätverksstationer.
Nya nätverksstationer uppmuntras inte.
Obemannade stationer tillåts endast i segmentet 144,800 - 144,990~MHz.
Utanför detta segment skall sidband inte överstiga -60~dB i 12~kHz band bredd.
Hänsyn skall tas till den använda bandbredden, så att sidband ej faller
utanför segmentet.
För stationer med 12,5~kHz bandbredd betyder det att kanalerna
144,8125 - 144,975 MHz då kan användas.

\item[(e)] En övergång till genuint i 2,5 kHz kanalsystem uppmuntras.
144,140 - 144,160 MHz utgör ett alternativt segment för EME.
\end{itemize}

\subsection{432 MHz bandplan}

\setlongtables
\begin{longtable}{llll}
\caption{432~MHz Användning: Amatörradio och radiolokalisering delat primär} \\
432,000 &        & & \\
        & CW (a) & 432,000 - 432,025 & Månstuds \\
        &        & 432,050           & CW aktivitetscenter \\
432,150 &        & & \\
        & SSB/CW & 432,200           & SSB aktivitetscenter \\
        &        & 432,350           & Mikrovågor "talk-back" center \\
432,500 & Linjära transpondrar, in & 432,500 & SSTV (smalband) \\
432,600 &                          & 432,600 & RTTY (FSK/PSK) \\
        & Linjära transpondrar, ut & 432,700 & FAX (FSK) \\
        &                          & 432,700- 432,775 & Digital kommunikation, ej mer än \\
        &                          &         & 25 kHz kanalseparation \\
432,800 & & & \\
        & \multicolumn{3}{l}{Fyrar (b)} \\
432,990 & & & \\
433,000 & RU368 & & \\
        & \multicolumn{3}{l}{NBFM repeater infrekvenser, 12,5 kHz kanalseparation, 1,6 MHz skift} \\
433,3875 & RU399 & & \\
433,400 & U272 & & \\
        & Simplex, i 2.5 kHz & 433,400 & SSTV (FM/AFSK) \\
        &                    & 433,500 & (Mobil) FM anropskanal \\
433,5875 & U287 & & \\
433,600 &            & 433,600           & RTTY (FM) \\
        &            & 433,625 - 433,775 & Digital kommunikation \\
        & Alla moder & 433,700           & FAX (FM/AFSK) \\
        &            & 434,450 - 434,575 & Digital kommunikation, ej mer än \\
434,575 &            &                   & 25 kHz kanalseparation \\
434,600 & RU 368 & & \\
        & \multicolumn{3}{l}{NBFM repeater utfrekvenser, 12,5 kHz kanalseparation, 1,6 MHz skift} \\
434,9875 & RU 399 & & \\
435,000 & & & \\
        & \multicolumn{3}{l}{satellitservice} \\
438,000 & & & \\
\end{longtable}

Förslag till packet duplex frekvenser: 432,700/434,500 - 432,775/434,575 MHz.

\textbf{ANMÄRKNINGAR}

\textbf{Generella}

\begin{enumerate}[label=\alph*.]
\item I Europa skall inga in- eller utfrekvenser för NBFM repeatrar
  förekomma inom segmentet 432 - 433 MHz.
\item Fyrar skall oavsett ERP placeras i fyrbandet.
\end{enumerate}

\textbf{Särskilda}

\begin{itemize}
\item[(a)] Telegrafi är tillåtet över hela bandet, exklusivt i segmentet
  144,035 - 144,150~MHz.
\item[(b)] Fyrar med ERP över 50~W koordineras av IARU Region 1 fyrkoordinator.
\end{itemize}

\subsection{1296 MHz bandplan}

\setlongtables
\begin{longtable}{llll}
\caption{1296 MHz Användning: Amatörradio sekundär} \\

1240.000 &            & 1240,000 - 1241,000 & Digital kommunikation \\
         & Alla moder & 1242,025 - 1242,700 & Repeater ut, RS1 - RS28 \\
         &            & 1242,725 - 1243,250 & Packet duplex, RS29 - RS50 \\
1243,250 & & & \\
         & Amatörtelevision & 1258,150 - 1259,350 & Repeater ut, R20 - R68 \\
1260,000 & & & \\
         & satellitservice & & \\
1270,000 & & & \\
         & Alla moder & 1270,025 - 1270,700 & Repeater in, RSI - RS28 \\
         &            & 1270,725 - 1271,250 & Packet duplex, RS29 - RS50 \\
1272,000 & & & \\
         & Amatörtelevision & & \\
1291,000 & RMO        & (används i Sverige) & \\
         & \multicolumn{3}{l}{NBFM repeater infrekvenser 25 kHz kanalseparation, 6 MHz skift} \\
1291,475 & RM19 & & \\
1291,500 & & & \\
         & Alla moder & 1293,150 - 1294,350 & Repeater in, R20 - R68 \\
1296,000 & & & \\
         & CW(a) & 1296,000 - 1296,025 & Månstuds \\
1296,150 & & & \\
         &     & 1296,200            & Smalbands aktivitetscenter \\
         &     & 1296,400 - 1296,600 & Linjär transponder infrekvens \\
         & SSB & 1296,500            & SSTV \\
         &     & 1296,600            & RTTY \\
         &     & 1296,600 - 1296,800 & Linjär transponder utfrekvens \\
         &     & 1296,700            & FAX \\
1296,800 & & & \\
         & Fyrar & & \\
1296,990 & & & \\
1297,000 & RMO & (används i Sverige) & \\
         & \multicolumn{3}{l}{NBFM repeater utfrekvenser, 25 kHz kanalseparation, 6 MHz skift} \\
1297,475 & RM19 & & \\
1297,500 & SM20 & 1297,500 & FM aktivitetscenter \\
         & \multicolumn{3}{l}{NBFM simplex kanaler, 25 kHz kanalseparation,} \\
1297,975 & SM39 & & \\
1298,000 & & & \\
         &            & 1298,025 - 1298,700 & Repeater ut, RS1 - RS28 \\
         & Alla moder & 1298,500 - 1300,000 & Digital kommunikation \\
         &            & 1298,725 - 1299,000 & Packet duplex, RS29 - RS40 \\
1300,000 & & & \\
\end{longtable}

\textbf{ANMÄRKNINGAR}

\textbf{Särskilda}

\begin{itemize}
\item[(a)] Telegrafi är tillåtet över hela smalbandsegmentet, exklusivt
i segmentet 1296,000 - 1296,150 MHz.
\item[(b)] Fyrar med ERP över 50 W koordineras av IARU Region 1 fyrkoordinator.
\end{itemize}

\subsection{2300 MHz bandplan}

\setlongtables
\begin{longtable}{llll}
\caption{2300 Mhz Användning: Amatörradio sekundär} \\

2300,000 & & & \\
         & \multicolumn{3}{l}{Subregional planering} \\
2320,000 & & & \\
         & CW     & 2320,000 - 2320,025 & Månstuds \\
2320,150 & & & \\
         & CW/SSB & 2320,200            & SSB aktivitetscenter \\
2320,800 & & & \\
         & Fyrar & & \\
2320,990 & & & \\
2321,000 & & & \\
         & \multicolumn{3}{l}{Simplex och repeater, NBFM} \\
2322,000 & & & \\
         &            & 2322,000 - 2355,000 & Amatörtelevision \\
         &            & 2355,000 - 2365,000 & Digital kommunikation \\
         & Alla moder & 2365,000 - 2370,000 & Repeatrar \\
         &            & 2370,000 - 2392,000 & Amatörtelevision \\
         &            & 2392,000 - 2400,000 & Digital kommunikation \\
2400,000 & & & \\
         & \multicolumn{3}{l}{Satellitservice} \\
2450,000 & & & \\
\end{longtable}

\subsection{5650 MHz bandplan}

\setlongtables
\begin{longtable}{llll}
\caption{5650 MHz Användning: Amatörradio sekundär} \\
5650,000 & & & \\
         & \multicolumn{3}{l}{Satellitservice, upplänk} \\
5670,000 & & & \\
5668,000 & & & \\
         & Smalband, CW/SSB/FM & 5668,200 & Aktivitetscenter \\
5670,000 & & & \\
         & Digital kommunikation & & \\
5700,000 & & & \\
         & Amatörtelevision & & \\
5720,000 & & & \\
         & Alla moder & & \\
5760,000 & & & \\
         & Smalband, CW/SSB/FM & 5760,200 & Aktivitetscenter \\
5762,000 & & & \\
         & Alla moder & & \\
5790,000 & & & \\
         & \multicolumn{3}{l}{Satellitservice, nerlänk} \\
5850,000 & & & \\
\end{longtable}

\subsection{10 GHz bandplan}

\setlongtables
\begin{longtable}{llll}
\caption{10000 MHz Användning: Amatörradio sekundär} \\
10000,000 & & & \\
          & \multicolumn{3}{l}{Digital kommunikation} \\
10150,000 & & & \\
          & \multicolumn{3}{l}{Alla moder: ATV, data, FM simplex/duplex/repeatrar} \\
10250,000 & & & \\
          & \multicolumn{3}{l}{Digital kommunikation} \\
10350,000 & & & \\
          & \multicolumn{3}{l}{Alla moder} \\
10368,000 & & & \\
          & Smalband CW/SSB/fyrar & 10368,200 & Aktivitetscenter \\
10370,000 & & & \\
          & \multicolumn{3}{l}{Alla moder} \\
10450,000 & & & \\
          & \multicolumn{3}{l}{Satellitservice} \\
10500,000 & & & \\
\end{longtable}

\subsection{24 GHz bandplan}

\setlongtables
\begin{longtable}{llll}
\caption{24000 MHz Användning: Amatörradio sekundär} \\
24000,000 & & & \\
          & \multicolumn{3}{l}{Satellitservice} \\
24048,000 & & & \\
          & CW/SSB/fyrar & 24048,200 & Aktivitetscenter, smalbandsmoder \\
24050,000 & & & \\
          & Alla moder   & 24125,000 & Aktivitetscenter, bredbandiga moder \\
24250,000 & & & \\
\end{longtable}

\subsection{47 GHz bandplan}

\setlongtables
\begin{longtable}{llll}
\caption{47000 MHz Användning: Amatörradio primär} \\
47000,000 & & & \\
          & & 47088,000 & Aktivitetscenter, smalbandsmoder \\
47200,000 & & & \\
\end{longtable}
