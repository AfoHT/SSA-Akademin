\chapter{IARU Region 1 bandplan}
\label{IARU bandplan}

\section{HF}
\label{HFbandplan}

Denna bandplan baseras på IARU Region~1 HF band plan 2016 \cite{IARU1}.

Den vänstra delen är själva bandplanen, medan den högra delen rekommenderar
användning/mötespunkter.
(PTS frekvensplan och status för amatörradio i Sverige, framgår av Kapitel
\ref{bandplaner} samt bilaga \ref{svensk frekvensplan} och
\ref{svenska repeatrar}.)

\setlongtables
\begin{longtable}{lcl}
Band & Segment & Trafiksätt \\
MHz  & kHz     & \\ \hline
\endhead

0,137 & 135,7--137,8 & CW, QRSS and narrow band digital modes\\

0,472 & 472--475 & CW\\
 & 475--479 & CW, digimodes\\

1,8  & 1810--1838 & CW, 1836~kHz QRP Centre of Activity\\
     & 1838--1840 & Narrow band modes\\
     & 1840--1843 & All modes--digimodes\\
     & 1843--2000 & All modes\\

3,5  & 3500--3510 & CW, priority for intercontinental operation\\
     & 3510--3560 & CW, contest preferred, 3555~kHz--QRS Centre of Activity\\
     & 3560--3570 & CW, 3560 kHz--QRP Centre of Activity\\
     & 3570--3580 & Narrow band modes--digimodes\\
     & 3580--3590 & Narrow band modes--digimodes\\
     & 3590--3600 & Narrow band modes--digimodes\\
     & & automatically controlled data stations (unattended)\\
     & 3600--3620 & All modes--digimodes\\
     & & automatically controlled data station (unattended\\
     & 3600--3650 & All modes, SSB contest preferred\\
     & & 3630~kHz--Digital Voice Centre of Activity\\
     & 3650--3700 & All modes, 3690~kHz--SSB QRP Centre of Activity\\
     & 3700--3775 & All modes, SSB contest preferred\\
     & & 3735~kHz--Image Centre of Activity\\
     & & 3760~kHz--Reg.1 Emergency Centre of Activity\\
     & 3775--3800 & All modes, SSB contest preferred\\
     & & priority for intercontinental operation\\

5,3  & & \textbf{Är inte ett amatörband i Sverige}\\
 & 535105--5354,0 & CW, Narrow band modes--digimodes\\
 & 5340,0--5366,0 & All modes, USB recommended for voice operation\\
 & 5366,0--5366,5 & Weak signal narrow band modes\\

7    & 7000--7040 & CW, 7030~kHz--QRP Centre of Activity\\
     & 7040--7047 & Narrow band modes--digimodes\\
     & 7047--7050 & Narrow band modes--digimodes\\
     & & automatically controlled data stations (unattended)\\
     & 7050--7053 & All modes--digimodes\\
     & & automatically controlled data stations (unattended)\\
     & 7053--7060 & All modes--digimodes\\
     & 7060--7100 & All modes, SSB contest preferred\\
     & & 7070~kHz--Digital Voice Centre of Activity\\
     & & 7090~kHz--SSB QRP Centre of Activity\\
     & 7100--7130 & All modes, 7110~kHz--Reg.1 Emergency Centre of Activity\\
     & 7130--7175 & All modes, SSB contest preferred\\
     & & 7165~kHz--Image Centre of Activity\\
     & 7175--7200 & All modes, SSB contest preferred\\
     & & priority for intercontinental operation\\

10   & 10100--10140 & CW, 10116~kHz--QRP Centre of Activity\\
     & 10140--10150 & Narrow band modes--digimodes\\

14   & 14000--14060 & CW, contest preferred, 14055~kHz--QRS Centre of Activity\\
     & 14060--14070 & CW, 14060~kHz--QRP Centre of Activity\\
     & 14070--14089 & Narrow band modes--digimodes\\
     & 14089--14099 & Narrow band modes--digimodes\\
     & & automatically controlled data stations (unattended)\\
     & 14099--14101 & IBP, exclusively for beacons\\
     & 14101--14112 & All modes--digimodes\\
     & & automatically controlled data stations (unattended)\\
     & 14112--14125 & All modes\\
     & 14125--14300 & All modes, SSB contest preferred\\
     & & 14130~kHz--Digital Voice Centre of Activity\\
     & & 14195~kHz \(\pm\) 5~kHz--Priority for Dxpeditions\\
     & & 14230~kHz--Image Centre of Activity\\
     & & 14285~kHz--SSB QRP Centre of Activity\\
     & 14300--14350 & All modes, 14300kHz--Global Emergency centre of activity\\

18   & 18068--18095 & CW, 18086~kHz--QRP Centre of Activity\\
     & 18095--18105 & Narrow band modes--digimodes\\
     & 18105--18109 & Narrow band modes--digimodes\\
     & & automatically controlled data stations (unattended)\\
     & 18109--18111 & IBP, exclusively for beacons\\
     & 18111--18120 & All modes--digimodes\\
     & & automatically controlled data stations (unattended)\\
     & 18120--18168 & All modes--digimodes\\
     & & automatically controlled data stations (unattended)\\
     & & 18130~kHz--SSB QRP Centre of Activity\\
     & & 18150~kHz--Digital Voice Centre of Activity\\
     & & 18160~kHz--Global Emergency Centre of Activity\\

21   & 21000--21070 & CW, 21055~kHz--QRS Centre of Activity\\
 & & 21060~kHz--QRP Centre of Activity\\
 & 21070--21090 & Narrow band modes--digimodes\\
     & 21090--21110 & Narrow band modes--digimodes\\
     & & automatically controlled data stations (unattended)\\
     & 21110--21120 & All modes (excluding SSB)\\
     & & digimodes, automatically controlled data stations (unattended)\\
     & 21120--21149 & Narrow band modes\\
     & 21149--21151 & All modes--digimodes\\
     & 21151--21450 & All modes, 21180~kHz--Digital Voice Centre of Activity \\
     & & 21285~kHz--SSB QRP Centre of Activity\\
     & & 21340~kHz--Image Centre of Activity\\
     & & 21360~kHz--Global Emergency Centre of Activity\\

24   & 24890--24915 & CW, 24906~kHz--QRP centre of activity\\
     & 24915--24925 & Narrow band modes--digimodes\\
     & 24925--27929 & Narrow band modes--digimodes\\
     & & automatically controlled data stations (unattended)\\
     & 24929--24931 & IBP, exclusively for beacons\\
     & 24931--24940 & All modes--digimodes\\
     & & automatically controlled data stations (unattended)\\
     & 24940--24990 & All modes, 24950kHz--SSB QRP Centre of Activity\\
     & & 24960 kHz--Digital Voice Centre of Activity\\

28   & 28000--28070 & CW, 28055~kHz--QRS Centre of Activity\\
 & & 28060~kHz--QRP Centre of Activity\\
     & 28070--28120 & Narrow band modes--digimodes\\
     & 28120--28150 & Narrow band modes--digimodes\\
     & & automatically controlled data stations (unattended)\\
     & 28150--28190 & Narrow band modes\\
     & 28190--28199 & IBP, regional time shared beacons\\
     & 28199--28201 & IBP, worldwide time shared beacons\\
     & 28201--28225 & IBP, continuous duty beacons\\
     & 28225--28300 & All modes--beacons\\
     & 28300--28320 & All modes--digimodes\\
     & & automatically controlled data stations (unattended)\\
     & 28320--2900 & All modes, 28330~kHz--Digital Voice Centre of Activity\\
     & & 28360~kHz--SSB QRP Centre of Activity\\
     & & 28680~kHz--Image Centre of Activity\\

29   & 29000--29100 & All modes\\
     & 29100--29200 & All modes--FM simplex--10 kHz channels\\
     & 29200--29300 & All modes--digimodes\\
     & & automatically controlled data stations (unattended)\\
     & 29300--29510 & Satellite Links\\
     & 29510--29520 & Guard channel\\
     & 29520--29590 & All modes--FM repeater input (RH1--RH8)\\
     & 29600 & All modes--FM calling channel\\
     & 29610 & All modes--FM simplex repeater (parrot input and output)\\
     & 29620--29700 & All modes--FM repeater outputs (RH1--RH8)\\
\end{longtable}

\subsection{ANMÄRKNINGAR}

\subsubsection{Definitioner}

\textbf{All modes} CW, SSB och AM samt de trafiksätt som angetts som
\emph{Centre of activity}.

\textbf{Image modes} Alla analoga eller digitala trafiksätt för bildöverföring
som ryms inom överenskommen bandbredd. Till exempel SSTV och FAX.

\textbf{Narrov band modes} Alla trafiksätt som använder en bandbredd upp till
500~Hz inklusive CW, RTTY, PSK med flera.

\textbf{Digimodes} Alla digitala trafiksätt som ryms inom överenskommen
bandbredd. Till exempel RTTY, PSK, MT63 med flera.

\subsubsection{Sändarfrekvenser}

I bandplanen angivna frekvenser är ''sändarfrekvenser''
(inte frekvensen för den undertryckta bärvågen).

\subsubsection{Digitala trafiksätt}

Omfattar BaudoVRTTY, AMTOR, PACTOR, CLOVER, ASCII, Packet Radio.
Observera undantagen för 1,8, 7 och 10~MHz där Packet Radio ej
ingår i Digitala Trafiksätt.

\subsubsection{Sidband}

Upp till 10~MHz ska lägre sidbandet (LSB) användas och över 10~MHz det övre
sidbandet (USB).

\subsubsection{Segment för tester}

Då DX-trafik ej är involverad ska testsegmenten ej innefatta
3500--3510 eller 3775--3800~kHz.

Frekvensbanden på 10, 18 och 24~MHz ska inte användas för tester.

\subsubsection{10 MHz bandet}

Vid nödtrafik får även SSB användas på detta band.
Nyhetsbulletiner ska ej sändas på 10~MHz bandet.
10120--10140~kHz får under dygnets ljusa timmar användas av
SSB-stationer i Afrika söder om ekvatorn.

\subsubsection{Obemannade sändare}

IARU:s medlemsföreningar är uppmanade att begränsa användningen av obemannade
sändare på kortvågsbanden.
Obemannade stationer på kortvåg ska endast aktiveras under kontroll av en
operatör som är ansvarig för att användningen inte orsakar störningar.

Detta är extra viktigt på 30~m där amatörradio har sekundär status.

Undantag gäller för fyrar och speciella experimentstationer.

\subsubsection{Fjärrstyrda radiostationer}

Med fjärrstyrda radiostationer (eng. \emph{remote controlled station}) menas en
sändare som fjärrstyrs av en radioamatör via någon typ av kontrollterminal.

Användningen av en fjärrstyrda radiostationer måste vara tillåten i det land där
stationen är placerad.

Anropssignalen som används vid fjärrstyrning måste vara tilldelad av myndigheten
i det land där stationen är placerad oavsett var i världen radioamatören som
använder den befinner sig.

Notera att överenskommelsen i CEPT T/R 61-01 \cite{TR6101} om användningen av
egen anropssignal med tillägg av landsprefix för besökt land endast gäller om
operatören befinner sig i landet, inte vid fjärrstyrning.
