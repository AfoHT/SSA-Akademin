\chapter{IARU Region 1 bandplan}

\section{HF bandplan}

Sammanfattad av SM3AVQ Lars

Denna bandplan reviderades vid lA RU Region i -konferensen i Tel-Aviv 1996.
Den vänstra delen är själva bandplanen, medan den högra delen rekommenderar användning/mötespunkter.
(PTS bandplan och status för amatörradio i Sverige, framgår av Kapitel III 1.6 samt Appendix G och H.)

\setlongtables
\begin{longtable}{lcl}
Band & Segment & Trafiksätt \\
MHz  & kHz     & \\ \hline
\endhead

1.8  & 1810 - 1838 & CW enbart \\
     & 1838 - 1840 & Digitala trafiksätt men ej Packet Radio, CW \\
     & 1840 - 1842 & Digitala trafiksätt men ej Packet Radio, Telefoni, CW \\
     & 1842 - 2000 & Telefoni, CW (i Sverige 1842-1850) \\

3.5  & 3500 - 3510 & CW enbart, DX-fönster för interkontinentala kontakter \\
     & 3500 - 3560 & CW enbart, segment för CW-tester \\
     & 3560 - 3580 & CW enbart \\
     & 3580 - 3590 & Digitala trafiksätt, CW \\
     & 3590 - 3600 & Digitala trafiksätt företrädesvis Packet Radio, CW \\
     & 3600 - 3620 & Telefoni, Digitala trafiksätt, CW \\
     & 3600 - 3650 & Telefoni, Segment för Telefoni-tester, CW \\
     & 3650 - 3775 & Telefoni, CW \\
     & 3700 - 3800 & Telefoni, Segment för Telefoni-tester, CW \\
     & 3730 - 3740 & SSTV \& FAX, Telefoni, CW \\
     & 3775 - 3800 & Telefoni, DX-fönster för interkontinentala kontakter \\

7    & 7000 - 7035 & CW enbart \\
     & 7035 - 7040 & Digitala trafiksätt men ej Packet Radio, SSTV \& FAX, CW \\
     & 7040 - 7045 & Digitala trafiksätt men ej Packet Radio, SSTV \& FAX, Telefoni, CW \\
     & 7045 - 7100 & Telefoni, CW \\

10   & 10100 - 10140 & CW enbart \\
     & 10140 - 10150 & Digitala trafiksätt men ej Packet Radio, CW \\

14   & 14000 - 14070 & CW enbart \\
     & 14000 - 14060 & CW enbart, Segment för CW-tester \\
     & 14070 - 14089 & Digitala trafiksätt, CW \\
     & 14089 - 14099 & Digitala trafiksätt företrädesvis Packet Radio, CW \\
     & 14099 - 14101 & Exklusivt fyrband IBP \\
     & 14101 - 14112 & Digitala trafiksätt företrädesvis Packet Radio forwarding, Telefoni, CW \\
     & 14112 - 14125 & Telefoni, CW \\
     & 14125 - 14300 & Telefoni, Segment för Telefoni-tester, CW \\
     & 14230         & SSTV \& FAX anropsfrekvens \\
     & 14300 - 14350 & Telefoni, CW \\

18   & 18068 - 18100 & CW enbart \\
     & 18100 - 18109 & Digitala trafiksätt, CW \\
     & 18109 - 18111 & Exklusivt fyrband IBP \\
     & 18111 - 18168 & Telefoni, CW \\

21   & 21000 - 21080 & CW enbart \\
     & 21080 - 21100 & Digitala trafiksätt, CW \\
     & 21100 - 21120 & Digitala trafiksätt företrädesvis Packet Radio, CW \\
     & 21120 - 21149 & CW enbart \\
     & 21149 - 21151 & Exklusivt fyrband IBP \\
     & 21151 - 21450 & Telefoni, CW \\
     & 21340         & SSTV \& FAX anropsfrekvens \\

24   & 24890 - 24920 & CW enbart \\
     & 24920 - 24929 & Digitala trafiksätt, CW \\
     & 24929 - 24931 & Exklusivt fyrband IBP \\
     & 24931 - 24990 & Telefoni, CW \\

28   & 28000 - 28050 & CW enbart \\
     & 28050 - 28120 & Digitala trafiksätt, CW \\
     & 28120 - 28150 & Digitala trafiksätt företrädesvis Packet Radio, CW \\
     & 28150 - 28190 & CW enbart \\
     & 28190 - 28199 & Regionella fyrar med tidsdelningsschema IBP \\
     & 28199 - 28201 & Världstäckande fyrnät med tidsdelnings-schema IBP \\
     & 28201 - 28225 & Kontinuerligt sändande fyrar IBP \\
     & 28225 - 29200 & Telefoni, CW \\
     & 28680         & SSTV \& FAX anropsfrekvens \\

29   & 29200 - 29300 & Digitala trafiksätt (NBFM Packet Radio), Telefoni, CW \\
     & 29300 - 29510 & satellit utfrekvens (nerlänk) \\
     & 29510 - 29700 & Telefoni (29 MHz FM-band, se nedan), CW \\
     &               & \textbf{FM Frekvensuppdelning} \\
     & 29510         & Del bandkant, används ej \\
     & 29520 - 29550 & FM Simplex \\
     & 29560 - 29590 & Repeater infrekvenser, 1O kHz frekvensavstånd \\
     & 29600         & Anropsfrekvens \\
     & 29610 - 29650 & FM Simplex \\
     & 29660 - 29690 & Repeater utfrekvenser, 1O kHz frekvensavstånd \\
     & 29700         & Bandkant, används ej \\
\end{longtable}

\subsection{ANMÄRKNINGAR}

\subsubsection{Prioritet:}

När flera trafiksätt förekommer på samma frekvenssegment, har det
trafiksätt företräde, som här nämns först. Detta sker dock under
vad som kallas Noninterference Basis, NIB (icke störande grundval).

\subsubsection{Digitala trafiksätt:}

Omfattar BaudoVRTTY, AMTOR, PACTOR, CLOVER, ASCII, Packet Radio.
Observera undantagen för 1.8, 7 och i OMHz där Packet Radio ej
ingår i Digitala Trafiksätt

\subsubsection{Telefoni:}

Alla slag av detta trafiksätt inkluderas. Upp till 10~MHz
skall lägre sidbandet (LSB) användas och över 10~MHz
det övre sidbandet (USB).

3500 - 3510 och 3775 - 3800 kHz:
Interkontinental trafik skall ges företräde på dessa segment.

\subsubsection{Segment för tester:}

Då DX-trafik ej är involverad skall testsegmenten ej innefatta
3500 - 3510 eller 3775 - 3800 kHz. Medlemsföreningarna tillåts
sätta andra, smalare, segment för sina nationella tester
(inom testsegmenten). Rekommendationen om testsegment gäller
inte tester med digitala trafiksätt.
Testaktivitet skall ej äga rum på 10, 18 och 24~MHz banden.

\subsubsection{7 och 10 MHz:}

Användande av Packet Radio på 7 och 10~MHz avråds.
7035-7045 får under dygnets ljusa timmar användas av
Packet Radio forwarding-stationer i Afrika söder om
ekvatorn.

\subsubsection{10 MHz bandet:}

Vid nödtrafik får även SSB användas på detta band. Obemannade
stationer som använder digitala trafiksätt skall undvika att
använda 10~MHz bandet.
Nyhetsbulletiner skall ej sändas på 10~MHz bandet.
10120 - 10140 kHz får under dygnets ljusa timmar användas av
SSB-stioner i Afrika söder om ekvatorn.

\subsubsection{SSTV / FAX:}

Frekvenserna 14230, 21340 och 28680 bör användas
som anropsfrekvenser för SSTV- och FAX-operatörer.
Efter att kontakt erhållits skall dessa flytta till annan
ledig frekvens inom telefonidelen av bandet.

\subsubsection{Satellitbandet 29300 - 29510 kHz:}

Medlemsländerna skall råda amatörerna att inte sända
FM på frekvenser mellan 29300 och 29510 kHz. Detta
för att undvika interferens med satelliternas nerlänk.

\subsubsection{Obemannade sändare:}

IARU:s medlemsländer äro uppmanade att begränsa
denna aktivitet på kortvågsbanden.
Obemannade stationer på kortvåg skall endast aktiveras under en operatörs kontroll.
Med detta menas att stationer ej skall aktiveras av
exempelvis ett program i en dator. Aktivering skall ske
av SYSOP eller av en uppropande station efter att de
avlyssnat frekvensen och funnit att den är ledig.
Undantag gäller för fyrar och speciella experimentstationer.

\subsubsection{Sändarfrekvenser:}

I bandplanen angivna frekvenser är "sändarfrekvenser"
(inte frekvensen för den undertryckta bärvågen).

\subsubsection{NBFM Packet Radio på 29 MHz-bandet:}

Rekommenderade frekvenser på varje 10 kHz f.o.m.
29210 t.o.m. 29290 kHz. En deviation av plus/minus 2.5
kHz skall användas med max 2.5 kHz modulationsfrekvens.
