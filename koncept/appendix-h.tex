\chapter{SVENSKA BANDPLANER}
\label{svenska bandplaner}

Vart och ett lands teleadministration utfärdar föreskrifter för amatörradio
i sitt land.
Dessa föreskrifter griper naturligtvis över IARU :s bandplaner, vilka endast
är rekommendationer för hur tilldelade frekvensband bör disponeras.
Post- och telestyrelsens föreskrifter PTSFS 2015:4 3 Kapitlet.

\begin{tabular}{clr|rr|l}
Frekvensband    &     & band   & 3 kap & Effekt       & Amatörradiostatus \\ \hline
  135,7 - 137,8 & kHz & 2200~m &  14§  & 1~W E.R.P..  & \\
  472   -   479 & kHz & 600~m  &  19§  & 1~W E.I.R.P. & \\
 1810   -  1850 & kHz & 160~m  &  22§  & 1~kW P.E.P.  & \\
 1850   -  1900 & kHz & 160~m  &  23§  & 10~W P.E.P.  & \\
 1900   -  1950 & kHz & 160~m  &  24§  & 100~W P.E.P. & \\
 1950   -  2000 & kHz & 160~m  &  25§  & 10~W P.E.P.  & \\
 3500   -  3800 & kHz &  80~m  &  27§  & 1~kW P.E.P.  & \\
 7000   -  7200 & kHz &  40~m  &  31§  & 1~kW P.E.P.  & \\
10100   - 10150 & kHz &  30~m  &  34§  & 150~W P.E.P. & \\
14000   - 14350 & kHz &  20~m  &  40§  & 1~kW P.E.P.  & \\
18068   - 18168 & kHz &  17~m  &  41§  & 1~kW P.E.P.  & \\
21000   - 21450 & kHz &  15~m  &  42§  & 1~kW P.E.P.  & \\
24890   - 24990 & kHz &  12~m  &  43§  & 1~kW P.E.P.  & \\
28000   - 29700 & kHz &  10~m  &  65§  & 1~kW P.E.P.  & \\
50000   - 52000 & kHz &   6~m  &  76§  & 200~W P.E.P. & \\ \hline
  144   -   146 & MHz &   2~m  &  80§  & 1~kW P.E.P.  & \\
  432   -   438 & MHz &  70~cm &  98§  & 1~kW P.E.P.  & \\
 1240   -  1300 & MHz &  23~cm & 123§  & 1~kW P.E.P.  & \\
 2400   -  2450 & MHz &  11~cm & 145§  & 100~mW P.E.P. & \\
 5650   -  5850 & MHz &   5~cm & 156§  & 1~kW P.E.P.  & \\
   10,0 -  10,5 & GHz &   3~cm & 164§  & 1~kW P.E.P.  & \\
   24,0 - 24,25 & GHz &  11~mm & 176§  & 1~kW P.E.P.  & \\
   47,0 - 47,2  & GHz &   6~mm & 188§  & 1~kW P.E.P.  & \\
   75,5 - 81,0  & GHz &   4~mm & 198§  & 1~kW P.E.P.  & \\
 122,25 - 123   & GHz &   2~mm & 203§  & 1~kW P.E.P.  & \\
    134 - 141   & GHz &   2~mm & 204§  & 1~kW P.E.P.  & \\
    241 - 250   & GHz &   1~mm & 205§  & 1~kW P.E.P.  & \\
\end{tabular}

Primär
sekundär

Primär tjänst har företräde före tjänst med lägre status.
Observera att flera tjänster kan ha delad primär status i ett band, som till'
exempel i 3500 kHz- och 432 MHz-bandet.
Ovanstående är den föreskrivna statusen för amatörradio i Sverige när
denna bok trycktes.
Mer om sändningsslagen per sändningsklass härovan på följande sida och
i Appendix E.

Primär tjänst har företräde före tjänst med lägre status.
Observera att flera tjänster kan ha delad primär status i ett band,
som till exempel i 3500 kHz- och 432 MHz-bandet.

