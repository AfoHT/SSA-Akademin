\chapter{Svenska bandplaner}
\label{svenska bandplaner}

Vart och ett lands teleadministration utfärdar föreskrifter för amatörradio
i sitt land.
Dessa föreskrifter griper naturligtvis över IARU:s bandplaner, vilka endast
är rekommendationer för hur tilldelade frekvensband bör disponeras.

Post- och telestyrelsens föreskrift PTSFS 2015:4 \cite{PTSFS2015:4} Kapitel 3
anger de band som i Sverige är tilldelade Amatörradio, och under vilka
vilkor, vilket främst avser effektbegränsning som finns återgiven.
Det är denna så kallade undantagsföreskrift som strikt reglerar vad som är
tillåtna band och effekter för amatörradio i Sverige.

Den svenska frekvensplanen PTSFS 2015:3 \cite{PTSFS2015:3} indikerar
rekommendation om primär och sekundär status för amatörradio på respektive
band.
Vissa band är markerade med pri/sek, och för de fallen behöver man
kontrollera i \cite{PTSFS2015:3} vad som gäller för vilken del av bandet.

\begin{tabular}{clr|rr|l}
Frekvensband    &     & band   & 3 kap & Effekt       & Amatörradio \\ \hline
  135,7--137,8 & kHz & 2200~m &  14§  & 1~W E.R.P.  & sekundär \\
    472--479 & kHz & 600~m  &  19§  & 1~W E.I.R.P. & sekundär \\
   1810--1850 & kHz & 160~m  &  22§  & 1~kW P.E.P.  & primär \\
   1850--1900 & kHz & 160~m  &  23§  & 10~W P.E.P.  & sekundär \\
   1900--1950 & kHz & 160~m  &  24§  & 100~W P.E.P. & sekundär \\
   1950--2000 & kHz & 160~m  &  25§  & 10~W P.E.P.  & sekunder \\
   3500--3800 & kHz &  80~m  &  27§  & 1~kW P.E.P.  & primär \\
   7000--7200 & kHz &  40~m  &  31§  & 1~kW P.E.P.  & primär \\
  10100--10150 & kHz &  30~m  &  34§  & 150~W P.E.P. & sekundär \\
  14000--14350 & kHz &  20~m  &  40§  & 1~kW P.E.P.  & primär \\
  18068--18168 & kHz &  17~m  &  41§  & 1~kW P.E.P.  & primär \\
  21000--21450 & kHz &  15~m  &  42§  & 1~kW P.E.P.  & primär \\
  24890--24990 & kHz &  12~m  &  43§  & 1~kW P.E.P.  & primär \\
  28000--29700 & kHz &  10~m  &  65§  & 1~kW P.E.P.  & primär \\
  50000--52000 & kHz &   6~m  &  76§  & 200~W P.E.P. & sekundär \\ \hline
    144--146 & MHz &   2~m  &  80§  & 1~kW P.E.P.  & primär \\
    432--438 & MHz &  70~cm &  98§  & 1~kW P.E.P.  & primär \\
   1240--1300 & MHz &  23~cm & 123§  & 1~kW P.E.P.  & sekundär \\
   2400--2450 & MHz &  11~cm & 145§  & 100~mW P.E.P. & sekundär \\
   5650--5850 & MHz &   5~cm & 156§  & 1~kW P.E.P.  & sekundär \\
   10,0--10,5 & GHz &   3~cm & 164§  & 1~kW P.E.P.  & sekundär \\
  24,00--24,25 & GHz &  11~mm & 176§  & 1~kW P.E.P.  & pri/sek \\
   47,0--47,2  & GHz &   6~mm & 188§  & 1~kW P.E.P.  & primär \\
   75,5--81,0  & GHz &   4~mm & 198§  & 1~kW P.E.P.  & pri/sek \\
 122,25--123,00   & GHz &   2~mm & 203§  & 1~kW P.E.P.  & sekundär \\
    134--141   & GHz &   2~mm & 204§  & 1~kW P.E.P.  & pri/sek \\
    241--250   & GHz &   1~mm & 205§  & 1~kW P.E.P.  & pri/sek \\
\end{tabular}

Primär tjänst har företräde före tjänst med lägre status.
Observera att flera tjänster kan ha delad primär status i ett band, som till'
exempel i 3500~kHz och 432~MHz-bandet.
Ovanstående är den föreskrivna statusen för amatörradio i Sverige när
denna bok trycktes.

Primär tjänst har företräde före tjänst med lägre status.
Observera att flera tjänster kan ha delad primär status i ett band,
som till exempel i 3500~kHz- och 432~MHz-bandet.
