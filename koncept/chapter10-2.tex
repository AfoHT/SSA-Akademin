\section{Störningar i elektronik}
\index{Electromagnetic Interference (EMI)}
\index{EMI}
\index{Electromagnetic Susceptibility (EMS)}
\index{EMS}

Liksom att radiomottagning kan ''störas'' av sändningar som inte är av
intresse, så kan störningar i form av radiovågor från olika slags
elektrisk utrustning försvåra mottagning eller andra funktioner.

Utstrålning från till exempel datorer, kabel-TV, hushållsmaskiner,
tändgnistor från oljebrännare, bilar och mopeder etc. är radiovågor.
Elektriska apparater kan alltså både störa och störas genom
radiovågor, även om de inte är definierade som \emph{radioanläggning},
det vill säga radiosändare och/eller radiomottagare.

Störningar som uppstår av elektromagnetiska fält kallas
\emph{Electromagnetic Interference (EMI)}.
Känsligheten för sådana störningar kallas för
\emph{Electromagnetic Susceptibility (EMS)}.

Äldre benämningar på störningar är till exempel BroadCasting Interference (BCI) och
TeleVision Interference (TVI).
Dessa återfinns dock inte i nu gällande terminologi.

\subsection{Blockering}
\textbf{
HAREC a.\ref{HAREC.a.9.1.1}\label{myHAREC.a.9.1.1}
}
\index{blockering}

I de flesta radiomottagare finns en automatisk förstärkningsreglering.
Om insignalerna blir för starka, så räcker regleringen inte till.
Då överstyrs förstärkarstegen så att de arbetar olinjärt.
Detta kallas blockering.

Ett sätt att undvika blockering är att koppla en \emph{dämpsats}
(eng. \emph{attenuator} till mottagaringången.
En sådan sänker dock signalstyrkan över hela frekvensområdet, inte bara för en
viss signalfrekvens.

\subsection{Interferens}
\textbf{
HAREC a.\ref{HAREC.a.9.1.2}\label{myHAREC.a.9.1.2}
}
\index{interferens}

När den önskade signalen störs av en annan signal nära i frekvens, kallas det
\emph{interferens} (eng. \emph{interference}).
I mottagaringången finns frekvensfilter, som undertrycker ej önskade signaler,
om de inte ligger alltför nära.
Om ingången inte är tillräckligt selektiv, kan det behövas en tillsats som
förbättrar selektiviteten.

\subsection{Intermodulation}
\textbf{
HAREC a.\ref{HAREC.a.9.1.3}\label{myHAREC.a.9.1.3}
}
\index{intermodulation}

(se även kapitel \ref{intermodulation})

Blandningsprodukter av signaler i en mottagare eller sändare kallas
för \emph{intermodulation}.

\subsection{LF-detektering}
\textbf{
HAREC a.\ref{HAREC.a.9.1.4}\label{myHAREC.a.9.1.4}
}

HF-signaler kan komma in genom in- och utgångarna för LF samt genom nätkabeln.

Dessutom förekommer direktinstrålning av radiovågor genom apparathöljet, om
detta inte har tillräckligt avskärmande verkan.

\emph{LF-detektering} uppstår när HF-signaler demoduleras i diodsträckor i den
störda apparatens komponenter.
Detta sker oavsett vilken frekvens som sändaren eller mottagaren är inställd på.

\emph{LF-detektering} uppstår särskilt vid AM- eller SSB-modulerade sändningar
samt av transienter vid bärvågsnyckling av sändare.

Ofta är det inte möjligt att förhindra LF-detektering utan ingrepp i den störda
apparaten.
Sådana ingrepp bör endast utföras av fackman.
