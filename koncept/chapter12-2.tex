\section{CEPT}

\subsection{Begreppet CEPT}

Vid sidan av folkrättsligt bindande avtal såsom den internationella
telekonventionen (ITC) --- har det internationella samarbetet lett till
överenskommelser som inte är tvingande. Sådana avtal görs bl.a. inom
\emph{CEPT}.

\emph{CEPT} betyder \emph{Conference Europeanne des Administrations
  des Postes et des Telecommunications}, d.v.s. Europeiska konferensen
förpost- och teleadministrationerna. ''Konferens'' är att förstå som
ett ständigt arbetande samarbetsorgan.

Arbetet inom CEPT har huvudsakligen karaktär av ömsesidiga
programförklaringar mellan länder. Trots att dessa viljeförklaringar
eller rekommendationer inte är bindande har de visat sig värdefulla
för utvecklingen av det internationella samarbetet.

Länder anslutna till CEPT förenklar handläggningen av ärenden
bl.a. rörande amatörradio genom att ömsesidigt bekräfta
rekommendationer inom området.

\subsection{CEPT-rekommendationerna}

Länder anslutna till CEPT förenklar numera handläggningen av
tillståndsärenden om amatörradio genom att ömsesidigt bekräfta och
inom sitt land tillämpa rekommendationer som länderna utformat i
samråd. Det innebär att svenska amatörradiobestämmelser kan
''harmoniseras'' till andra länders.  För kompetenskrav vid
examinering av radioamatörer finns CEPT-rekommendationerna T/R~61-01
och T/R~61-02.

\subsubsection{CEPT-rekommendation T/R 61-01}
\textbf{
HAREC b.\ref{HAREC.c.2.1}\label{myHAREC.c.2.1},
 b.\ref{HAREC.c.2.2}\label{myHAREC.c.2.2},
 b.\ref{HAREC.c.2.3}\label{myHAREC.c.2.3}
}

Rekommendationen T/R 61-01 som första gången godkändes år 1985 möjliggör för
radioamatörer från CEPT-länderna att utöva amatörradio under korta besök i
andra CEPT-länder, utan att behöva ett tillfälligt tillstånd från det besökta
CEPT-landet. Erfarenheterna med detta system är goda.

\subsubsection{CEPT-rekommendation T/R 61-02}

Rekommendationen T/R 61-02 innebär att administrationerna i CEPT-länder utger
ömsesidigt erkända Harmoniserade Amatörradio Examinerings Certifikat (HAREC)
till de personer som vid nationella prov uppfyller rekommendationens
kunskapskrav. Denna HAREC-nivå motsvarar kravet för det svenska certifikatet.
Radioamatörer med ett sådant certifikat får utöva amatörradio i annat
CEPT-land, som godkänt T/R 61-02 och får tilldelas ett CEPT-certifikat av det
landet utan att behöva genomgå ytterligare kunskapsprov.

Det medger också att en person som uppvisar ett CEPT-certifikat
(HAREC), utfärdat av ett annat CEPT-land, tilldelas ett motsvarande
tillstånd vid återkomsten till hemlandet utan att behöva genomgå
ytterligare kunskapsprov.

Sverige tillämpar T/R 61-01 och T/R 61-02.


