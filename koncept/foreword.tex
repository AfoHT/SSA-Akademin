\chapter*{Förord}
\section*{Amatörradio}
Amatörradio är en teknisk hobby med inriktning på kommunikation och experiment
med radioanläggningar samt radiovågors utbredning. Det är en verksamhet som
utövas över hela världen av licensierade radioamatörer, även kallade
sändaramatörer.

Syftet med amatörradio är att främja personlig utveckling och internationell
förståelse samt teknisk färdighet och erfarenhetsutbyte inom området.
Amatörradio kan därtill vara en tillgång då samhällets normala resurser för
radiokommunikation behöver förstärkas.

\section*{En hobby med krav}

För att inneha och använda en radioläggning i ett land, krävs tillstånd (licens)
från dess teleadministration. För ett amatörradiotillstånd föreskrivs
i det internationella radioreglementet bland annat handhavandemässiga och
tekniska kvalifikationer hos varje person som önskar använda en
amatörradiostation. De nationella teleadministrationerna tillser detta genom
kompetensprov.

\section*{Utbildningsställen}

Amatörradioklubbarna bedriver huvuddelen av utbildningen med
amatörradiocertifikat som mål. Även vissa skolor, militära förband m.fl. har
amatörradio på programmet. I någon utsträckning förekommer även självstudier.
Rekrytering av handledare för terminslånga kurser är en nyckelfråga för
kursarrangören, liksom målinriktade, anpassade läromedel.

Tanken med denna bok är att leverera ett material som kan vara grunden till
denna utbildning samt även för viss fördjupning och förståelse för de koncept
som man vanligtvis stöter på inom hobbyn.

\section*{Andra förutsättningar}

Den svenska teleadministrationen har omdanats på senare tid. Därvid har även
amatörradioanvändningen berörts, främst genom att provförrättningarna för
amatörradiocertifikat delegerats till av myndigheten utsedda, ideellt arbetande
förrättare. Vidare genom att teleadministrationerna inom CEPT har infört
harmoniserade certifikats- och tillståndsklasser för amatörradio. Främst av
dessa anledningar har det uppstått behov av samordnade hjälpmedel för utbildning
och examinering, vilket amatörradiorörelsen själv har att tillgodose.

\section*{Föreningen Sveriges Sändareamatörer -- SSA}

SSA är en ideell förening för personer med intresse för amatörradio.
Verksamheten är religiöst och politiskt obunden. Ett av syftena är att bland
medlemmarna verka för ökade tekniska kunskaper och god radiotrafikkultur för att
därigenom skapa en kår av kunniga radioamatörer. SSA representerar Sverige som
nationell förening i International Amateur Radio Union (IARU), Region~1.

\section*{Internationell samverkan}

De nationella föreningarna inom IARU samarbetar över nationsgränserna. Ett
exempel är när DARC (Deutscher Amateur-Radio-Club e.V.) för några år sedan
ställde sina Ausbildungsunterlagen till SSA:s förfogande som källmaterial för
boken El-lära och Radioteknik. Detta material har till stor del kunnat utnyttjas
även i här föreliggande bok.


\clearpage

Förord andra upplagan:

Under 2016 kom en grupp att sammankallas i en vilja att modernisera
utbildningen av radioamatörer. Denna grupp består av SSA:s utbildningsansvariga
Jonas Hultin SM5PHU, samt Magnus Danielson SA0MAD, Hans Insulander SM0UTY,
Petter Karkea SA2PKA samt Peter Lundberg SA2BLV. Utöver denna kärngrupp har
ett antal andra bidragit i stort och smått.

Dels fanns ett uppdämt behov att adressera brister i existerande
utbildningsmaterial, dels för att anpassa det till ett modernare sätt att
utbilda, vilket inkluderar att kunna nyttja moderna webbaserade
utbildningssystem. En sådan pedagogik är koncepted ''flipped classroom''
där man istället för att ha föreläsningar snarare läser hemmavid innan
lektionen och sedan försöker reda ut oklarheter och försöker
illustrera det under lektionstid. Sådan utbildningspedagogik har använts länge
t.ex. i dykutbildning hos PADI dykcenter.

En aspekt har varit att viktig är att se till att uppdatera materialet för
att täcka hela CEPT HAREC, som uppdaterats att inkludera bland annat sampling
och DSP:er, vilket nu mer är en naturlig del av hobbyn. Även andra delar av
materialet har varit i behov av uppdatering.

Arbetet med att omvandla KonCEPT till \LaTeX, som bygger på den scannade och
OCR:ade versionen på första upplagan, inleddes av Magnus Danielson SA0MAD och
Hans Insulander SM0UTY. Arbetet har understötts av Thorbiörn Fritzon SA0LAT och
Täpp-Anders Sikvall SM0UEI som bidragit med sin ovärderliga \LaTeX -kunskap och
SA0LAT har även gjort det stora jobbet att extrahera alla bilder från den
scannade boken och göra dem tillgängliga så att de kan läggas in.

Elsäkerhetsavsnittet has fått en ordentlig genomgång av Lorentz Björklund SM7NTJ
som även granskat igenom den övriga texten och lämnat konstruktiva synpunkter.
Regler och regelverk mm har granskats av Christer Jonson SA0BFC.

Magnus Danielson SA0MAD har skrivit nya avsnitt för digital modulation, DSP,
isolation och jordning mm. för att komplettera de hål i HAREC och materialet
som varit uppenbara.

Philip Eriksson, FRO Gotland, har granskat och uppdaterat texten med många
rättningar på språk och format.

Kenneth Becker SM0GXZ har bidragit med text om digitala modulationer och
mätinstrument så som de används av radioamatörer.

Under arbetet har det varit viktigt att hålla spårbarhet till HAREC-krav,
vilket resulterat till bilaga \ref{CEPT HAREC}.
Sakregister är också helt omarbetad för att vara till hjälp.

\emph{Magnus Danielson SA0MAD}

\hilight{TODO: Uppdatera förord med utvecklingen.}

Förord första upplagan:

TACK!

Ett stort tack till alla dem, som på olika sätt bidragit till att förverkliga
boken.

\emph{Författaren}
