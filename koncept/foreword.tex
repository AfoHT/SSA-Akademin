\chapter*{Förord}
\section*{Amatörradio}
Amatörradio är en teknisk hobby med inriktning på kommunikation och experiment
med radioanläggningar samt radiovågors utbredning. Det är en verksamhet som
utövas över hela världen av licensierade radioamatörer, även kallade
sändaramatörer.

Syftet med amatörradio är att främja personlig utveckling och internationell
förståelse samt teknisk färdighet och erfarenhetsutbyte inom området.
Amatörradio kan därtill vara en tillgång då samhällets normala resurser för
radiokommunikation behöver förstärkas.

\section*{En hobby med krav}

För att inneha och använda en radioläggning i ett land, krävs tillstånd (licens)
från dess teleadministration. För ett amatörradiotillstånd föreskrivs
i det internationella radioreglementet bland annat handhavandemässiga och
tekniska kvalifikationer hos varje person som önskar använda en
amatörradiostation. De nationella teleadministrationerna tillser detta genom
kompetensprov.

\section*{Utbildningsställen}

Amatörradioklubbarna bedriver huvuddelen av utbildningen med
amatörradiocertifikat som mål. Även vissa skolor, militära förband m.fl. har
amatörradio på programmet. I någon utsträckning förekommer även självstudier.
Rekrytering av handledare för terminslånga kurser är en nyckelfråga för
kursarrangören, liksom målinriktade, anpassade läromedel.

Tanken med denna bok är att leverera ett material som kan vara grunden till
denna utbildning samt även för viss fördjupning och förståelse för de koncept
som man vanligtvis stöter på inom hobbyn.

\section*{Andra förutsättningar}

Den svenska teleadministrationen har genom åren omdanats. Därvid har även
amatörradioanvändningen berörts, genom att all provförrättning för
amatörradiocertifikat, liksom administration av anropssignaler, har delegerats till Föreningen Sveriges Sändareamatörer. Vidare har teleadministrationerna inom CEPT infört
harmoniserade certifikatsklasser för amatörradio -- HAREC \ref{CEPT HAREC}. Främst av dessa anledningar har det uppstått behov av samordnade hjälpmedel för utbildning och certifiering, vilket amatörradiorörelsen själv har att tillgodose.

\section*{Föreningen Sveriges Sändareamatörer -- SSA}

SSA är en ideell förening för personer med intresse för amatörradio.
Verksamheten är religiöst och politiskt obunden. Ett av syftena är att bland
medlemmarna verka för ökade tekniska kunskaper och god radiotrafikkultur för att
därigenom skapa en kår av kunniga radioamatörer. SSA representerar Sverige som
nationell förening i International Amateur Radio Union (IARU), Region~1.

\section*{Internationell samverkan}

De nationella föreningarna inom IARU samarbetar över nationsgränserna. Ett
exempel är när DARC (Deutscher Amateur-Radio-Club e.V.) för ett antal år sedan
ställde sina Ausbildungsunterlagen till SSA:s förfogande som källmaterial till föregångaren till denna bok.

\section*{Denna bok}

Denna faktabok omfattar det av Post- och telestyrelsen anvisade
kompetensområdet för amatörradiocertifikat.

Innehållet är delat i två ämnesgrupper; grundläggande radioteknik
samt regler och trafikmetoder.
Det finns även inlärningsanvisningar för morsesignalering för den
som vill lära sig telegrafi.

I bilagorna finns bland annat grundläggande matematik
och frekvensplaner för amatörradiotrafik.


\clearpage

Förord till andra upplagan:


Boken bygger till mycket stor del på det arbete som till första upplagan utfördes av Lennart Wiberg SM7KHF, med flera.

Med tiden har uppstått ett behov av att bredda det existerande utbildningsmaterialet, och att anpassa det till ett modernare sätt att utbilda, inte minst för att kunna utnyttja moderna webbaserade utbildningssystem.

En viktig aspekt har varit att materialet ska täcka hela CEPT HAREC och vara spårbart till dessa krav.

Till denna andra upplaga har allt tidigare material granskats och uppdaterats. Nya kapitel har lagts till, bland annat om elektromagnetiska fält, digitala trafiksätt och digital signalbehandling. Avsnitten om elsäkerhet och nödtrafik har omarbetats och samtliga referenser till lagar och föreskrifter är i skrivande stund aktuella.

Den nu föreliggande andra upplagan finns tillgänglig på digitalt format. Detta underlättar inte bara för läsaren att söka efter specifik information, men utgör också en grund för kommande webbaserad utbildning.



TACK!

Ett stort tack till alla dem, som på olika sätt bidragit till att förverkliga
boken. Ett särskilt tack riktas till Magnus Danielsson SA0MAD, Petter Karkea SA2PKA, Lorentz Björklund SM7NTJ, Jonas Hultin SM5PHU och Philip Eriksson.

\emph{Författarna}
