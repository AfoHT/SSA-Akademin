\section{Internationell nödtrafik och trafik vid naturkatastrofer}
Nödsignaler
l ITU Radioreglemente (RR) framgår av Artikel 39 om ``Distress Communications'' hur
nödsignaler skall vara formulerade. Dessa
signaler är internationella och sänds när ett
skepp, flygplan eller annan farkost hotas av
allvarlig och omedelbarfara och begär hjälp.
Nödsignalen på morsetelegrafi består av
teckendelarna -- .. - - - .. -- sända i en följd,
där längden på de långa teckendelarna betonas så att de klart skiljer sig från de korta.
Signalen skrivs som bokstäverna SOS med
ett streck ovanför.
Nödsignalen på radiotelefoni består av
ordet MAYDAY uttalat som det franska uttrycket ``m'aider''.
På amatörradiofrekvenserna förekommer
även CQ EMERGENCY som internationellt
nq~anrop. l Sverige kan man även ropa
NODANROP på svenska.
Nödtrafik
Vid 1979 års världsradiokonferens (WARC)
antogs bl.a. Resolution 640, vilken avsåg
internationell radiokommunikation på frekvensband upplåtna åt amatörradion, i händelse av naturkatastrofer.
Resolutionen är inskriven i RR. l CEPTrekommendation T/R 61-02 nämns beslutspunkterna 1-5. För orientering återges resolutionen med dessa punkter kursiverade.
l betraktande av
a. att i händelse av naturkatastrofer de normala kommunikationssystemen ofta är
överbelastade, skadade eller helt avskurna
b. att snabbt upprättande av kommunikationer är absolut nödvändigt för att möjliggöra världsomspännande hjälpaktioner
c. att amatörbanden inte är bundna av fasta
bandplaner eller kungörelser och därför
är vällämpade för korttidsanvändning vid
nödtillfällen
d. att internationell nödtrafik kan underlättas genom tillfällig användning av vissa
frekvensband upplåtna åt amatörradiotrafiken

1111 - 6

e. att i sådana situationer amatörradiostationer p.g .a. deras stora geografiska spridning och påvisade kapacitet kan hjälpa till
att upprätthålla viktiga radioförbindelser
f. att det existerar nationella och regionala
amatörnödtrafiknät som använder frekvenser inom gällande bandplan för amatör rad iot ra f ik
g. att i händelse av en naturkatastrof direkt
förbindelse mellan amatörradiostationer
och andra radiotjänster möjliggör överförandet av livsviktiga meddelanden tills
normala radioförbindelser åter kan upprättas.
Med insikt om att befogenheter och ansvar
för sådan radiotrafik vid naturkatastrofer vilar på berörda länders myndigheter, beslutar
konferensen
1. att frekvensbanden specificerade i No.
51 O *får användas av myndigheter inom
ramen för internationell nödtrafik
2. att sådan användning av amatörbanden
skall begränsas till nödtrafik i samband
med naturkatastrofer
3. att i sådana fall trafiken av icke-amatörradiostationer skall inskränkas till nödtillfället inom det speciella område som
anges av resp myndighet i det drabbade
landet
4. att nödtrafiken skall äga rum inom
katastrofområdet och mellan detta och
vederbörande hjälporganisations högkvarter
5. att sådan nödtrafik endast får upprättas
efter medgivande ifrån det drabbade landets myndighet
6. att nödtrafik ifrån länder utanför inte får
upphäva redan befintliga nationella eller
internationella amatörnödtrafiknät
7. att ett nära samarbete mellan amatörradiostationer och andra radiotjänster,
som i en framtid kan finna det nödvändigt
att för nödtrafik använda amatörbanden,
är önskvärt
8. att vid sådan internationell trafik såvitt
möjligt skall undvikas att störa amatörradiotrafik.

ET DER
Anhåller konferensen hos myndigheterna**
1. att skapa sådana förutsättningar som tillåter genomförandet av internationell nödtrafik
2. att i sina radioreglementen upptaga föreskrifter som tillåter genomförande av nödtrafik.

Om Du hör en nödsignal på radio

* 51 Oär den fotnot i frekvenstilldelningstabellen, som hänvisar till Resolution 640.
De amatörradioband som specificeras för
användning i händelse av naturkastrater
är 3.5, 7.0, 1O. i, i 4.0, i 8.068, 21.0, 24.89
och 144 MHz.
** Med myndigheterna avses respektive
lands teleadministration.

Du själv sänder nödsignal över radio

Avbryt omedelbart din egen sändning när du
hören nödsignal. Lyssna på nödmeddelandet
och SKRIV NER vad som sägs. Notera position, frekvens, tidpunkt etc. Anmäl vad du
hört på följande sätt.

Uppträd lugnt och sansat, när du kallar på
hjälp över radion. Tänk först och sänd sedan. Som ovan sagts måste den som svarar
dig och sedan ringer 112 (förut 90000) meddela larmoperatören att Ditt nödanrop kommit via radio.

Nödsignal från radioamatör i utlandet
Nödsignal från en radioamatör i ett katastrofområde utomlands ska anmälas till UD, d.v.s.
Utrikesdepartementet.
På dagtid kl. 8 - 17
te l. 08-40 55 950.
te l. 08- 40 55 001 .
På övrig tid

Nyckelordet för dina åtgärder är LARMA:
läge
Ange olycksplatsens läge. Du kan
ange gatu- eller vägnamn eller
riktmärken som t.ex. vägkorset,
gränsen, bron, järnvägen etc.
Analysera Gör en överblick över olycksplatsen och tala om vad som hänt.
Några skadade? Några innestängda? Brinner det? Släpps
farliga ämnen ut?
Ropa
Ropa på hjälp. Använd gärna en
repeater på 2-metersbandet så
att du når många, men även andra frekvenser kan användas.
Anropamed NÖDANROPFRÅN
SMXxxx. Fråga efter någon med
telefon. Ge inte upp om du inte får
svar genast.
Meddela Meddela när du fått kontakt med
någon med telefon, sänd NÖDTRAFIKPÅGÅR för att freda frekvensen och NÖDMEDDELANDET med de viktigaste uppgifterna. Begär att uppgifterna repeteras och ta löfte på att de sänds
vidare. Begär att få veta när så
har skett. Påminn annars!
Avvakta Vänta på platsen tills hjälp har
anlänt. Passa radion så att du
kan svara på frågor. Behövs inte
lätJgre din hjälp, avsll!~a då m~d
NODTRAFIK UPPHOR FRAN
SMXxxx .. KLART SLUT.

Nödsignal från svenskt landområde
l Sverige bör du ringa 112 (förut 90 000) för
att kalla på Ambulans, Polis, Räddningskår,
Sjöräddning, Flygräddning etc. Ditt telefonnummer visas automatiskt i larmoperatörens display.
För att undvika missförstånd och feldirigering av räddningsinsatserna MÅSTE
du meddela operatören att nödanropet kommit via radio. Själva olycksplatsen kan ju
ligga i ett helt annat riktnummerområde, än
som både Ditttelefonsamtal och nödanropet
kommer ifrån.
Nödsignal från fartyg eller luftfarkost
Om nödsignalen inte besvaras av någon
kust- eller markstatio n, ring 1 i 2 (förut 90000)
och begär Sjöräddning respektive Flygräddning och meddela dina iaktagelser. Du
kan även rapportera direkt till centralerna
Sjöräddning i Stockholm 08 - 601 79 00,
Sjöräddning i Göteborg 031 - 64 80 20 och
Flygräddning 031 - 64 80 00.
Vidarebefordra nödmeddelandet utan att
ändra på det!

1111 -7

R

l

H TRAFIKMET DER

~©~

PT
