\section{Svenska lagar, bestämmelser}

$>>>>>$ TODO: Behöver uppdateras med aktuella lagar och förordningar.

\textbf{Lagar, föreskrifter och anvisningar tillämpas för
  amatörradioanvändning.}

\textbf{Märk, att ändringar kan förekomma.}

\textbf{Använd därför aktuella versioner!}


\subsection{Lagen om radiokommunikation m.fl.}
\textbf{
HAREC b.\ref{HAREC.c.3.1}\label{myHAREC.c.3.1}
}

Denna lag reglerar all i radiokommunikation Sverige. Tillstånd behövs
i princip för all slags radiosändning.

\emph{Post- och telestyrelsen - PTS} - är enligt Förordning om
radiokommunikation den svenska myndigheten (administrationen) för
telekommunikation. PTS skall bland annat svara för att möjligheterna
till radiokommunikationer utnyttjas effektivt och har därvid att
beakta den internationella regleringen inom området. Regleringen av
amatörradioanvändningen begränsas nu till den minsta omfattning som
följer av internationella avtal och europeiska rekommendationer,
CEPT-rekommendationer.

\subsection{Post- och telestyrelsens föreskrifter om innehav och
användning av amatörradioanläggningar m.m.}
\textbf{
HAREC b.\ref{HAREC.c.3.2}\label{myHAREC.c.3.2}
}

Post- och telestyrelsens styrelse beslutar om Post- och telestyrelsens
föreskrifter om innehav och användning av amatörradioanläggningar.
Dessa föreskrifter är anpassade till lagen om radiokommunikation.

Enligt radiolagen kan ett tillstånd att inneha och använda
radiosändare förenas med villkor angående kompetenskrav för den som
skall handha radioanläggningen.

För att få ett amatörradiotillstånd måste man ha ett
radioamatörcertifikat, som är ett kompetensbevis från Post- och
telestyrelsen. Med stöd av lagen ställer PTS kompetenskrav, att jämföras
med artikel S25 i internationella radioreglementet åberopar därvid
CEPT-rekommendationen T/R 61-02 som kunskapsnorm för de svenska
klasserna CEPT 1 och CEPT 2.

Det innebär att PTS numera tillhandahåller endast dessa två
certifikatsklasser.

\subsection{Föreningen Sveriges Sändareamatörers anvisningar
1995: 1 om innehav och av amatörradioanläggningar}

CEPT-rekommendationer för nybörjarcertifikat däremot, finns inte
f.n. (år 1997).

Bland CEPT-länderna är nämligen meningarna delade om de krav som
skulle rekommenderas. Eftersom CEPT-rekommendationer för
nybörjarcertifikat ej finns, tillgodoses behovet av svenska sådana
certifikat på annat sätt.

Med stöd av Post- och telestyrelsens föreskrifter har Föreningen
Sveriges Sändareamatörer- SSA - tillstånd att inneha och använda
amatörradiosändare för Föreningens utbildningsverksamhet inom
amatörradioområdet. Mot denna bakgrund beslutar SSA:s styrelse en serie
anvisningar där villkoren för SSA-certifikat och SSAtillstånd
specifiseras.

\subsection{Litteraturhänvisning om lagar och föreskrifter}

Följande kan lånas på de flesta större bibliotek eller kan beställas
från Fritzes Förlag eller Föreningen Sveriges Sändareamatörer:
\begin{itemize}
\item Telelag,
\item Lag om radiokommunikation,
\item Förordning om radiokommunikation,
\item Post-och telestyrelsens föreskitter om godkännande av
  provförrättare m.m.,
\item Post- och telestyrelsens föreskifterom innehav och användning av
  amatörradioanläggningar m.m.,
\item Föreskrifter om ändring i Post- och telestyrelsens föreskitter
  om innehav och användning av amatörradioanläggningar m.m.,
\item Post- och telestyrelsens föreskrifter om avgifter för certifikat
  m.m. inom radioområdet
\end{itemize}

Följande kan beställas från Föreningen Sveriges Sändareamatörer:
\begin{itemize}
\item SSA:s anvisningar om amatörradioanläggningar vid SSA-tillstånd,
\item SSA:s anvisningar om kunskapskrav för SSA-certifikat,
\item SSA:s anvisningar om provförrättning för SSA-certifikat.
\end{itemize}
