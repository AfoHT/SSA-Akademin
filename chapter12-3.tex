\section{Svenska lagar, bestämmelser}

Lagar, föreskrifter och anvisningar
tillämpas för amatörradioanvändning.
Märk, att ändringar kan tor'eKt'JmJma.
Använd
lagen om radiokommunikation m.fl.
Denna lag reglerar all
i
Sverige. Tillstånd behövs i princip för all
slags radiosändning.
Post- och telestyrelsen - PTS - är enligt
Förordning om radiokommunikation den
svenska myndigheten (administrationen) för
telekommunikation. PTS skall bland annat
svara för att möjligheterna till radiokommunikationer utnyttjas effektivt och har därvid att
beakta den internationella regleringen inom
området. Regleringen av amatörradioanvändningen begränsas nu till den minsta
omfattning som följer av internationella avtal
och europeiska rekommendationer,
CEPT-rekommendationer.

Post- och telestyrelsens föreskrifter om
innehav och användning av amatörrad i om
anläggningar m.m.
Post- och telestyrelsens styrelse beslutar
om Post- och telestyrelsens föreskrifter om
innehav och användning av amatörradioanläggningar. Dessa föreskrifter är anpassade tilllagen om radiokommunikation.
Enligt radiolagen kan ett tillstånd att
inneha och använda radiosändare +r. ..."'"''"'"
med villkor angående kompetenskrav för
den som skall handha radioanläggningen.
För att få ett amatörradiotillstånd måste
man ha ett radioamatörcertifikat, som är ett
kompetensbevis från Post- och telestyrelsen. Med stöd av lagen ställer PTS
tenskrav, att jämföras med artikel S25 i
internationella radioreglementet
åberopar därvid CEPT-rekommendationen
T/R 61-02 som kunskapsnorm för de svenska klasserna CEPT 1 och CEPT 2.
Det innebär att PTS numera tillhandahåller endast dessa två certifikatsklasser.

tillståndsvillkor
Föreningen Sveriges Sändareamatörers
1995: 1 om innehav och
av amatörradioanläggningar
CEPT-rekommendationer för nybörjarcertifikat däremot, finns inte f.n. (år 1997).
CEPT-Iänderna är nämligen meningarna delade om de krav som skulle
rekommenderas. Eftersom CEPT-rekommendationer för nybörjarcertifikat ej finns,
tillgodoses behovet av svenska sådana certifikat på annat sätt.
Med stöd av Post- och telestyrelsens
föreskrifter har Föreningen Sveriges Sändareamatörer- SSA - tillstånd att inneha
och använda amatörradiosändare för Föreningens utbildningsverksamhet inom amatörradioområdet Mot denna bakgrund beslutar SSA:s styrelse en serie anvisningar
där villkoren för SSA-certifikat och SSAtillstånd specifiseras.
.,,.., ..,,.,l!"'',fti!"!!U"'\ 11"1!

litteraturhänvisning om lagar och föreskrifter
Följande kan lånas på de flesta större bibliotek eller kan beställas från Fritzes Förlag
eller Föreningen Sveriges Sändareamatörer:
e Telelag,
• Lag om radiokommunikation,
e Förordning om radiokommunikation,
• Post-och telestyrelsens föreskitter om godkännande av provförrättare m.m.,
• Post- och telestyrelsens föreskifterom innehav och användning av amatörradioanläggningar m. m.,
• Föreskrifter om ändring i Post- och telestyrelsens föreskitter om innehav och användning av amatörradioanläggningar
m.m.,
e Post- och telestyrelsens föreskrifter om
avgifter för certifikat m.m. inom radioområdet
Följande kan beställas från
~o-r.v·on1nncln Sveriges Sändareamatörer:
• SSA:s anvisningar om amatörradioanläggningar vid SSA-tillstånd,
e SSA:s anvisningar om kunskapskrav för
SSA-certifikat,
• SSA:s anvisningar om provförrättning för
SSA-certifikat.

1112-5


1112-6

