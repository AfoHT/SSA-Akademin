\section{Fonetiska alfabeten}
\textbf{
HAREC b.\ref{HAREC.b.1.1}\label{myHAREC.b.1.1},
 b.\ref{HAREC.b.1.2}\label{myHAREC.b.1.2},
 b.\ref{HAREC.b.1.3}\label{myHAREC.b.1.3},
 b.\ref{HAREC.b.1.4}\label{myHAREC.b.1.4},
 b.\ref{HAREC.b.1.5}\label{myHAREC.b.1.5},
 b.\ref{HAREC.b.1.6}\label{myHAREC.b.1.6},
 b.\ref{HAREC.b.1.7}\label{myHAREC.b.1.7},
 b.\ref{HAREC.b.1.8}\label{myHAREC.b.1.8},
 b.\ref{HAREC.b.1.9}\label{myHAREC.b.1.9},
 b.\ref{HAREC.b.1.10}\label{myHAREC.b.1.10},
 b.\ref{HAREC.b.1.11}\label{myHAREC.b.1.11},
 b.\ref{HAREC.b.1.12}\label{myHAREC.b.1.12},
 b.\ref{HAREC.b.1.13}\label{myHAREC.b.1.13},
 b.\ref{HAREC.b.1.14}\label{myHAREC.b.1.14},
 b.\ref{HAREC.b.1.15}\label{myHAREC.b.1.15},
 b.\ref{HAREC.b.1.16}\label{myHAREC.b.1.16},
 b.\ref{HAREC.b.1.17}\label{myHAREC.b.1.17},
 b.\ref{HAREC.b.1.18}\label{myHAREC.b.1.18},
 b.\ref{HAREC.b.1.19}\label{myHAREC.b.1.19},
 b.\ref{HAREC.b.1.20}\label{myHAREC.b.1.20},
 b.\ref{HAREC.b.1.21}\label{myHAREC.b.1.21},
 b.\ref{HAREC.b.1.22}\label{myHAREC.b.1.22},
 b.\ref{HAREC.b.1.23}\label{myHAREC.b.1.23},
 b.\ref{HAREC.b.1.24}\label{myHAREC.b.1.24},
 b.\ref{HAREC.b.1.25}\label{myHAREC.b.1.25},
 b.\ref{HAREC.b.1.26}\label{myHAREC.b.1.26}
}

Ibland behöver man göra förtydliganden genom att bokstavera.

Svenska radioamatörer skall kunna två fonetiska alfabeten.

% TODO Få tabellerna att hamna på ett vettigt ställe

\begin{table}[htbp]
  \label{tab:bokstavering-internationell}
  \caption{Det internationella fonetiska alfabetet}
  \begin{tabular}{lll}
    A & Alfa  & \underline{ALL} FA \\
    B & Bravo & \underline{BRA} VO \\
    C & Charlie & \underline{TJAR} LI \\
    D & Delta & \underline{DELL} TA \\
    E & Echo & \underline{ECK} Å \\
    F & Foxtrot & \underline{FÅCKS} TRÅTT \\
    G & Golf & \underline{GÅLF} \\
    H & Hotel & HÅ \underline{TELL} \\
    I & India & \underline{IN} DIA \\
    J & Juliett & \underline{DJO} LI \underline{ETT} \\
    K & Kilo & \underline{KI} LÅ \\
    L & Lima & \underline{LI} MA \\
    M & Mike & MAJK \\
    N & November & NO \underline{VEM} BÖ(RR) \\
    O & Oscar & \underline{ÅSSK} A(RR) \\
    P & Papa & PA \underline{PA} \\
    Q & Quebec & KE \underline{BECK} \\
    R & Romeo & \underline{RÅ} MIO \\
    S & Sierra & SI \underline{ERR} RA \\
    T & Tango & \underline{TÄNG} GÅ \\
    U & Uniform & \underline{JO} NI FORM \\
    V & Victor & \underline{VICK} TÖ(RR) \\
    W & Whiskey & \underline{OISS} KI \\
    X & X-ray & \underline{ECKS} REJ \\
    Y & Yankee & \underline{JÄNG} KI \\
    Z & Zulu & \underline{ZO} LO \\
    Å & Alfa Alfa & \underline{ALL} FA \underline{ALL} FA \\
    Ä & Alfa Echo & \underline{ALL} FA \underline{ECK} Å \\
    Ö & Oscar Echo & \underline{ÅSSK} A \underline{ECK} Å \\
    & & \\
    0 & Zero & \underline{ZE} RO \\
    1 & One & O \underline{ANN} \\
    2 & Two & TO \\
    3 & Three & TRI \\
    4 & Four & FÅR \\
    5 & Five & FAJV \\
    6 & Six & SICKS \\
    7 & Seven & \underline{SE} VEN \\
    9 & Nine & \underline{NAJ} NÖ(RR) \\
    & & \\
    , & Decimal & \underline{DE} SI MAL \\
    . & Stop & STOPP \\
    \multicolumn{3}{l}{Ungefärligt uttal. Betona det understrukna.}
  \end{tabular}
\end{table}


\begin{table}[htbp]
  \label{tab:bokstavering-svenska}
  \caption{Det svenska fonetiska alfabetet}
  \begin{tabular}{ll}
    A & Adam \\
    B & Bertil \\
    C & Ceasar \\
    D & David \\
    E & Erik \\
    F & Filip \\
    G & Gustav \\
    H & Helge \\
    I & Ivar \\
    J & Johan \\
    K & Kalle \\
    L & Ludvig \\
    M & Martin \\
    N & Niklas \\
    O & Olof \\
    P & Petter \\
    Q & Qvintus \\
    R & Rudolf \\
    S & Sigurd \\
    T & Tore \\
    U & Urban \\
    V & Viktor \\
    W & Wilhelm \\
    X & Xerxes \\
    Y & Yngve \\
    Å & Åke \\
    Ä & Ärlig \\
    Ö & Östen \\
    & \\
    0 & Nolla \\
    1 & Ett (inte etta) \\
    2 & Tvåa \\
    3 & Trea \\
    4 & Fyra \\
    5 & Femma \\
    6 & Sexa \\
    7 & Sju (inte sjua) \\
    8 & Åtta \\
    9 & Nia \\
    & \\
    , & Komma \\
    . & Punkt
  \end{tabular}
\end{table}
