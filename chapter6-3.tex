\subsection{Antenner för kortvåg}

\subsection{Mittmatad halvvågsantenn}

Se föregående avsnitt

\subsection{Ändmatad halvvågsantenn}

Utstrålningen från en halvvågsantenn är i
princip lika hur den än matas. En än d matat
halvvågsantenn fungerar m.a.p. strålningsriktningar på samma sätt som en mittmatat
Vid längre antenner blir strålningskaraktären däremot en annan.
Skillnaden mellan änd- och mittmatade
halvvågsdipoler är att anslutningsimpedansen är mycket högre i ändarna än i
mitten. För att mata antennen längst ut i ena
änden behövs en transmissionsledning med
hög impedans, varvid ledningens ena part
ansluts till antennen och den andra parten
lämnas fri. En sådan anordning kallas zeppantenn och användes först i luftskepp, s.k.
zeppelinare.

\subsection{Omvikt dipol (folded dipole)}

Bild II 6-12
En omvikt di pol kan ses som två eller flera
parallella element, som är sammankopplade i ändarna. Mittpunkten på ett av elementen är ansluten till antennledningen.
Matningsimpedansen för en omvikt /J2dipol med två element är c:a fyra gånger
högre än den för en enkel dipol, d.v.s. 200
- 300 Q. Den omvikta dipolen, som endast
fungerar på grundfrekvensen och på dess
udda övertoner, är relativt bredbandig. Matningsimpedansen kan ändras med sinsemellan olika diametrar på de ingående elementen samt med antalet parallellkopplade
element.

Bild II 6-12 Omvikt dipol

\subsection{Jordplanantenn}

Bild II 6-13
Jordplanantennen eller GP-antennen (GP
av ground plane) består av en lodrätstrålare
som den ena polen och flera sammankopplade A./4-radialer eller markplanet som den
andra polen.
GP-antennen är rundstrålande och har
vertikal polarisering. Dess relativt flacka utstrålning, i jämförelse med en horisontell
antenn, gör den lämpad för långa distanser.
Av mekaniska skäl används den mest på
högre frekvenser (14 MHz och högre).
Med horisontella radialer som jordplan
är matningsimpedansen c:a 35 Q. För att få
god impedansanpassning, t.ex. till en 50 Q
koaxialkabel som matarledning, görs radialerna sluttande nedåt i en lämplig vinkel.
Koaxialkabelns innerledare ansluts till
antennen och kabelskärmen till radialerna.
Om antennen placeras omedelbart ovan
markytan, kan marken användas som jordplan, särskilt om dess elektriska ledningsförmåga är god.
Bild II 6-14
Om antennelementet inte har en elektrisk längd av ')J4, kan längden anpassas
elektriskt på liknande sätt som beskrivits
tidigare i detta kapitel för dipolantenner.

Bild II 6-13 GP-antenn

116-9

GP med
seriekondensator

l >

GPmed
toppkapacitans

t

Bild II 6-14 GP-antenner med elektrisk längdanpassning

\subsection{Flerbands GP-antenner}

Antennen fungerar som 'A/4 GP-antenn
åtminstone på de lägsta banden. Den mekaniska längden på en flerbands GP för
kortvåg blir kort, 4 6.5 meter, vilket på de
lägre banden innebär dålig verkningsgrad
och liten bandbredd. Jämför med SVF-kurvorna på bilden. Flerbands G P-antenner för
upp till sju kortvågsband tillverkas.

En GP-antenn kan fås att fungera på flera
band genom inbyggnad av en spärrkrets i
antennelementetför tillkommande band och
av jordplansradialer med anpassad längd
eller med spärrkretsar även i jordplanet för
de banden.
Bild II 6-15

a

SVF

SVF

'

l

1\ J

3.5

- 1 - - - --

-r---··- --- -·· - -

MHz

3

'l'..

5
1.u

.B

.1

svF

SVF
3

.6

28

2 .4

.6

.8 29 .2 .4

MHz

1

-

l

.05

J
MHz

.10

SVF
J

GP
( 80 /40 l 20 /15 /10m )

Bild II 6-15 SVF-kurvor för flerbands GP-antenn
116-10

1.5
1.2
1.0

i""'oo.

14

-

.2

...,..,.,.
.3

MHz

Det finns flera principer för denna
antenntyp. l den typ som visas här
används spärrkretsar.

\subsection{Flerbands halvvågsantenner}
Bild II 6-i6

En vanligt förekommande flerbandsantenn
är W3DZZ-antennen (namnet efter konstruktörens anropssignal). Den är en oftast horisontellt upphängd dipolantenn för 80, 40,
20, i 5 och i O m-banden.
W3DZZ-antennen är c:a 33.6 meter lång
och har två spärrkretsar, symmetriskt utplacerade omkring matningspunkten. Matningen sker med koaxialkabel och balun.

Antennen har en matnings impedans av
c:a 50 Q på 80- och 40-metersbanden På de
högre banden är anpassningen inte optimal
- matningsimpedansen stiger där upp till c:a
120 Q. Många använder bl. a. av den anledningen inte W3DZZ-antennen på höga kortvågsband utan föredrar där en flerbandig
GP-antenn eller en riktantenn (Yagi, quad
m.fl.).

Fysiska data
6,7tm

Praktiskt utförande
Balun1:1

spärrkrets

l

spärrkrets

om möjligt 6 mtr
lodrätt nedåt

koaxialkabel

Strömfördelning

80m

40m

./---o------------~~
~~~~

Bild II 6-16 W3DZZ-antennen

116-11

ANTENNSYSTE
W3DZZ-antennens arbetssätt:
80m-bandet
Hela antennen fungerar som en A/2dipol med resonansfrekvensen 3.7 MHz.
Den mekaniska längden är 2 · 16.8 meter
och förlängs elektriskt med induktanserna
i spärrkretsarna, vilka f.ö. är ur resonans
på detta band.
40 m-bandet
Spärrkretsarna är i resonans och "kopplar bort" antenndelen utanför dem. Delen
där innanför fungerar som en A/2-dipol
med resonansfrekvensen 7.05 MHz.
20m-bandet
Hela antennen fungerar som 3A/2-dipol
med resonansfrekvensen 14.1 MHz.
15m-bandet
Hela antennen fungerar som 5A/2-dipol
med resonansfrekvensen 21.2 MHz.
10m-bandet
Hela antennen fungerar som 7A/2-dipol
med resonansfrekvensen 28.4 MHz.

